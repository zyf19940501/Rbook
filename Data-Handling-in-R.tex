% Options for packages loaded elsewhere
\PassOptionsToPackage{unicode}{hyperref}
\PassOptionsToPackage{hyphens}{url}
%
\documentclass[
]{book}
\usepackage{amsmath,amssymb}
\usepackage{lmodern}
\usepackage{ifxetex,ifluatex}
\ifnum 0\ifxetex 1\fi\ifluatex 1\fi=0 % if pdftex
  \usepackage[T1]{fontenc}
  \usepackage[utf8]{inputenc}
  \usepackage{textcomp} % provide euro and other symbols
\else % if luatex or xetex
  \usepackage{unicode-math}
  \defaultfontfeatures{Scale=MatchLowercase}
  \defaultfontfeatures[\rmfamily]{Ligatures=TeX,Scale=1}
\fi
% Use upquote if available, for straight quotes in verbatim environments
\IfFileExists{upquote.sty}{\usepackage{upquote}}{}
\IfFileExists{microtype.sty}{% use microtype if available
  \usepackage[]{microtype}
  \UseMicrotypeSet[protrusion]{basicmath} % disable protrusion for tt fonts
}{}
\makeatletter
\@ifundefined{KOMAClassName}{% if non-KOMA class
  \IfFileExists{parskip.sty}{%
    \usepackage{parskip}
  }{% else
    \setlength{\parindent}{0pt}
    \setlength{\parskip}{6pt plus 2pt minus 1pt}}
}{% if KOMA class
  \KOMAoptions{parskip=half}}
\makeatother
\usepackage{xcolor}
\IfFileExists{xurl.sty}{\usepackage{xurl}}{} % add URL line breaks if available
\IfFileExists{bookmark.sty}{\usepackage{bookmark}}{\usepackage{hyperref}}
\hypersetup{
  pdftitle={R语言学习笔记},
  pdfauthor={Yufei Zhong},
  hidelinks,
  pdfcreator={LaTeX via pandoc}}
\urlstyle{same} % disable monospaced font for URLs
\usepackage{color}
\usepackage{fancyvrb}
\newcommand{\VerbBar}{|}
\newcommand{\VERB}{\Verb[commandchars=\\\{\}]}
\DefineVerbatimEnvironment{Highlighting}{Verbatim}{commandchars=\\\{\}}
% Add ',fontsize=\small' for more characters per line
\usepackage{framed}
\definecolor{shadecolor}{RGB}{248,248,248}
\newenvironment{Shaded}{\begin{snugshade}}{\end{snugshade}}
\newcommand{\AlertTok}[1]{\textcolor[rgb]{0.94,0.16,0.16}{#1}}
\newcommand{\AnnotationTok}[1]{\textcolor[rgb]{0.56,0.35,0.01}{\textbf{\textit{#1}}}}
\newcommand{\AttributeTok}[1]{\textcolor[rgb]{0.77,0.63,0.00}{#1}}
\newcommand{\BaseNTok}[1]{\textcolor[rgb]{0.00,0.00,0.81}{#1}}
\newcommand{\BuiltInTok}[1]{#1}
\newcommand{\CharTok}[1]{\textcolor[rgb]{0.31,0.60,0.02}{#1}}
\newcommand{\CommentTok}[1]{\textcolor[rgb]{0.56,0.35,0.01}{\textit{#1}}}
\newcommand{\CommentVarTok}[1]{\textcolor[rgb]{0.56,0.35,0.01}{\textbf{\textit{#1}}}}
\newcommand{\ConstantTok}[1]{\textcolor[rgb]{0.00,0.00,0.00}{#1}}
\newcommand{\ControlFlowTok}[1]{\textcolor[rgb]{0.13,0.29,0.53}{\textbf{#1}}}
\newcommand{\DataTypeTok}[1]{\textcolor[rgb]{0.13,0.29,0.53}{#1}}
\newcommand{\DecValTok}[1]{\textcolor[rgb]{0.00,0.00,0.81}{#1}}
\newcommand{\DocumentationTok}[1]{\textcolor[rgb]{0.56,0.35,0.01}{\textbf{\textit{#1}}}}
\newcommand{\ErrorTok}[1]{\textcolor[rgb]{0.64,0.00,0.00}{\textbf{#1}}}
\newcommand{\ExtensionTok}[1]{#1}
\newcommand{\FloatTok}[1]{\textcolor[rgb]{0.00,0.00,0.81}{#1}}
\newcommand{\FunctionTok}[1]{\textcolor[rgb]{0.00,0.00,0.00}{#1}}
\newcommand{\ImportTok}[1]{#1}
\newcommand{\InformationTok}[1]{\textcolor[rgb]{0.56,0.35,0.01}{\textbf{\textit{#1}}}}
\newcommand{\KeywordTok}[1]{\textcolor[rgb]{0.13,0.29,0.53}{\textbf{#1}}}
\newcommand{\NormalTok}[1]{#1}
\newcommand{\OperatorTok}[1]{\textcolor[rgb]{0.81,0.36,0.00}{\textbf{#1}}}
\newcommand{\OtherTok}[1]{\textcolor[rgb]{0.56,0.35,0.01}{#1}}
\newcommand{\PreprocessorTok}[1]{\textcolor[rgb]{0.56,0.35,0.01}{\textit{#1}}}
\newcommand{\RegionMarkerTok}[1]{#1}
\newcommand{\SpecialCharTok}[1]{\textcolor[rgb]{0.00,0.00,0.00}{#1}}
\newcommand{\SpecialStringTok}[1]{\textcolor[rgb]{0.31,0.60,0.02}{#1}}
\newcommand{\StringTok}[1]{\textcolor[rgb]{0.31,0.60,0.02}{#1}}
\newcommand{\VariableTok}[1]{\textcolor[rgb]{0.00,0.00,0.00}{#1}}
\newcommand{\VerbatimStringTok}[1]{\textcolor[rgb]{0.31,0.60,0.02}{#1}}
\newcommand{\WarningTok}[1]{\textcolor[rgb]{0.56,0.35,0.01}{\textbf{\textit{#1}}}}
\usepackage{longtable,booktabs,array}
\usepackage{calc} % for calculating minipage widths
% Correct order of tables after \paragraph or \subparagraph
\usepackage{etoolbox}
\makeatletter
\patchcmd\longtable{\par}{\if@noskipsec\mbox{}\fi\par}{}{}
\makeatother
% Allow footnotes in longtable head/foot
\IfFileExists{footnotehyper.sty}{\usepackage{footnotehyper}}{\usepackage{footnote}}
\makesavenoteenv{longtable}
\usepackage{graphicx}
\makeatletter
\def\maxwidth{\ifdim\Gin@nat@width>\linewidth\linewidth\else\Gin@nat@width\fi}
\def\maxheight{\ifdim\Gin@nat@height>\textheight\textheight\else\Gin@nat@height\fi}
\makeatother
% Scale images if necessary, so that they will not overflow the page
% margins by default, and it is still possible to overwrite the defaults
% using explicit options in \includegraphics[width, height, ...]{}
\setkeys{Gin}{width=\maxwidth,height=\maxheight,keepaspectratio}
% Set default figure placement to htbp
\makeatletter
\def\fps@figure{htbp}
\makeatother
\usepackage[normalem]{ulem}
% Avoid problems with \sout in headers with hyperref
\pdfstringdefDisableCommands{\renewcommand{\sout}{}}
\setlength{\emergencystretch}{3em} % prevent overfull lines
\providecommand{\tightlist}{%
  \setlength{\itemsep}{0pt}\setlength{\parskip}{0pt}}
\setcounter{secnumdepth}{5}
\usepackage{booktabs}
\usepackage{fontspec}
\usepackage{multirow}
\usepackage{multicol}
\usepackage{colortbl}
\usepackage{hhline}
\usepackage{longtable}
\usepackage{array}
\usepackage{hyperref}
\ifluatex
  \usepackage{selnolig}  % disable illegal ligatures
\fi
\usepackage[]{natbib}
\bibliographystyle{apalike}

\title{R语言学习笔记}
\author{Yufei Zhong}
\date{2021-05-21}

\begin{document}
\maketitle

{
\setcounter{tocdepth}{1}
\tableofcontents
}
\hypertarget{welcome}{%
\chapter*{欢迎}\label{welcome}}
\addcontentsline{toc}{chapter}{欢迎}

本文主要是我作为商业数据分析师的R语言学习笔记,主要是数据清洗相关包的介绍。

使用R语言自动完成如下报表:

\providecommand{\docline}[3]{\noalign{\global\setlength{\arrayrulewidth}{#1}}\arrayrulecolor[HTML]{#2}\cline{#3}}

\setlength{\tabcolsep}{2pt}

\renewcommand*{\arraystretch}{1.5}

\begin{longtable}[c]{|p{0.86in}|p{0.75in}|p{1.13in}|p{0.80in}|p{1.13in}|p{0.80in}|p{0.88in}|p{0.84in}|p{0.84in}|p{0.88in}|p{0.75in}|p{0.75in}}



\hhline{>{\arrayrulecolor[HTML]{666666}\global\arrayrulewidth=1pt}->{\arrayrulecolor[HTML]{666666}\global\arrayrulewidth=1pt}->{\arrayrulecolor[HTML]{666666}\global\arrayrulewidth=1pt}->{\arrayrulecolor[HTML]{666666}\global\arrayrulewidth=1pt}->{\arrayrulecolor[HTML]{666666}\global\arrayrulewidth=1pt}->{\arrayrulecolor[HTML]{666666}\global\arrayrulewidth=1pt}->{\arrayrulecolor[HTML]{666666}\global\arrayrulewidth=1pt}->{\arrayrulecolor[HTML]{666666}\global\arrayrulewidth=1pt}->{\arrayrulecolor[HTML]{666666}\global\arrayrulewidth=1pt}->{\arrayrulecolor[HTML]{666666}\global\arrayrulewidth=1pt}->{\arrayrulecolor[HTML]{666666}\global\arrayrulewidth=1pt}->{\arrayrulecolor[HTML]{666666}\global\arrayrulewidth=1pt}-}

\multicolumn{1}{!{\color[HTML]{666666}\vrule width 1pt}>{\cellcolor[HTML]{E05297}\raggedright}p{\dimexpr 0.86in+0\tabcolsep+0\arrayrulewidth}}{\fontsize{11}{11}\selectfont{\textcolor[HTML]{FFFFFF}{\global\setmainfont{Arial}\textbf{一级部门}}}} & \multicolumn{1}{!{\color[HTML]{666666}\vrule width 1pt}>{\cellcolor[HTML]{E05297}\raggedright}p{\dimexpr 0.75in+0\tabcolsep+0\arrayrulewidth}}{\fontsize{11}{11}\selectfont{\textcolor[HTML]{FFFFFF}{\global\setmainfont{Arial}\textbf{分析大类}}}} & \multicolumn{1}{!{\color[HTML]{666666}\vrule width 1pt}>{\cellcolor[HTML]{E05297}\raggedleft}p{\dimexpr 1.13in+0\tabcolsep+0\arrayrulewidth}}{\fontsize{11}{11}\selectfont{\textcolor[HTML]{FFFFFF}{\global\setmainfont{Arial}\textbf{当前销额}}}} & \multicolumn{1}{!{\color[HTML]{666666}\vrule width 1pt}>{\cellcolor[HTML]{E05297}\raggedleft}p{\dimexpr 0.8in+0\tabcolsep+0\arrayrulewidth}}{\fontsize{11}{11}\selectfont{\textcolor[HTML]{FFFFFF}{\global\setmainfont{Arial}\textbf{当前\%}}}} & \multicolumn{1}{!{\color[HTML]{666666}\vrule width 1pt}>{\cellcolor[HTML]{E05297}\raggedleft}p{\dimexpr 1.13in+0\tabcolsep+0\arrayrulewidth}}{\fontsize{11}{11}\selectfont{\textcolor[HTML]{FFFFFF}{\global\setmainfont{Arial}\textbf{同比销额}}}} & \multicolumn{1}{!{\color[HTML]{666666}\vrule width 1pt}>{\cellcolor[HTML]{E05297}\raggedleft}p{\dimexpr 0.8in+0\tabcolsep+0\arrayrulewidth}}{\fontsize{11}{11}\selectfont{\textcolor[HTML]{FFFFFF}{\global\setmainfont{Arial}\textbf{同比\%}}}} & \multicolumn{1}{!{\color[HTML]{666666}\vrule width 1pt}>{\cellcolor[HTML]{E05297}\raggedleft}p{\dimexpr 0.88in+0\tabcolsep+0\arrayrulewidth}}{\fontsize{11}{11}\selectfont{\textcolor[HTML]{FFFFFF}{\global\setmainfont{Arial}\textbf{金额增长\%}}}} & \multicolumn{1}{!{\color[HTML]{666666}\vrule width 1pt}>{\cellcolor[HTML]{E05297}\raggedleft}p{\dimexpr 0.84in+0\tabcolsep+0\arrayrulewidth}}{\fontsize{11}{11}\selectfont{\textcolor[HTML]{FFFFFF}{\global\setmainfont{Arial}\textbf{当前销量}}}} & \multicolumn{1}{!{\color[HTML]{666666}\vrule width 1pt}>{\cellcolor[HTML]{E05297}\raggedleft}p{\dimexpr 0.84in+0\tabcolsep+0\arrayrulewidth}}{\fontsize{11}{11}\selectfont{\textcolor[HTML]{FFFFFF}{\global\setmainfont{Arial}\textbf{同比销量}}}} & \multicolumn{1}{!{\color[HTML]{666666}\vrule width 1pt}>{\cellcolor[HTML]{E05297}\raggedleft}p{\dimexpr 0.88in+0\tabcolsep+0\arrayrulewidth}}{\fontsize{11}{11}\selectfont{\textcolor[HTML]{FFFFFF}{\global\setmainfont{Arial}\textbf{销量同比\%}}}} & \multicolumn{1}{!{\color[HTML]{666666}\vrule width 1pt}>{\cellcolor[HTML]{E05297}\raggedleft}p{\dimexpr 0.75in+0\tabcolsep+0\arrayrulewidth}}{\fontsize{11}{11}\selectfont{\textcolor[HTML]{FFFFFF}{\global\setmainfont{Arial}\textbf{当前折扣}}}} & \multicolumn{1}{!{\color[HTML]{666666}\vrule width 1pt}>{\cellcolor[HTML]{E05297}\raggedleft}p{\dimexpr 0.75in+0\tabcolsep+0\arrayrulewidth}!{\color[HTML]{666666}\vrule width 1pt}}{\fontsize{11}{11}\selectfont{\textcolor[HTML]{FFFFFF}{\global\setmainfont{Arial}\textbf{同比折扣}}}} \\

\hhline{>{\arrayrulecolor[HTML]{666666}\global\arrayrulewidth=1pt}->{\arrayrulecolor[HTML]{666666}\global\arrayrulewidth=1pt}->{\arrayrulecolor[HTML]{666666}\global\arrayrulewidth=1pt}->{\arrayrulecolor[HTML]{666666}\global\arrayrulewidth=1pt}->{\arrayrulecolor[HTML]{666666}\global\arrayrulewidth=1pt}->{\arrayrulecolor[HTML]{666666}\global\arrayrulewidth=1pt}->{\arrayrulecolor[HTML]{666666}\global\arrayrulewidth=1pt}->{\arrayrulecolor[HTML]{666666}\global\arrayrulewidth=1pt}->{\arrayrulecolor[HTML]{666666}\global\arrayrulewidth=1pt}->{\arrayrulecolor[HTML]{666666}\global\arrayrulewidth=1pt}->{\arrayrulecolor[HTML]{666666}\global\arrayrulewidth=1pt}->{\arrayrulecolor[HTML]{666666}\global\arrayrulewidth=1pt}-}

\endfirsthead

\hhline{>{\arrayrulecolor[HTML]{666666}\global\arrayrulewidth=1pt}->{\arrayrulecolor[HTML]{666666}\global\arrayrulewidth=1pt}->{\arrayrulecolor[HTML]{666666}\global\arrayrulewidth=1pt}->{\arrayrulecolor[HTML]{666666}\global\arrayrulewidth=1pt}->{\arrayrulecolor[HTML]{666666}\global\arrayrulewidth=1pt}->{\arrayrulecolor[HTML]{666666}\global\arrayrulewidth=1pt}->{\arrayrulecolor[HTML]{666666}\global\arrayrulewidth=1pt}->{\arrayrulecolor[HTML]{666666}\global\arrayrulewidth=1pt}->{\arrayrulecolor[HTML]{666666}\global\arrayrulewidth=1pt}->{\arrayrulecolor[HTML]{666666}\global\arrayrulewidth=1pt}->{\arrayrulecolor[HTML]{666666}\global\arrayrulewidth=1pt}->{\arrayrulecolor[HTML]{666666}\global\arrayrulewidth=1pt}-}

\multicolumn{1}{!{\color[HTML]{666666}\vrule width 1pt}>{\cellcolor[HTML]{E05297}\raggedright}p{\dimexpr 0.86in+0\tabcolsep+0\arrayrulewidth}}{\fontsize{11}{11}\selectfont{\textcolor[HTML]{FFFFFF}{\global\setmainfont{Arial}\textbf{一级部门}}}} & \multicolumn{1}{!{\color[HTML]{666666}\vrule width 1pt}>{\cellcolor[HTML]{E05297}\raggedright}p{\dimexpr 0.75in+0\tabcolsep+0\arrayrulewidth}}{\fontsize{11}{11}\selectfont{\textcolor[HTML]{FFFFFF}{\global\setmainfont{Arial}\textbf{分析大类}}}} & \multicolumn{1}{!{\color[HTML]{666666}\vrule width 1pt}>{\cellcolor[HTML]{E05297}\raggedleft}p{\dimexpr 1.13in+0\tabcolsep+0\arrayrulewidth}}{\fontsize{11}{11}\selectfont{\textcolor[HTML]{FFFFFF}{\global\setmainfont{Arial}\textbf{当前销额}}}} & \multicolumn{1}{!{\color[HTML]{666666}\vrule width 1pt}>{\cellcolor[HTML]{E05297}\raggedleft}p{\dimexpr 0.8in+0\tabcolsep+0\arrayrulewidth}}{\fontsize{11}{11}\selectfont{\textcolor[HTML]{FFFFFF}{\global\setmainfont{Arial}\textbf{当前\%}}}} & \multicolumn{1}{!{\color[HTML]{666666}\vrule width 1pt}>{\cellcolor[HTML]{E05297}\raggedleft}p{\dimexpr 1.13in+0\tabcolsep+0\arrayrulewidth}}{\fontsize{11}{11}\selectfont{\textcolor[HTML]{FFFFFF}{\global\setmainfont{Arial}\textbf{同比销额}}}} & \multicolumn{1}{!{\color[HTML]{666666}\vrule width 1pt}>{\cellcolor[HTML]{E05297}\raggedleft}p{\dimexpr 0.8in+0\tabcolsep+0\arrayrulewidth}}{\fontsize{11}{11}\selectfont{\textcolor[HTML]{FFFFFF}{\global\setmainfont{Arial}\textbf{同比\%}}}} & \multicolumn{1}{!{\color[HTML]{666666}\vrule width 1pt}>{\cellcolor[HTML]{E05297}\raggedleft}p{\dimexpr 0.88in+0\tabcolsep+0\arrayrulewidth}}{\fontsize{11}{11}\selectfont{\textcolor[HTML]{FFFFFF}{\global\setmainfont{Arial}\textbf{金额增长\%}}}} & \multicolumn{1}{!{\color[HTML]{666666}\vrule width 1pt}>{\cellcolor[HTML]{E05297}\raggedleft}p{\dimexpr 0.84in+0\tabcolsep+0\arrayrulewidth}}{\fontsize{11}{11}\selectfont{\textcolor[HTML]{FFFFFF}{\global\setmainfont{Arial}\textbf{当前销量}}}} & \multicolumn{1}{!{\color[HTML]{666666}\vrule width 1pt}>{\cellcolor[HTML]{E05297}\raggedleft}p{\dimexpr 0.84in+0\tabcolsep+0\arrayrulewidth}}{\fontsize{11}{11}\selectfont{\textcolor[HTML]{FFFFFF}{\global\setmainfont{Arial}\textbf{同比销量}}}} & \multicolumn{1}{!{\color[HTML]{666666}\vrule width 1pt}>{\cellcolor[HTML]{E05297}\raggedleft}p{\dimexpr 0.88in+0\tabcolsep+0\arrayrulewidth}}{\fontsize{11}{11}\selectfont{\textcolor[HTML]{FFFFFF}{\global\setmainfont{Arial}\textbf{销量同比\%}}}} & \multicolumn{1}{!{\color[HTML]{666666}\vrule width 1pt}>{\cellcolor[HTML]{E05297}\raggedleft}p{\dimexpr 0.75in+0\tabcolsep+0\arrayrulewidth}}{\fontsize{11}{11}\selectfont{\textcolor[HTML]{FFFFFF}{\global\setmainfont{Arial}\textbf{当前折扣}}}} & \multicolumn{1}{!{\color[HTML]{666666}\vrule width 1pt}>{\cellcolor[HTML]{E05297}\raggedleft}p{\dimexpr 0.75in+0\tabcolsep+0\arrayrulewidth}!{\color[HTML]{666666}\vrule width 1pt}}{\fontsize{11}{11}\selectfont{\textcolor[HTML]{FFFFFF}{\global\setmainfont{Arial}\textbf{同比折扣}}}} \\

\hhline{>{\arrayrulecolor[HTML]{666666}\global\arrayrulewidth=1pt}->{\arrayrulecolor[HTML]{666666}\global\arrayrulewidth=1pt}->{\arrayrulecolor[HTML]{666666}\global\arrayrulewidth=1pt}->{\arrayrulecolor[HTML]{666666}\global\arrayrulewidth=1pt}->{\arrayrulecolor[HTML]{666666}\global\arrayrulewidth=1pt}->{\arrayrulecolor[HTML]{666666}\global\arrayrulewidth=1pt}->{\arrayrulecolor[HTML]{666666}\global\arrayrulewidth=1pt}->{\arrayrulecolor[HTML]{666666}\global\arrayrulewidth=1pt}->{\arrayrulecolor[HTML]{666666}\global\arrayrulewidth=1pt}->{\arrayrulecolor[HTML]{666666}\global\arrayrulewidth=1pt}->{\arrayrulecolor[HTML]{666666}\global\arrayrulewidth=1pt}->{\arrayrulecolor[HTML]{666666}\global\arrayrulewidth=1pt}-}\endhead



\multicolumn{12}{!{\color[HTML]{666666}\vrule width 1pt}>{\raggedright}p{\dimexpr 10.42in+22\tabcolsep+11\arrayrulewidth}}{\fontsize{11}{11}\selectfont{\textcolor[HTML]{000000}{\global\setmainfont{Arial}\textit{数据更新时间:2021-05-21}}}} \\

\hhline{>{\arrayrulecolor[HTML]{666666}\global\arrayrulewidth=1pt}->{\arrayrulecolor[HTML]{666666}\global\arrayrulewidth=1pt}->{\arrayrulecolor[HTML]{666666}\global\arrayrulewidth=1pt}->{\arrayrulecolor[HTML]{666666}\global\arrayrulewidth=1pt}->{\arrayrulecolor[HTML]{666666}\global\arrayrulewidth=1pt}->{\arrayrulecolor[HTML]{666666}\global\arrayrulewidth=1pt}->{\arrayrulecolor[HTML]{666666}\global\arrayrulewidth=1pt}->{\arrayrulecolor[HTML]{666666}\global\arrayrulewidth=1pt}->{\arrayrulecolor[HTML]{666666}\global\arrayrulewidth=1pt}->{\arrayrulecolor[HTML]{666666}\global\arrayrulewidth=1pt}->{\arrayrulecolor[HTML]{666666}\global\arrayrulewidth=1pt}->{\arrayrulecolor[HTML]{666666}\global\arrayrulewidth=1pt}-}\endfoot



\multicolumn{1}{!{\color[HTML]{666666}\vrule width 1pt}>{\centering}p{\dimexpr 0.86in+0\tabcolsep+0\arrayrulewidth}}{} & \multicolumn{1}{!{\color[HTML]{666666}\vrule width 1pt}>{\centering}p{\dimexpr 0.75in+0\tabcolsep+0\arrayrulewidth}}{\fontsize{11}{11}\selectfont{\textcolor[HTML]{000000}{\global\setmainfont{Arial}衬衣}}} & \multicolumn{1}{!{\color[HTML]{666666}\vrule width 1pt}>{\centering}p{\dimexpr 1.13in+0\tabcolsep+0\arrayrulewidth}}{\fontsize{11}{11}\selectfont{\textcolor[HTML]{000000}{\global\setmainfont{Arial}55,148,056}}} & \multicolumn{1}{!{\color[HTML]{666666}\vrule width 1pt}>{\centering}p{\dimexpr 0.8in+0\tabcolsep+0\arrayrulewidth}}{\fontsize{11}{11}\selectfont{\textcolor[HTML]{000000}{\global\setmainfont{Arial}35.4\%}}} & \multicolumn{1}{!{\color[HTML]{666666}\vrule width 1pt}>{\centering}p{\dimexpr 1.13in+0\tabcolsep+0\arrayrulewidth}}{\fontsize{11}{11}\selectfont{\textcolor[HTML]{000000}{\global\setmainfont{Arial}51,999,209}}} & \multicolumn{1}{!{\color[HTML]{666666}\vrule width 1pt}>{\centering}p{\dimexpr 0.8in+0\tabcolsep+0\arrayrulewidth}}{\fontsize{11}{11}\selectfont{\textcolor[HTML]{000000}{\global\setmainfont{Arial}28.1\%}}} & \multicolumn{1}{!{\color[HTML]{666666}\vrule width 1pt}>{\centering}p{\dimexpr 0.88in+0\tabcolsep+0\arrayrulewidth}}{\fontsize{11}{11}\selectfont{\textcolor[HTML]{000000}{\global\setmainfont{Arial}6.1\%}}} & \multicolumn{1}{!{\color[HTML]{666666}\vrule width 1pt}>{\centering}p{\dimexpr 0.84in+0\tabcolsep+0\arrayrulewidth}}{\fontsize{11}{11}\selectfont{\textcolor[HTML]{000000}{\global\setmainfont{Arial}282,426}}} & \multicolumn{1}{!{\color[HTML]{666666}\vrule width 1pt}>{\centering}p{\dimexpr 0.84in+0\tabcolsep+0\arrayrulewidth}}{\fontsize{11}{11}\selectfont{\textcolor[HTML]{000000}{\global\setmainfont{Arial}358,658}}} & \multicolumn{1}{!{\color[HTML]{666666}\vrule width 1pt}>{\centering}p{\dimexpr 0.88in+0\tabcolsep+0\arrayrulewidth}}{\fontsize{11}{11}\selectfont{\textcolor[HTML]{000000}{\global\setmainfont{Arial}-21.3\%}}} & \multicolumn{1}{!{\color[HTML]{666666}\vrule width 1pt}>{\centering}p{\dimexpr 0.75in+0\tabcolsep+0\arrayrulewidth}}{\fontsize{11}{11}\selectfont{\textcolor[HTML]{000000}{\global\setmainfont{Arial}81.0\%}}} & \multicolumn{1}{!{\color[HTML]{666666}\vrule width 1pt}>{\centering}p{\dimexpr 0.75in+0\tabcolsep+0\arrayrulewidth}!{\color[HTML]{666666}\vrule width 1pt}}{\fontsize{11}{11}\selectfont{\textcolor[HTML]{000000}{\global\setmainfont{Arial}60.1\%}}} \\

\hhline{~>{\arrayrulecolor[HTML]{666666}\global\arrayrulewidth=1pt}->{\arrayrulecolor[HTML]{666666}\global\arrayrulewidth=1pt}->{\arrayrulecolor[HTML]{666666}\global\arrayrulewidth=1pt}->{\arrayrulecolor[HTML]{666666}\global\arrayrulewidth=1pt}->{\arrayrulecolor[HTML]{666666}\global\arrayrulewidth=1pt}->{\arrayrulecolor[HTML]{666666}\global\arrayrulewidth=1pt}->{\arrayrulecolor[HTML]{666666}\global\arrayrulewidth=1pt}->{\arrayrulecolor[HTML]{666666}\global\arrayrulewidth=1pt}->{\arrayrulecolor[HTML]{666666}\global\arrayrulewidth=1pt}->{\arrayrulecolor[HTML]{666666}\global\arrayrulewidth=1pt}->{\arrayrulecolor[HTML]{666666}\global\arrayrulewidth=1pt}-}



\multicolumn{1}{!{\color[HTML]{666666}\vrule width 1pt}>{\centering}p{\dimexpr 0.86in+0\tabcolsep+0\arrayrulewidth}}{} & \multicolumn{1}{!{\color[HTML]{666666}\vrule width 1pt}>{\centering}p{\dimexpr 0.75in+0\tabcolsep+0\arrayrulewidth}}{\fontsize{11}{11}\selectfont{\textcolor[HTML]{000000}{\global\setmainfont{Arial}箱包}}} & \multicolumn{1}{!{\color[HTML]{666666}\vrule width 1pt}>{\centering}p{\dimexpr 1.13in+0\tabcolsep+0\arrayrulewidth}}{\fontsize{11}{11}\selectfont{\textcolor[HTML]{000000}{\global\setmainfont{Arial}72,439,038}}} & \multicolumn{1}{!{\color[HTML]{666666}\vrule width 1pt}>{\centering}p{\dimexpr 0.8in+0\tabcolsep+0\arrayrulewidth}}{\fontsize{11}{11}\selectfont{\textcolor[HTML]{000000}{\global\setmainfont{Arial}46.5\%}}} & \multicolumn{1}{!{\color[HTML]{666666}\vrule width 1pt}>{\centering}p{\dimexpr 1.13in+0\tabcolsep+0\arrayrulewidth}}{\fontsize{11}{11}\selectfont{\textcolor[HTML]{000000}{\global\setmainfont{Arial}92,585,626}}} & \multicolumn{1}{!{\color[HTML]{666666}\vrule width 1pt}>{\centering}p{\dimexpr 0.8in+0\tabcolsep+0\arrayrulewidth}}{\fontsize{11}{11}\selectfont{\textcolor[HTML]{000000}{\global\setmainfont{Arial}50.0\%}}} & \multicolumn{1}{!{\color[HTML]{666666}\vrule width 1pt}>{\centering}p{\dimexpr 0.88in+0\tabcolsep+0\arrayrulewidth}}{\fontsize{11}{11}\selectfont{\textcolor[HTML]{000000}{\global\setmainfont{Arial}-21.8\%}}} & \multicolumn{1}{!{\color[HTML]{666666}\vrule width 1pt}>{\centering}p{\dimexpr 0.84in+0\tabcolsep+0\arrayrulewidth}}{\fontsize{11}{11}\selectfont{\textcolor[HTML]{000000}{\global\setmainfont{Arial}124,584}}} & \multicolumn{1}{!{\color[HTML]{666666}\vrule width 1pt}>{\centering}p{\dimexpr 0.84in+0\tabcolsep+0\arrayrulewidth}}{\fontsize{11}{11}\selectfont{\textcolor[HTML]{000000}{\global\setmainfont{Arial}163,813}}} & \multicolumn{1}{!{\color[HTML]{666666}\vrule width 1pt}>{\centering}p{\dimexpr 0.88in+0\tabcolsep+0\arrayrulewidth}}{\fontsize{11}{11}\selectfont{\textcolor[HTML]{000000}{\global\setmainfont{Arial}-23.9\%}}} & \multicolumn{1}{!{\color[HTML]{666666}\vrule width 1pt}>{\centering}p{\dimexpr 0.75in+0\tabcolsep+0\arrayrulewidth}}{\fontsize{11}{11}\selectfont{\textcolor[HTML]{000000}{\global\setmainfont{Arial}92.4\%}}} & \multicolumn{1}{!{\color[HTML]{666666}\vrule width 1pt}>{\centering}p{\dimexpr 0.75in+0\tabcolsep+0\arrayrulewidth}!{\color[HTML]{666666}\vrule width 1pt}}{\fontsize{11}{11}\selectfont{\textcolor[HTML]{000000}{\global\setmainfont{Arial}93.0\%}}} \\

\hhline{~>{\arrayrulecolor[HTML]{666666}\global\arrayrulewidth=1pt}->{\arrayrulecolor[HTML]{666666}\global\arrayrulewidth=1pt}->{\arrayrulecolor[HTML]{666666}\global\arrayrulewidth=1pt}->{\arrayrulecolor[HTML]{666666}\global\arrayrulewidth=1pt}->{\arrayrulecolor[HTML]{666666}\global\arrayrulewidth=1pt}->{\arrayrulecolor[HTML]{666666}\global\arrayrulewidth=1pt}->{\arrayrulecolor[HTML]{666666}\global\arrayrulewidth=1pt}->{\arrayrulecolor[HTML]{666666}\global\arrayrulewidth=1pt}->{\arrayrulecolor[HTML]{666666}\global\arrayrulewidth=1pt}->{\arrayrulecolor[HTML]{666666}\global\arrayrulewidth=1pt}->{\arrayrulecolor[HTML]{666666}\global\arrayrulewidth=1pt}-}



\multicolumn{1}{!{\color[HTML]{666666}\vrule width 1pt}>{\centering}p{\dimexpr 0.86in+0\tabcolsep+0\arrayrulewidth}}{} & \multicolumn{1}{!{\color[HTML]{666666}\vrule width 1pt}>{\centering}p{\dimexpr 0.75in+0\tabcolsep+0\arrayrulewidth}}{\fontsize{11}{11}\selectfont{\textcolor[HTML]{000000}{\global\setmainfont{Arial}篮球鞋}}} & \multicolumn{1}{!{\color[HTML]{666666}\vrule width 1pt}>{\centering}p{\dimexpr 1.13in+0\tabcolsep+0\arrayrulewidth}}{\fontsize{11}{11}\selectfont{\textcolor[HTML]{000000}{\global\setmainfont{Arial}21,872,495}}} & \multicolumn{1}{!{\color[HTML]{666666}\vrule width 1pt}>{\centering}p{\dimexpr 0.8in+0\tabcolsep+0\arrayrulewidth}}{\fontsize{11}{11}\selectfont{\textcolor[HTML]{000000}{\global\setmainfont{Arial}14.0\%}}} & \multicolumn{1}{!{\color[HTML]{666666}\vrule width 1pt}>{\centering}p{\dimexpr 1.13in+0\tabcolsep+0\arrayrulewidth}}{\fontsize{11}{11}\selectfont{\textcolor[HTML]{000000}{\global\setmainfont{Arial}33,590,835}}} & \multicolumn{1}{!{\color[HTML]{666666}\vrule width 1pt}>{\centering}p{\dimexpr 0.8in+0\tabcolsep+0\arrayrulewidth}}{\fontsize{11}{11}\selectfont{\textcolor[HTML]{000000}{\global\setmainfont{Arial}18.2\%}}} & \multicolumn{1}{!{\color[HTML]{666666}\vrule width 1pt}>{\centering}p{\dimexpr 0.88in+0\tabcolsep+0\arrayrulewidth}}{\fontsize{11}{11}\selectfont{\textcolor[HTML]{000000}{\global\setmainfont{Arial}-34.9\%}}} & \multicolumn{1}{!{\color[HTML]{666666}\vrule width 1pt}>{\centering}p{\dimexpr 0.84in+0\tabcolsep+0\arrayrulewidth}}{\fontsize{11}{11}\selectfont{\textcolor[HTML]{000000}{\global\setmainfont{Arial}44,260}}} & \multicolumn{1}{!{\color[HTML]{666666}\vrule width 1pt}>{\centering}p{\dimexpr 0.84in+0\tabcolsep+0\arrayrulewidth}}{\fontsize{11}{11}\selectfont{\textcolor[HTML]{000000}{\global\setmainfont{Arial}70,666}}} & \multicolumn{1}{!{\color[HTML]{666666}\vrule width 1pt}>{\centering}p{\dimexpr 0.88in+0\tabcolsep+0\arrayrulewidth}}{\fontsize{11}{11}\selectfont{\textcolor[HTML]{000000}{\global\setmainfont{Arial}-37.4\%}}} & \multicolumn{1}{!{\color[HTML]{666666}\vrule width 1pt}>{\centering}p{\dimexpr 0.75in+0\tabcolsep+0\arrayrulewidth}}{\fontsize{11}{11}\selectfont{\textcolor[HTML]{000000}{\global\setmainfont{Arial}87.3\%}}} & \multicolumn{1}{!{\color[HTML]{666666}\vrule width 1pt}>{\centering}p{\dimexpr 0.75in+0\tabcolsep+0\arrayrulewidth}!{\color[HTML]{666666}\vrule width 1pt}}{\fontsize{11}{11}\selectfont{\textcolor[HTML]{000000}{\global\setmainfont{Arial}89.0\%}}} \\

\hhline{~>{\arrayrulecolor[HTML]{666666}\global\arrayrulewidth=1pt}->{\arrayrulecolor[HTML]{666666}\global\arrayrulewidth=1pt}->{\arrayrulecolor[HTML]{666666}\global\arrayrulewidth=1pt}->{\arrayrulecolor[HTML]{666666}\global\arrayrulewidth=1pt}->{\arrayrulecolor[HTML]{666666}\global\arrayrulewidth=1pt}->{\arrayrulecolor[HTML]{666666}\global\arrayrulewidth=1pt}->{\arrayrulecolor[HTML]{666666}\global\arrayrulewidth=1pt}->{\arrayrulecolor[HTML]{666666}\global\arrayrulewidth=1pt}->{\arrayrulecolor[HTML]{666666}\global\arrayrulewidth=1pt}->{\arrayrulecolor[HTML]{666666}\global\arrayrulewidth=1pt}->{\arrayrulecolor[HTML]{666666}\global\arrayrulewidth=1pt}-}



\multicolumn{1}{!{\color[HTML]{666666}\vrule width 1pt}>{\centering}p{\dimexpr 0.86in+0\tabcolsep+0\arrayrulewidth}}{} & \multicolumn{1}{!{\color[HTML]{666666}\vrule width 1pt}>{\centering}p{\dimexpr 0.75in+0\tabcolsep+0\arrayrulewidth}}{\fontsize{11}{11}\selectfont{\textcolor[HTML]{000000}{\global\setmainfont{Arial}太阳镜}}} & \multicolumn{1}{!{\color[HTML]{666666}\vrule width 1pt}>{\centering}p{\dimexpr 1.13in+0\tabcolsep+0\arrayrulewidth}}{\fontsize{11}{11}\selectfont{\textcolor[HTML]{000000}{\global\setmainfont{Arial}1,526,908}}} & \multicolumn{1}{!{\color[HTML]{666666}\vrule width 1pt}>{\centering}p{\dimexpr 0.8in+0\tabcolsep+0\arrayrulewidth}}{\fontsize{11}{11}\selectfont{\textcolor[HTML]{000000}{\global\setmainfont{Arial}1.0\%}}} & \multicolumn{1}{!{\color[HTML]{666666}\vrule width 1pt}>{\centering}p{\dimexpr 1.13in+0\tabcolsep+0\arrayrulewidth}}{\fontsize{11}{11}\selectfont{\textcolor[HTML]{000000}{\global\setmainfont{Arial}952,934}}} & \multicolumn{1}{!{\color[HTML]{666666}\vrule width 1pt}>{\centering}p{\dimexpr 0.8in+0\tabcolsep+0\arrayrulewidth}}{\fontsize{11}{11}\selectfont{\textcolor[HTML]{000000}{\global\setmainfont{Arial}0.5\%}}} & \multicolumn{1}{!{\color[HTML]{666666}\vrule width 1pt}>{\centering}p{\dimexpr 0.88in+0\tabcolsep+0\arrayrulewidth}}{\fontsize{11}{11}\selectfont{\textcolor[HTML]{000000}{\global\setmainfont{Arial}60.2\%}}} & \multicolumn{1}{!{\color[HTML]{666666}\vrule width 1pt}>{\centering}p{\dimexpr 0.84in+0\tabcolsep+0\arrayrulewidth}}{\fontsize{11}{11}\selectfont{\textcolor[HTML]{000000}{\global\setmainfont{Arial}6,609}}} & \multicolumn{1}{!{\color[HTML]{666666}\vrule width 1pt}>{\centering}p{\dimexpr 0.84in+0\tabcolsep+0\arrayrulewidth}}{\fontsize{11}{11}\selectfont{\textcolor[HTML]{000000}{\global\setmainfont{Arial}3,944}}} & \multicolumn{1}{!{\color[HTML]{666666}\vrule width 1pt}>{\centering}p{\dimexpr 0.88in+0\tabcolsep+0\arrayrulewidth}}{\fontsize{11}{11}\selectfont{\textcolor[HTML]{000000}{\global\setmainfont{Arial}67.6\%}}} & \multicolumn{1}{!{\color[HTML]{666666}\vrule width 1pt}>{\centering}p{\dimexpr 0.75in+0\tabcolsep+0\arrayrulewidth}}{\fontsize{11}{11}\selectfont{\textcolor[HTML]{000000}{\global\setmainfont{Arial}94.0\%}}} & \multicolumn{1}{!{\color[HTML]{666666}\vrule width 1pt}>{\centering}p{\dimexpr 0.75in+0\tabcolsep+0\arrayrulewidth}!{\color[HTML]{666666}\vrule width 1pt}}{\fontsize{11}{11}\selectfont{\textcolor[HTML]{000000}{\global\setmainfont{Arial}97.0\%}}} \\

\hhline{~>{\arrayrulecolor[HTML]{666666}\global\arrayrulewidth=1pt}->{\arrayrulecolor[HTML]{666666}\global\arrayrulewidth=1pt}->{\arrayrulecolor[HTML]{666666}\global\arrayrulewidth=1pt}->{\arrayrulecolor[HTML]{666666}\global\arrayrulewidth=1pt}->{\arrayrulecolor[HTML]{666666}\global\arrayrulewidth=1pt}->{\arrayrulecolor[HTML]{666666}\global\arrayrulewidth=1pt}->{\arrayrulecolor[HTML]{666666}\global\arrayrulewidth=1pt}->{\arrayrulecolor[HTML]{666666}\global\arrayrulewidth=1pt}->{\arrayrulecolor[HTML]{666666}\global\arrayrulewidth=1pt}->{\arrayrulecolor[HTML]{666666}\global\arrayrulewidth=1pt}->{\arrayrulecolor[HTML]{666666}\global\arrayrulewidth=1pt}-}



\multicolumn{1}{!{\color[HTML]{666666}\vrule width 1pt}>{\centering}p{\dimexpr 0.86in+0\tabcolsep+0\arrayrulewidth}}{} & \multicolumn{1}{!{\color[HTML]{666666}\vrule width 1pt}>{\centering}p{\dimexpr 0.75in+0\tabcolsep+0\arrayrulewidth}}{\fontsize{11}{11}\selectfont{\textcolor[HTML]{000000}{\global\setmainfont{Arial}周边商品}}} & \multicolumn{1}{!{\color[HTML]{666666}\vrule width 1pt}>{\centering}p{\dimexpr 1.13in+0\tabcolsep+0\arrayrulewidth}}{\fontsize{11}{11}\selectfont{\textcolor[HTML]{000000}{\global\setmainfont{Arial}1,691,810}}} & \multicolumn{1}{!{\color[HTML]{666666}\vrule width 1pt}>{\centering}p{\dimexpr 0.8in+0\tabcolsep+0\arrayrulewidth}}{\fontsize{11}{11}\selectfont{\textcolor[HTML]{000000}{\global\setmainfont{Arial}1.1\%}}} & \multicolumn{1}{!{\color[HTML]{666666}\vrule width 1pt}>{\centering}p{\dimexpr 1.13in+0\tabcolsep+0\arrayrulewidth}}{\fontsize{11}{11}\selectfont{\textcolor[HTML]{000000}{\global\setmainfont{Arial}372,050}}} & \multicolumn{1}{!{\color[HTML]{666666}\vrule width 1pt}>{\centering}p{\dimexpr 0.8in+0\tabcolsep+0\arrayrulewidth}}{\fontsize{11}{11}\selectfont{\textcolor[HTML]{000000}{\global\setmainfont{Arial}0.2\%}}} & \multicolumn{1}{!{\color[HTML]{666666}\vrule width 1pt}>{\centering}p{\dimexpr 0.88in+0\tabcolsep+0\arrayrulewidth}}{\fontsize{11}{11}\selectfont{\textcolor[HTML]{000000}{\global\setmainfont{Arial}354.7\%}}} & \multicolumn{1}{!{\color[HTML]{666666}\vrule width 1pt}>{\centering}p{\dimexpr 0.84in+0\tabcolsep+0\arrayrulewidth}}{\fontsize{11}{11}\selectfont{\textcolor[HTML]{000000}{\global\setmainfont{Arial}11,777}}} & \multicolumn{1}{!{\color[HTML]{666666}\vrule width 1pt}>{\centering}p{\dimexpr 0.84in+0\tabcolsep+0\arrayrulewidth}}{\fontsize{11}{11}\selectfont{\textcolor[HTML]{000000}{\global\setmainfont{Arial}2,249}}} & \multicolumn{1}{!{\color[HTML]{666666}\vrule width 1pt}>{\centering}p{\dimexpr 0.88in+0\tabcolsep+0\arrayrulewidth}}{\fontsize{11}{11}\selectfont{\textcolor[HTML]{000000}{\global\setmainfont{Arial}423.7\%}}} & \multicolumn{1}{!{\color[HTML]{666666}\vrule width 1pt}>{\centering}p{\dimexpr 0.75in+0\tabcolsep+0\arrayrulewidth}}{\fontsize{11}{11}\selectfont{\textcolor[HTML]{000000}{\global\setmainfont{Arial}87.0\%}}} & \multicolumn{1}{!{\color[HTML]{666666}\vrule width 1pt}>{\centering}p{\dimexpr 0.75in+0\tabcolsep+0\arrayrulewidth}!{\color[HTML]{666666}\vrule width 1pt}}{\fontsize{11}{11}\selectfont{\textcolor[HTML]{000000}{\global\setmainfont{Arial}87.0\%}}} \\

\hhline{~>{\arrayrulecolor[HTML]{666666}\global\arrayrulewidth=1pt}->{\arrayrulecolor[HTML]{666666}\global\arrayrulewidth=1pt}->{\arrayrulecolor[HTML]{666666}\global\arrayrulewidth=1pt}->{\arrayrulecolor[HTML]{666666}\global\arrayrulewidth=1pt}->{\arrayrulecolor[HTML]{666666}\global\arrayrulewidth=1pt}->{\arrayrulecolor[HTML]{666666}\global\arrayrulewidth=1pt}->{\arrayrulecolor[HTML]{666666}\global\arrayrulewidth=1pt}->{\arrayrulecolor[HTML]{666666}\global\arrayrulewidth=1pt}->{\arrayrulecolor[HTML]{666666}\global\arrayrulewidth=1pt}->{\arrayrulecolor[HTML]{666666}\global\arrayrulewidth=1pt}->{\arrayrulecolor[HTML]{666666}\global\arrayrulewidth=1pt}-}



\multicolumn{1}{!{\color[HTML]{666666}\vrule width 1pt}>{\centering}p{\dimexpr 0.86in+0\tabcolsep+0\arrayrulewidth}}{} & \multicolumn{1}{!{\color[HTML]{666666}\vrule width 1pt}>{\centering}p{\dimexpr 0.75in+0\tabcolsep+0\arrayrulewidth}}{\fontsize{11}{11}\selectfont{\textcolor[HTML]{000000}{\global\setmainfont{Arial}袜子}}} & \multicolumn{1}{!{\color[HTML]{666666}\vrule width 1pt}>{\centering}p{\dimexpr 1.13in+0\tabcolsep+0\arrayrulewidth}}{\fontsize{11}{11}\selectfont{\textcolor[HTML]{000000}{\global\setmainfont{Arial}692,591}}} & \multicolumn{1}{!{\color[HTML]{666666}\vrule width 1pt}>{\centering}p{\dimexpr 0.8in+0\tabcolsep+0\arrayrulewidth}}{\fontsize{11}{11}\selectfont{\textcolor[HTML]{000000}{\global\setmainfont{Arial}0.4\%}}} & \multicolumn{1}{!{\color[HTML]{666666}\vrule width 1pt}>{\centering}p{\dimexpr 1.13in+0\tabcolsep+0\arrayrulewidth}}{\fontsize{11}{11}\selectfont{\textcolor[HTML]{000000}{\global\setmainfont{Arial}587,555}}} & \multicolumn{1}{!{\color[HTML]{666666}\vrule width 1pt}>{\centering}p{\dimexpr 0.8in+0\tabcolsep+0\arrayrulewidth}}{\fontsize{11}{11}\selectfont{\textcolor[HTML]{000000}{\global\setmainfont{Arial}0.3\%}}} & \multicolumn{1}{!{\color[HTML]{666666}\vrule width 1pt}>{\centering}p{\dimexpr 0.88in+0\tabcolsep+0\arrayrulewidth}}{\fontsize{11}{11}\selectfont{\textcolor[HTML]{000000}{\global\setmainfont{Arial}17.9\%}}} & \multicolumn{1}{!{\color[HTML]{666666}\vrule width 1pt}>{\centering}p{\dimexpr 0.84in+0\tabcolsep+0\arrayrulewidth}}{\fontsize{11}{11}\selectfont{\textcolor[HTML]{000000}{\global\setmainfont{Arial}243,679}}} & \multicolumn{1}{!{\color[HTML]{666666}\vrule width 1pt}>{\centering}p{\dimexpr 0.84in+0\tabcolsep+0\arrayrulewidth}}{\fontsize{11}{11}\selectfont{\textcolor[HTML]{000000}{\global\setmainfont{Arial}271,089}}} & \multicolumn{1}{!{\color[HTML]{666666}\vrule width 1pt}>{\centering}p{\dimexpr 0.88in+0\tabcolsep+0\arrayrulewidth}}{\fontsize{11}{11}\selectfont{\textcolor[HTML]{000000}{\global\setmainfont{Arial}-10.1\%}}} & \multicolumn{1}{!{\color[HTML]{666666}\vrule width 1pt}>{\centering}p{\dimexpr 0.75in+0\tabcolsep+0\arrayrulewidth}}{\fontsize{11}{11}\selectfont{\textcolor[HTML]{000000}{\global\setmainfont{Arial}94.7\%}}} & \multicolumn{1}{!{\color[HTML]{666666}\vrule width 1pt}>{\centering}p{\dimexpr 0.75in+0\tabcolsep+0\arrayrulewidth}!{\color[HTML]{666666}\vrule width 1pt}}{\fontsize{11}{11}\selectfont{\textcolor[HTML]{000000}{\global\setmainfont{Arial}22.2\%}}} \\

\hhline{~>{\arrayrulecolor[HTML]{666666}\global\arrayrulewidth=1pt}->{\arrayrulecolor[HTML]{666666}\global\arrayrulewidth=1pt}->{\arrayrulecolor[HTML]{666666}\global\arrayrulewidth=1pt}->{\arrayrulecolor[HTML]{666666}\global\arrayrulewidth=1pt}->{\arrayrulecolor[HTML]{666666}\global\arrayrulewidth=1pt}->{\arrayrulecolor[HTML]{666666}\global\arrayrulewidth=1pt}->{\arrayrulecolor[HTML]{666666}\global\arrayrulewidth=1pt}->{\arrayrulecolor[HTML]{666666}\global\arrayrulewidth=1pt}->{\arrayrulecolor[HTML]{666666}\global\arrayrulewidth=1pt}->{\arrayrulecolor[HTML]{666666}\global\arrayrulewidth=1pt}->{\arrayrulecolor[HTML]{666666}\global\arrayrulewidth=1pt}-}



\multicolumn{1}{!{\color[HTML]{666666}\vrule width 1pt}>{\centering}p{\dimexpr 0.86in+0\tabcolsep+0\arrayrulewidth}}{\multirow[c]{-7}{*}{\fontsize{11}{11}\selectfont{\textcolor[HTML]{000000}{\global\setmainfont{Arial}事业部}}}} & \multicolumn{1}{!{\color[HTML]{666666}\vrule width 1pt}>{\centering}p{\dimexpr 0.75in+0\tabcolsep+0\arrayrulewidth}}{\fontsize{11}{11}\selectfont{\textcolor[HTML]{000000}{\global\setmainfont{Arial}其它}}} & \multicolumn{1}{!{\color[HTML]{666666}\vrule width 1pt}>{\centering}p{\dimexpr 1.13in+0\tabcolsep+0\arrayrulewidth}}{\fontsize{11}{11}\selectfont{\textcolor[HTML]{000000}{\global\setmainfont{Arial}2,312,950}}} & \multicolumn{1}{!{\color[HTML]{666666}\vrule width 1pt}>{\centering}p{\dimexpr 0.8in+0\tabcolsep+0\arrayrulewidth}}{\fontsize{11}{11}\selectfont{\textcolor[HTML]{000000}{\global\setmainfont{Arial}1.5\%}}} & \multicolumn{1}{!{\color[HTML]{666666}\vrule width 1pt}>{\centering}p{\dimexpr 1.13in+0\tabcolsep+0\arrayrulewidth}}{\fontsize{11}{11}\selectfont{\textcolor[HTML]{000000}{\global\setmainfont{Arial}4,946,510}}} & \multicolumn{1}{!{\color[HTML]{666666}\vrule width 1pt}>{\centering}p{\dimexpr 0.8in+0\tabcolsep+0\arrayrulewidth}}{\fontsize{11}{11}\selectfont{\textcolor[HTML]{000000}{\global\setmainfont{Arial}2.7\%}}} & \multicolumn{1}{!{\color[HTML]{666666}\vrule width 1pt}>{\centering}p{\dimexpr 0.88in+0\tabcolsep+0\arrayrulewidth}}{\fontsize{11}{11}\selectfont{\textcolor[HTML]{000000}{\global\setmainfont{Arial}-53.2\%}}} & \multicolumn{1}{!{\color[HTML]{666666}\vrule width 1pt}>{\centering}p{\dimexpr 0.84in+0\tabcolsep+0\arrayrulewidth}}{\fontsize{11}{11}\selectfont{\textcolor[HTML]{000000}{\global\setmainfont{Arial}5,581}}} & \multicolumn{1}{!{\color[HTML]{666666}\vrule width 1pt}>{\centering}p{\dimexpr 0.84in+0\tabcolsep+0\arrayrulewidth}}{\fontsize{11}{11}\selectfont{\textcolor[HTML]{000000}{\global\setmainfont{Arial}10,128}}} & \multicolumn{1}{!{\color[HTML]{666666}\vrule width 1pt}>{\centering}p{\dimexpr 0.88in+0\tabcolsep+0\arrayrulewidth}}{\fontsize{11}{11}\selectfont{\textcolor[HTML]{000000}{\global\setmainfont{Arial}-44.9\%}}} & \multicolumn{1}{!{\color[HTML]{666666}\vrule width 1pt}>{\centering}p{\dimexpr 0.75in+0\tabcolsep+0\arrayrulewidth}}{\fontsize{11}{11}\selectfont{\textcolor[HTML]{000000}{\global\setmainfont{Arial}94.0\%}}} & \multicolumn{1}{!{\color[HTML]{666666}\vrule width 1pt}>{\centering}p{\dimexpr 0.75in+0\tabcolsep+0\arrayrulewidth}!{\color[HTML]{666666}\vrule width 1pt}}{\fontsize{11}{11}\selectfont{\textcolor[HTML]{000000}{\global\setmainfont{Arial}93.1\%}}} \\

\hhline{>{\arrayrulecolor[HTML]{666666}\global\arrayrulewidth=1pt}->{\arrayrulecolor[HTML]{666666}\global\arrayrulewidth=1pt}->{\arrayrulecolor[HTML]{666666}\global\arrayrulewidth=1pt}->{\arrayrulecolor[HTML]{666666}\global\arrayrulewidth=1pt}->{\arrayrulecolor[HTML]{666666}\global\arrayrulewidth=1pt}->{\arrayrulecolor[HTML]{666666}\global\arrayrulewidth=1pt}->{\arrayrulecolor[HTML]{666666}\global\arrayrulewidth=1pt}->{\arrayrulecolor[HTML]{666666}\global\arrayrulewidth=1pt}->{\arrayrulecolor[HTML]{666666}\global\arrayrulewidth=1pt}->{\arrayrulecolor[HTML]{666666}\global\arrayrulewidth=1pt}->{\arrayrulecolor[HTML]{666666}\global\arrayrulewidth=1pt}->{\arrayrulecolor[HTML]{666666}\global\arrayrulewidth=1pt}-}



\multicolumn{1}{!{\color[HTML]{666666}\vrule width 1pt}>{\centering}p{\dimexpr 0.86in+0\tabcolsep+0\arrayrulewidth}}{\fontsize{11}{11}\selectfont{\textcolor[HTML]{000000}{\global\setmainfont{Arial}事业部汇总}}} & \multicolumn{1}{!{\color[HTML]{666666}\vrule width 1pt}>{\centering}p{\dimexpr 0.75in+0\tabcolsep+0\arrayrulewidth}}{\fontsize{11}{11}\selectfont{\textcolor[HTML]{000000}{\global\setmainfont{Arial}}}} & \multicolumn{1}{!{\color[HTML]{666666}\vrule width 1pt}>{\centering}p{\dimexpr 1.13in+0\tabcolsep+0\arrayrulewidth}}{\fontsize{11}{11}\selectfont{\textcolor[HTML]{000000}{\global\setmainfont{Arial}155,683,848}}} & \multicolumn{1}{!{\color[HTML]{666666}\vrule width 1pt}>{\centering}p{\dimexpr 0.8in+0\tabcolsep+0\arrayrulewidth}}{\fontsize{11}{11}\selectfont{\textcolor[HTML]{000000}{\global\setmainfont{Arial}100.0\%}}} & \multicolumn{1}{!{\color[HTML]{666666}\vrule width 1pt}>{\centering}p{\dimexpr 1.13in+0\tabcolsep+0\arrayrulewidth}}{\fontsize{11}{11}\selectfont{\textcolor[HTML]{000000}{\global\setmainfont{Arial}185,034,718}}} & \multicolumn{1}{!{\color[HTML]{666666}\vrule width 1pt}>{\centering}p{\dimexpr 0.8in+0\tabcolsep+0\arrayrulewidth}}{\fontsize{11}{11}\selectfont{\textcolor[HTML]{000000}{\global\setmainfont{Arial}100.0\%}}} & \multicolumn{1}{!{\color[HTML]{666666}\vrule width 1pt}>{\centering}p{\dimexpr 0.88in+0\tabcolsep+0\arrayrulewidth}}{\fontsize{11}{11}\selectfont{\textcolor[HTML]{000000}{\global\setmainfont{Arial}-15.9\%}}} & \multicolumn{1}{!{\color[HTML]{666666}\vrule width 1pt}>{\centering}p{\dimexpr 0.84in+0\tabcolsep+0\arrayrulewidth}}{\fontsize{11}{11}\selectfont{\textcolor[HTML]{000000}{\global\setmainfont{Arial}718,916}}} & \multicolumn{1}{!{\color[HTML]{666666}\vrule width 1pt}>{\centering}p{\dimexpr 0.84in+0\tabcolsep+0\arrayrulewidth}}{\fontsize{11}{11}\selectfont{\textcolor[HTML]{000000}{\global\setmainfont{Arial}880,547}}} & \multicolumn{1}{!{\color[HTML]{666666}\vrule width 1pt}>{\centering}p{\dimexpr 0.88in+0\tabcolsep+0\arrayrulewidth}}{\fontsize{11}{11}\selectfont{\textcolor[HTML]{000000}{\global\setmainfont{Arial}-18.4\%}}} & \multicolumn{1}{!{\color[HTML]{666666}\vrule width 1pt}>{\centering}p{\dimexpr 0.75in+0\tabcolsep+0\arrayrulewidth}}{\fontsize{11}{11}\selectfont{\textcolor[HTML]{000000}{\global\setmainfont{Arial}87.3\%}}} & \multicolumn{1}{!{\color[HTML]{666666}\vrule width 1pt}>{\centering}p{\dimexpr 0.75in+0\tabcolsep+0\arrayrulewidth}!{\color[HTML]{666666}\vrule width 1pt}}{\fontsize{11}{11}\selectfont{\textcolor[HTML]{000000}{\global\setmainfont{Arial}79.3\%}}} \\

\hhline{>{\arrayrulecolor[HTML]{666666}\global\arrayrulewidth=1pt}->{\arrayrulecolor[HTML]{666666}\global\arrayrulewidth=1pt}->{\arrayrulecolor[HTML]{666666}\global\arrayrulewidth=1pt}->{\arrayrulecolor[HTML]{666666}\global\arrayrulewidth=1pt}->{\arrayrulecolor[HTML]{666666}\global\arrayrulewidth=1pt}->{\arrayrulecolor[HTML]{666666}\global\arrayrulewidth=1pt}->{\arrayrulecolor[HTML]{666666}\global\arrayrulewidth=1pt}->{\arrayrulecolor[HTML]{666666}\global\arrayrulewidth=1pt}->{\arrayrulecolor[HTML]{666666}\global\arrayrulewidth=1pt}->{\arrayrulecolor[HTML]{666666}\global\arrayrulewidth=1pt}->{\arrayrulecolor[HTML]{666666}\global\arrayrulewidth=1pt}->{\arrayrulecolor[HTML]{666666}\global\arrayrulewidth=1pt}-}

\end{longtable}

\hypertarget{abstract}{%
\section*{内容概要}\label{abstract}}
\addcontentsline{toc}{section}{内容概要}

主要内容是为了将数据报表,数据报告,数据可视化等需求利用R语言自动化。

\begin{itemize}
\tightlist
\item
  数据导入导出
\item
  数据操作 dplyr
\item
  数据整洁 tidyr
\item
  字符处理
\item
  日期时间处理
\item
  因子处理
\item
  数据处理利器 data.table
\item
  数据库使用
\item
  循环结构
\item
  循环迭代之purrr包介绍
\item
  自定义函数功能
\end{itemize}

\hypertarget{sec:licenses}{%
\section*{授权说明}\label{sec:licenses}}
\addcontentsline{toc}{section}{授权说明}

\begin{rmdwarn}{警告}
本书采用 \href{https://creativecommons.org/licenses/by-nc-nd/4.0/}{知识共享署名-非商业性使用-禁止演绎4.0国际许可协议} 许可,请君自重。
项目中代码使用 \href{https://github.com/zyf19940501/Rbook}{MIT协议} 开源。

\end{rmdwarn}

\hypertarget{session}{%
\section*{运行信息}\label{session}}
\addcontentsline{toc}{section}{运行信息}

\begin{Shaded}
\begin{Highlighting}[]
\NormalTok{xfun}\SpecialCharTok{::}\FunctionTok{session\_info}\NormalTok{(}\AttributeTok{packages =} \FunctionTok{c}\NormalTok{(}
  \StringTok{"knitr"}\NormalTok{, }\StringTok{"rmarkdown"}\NormalTok{, }\StringTok{"bookdown"}\NormalTok{,}\StringTok{"collapse"}\NormalTok{,}
  \StringTok{"data.table"}\NormalTok{, }\StringTok{"DT"}\NormalTok{, }\StringTok{"reactable"}\NormalTok{,}\StringTok{"flextable"}\NormalTok{,}
  \StringTok{"patchwork"}\NormalTok{, }\StringTok{"plotly"}\NormalTok{, }\StringTok{"shiny"}\NormalTok{,}\StringTok{"formattable"}\NormalTok{,}
  \StringTok{"ggplot2"}\NormalTok{, }\StringTok{"dplyr"}\NormalTok{, }\StringTok{"tidyverse"}\NormalTok{,}\StringTok{"DBI"}\NormalTok{,}\StringTok{"ROracle"}\NormalTok{,}\StringTok{"dbplyr"}
\NormalTok{), }\AttributeTok{dependencies =} \ConstantTok{FALSE}\NormalTok{)}
\CommentTok{\#\textgreater{} R version 4.1.0 (2021{-}05{-}18)}
\CommentTok{\#\textgreater{} Platform: x86\_64{-}w64{-}mingw32/x64 (64{-}bit)}
\CommentTok{\#\textgreater{} Running under: Windows 10 x64 (build 19041)}
\CommentTok{\#\textgreater{} }
\CommentTok{\#\textgreater{} Locale:}
\CommentTok{\#\textgreater{}   LC\_COLLATE=Chinese (Simplified)\_China.936 }
\CommentTok{\#\textgreater{}   LC\_CTYPE=Chinese (Simplified)\_China.936   }
\CommentTok{\#\textgreater{}   LC\_MONETARY=Chinese (Simplified)\_China.936}
\CommentTok{\#\textgreater{}   LC\_NUMERIC=C                              }
\CommentTok{\#\textgreater{}   LC\_TIME=Chinese (Simplified)\_China.936    }
\CommentTok{\#\textgreater{} }
\CommentTok{\#\textgreater{} Package version:}
\CommentTok{\#\textgreater{}   bookdown\_0.22     collapse\_1.5.3    data.table\_1.14.0 DBI\_1.1.1        }
\CommentTok{\#\textgreater{}   dbplyr\_2.1.1      dplyr\_1.0.6       DT\_0.18           flextable\_0.6.6  }
\CommentTok{\#\textgreater{}   formattable\_0.2.1 ggplot2\_3.3.3     knitr\_1.33        patchwork\_1.1.1  }
\CommentTok{\#\textgreater{}   plotly\_4.9.3      reactable\_0.2.3   rmarkdown\_2.8     ROracle\_1.3.1    }
\CommentTok{\#\textgreater{}   shiny\_1.6.0       tidyverse\_1.3.1  }
\CommentTok{\#\textgreater{} }
\CommentTok{\#\textgreater{} Pandoc version: 2.11.4}
\end{Highlighting}
\end{Shaded}

\hypertarget{author}{%
\section*{关于本人}\label{author}}
\addcontentsline{toc}{section}{关于本人}

一名热爱R语言的商业数据分析师。\texttt{R}极大拓展了我数据处理能力,让我很轻松方便处理数据,有更多精力时间聚焦在具体问题上。

因个人能力有限,本书难免出现错误,如发现错误,欢迎联系本人更正。

Email: \href{mailto:598253220@qq.com}{\nolinkurl{598253220@qq.com}}

微信公众号: 宇飞的世界

语雀: \url{https://www.yuque.com/zyufei}

\hypertarget{data:read-write-description}{%
\chapter{数据导入导出}\label{data:read-write-description}}

作为一名普通的数据分析师,我日常接触最多的数据是业务系统中的销售订单表、商品库存表、会员信息表,门店信息表,商品信息表等之类的业务表,但最初接触R时,看到的演示代码以及数据集大部分都是R包中内置的数据集,没有很明确操作数据的意义,没有代入感。在刚开始学习使用R做数据处理后,我就想使用自己的数据集来操作数据,用R去实现Excel的透视表或sql功能。这时就首先需要将原始数据导入\footnote{由于R是将数据加载至内存中,故数据集大小超过内存大小将导入失败。}R中。

现实生活中数据来源复杂,商业环境中数据源同样具有多样性,如SAP,不同的数据库、OA系统、EXCEL手工文件等;我们想要统一数据做分析,就需要将不同的数据源整合导入R中。

我们从读取方式简单区分为本地文件数据、数据库数据,本章主要说明常用的Excel文件和csv\footnote{csv即Comma-Separated Values,逗号分隔值,分隔符也可是不是逗号。
  csv文件是一种以纯文本形式存储的表格数据,可以通过记事本打开。与Excel不同的是,CSV是一种文本格式,也不受Excel最大行数(1048576)限制。
  csv文件也被称作平面文件,结构简单,平面文件比结构文件占用更少的空间;平面文件在数据仓库项目中广泛用于导入数据。}、txt等文本文件的读写方式。关于数据库的数据的读取,可以参照后续database章节。

相信大家随便使用搜索引擎搜索诸如``将Excel导入R''的关键词都能得到一些行之有效的方法,但是不够系统全面。本章主要简述R中数据导入导出的相关R包,如\texttt{readxl},\texttt{writexl},\texttt{openxlsx},\texttt{readr}, \texttt{vroom}等主要处理csv或Excel的R包。

\begin{quote}
当有其它数据格式需求的时候,那时候的你肯定已经会自行查找相关R包使用了。
\end{quote}

在本章开始前,假定已经有一些R相关基础。如使用Rstudio查看导入的数据,R的数据结构等有一定认识。本章节主要分为:

\begin{itemize}
\tightlist
\item
  excel读写
\item
  csv等平面文件读写
\item
  文件路径
\end{itemize}

\hypertarget{readxl:description}{%
\section{readxl}\label{readxl:description}}

readxl软件包使R获取Excel数据变得方便简洁。与现有的软件包(例如:xlsx)相比,readxl没有外部依赖性,xlsx等包依赖java环境。readxl包容易在所有的操作系统安装使用。

readxl\href{https://readxl.tidyverse.org/}{项目地址},本节大部分代码来源项目官网介绍,可自行查阅官网。

\hypertarget{readxl:install}{%
\subsection{安装}\label{readxl:install}}

从CRAN安装最新发行版本的最简单方法是安装整个tidyverse。

\begin{Shaded}
\begin{Highlighting}[]
\FunctionTok{install.packages}\NormalTok{(}\StringTok{"tidyverse"}\NormalTok{)}
\end{Highlighting}
\end{Shaded}

\begin{quote}
由于readxl不是tidyverse核心加载包,使用时仅需加载library(readxl)
\end{quote}

或者是从CRAN仅安装readxl;

\begin{Shaded}
\begin{Highlighting}[]
\FunctionTok{install.packages}\NormalTok{(}\StringTok{"readxl"}\NormalTok{)}
\end{Highlighting}
\end{Shaded}

从github安装开发版:

\begin{Shaded}
\begin{Highlighting}[]
\CommentTok{\# install.packages("devtools")}
\NormalTok{devtools}\SpecialCharTok{::}\FunctionTok{install\_github}\NormalTok{(}\StringTok{"tidyverse/readxl"}\NormalTok{)}
\end{Highlighting}
\end{Shaded}

\hypertarget{readxl:usage}{%
\subsection{用法}\label{readxl:usage}}

1.读取

readxl包中包含了几个示例文件,我们在接下来的案例中使用。

\begin{itemize}
\tightlist
\item
  查看readxl包中自带xlsx文件
\end{itemize}

\begin{Shaded}
\begin{Highlighting}[]
\FunctionTok{library}\NormalTok{(readxl)}
\FunctionTok{readxl\_example}\NormalTok{()}
\CommentTok{\#\textgreater{}  [1] "clippy.xls"    "clippy.xlsx"   "datasets.xls"  "datasets.xlsx"}
\CommentTok{\#\textgreater{}  [5] "deaths.xls"    "deaths.xlsx"   "geometry.xls"  "geometry.xlsx"}
\CommentTok{\#\textgreater{}  [9] "type{-}me.xls"   "type{-}me.xlsx"}
\FunctionTok{readxl\_example}\NormalTok{(}\StringTok{"clippy.xls"}\NormalTok{)}
\CommentTok{\#\textgreater{} [1] "C:/R/R{-}4.1.0/library/readxl/extdata/clippy.xls"}
\end{Highlighting}
\end{Shaded}

\texttt{read\_excel()}可读取xls和xlsx文件。

\begin{Shaded}
\begin{Highlighting}[]
\NormalTok{xlsx\_example }\OtherTok{\textless{}{-}} \FunctionTok{readxl\_example}\NormalTok{(}\StringTok{"datasets.xlsx"}\NormalTok{)}
\NormalTok{dt }\OtherTok{\textless{}{-}} \FunctionTok{read\_excel}\NormalTok{(xlsx\_example)}

\CommentTok{\# 查看数据}
\FunctionTok{head}\NormalTok{(dt)}
\CommentTok{\#\textgreater{} \# A tibble: 6 x 5}
\CommentTok{\#\textgreater{}   Sepal.Length Sepal.Width Petal.Length Petal.Width Species}
\CommentTok{\#\textgreater{}          \textless{}dbl\textgreater{}       \textless{}dbl\textgreater{}        \textless{}dbl\textgreater{}       \textless{}dbl\textgreater{} \textless{}chr\textgreater{}  }
\CommentTok{\#\textgreater{} 1          5.1         3.5          1.4         0.2 setosa }
\CommentTok{\#\textgreater{} 2          4.9         3            1.4         0.2 setosa }
\CommentTok{\#\textgreater{} 3          4.7         3.2          1.3         0.2 setosa }
\CommentTok{\#\textgreater{} 4          4.6         3.1          1.5         0.2 setosa }
\CommentTok{\#\textgreater{} 5          5           3.6          1.4         0.2 setosa }
\CommentTok{\#\textgreater{} 6          5.4         3.9          1.7         0.4 setosa}

\CommentTok{\# 查看数据类型}
\FunctionTok{str}\NormalTok{(dt)}
\CommentTok{\#\textgreater{} tibble [150 x 5] (S3: tbl\_df/tbl/data.frame)}
\CommentTok{\#\textgreater{}  $ Sepal.Length: num [1:150] 5.1 4.9 4.7 4.6 5 5.4 4.6 5 4.4 4.9 ...}
\CommentTok{\#\textgreater{}  $ Sepal.Width : num [1:150] 3.5 3 3.2 3.1 3.6 3.9 3.4 3.4 2.9 3.1 ...}
\CommentTok{\#\textgreater{}  $ Petal.Length: num [1:150] 1.4 1.4 1.3 1.5 1.4 1.7 1.4 1.5 1.4 1.5 ...}
\CommentTok{\#\textgreater{}  $ Petal.Width : num [1:150] 0.2 0.2 0.2 0.2 0.2 0.4 0.3 0.2 0.2 0.1 ...}
\CommentTok{\#\textgreater{}  $ Species     : chr [1:150] "setosa" "setosa" "setosa" "setosa" ...}
\end{Highlighting}
\end{Shaded}

通过函数\texttt{excel\_sheets()}查看Excel的sheet名称

\begin{Shaded}
\begin{Highlighting}[]
\FunctionTok{excel\_sheets}\NormalTok{(xlsx\_example)}
\CommentTok{\#\textgreater{} [1] "iris"     "mtcars"   "chickwts" "quakes"}
\end{Highlighting}
\end{Shaded}

指定worksheet的名字读取,可以是sheet的名字或序号。

当我们要读取的本地xlsx文件有多个sheets时,通过指定sheet参数读取指定的sheet。

\begin{Shaded}
\begin{Highlighting}[]
\FunctionTok{read\_excel}\NormalTok{(xlsx\_example, }\AttributeTok{sheet =} \StringTok{"chickwts"}\NormalTok{)}
\CommentTok{\#\textgreater{} \# A tibble: 71 x 2}
\CommentTok{\#\textgreater{}   weight feed     }
\CommentTok{\#\textgreater{}    \textless{}dbl\textgreater{} \textless{}chr\textgreater{}    }
\CommentTok{\#\textgreater{} 1    179 horsebean}
\CommentTok{\#\textgreater{} 2    160 horsebean}
\CommentTok{\#\textgreater{} 3    136 horsebean}
\CommentTok{\#\textgreater{} 4    227 horsebean}
\CommentTok{\#\textgreater{} 5    217 horsebean}
\CommentTok{\#\textgreater{} 6    168 horsebean}
\CommentTok{\#\textgreater{} \# ... with 65 more rows}
\CommentTok{\# not run}
\CommentTok{\#read\_excel(xlsx\_example, sheet = 1)}
\CommentTok{\#read\_excel(xlsx\_example, sheet = 3)}
\end{Highlighting}
\end{Shaded}

读取xlsx文件的指定范围,有多种方法控制。

本处提供几个案例,详情请\texttt{?read\_excel()}查看帮助。

readxl::read\_excel参数如下:

\begin{Shaded}
\begin{Highlighting}[]
\FunctionTok{read\_excel}\NormalTok{(path, }\AttributeTok{sheet =} \ConstantTok{NULL}\NormalTok{, }\AttributeTok{range =} \ConstantTok{NULL}\NormalTok{, }\AttributeTok{col\_names =} \ConstantTok{TRUE}\NormalTok{,}
  \AttributeTok{col\_types =} \ConstantTok{NULL}\NormalTok{, }\AttributeTok{na =} \StringTok{""}\NormalTok{, }\AttributeTok{trim\_ws =} \ConstantTok{TRUE}\NormalTok{, }\AttributeTok{skip =} \DecValTok{0}\NormalTok{,}
  \AttributeTok{n\_max =} \ConstantTok{Inf}\NormalTok{, }\AttributeTok{guess\_max =} \FunctionTok{min}\NormalTok{(}\DecValTok{1000}\NormalTok{, n\_max),}
  \AttributeTok{progress =} \FunctionTok{readxl\_progress}\NormalTok{(), }\AttributeTok{.name\_repair =} \StringTok{"unique"}\NormalTok{)}
\end{Highlighting}
\end{Shaded}

range接受单元格范围,最简单的表示方式即Excle中单元格表示方法,如as range = ``D12:F15'' or range = ``R1C12:R6C15''。

其余参数中,个人觉得col\_types比较重要,可以指定列的类型。可用选项:``skip'', ``guess'', ``logical'', ``numeric'', ``date'', ``text'' or ``list''。
.name\_repair 参数能自动避免重复字段,可避免手工Excle出现的字段名不唯一的情况。

\hypertarget{ux6279ux91cfux8bfbux53d6}{%
\subsection{批量读取}\label{ux6279ux91cfux8bfbux53d6}}

某文件夹下有大量相同的Excel文件(sheet名称以及列字段相同),要合并全部Excel数据,代码如下:

\begin{Shaded}
\begin{Highlighting}[]
\NormalTok{allfiles }\OtherTok{\textless{}{-}} \FunctionTok{list.files}\NormalTok{(}\AttributeTok{path =} \StringTok{\textquotesingle{}./data/\textquotesingle{}}\NormalTok{,}\AttributeTok{pattern =} \StringTok{\textquotesingle{}.xlsx$\textquotesingle{}}\NormalTok{,}\AttributeTok{full.names =}\NormalTok{ T)}
\NormalTok{purrr}\SpecialCharTok{::}\FunctionTok{map\_dfr}\NormalTok{(allfiles,read\_excel)}
\end{Highlighting}
\end{Shaded}

\hypertarget{ux6279ux91cfux8f93ux51fa}{%
\subsection{批量输出}\label{ux6279ux91cfux8f93ux51fa}}

我们按照一定条件拆解数据集,分别输出,代码如下:

\begin{Shaded}
\begin{Highlighting}[]
\FunctionTok{library}\NormalTok{(tidyverse)}
\FunctionTok{library}\NormalTok{(readxl)}
\NormalTok{dt }\OtherTok{\textless{}{-}} \FunctionTok{read\_xlsx}\NormalTok{(}\AttributeTok{path =} \StringTok{\textquotesingle{}./data/read{-}write/批量读写.xlsx\textquotesingle{}}\NormalTok{)}


\NormalTok{dt }\SpecialCharTok{\%\textgreater{}\%} 
  \FunctionTok{group\_by}\NormalTok{(name) }\SpecialCharTok{\%\textgreater{}\%} 
  \FunctionTok{group\_walk}\NormalTok{(}\SpecialCharTok{\textasciitilde{}} \FunctionTok{write.csv}\NormalTok{(.x,}\AttributeTok{file =} \FunctionTok{file.path}\NormalTok{(}\StringTok{\textquotesingle{}data/read{-}write\textquotesingle{}}\NormalTok{,}\FunctionTok{paste0}\NormalTok{(.y}\SpecialCharTok{$}\NormalTok{name,}\StringTok{\textquotesingle{}.csv\textquotesingle{}}\NormalTok{))))}
\FunctionTok{list.files}\NormalTok{(}\AttributeTok{path =} \StringTok{\textquotesingle{}data/read{-}write/\textquotesingle{}}\NormalTok{)}
\CommentTok{\#\textgreater{} [1] "a.csv"         "b.csv"         "d.csv"         "批量读写.xlsx"}
\end{Highlighting}
\end{Shaded}

\begin{quote}
暂时不用理解批量读取和输出的代码具体含义,可以先记住用法。
\end{quote}

\hypertarget{writexl}{%
\section{writexl}\label{writexl}}

截止到2021年5月17日,writexl包功能比较简单,仅有输出Excel功能。快速、不依赖java和Excle是它绝对的优势,并且输出文件相比\texttt{openxlsx}包较小。

\href{https://docs.ropensci.org/writexl/}{项目地址}

\hypertarget{writexl:usage}{%
\subsection{用法}\label{writexl:usage}}

安装

\begin{Shaded}
\begin{Highlighting}[]
\FunctionTok{install.packages}\NormalTok{(}\StringTok{"writexl"}\NormalTok{)}
\end{Highlighting}
\end{Shaded}

参数

\begin{Shaded}
\begin{Highlighting}[]
\FunctionTok{write\_xlsx}\NormalTok{(}
\NormalTok{  x,}
  \AttributeTok{path =} \FunctionTok{tempfile}\NormalTok{(}\AttributeTok{fileext =} \StringTok{".xlsx"}\NormalTok{),}
  \AttributeTok{col\_names =} \ConstantTok{TRUE}\NormalTok{,}
  \AttributeTok{format\_headers =} \ConstantTok{TRUE}\NormalTok{,}
  \AttributeTok{use\_zip64 =} \ConstantTok{FALSE}
\NormalTok{)}
\end{Highlighting}
\end{Shaded}

输出Excel

\begin{Shaded}
\begin{Highlighting}[]
\FunctionTok{library}\NormalTok{(writexl)}
\NormalTok{writexl}\SpecialCharTok{::}\FunctionTok{write\_xlsx}\NormalTok{(iris,}\AttributeTok{path =} \StringTok{\textquotesingle{}iris.xlsx\textquotesingle{}}\NormalTok{)}
\FunctionTok{write\_xlsx}\NormalTok{(}\FunctionTok{list}\NormalTok{(}\AttributeTok{mysheet1 =}\NormalTok{ iris,}\AttributeTok{mysheet2 =}\NormalTok{ iris),}\AttributeTok{path =} \StringTok{\textquotesingle{}iris.xlsx\textquotesingle{}}\NormalTok{)}
\end{Highlighting}
\end{Shaded}

效率比较

\begin{Shaded}
\begin{Highlighting}[]
\FunctionTok{library}\NormalTok{(microbenchmark)}
\FunctionTok{library}\NormalTok{(nycflights13)}
\FunctionTok{microbenchmark}\NormalTok{(}
  \AttributeTok{writexl =}\NormalTok{ writexl}\SpecialCharTok{::}\FunctionTok{write\_xlsx}\NormalTok{(flights, }\FunctionTok{tempfile}\NormalTok{()),}
  \AttributeTok{openxlsx =}\NormalTok{ openxlsx}\SpecialCharTok{::}\FunctionTok{write.xlsx}\NormalTok{(flights, }\FunctionTok{tempfile}\NormalTok{()),}
  \AttributeTok{times =} \DecValTok{2}
\NormalTok{)}
\end{Highlighting}
\end{Shaded}

文件大小比较

\begin{Shaded}
\begin{Highlighting}[]
\FunctionTok{library}\NormalTok{(nycflights13)}
\NormalTok{writexl}\SpecialCharTok{::}\FunctionTok{write\_xlsx}\NormalTok{(flights, tmp1 }\OtherTok{\textless{}{-}} \FunctionTok{tempfile}\NormalTok{())}
\FunctionTok{file.info}\NormalTok{(tmp1)}\SpecialCharTok{$}\NormalTok{size}
\CommentTok{\#\textgreater{} [1] 29139352}
\end{Highlighting}
\end{Shaded}

\begin{Shaded}
\begin{Highlighting}[]
\NormalTok{openxlsx}\SpecialCharTok{::}\FunctionTok{write.xlsx}\NormalTok{(flights, tmp2 }\OtherTok{\textless{}{-}} \FunctionTok{tempfile}\NormalTok{())}
\FunctionTok{file.info}\NormalTok{(tmp2)}\SpecialCharTok{$}\NormalTok{size}
\CommentTok{\#\textgreater{} [1] 26833693}
\end{Highlighting}
\end{Shaded}

其它功能

\begin{Shaded}
\begin{Highlighting}[]
\NormalTok{df }\OtherTok{\textless{}{-}} \FunctionTok{data.frame}\NormalTok{(}
  \AttributeTok{name =} \FunctionTok{c}\NormalTok{(}\StringTok{"UCLA"}\NormalTok{, }\StringTok{"Berkeley"}\NormalTok{, }\StringTok{"Jeroen"}\NormalTok{),}
  \AttributeTok{founded =} \FunctionTok{c}\NormalTok{(}\DecValTok{1919}\NormalTok{, }\DecValTok{1868}\NormalTok{, }\DecValTok{2030}\NormalTok{),}
  \AttributeTok{website =} \FunctionTok{xl\_hyperlink}\NormalTok{(}\FunctionTok{c}\NormalTok{(}\StringTok{"http://www.ucla.edu"}\NormalTok{, }\StringTok{"http://www.berkeley.edu"}\NormalTok{, }\ConstantTok{NA}\NormalTok{), }\StringTok{"homepage"}\NormalTok{)}
\NormalTok{)}
\NormalTok{df}\SpecialCharTok{$}\NormalTok{age }\OtherTok{\textless{}{-}} \FunctionTok{xl\_formula}\NormalTok{(}\StringTok{\textquotesingle{}=(YEAR(TODAY()) {-} INDIRECT("B" \& ROW()))\textquotesingle{}}\NormalTok{)}
\FunctionTok{write\_xlsx}\NormalTok{(df, }\StringTok{\textquotesingle{}universities.xlsx\textquotesingle{}}\NormalTok{)}

\CommentTok{\# cleanup}
\FunctionTok{unlink}\NormalTok{(}\StringTok{\textquotesingle{}universities.xlsx\textquotesingle{}}\NormalTok{)}
\end{Highlighting}
\end{Shaded}

\hypertarget{openxlsx:description}{%
\section{openxlsx}\label{openxlsx:description}}

openxlsx是当我需要定制输出Excel表格或报表时常用R包。目前该包的版本4.2.3,通过使用Rcpp加速,包的读写速度在Excel的百万级下是可接受状态,包的相关函数功能完善且简易好用,并且正在积极开发中,相信它以后功能会越来越强大。

项目官方地址:\url{https://ycphs.github.io/openxlsx/index.html}

个人感觉主要优势:

\begin{itemize}
\tightlist
\item
  不依赖java环境
\item
  读写速度可接受
\item
  可设置条件格式,与Excel中『开始』选项卡的条件格式功能接近
\item
  可批量插入ggplot2图
\item
  可插入公式
\item
  可渲染大部分Excel格式,并且效率相比部分python包高效
\item
  可添加页眉页脚以及其他格式,方便直接打印
\item
  功能稳定可用并且在积极开发中
\end{itemize}

版本信息查看

\begin{Shaded}
\begin{Highlighting}[]
\FunctionTok{packageVersion}\NormalTok{(}\StringTok{"openxlsx"}\NormalTok{)}
\CommentTok{\#\textgreater{} [1] \textquotesingle{}4.2.3\textquotesingle{}}
\end{Highlighting}
\end{Shaded}

本人公众号:宇飞的世界中有更加详细的阐述:\url{https://mp.weixin.qq.com/s/ZD0dJb0y8fsWGI1dCPh2mQ}

\hypertarget{openxlsx:install}{%
\subsection{安装}\label{openxlsx:install}}

稳定版

\begin{Shaded}
\begin{Highlighting}[]
\CommentTok{\# 稳定版}
\FunctionTok{install.packages}\NormalTok{(}\StringTok{"openxlsx"}\NormalTok{, }\AttributeTok{dependencies =} \ConstantTok{TRUE}\NormalTok{, }\AttributeTok{repos =} \StringTok{"https://mirrors.tuna.tsinghua.edu.cn/CRAN/"}\NormalTok{)}
\end{Highlighting}
\end{Shaded}

开发版

\begin{Shaded}
\begin{Highlighting}[]
\FunctionTok{install.packages}\NormalTok{(}\FunctionTok{c}\NormalTok{(}\StringTok{"Rcpp"}\NormalTok{, }\StringTok{"devtools"}\NormalTok{), }\AttributeTok{dependencies =} \ConstantTok{TRUE}\NormalTok{)}
\FunctionTok{library}\NormalTok{(devtools)}
\FunctionTok{install\_github}\NormalTok{(}\StringTok{"ycphs/openxlsx"}\NormalTok{)}
\end{Highlighting}
\end{Shaded}

\hypertarget{openxlsx:functions}{%
\subsection{基础功能}\label{openxlsx:functions}}

本文仅呈现基础功能部分,即读写EXCEL文件。其它功能,请查阅项目官方地址或微信公众号文章\href{https://mp.weixin.qq.com/s/ZD0dJb0y8fsWGI1dCPh2mQ}{R包-openxlsx-学习笔记}

\hypertarget{openxlsx:read-function}{%
\subsubsection{读取Excel}\label{openxlsx:read-function}}

read.xlsx()是读取函数,主要参数如下:

\begin{Shaded}
\begin{Highlighting}[]
\FunctionTok{library}\NormalTok{(openxlsx)}
\FunctionTok{read.xlsx}\NormalTok{(}
\NormalTok{  xlsxFile,}
  \AttributeTok{sheet =} \DecValTok{1}\NormalTok{,}
  \AttributeTok{startRow =} \DecValTok{1}\NormalTok{,}
  \AttributeTok{colNames =} \ConstantTok{TRUE}\NormalTok{,}
  \AttributeTok{rowNames =} \ConstantTok{FALSE}\NormalTok{,}
  \AttributeTok{detectDates =} \ConstantTok{FALSE}\NormalTok{,}
  \AttributeTok{skipEmptyRows =} \ConstantTok{TRUE}\NormalTok{,}
  \AttributeTok{skipEmptyCols =} \ConstantTok{TRUE}\NormalTok{,}
  \AttributeTok{rows =} \ConstantTok{NULL}\NormalTok{,}
  \AttributeTok{cols =} \ConstantTok{NULL}\NormalTok{,}
  \AttributeTok{check.names =} \ConstantTok{FALSE}\NormalTok{,}
  \AttributeTok{sep.names =} \StringTok{"."}\NormalTok{,}
  \AttributeTok{namedRegion =} \ConstantTok{NULL}\NormalTok{,}
  \AttributeTok{na.strings =} \StringTok{"NA"}\NormalTok{,}
  \AttributeTok{fillMergedCells =} \ConstantTok{FALSE}
\NormalTok{)}
\end{Highlighting}
\end{Shaded}

以上参数中需要注意 :

detecDates参数,当你的Excel表格中带日期列时需要将参数设置为TRUE,不然将会把日期识别为数字读入。

fillMergedCells参数,当你读取的表格中存在合并单元格,将用值填充其他全部单元格,如下所示:

\includegraphics{https://gitee.com/zhongyufei/photo-bed/raw/pic/img/merge-cell-xlsx.png}

\begin{Shaded}
\begin{Highlighting}[]
\FunctionTok{read.xlsx}\NormalTok{(}\StringTok{\textquotesingle{}./test.xlsx\textquotesingle{}}\NormalTok{,}\AttributeTok{detectDates =} \ConstantTok{TRUE}\NormalTok{,}\AttributeTok{fillMergedCells =} \ConstantTok{TRUE}\NormalTok{)}
\end{Highlighting}
\end{Shaded}

读取后如下所示:

\begin{figure}
\centering
\includegraphics{https://gitee.com/zhongyufei/photo-bed/raw/pic/img/R-read-merge-xlsx.png}
\caption{openxlsx-merge-xlsx}
\end{figure}

readWorkbook()也可以读取Excel表格数据,参数与read.xlsx基本一致。

\begin{Shaded}
\begin{Highlighting}[]
\NormalTok{xlsxFile }\OtherTok{\textless{}{-}} \FunctionTok{system.file}\NormalTok{(}\StringTok{"extdata"}\NormalTok{, }\StringTok{"readTest.xlsx"}\NormalTok{, }\AttributeTok{package =} \StringTok{"openxlsx"}\NormalTok{)}
\NormalTok{df1 }\OtherTok{\textless{}{-}} \FunctionTok{readWorkbook}\NormalTok{(}\AttributeTok{xlsxFile =}\NormalTok{ xlsxFile, }\AttributeTok{sheet =} \DecValTok{1}\NormalTok{)}
\end{Highlighting}
\end{Shaded}

\hypertarget{openxlsx:write-function}{%
\subsubsection{写入Excel}\label{openxlsx:write-function}}

数据清洗完之后,或者是透视表已经完成,需要将结果从R导出到Excle,这时就利用函数将结果数据集data.frame写入Excle中。

write.xlsx()函数写入

\begin{Shaded}
\begin{Highlighting}[]
\FunctionTok{write.xlsx}\NormalTok{(iris, }\AttributeTok{file =} \StringTok{"writeXLSX1.xlsx"}\NormalTok{, }\AttributeTok{colNames =} \ConstantTok{TRUE}\NormalTok{, }\AttributeTok{borders =} \StringTok{"columns"}\NormalTok{)}
\end{Highlighting}
\end{Shaded}

带格式输出

\begin{Shaded}
\begin{Highlighting}[]
\NormalTok{hs }\OtherTok{\textless{}{-}} \FunctionTok{createStyle}\NormalTok{(}
  \AttributeTok{textDecoration =} \StringTok{"BOLD"}\NormalTok{, }\AttributeTok{fontColour =} \StringTok{"\#FFFFFF"}\NormalTok{, }\AttributeTok{fontSize =} \DecValTok{12}\NormalTok{,}
  \AttributeTok{fontName =} \StringTok{"Arial Narrow"}\NormalTok{, }\AttributeTok{fgFill =} \StringTok{"\#4F80BD"}
\NormalTok{)}
\DocumentationTok{\#\# Not run: }
\FunctionTok{write.xlsx}\NormalTok{(iris,}
  \AttributeTok{file =} \StringTok{"writeXLSX3.xlsx"}\NormalTok{,}
  \AttributeTok{colNames =} \ConstantTok{TRUE}\NormalTok{, }\AttributeTok{borders =} \StringTok{"rows"}\NormalTok{, }\AttributeTok{headerStyle =}\NormalTok{ hs}
\NormalTok{)}
\end{Highlighting}
\end{Shaded}

\hypertarget{ux5e26ux683cux5f0fux8f93ux51fa}{%
\subsection{带格式输出}\label{ux5e26ux683cux5f0fux8f93ux51fa}}

输出过程共四步,第一步创建workbook,第二步添加sheet,第三步写入数据,第四步保存workbook。在输出的过程中可以通过\texttt{addStyle()}、\texttt{createStyle()}或\texttt{conditionalFormatting}添加格式或条件格式。

\begin{Shaded}
\begin{Highlighting}[]
\NormalTok{df }\OtherTok{\textless{}{-}} \FunctionTok{data.frame}\NormalTok{(}\AttributeTok{a=}\DecValTok{1}\SpecialCharTok{:}\DecValTok{10}\NormalTok{,}\AttributeTok{b=}\DecValTok{1}\SpecialCharTok{:}\DecValTok{10}\NormalTok{,}\AttributeTok{d=}\DecValTok{1}\SpecialCharTok{:}\DecValTok{10}\NormalTok{)}
\NormalTok{wb }\OtherTok{\textless{}{-}} \FunctionTok{createWorkbook}\NormalTok{(}\AttributeTok{creator =} \StringTok{\textquotesingle{}zhongyf\textquotesingle{}}\NormalTok{,}\AttributeTok{title =} \StringTok{\textquotesingle{}test\textquotesingle{}}\NormalTok{)}
\FunctionTok{addWorksheet}\NormalTok{(wb,}\AttributeTok{sheetName =} \StringTok{\textquotesingle{}test\textquotesingle{}}\NormalTok{)}
\FunctionTok{writeData}\NormalTok{(wb,}\AttributeTok{sheet =} \StringTok{\textquotesingle{}test\textquotesingle{}}\NormalTok{,}\AttributeTok{x =}\NormalTok{ df)}
\FunctionTok{saveWorkbook}\NormalTok{(wb, }\StringTok{"test.xlsx"}\NormalTok{, }\AttributeTok{overwrite =} \ConstantTok{TRUE}\NormalTok{)}
\end{Highlighting}
\end{Shaded}

我们以上面四步输出的方式,查看包自带的例子。

\begin{itemize}
\item
  createWorkbook()
\item
  addWorksheet()
\item
  writeData()
\item
  saveWorkbook()
\end{itemize}

\begin{Shaded}
\begin{Highlighting}[]
\NormalTok{wb }\OtherTok{\textless{}{-}} \FunctionTok{createWorkbook}\NormalTok{(}\StringTok{"Fred"}\NormalTok{)}

\DocumentationTok{\#\# Add 3 worksheets}
\FunctionTok{addWorksheet}\NormalTok{(wb, }\StringTok{"Sheet 1"}\NormalTok{)}
\FunctionTok{addWorksheet}\NormalTok{(wb, }\StringTok{"Sheet 2"}\NormalTok{, }\AttributeTok{gridLines =} \ConstantTok{FALSE}\NormalTok{)}
\FunctionTok{addWorksheet}\NormalTok{(wb, }\StringTok{"Sheet 3"}\NormalTok{, }\AttributeTok{tabColour =} \StringTok{"red"}\NormalTok{)}
\FunctionTok{addWorksheet}\NormalTok{(wb, }\StringTok{"Sheet 4"}\NormalTok{, }\AttributeTok{gridLines =} \ConstantTok{FALSE}\NormalTok{, }\AttributeTok{tabColour =} \StringTok{"\#4F81BD"}\NormalTok{)}

\DocumentationTok{\#\# Headers and Footers}
\FunctionTok{addWorksheet}\NormalTok{(wb, }\StringTok{"Sheet 5"}\NormalTok{,}
  \AttributeTok{header =} \FunctionTok{c}\NormalTok{(}\StringTok{"ODD HEAD LEFT"}\NormalTok{, }\StringTok{"ODD HEAD CENTER"}\NormalTok{, }\StringTok{"ODD HEAD RIGHT"}\NormalTok{),}
  \AttributeTok{footer =} \FunctionTok{c}\NormalTok{(}\StringTok{"ODD FOOT RIGHT"}\NormalTok{, }\StringTok{"ODD FOOT CENTER"}\NormalTok{, }\StringTok{"ODD FOOT RIGHT"}\NormalTok{),}
  \AttributeTok{evenHeader =} \FunctionTok{c}\NormalTok{(}\StringTok{"EVEN HEAD LEFT"}\NormalTok{, }\StringTok{"EVEN HEAD CENTER"}\NormalTok{, }\StringTok{"EVEN HEAD RIGHT"}\NormalTok{),}
  \AttributeTok{evenFooter =} \FunctionTok{c}\NormalTok{(}\StringTok{"EVEN FOOT RIGHT"}\NormalTok{, }\StringTok{"EVEN FOOT CENTER"}\NormalTok{, }\StringTok{"EVEN FOOT RIGHT"}\NormalTok{),}
  \AttributeTok{firstHeader =} \FunctionTok{c}\NormalTok{(}\StringTok{"TOP"}\NormalTok{, }\StringTok{"OF FIRST"}\NormalTok{, }\StringTok{"PAGE"}\NormalTok{),}
  \AttributeTok{firstFooter =} \FunctionTok{c}\NormalTok{(}\StringTok{"BOTTOM"}\NormalTok{, }\StringTok{"OF FIRST"}\NormalTok{, }\StringTok{"PAGE"}\NormalTok{)}
\NormalTok{)}

\FunctionTok{addWorksheet}\NormalTok{(wb, }\StringTok{"Sheet 6"}\NormalTok{,}
  \AttributeTok{header =} \FunctionTok{c}\NormalTok{(}\StringTok{"\&[Date]"}\NormalTok{, }\StringTok{"ALL HEAD CENTER 2"}\NormalTok{, }\StringTok{"\&[Page] / \&[Pages]"}\NormalTok{),}
  \AttributeTok{footer =} \FunctionTok{c}\NormalTok{(}\StringTok{"\&[Path]\&[File]"}\NormalTok{, }\ConstantTok{NA}\NormalTok{, }\StringTok{"\&[Tab]"}\NormalTok{),}
  \AttributeTok{firstHeader =} \FunctionTok{c}\NormalTok{(}\ConstantTok{NA}\NormalTok{, }\StringTok{"Center Header of First Page"}\NormalTok{, }\ConstantTok{NA}\NormalTok{),}
  \AttributeTok{firstFooter =} \FunctionTok{c}\NormalTok{(}\ConstantTok{NA}\NormalTok{, }\StringTok{"Center Footer of First Page"}\NormalTok{, }\ConstantTok{NA}\NormalTok{)}
\NormalTok{)}

\FunctionTok{addWorksheet}\NormalTok{(wb, }\StringTok{"Sheet 7"}\NormalTok{,}
  \AttributeTok{header =} \FunctionTok{c}\NormalTok{(}\StringTok{"ALL HEAD LEFT 2"}\NormalTok{, }\StringTok{"ALL HEAD CENTER 2"}\NormalTok{, }\StringTok{"ALL HEAD RIGHT 2"}\NormalTok{),}
  \AttributeTok{footer =} \FunctionTok{c}\NormalTok{(}\StringTok{"ALL FOOT RIGHT 2"}\NormalTok{, }\StringTok{"ALL FOOT CENTER 2"}\NormalTok{, }\StringTok{"ALL FOOT RIGHT 2"}\NormalTok{)}
\NormalTok{)}

\FunctionTok{addWorksheet}\NormalTok{(wb, }\StringTok{"Sheet 8"}\NormalTok{,}
  \AttributeTok{firstHeader =} \FunctionTok{c}\NormalTok{(}\StringTok{"FIRST ONLY L"}\NormalTok{, }\ConstantTok{NA}\NormalTok{, }\StringTok{"FIRST ONLY R"}\NormalTok{),}
  \AttributeTok{firstFooter =} \FunctionTok{c}\NormalTok{(}\StringTok{"FIRST ONLY L"}\NormalTok{, }\ConstantTok{NA}\NormalTok{, }\StringTok{"FIRST ONLY R"}\NormalTok{)}
\NormalTok{)}

\DocumentationTok{\#\# Need data on worksheet to see all headers and footers}
\FunctionTok{writeData}\NormalTok{(wb, }\AttributeTok{sheet =} \DecValTok{5}\NormalTok{, }\DecValTok{1}\SpecialCharTok{:}\DecValTok{400}\NormalTok{)}
\FunctionTok{writeData}\NormalTok{(wb, }\AttributeTok{sheet =} \DecValTok{6}\NormalTok{, }\DecValTok{1}\SpecialCharTok{:}\DecValTok{400}\NormalTok{)}
\FunctionTok{writeData}\NormalTok{(wb, }\AttributeTok{sheet =} \DecValTok{7}\NormalTok{, }\DecValTok{1}\SpecialCharTok{:}\DecValTok{400}\NormalTok{)}
\FunctionTok{writeData}\NormalTok{(wb, }\AttributeTok{sheet =} \DecValTok{8}\NormalTok{, }\DecValTok{1}\SpecialCharTok{:}\DecValTok{400}\NormalTok{)}

\DocumentationTok{\#\# Save workbook}
\DocumentationTok{\#\# Not run: }
\FunctionTok{saveWorkbook}\NormalTok{(wb, }\StringTok{"addWorksheetExample.xlsx"}\NormalTok{, }\AttributeTok{overwrite =} \ConstantTok{TRUE}\NormalTok{)}
\end{Highlighting}
\end{Shaded}

\hypertarget{ux51fdux6570ux53c2ux6570}{%
\subsection{函数参数}\label{ux51fdux6570ux53c2ux6570}}

输出Excel的过程分为四步,本小节主要拆解\texttt{createWorkbook},\texttt{addWorksheet},\texttt{writeDataTable},\texttt{saveWorkbook}四个函数的参数以及用法。

\begin{itemize}
\tightlist
\item
  createWorkbook
\end{itemize}

\begin{Shaded}
\begin{Highlighting}[]
\FunctionTok{createWorkbook}\NormalTok{(}
  \AttributeTok{creator =} \FunctionTok{ifelse}\NormalTok{(.Platform}\SpecialCharTok{$}\NormalTok{OS.type }\SpecialCharTok{==} \StringTok{"windows"}\NormalTok{, }\FunctionTok{Sys.getenv}\NormalTok{(}\StringTok{"USERNAME"}\NormalTok{),}
    \FunctionTok{Sys.getenv}\NormalTok{(}\StringTok{"USER"}\NormalTok{)),}
  \AttributeTok{title =} \ConstantTok{NULL}\NormalTok{,}
  \AttributeTok{subject =} \ConstantTok{NULL}\NormalTok{,}
  \AttributeTok{category =} \ConstantTok{NULL}
\NormalTok{)}
\end{Highlighting}
\end{Shaded}

\begin{Shaded}
\begin{Highlighting}[]
\NormalTok{wb }\OtherTok{\textless{}{-}} \FunctionTok{createWorkbook}\NormalTok{(}
  \AttributeTok{creator =} \StringTok{"宇飞的世界"}\NormalTok{,}
  \AttributeTok{title =} \StringTok{"标题"}\NormalTok{,}
  \AttributeTok{subject =} \StringTok{"主题"}\NormalTok{,}
  \AttributeTok{category =} \StringTok{"类别目录"}
\NormalTok{)}
\end{Highlighting}
\end{Shaded}

\begin{itemize}
\tightlist
\item
  addWorksheet
\end{itemize}

\begin{Shaded}
\begin{Highlighting}[]
\FunctionTok{addWorksheet}\NormalTok{(}
\NormalTok{  wb,}
\NormalTok{  sheetName,}
  \AttributeTok{gridLines =} \ConstantTok{TRUE}\NormalTok{,}
  \AttributeTok{tabColour =} \ConstantTok{NULL}\NormalTok{,}
  \AttributeTok{zoom =} \DecValTok{100}\NormalTok{,}
  \AttributeTok{header =} \ConstantTok{NULL}\NormalTok{,}
  \AttributeTok{footer =} \ConstantTok{NULL}\NormalTok{,}
  \AttributeTok{evenHeader =} \ConstantTok{NULL}\NormalTok{,}
  \AttributeTok{evenFooter =} \ConstantTok{NULL}\NormalTok{,}
  \AttributeTok{firstHeader =} \ConstantTok{NULL}\NormalTok{,}
  \AttributeTok{firstFooter =} \ConstantTok{NULL}\NormalTok{,}
  \AttributeTok{visible =} \ConstantTok{TRUE}\NormalTok{,}
  \AttributeTok{paperSize =} \FunctionTok{getOption}\NormalTok{(}\StringTok{"openxlsx.paperSize"}\NormalTok{, }\AttributeTok{default =} \DecValTok{9}\NormalTok{),}
  \AttributeTok{orientation =} \FunctionTok{getOption}\NormalTok{(}\StringTok{"openxlsx.orientation"}\NormalTok{, }\AttributeTok{default =} \StringTok{"portrait"}\NormalTok{),}
  \AttributeTok{vdpi =} \FunctionTok{getOption}\NormalTok{(}\StringTok{"openxlsx.vdpi"}\NormalTok{, }\AttributeTok{default =} \FunctionTok{getOption}\NormalTok{(}\StringTok{"openxlsx.dpi"}\NormalTok{, }\AttributeTok{default =} \DecValTok{300}\NormalTok{)),}
  \AttributeTok{hdpi =} \FunctionTok{getOption}\NormalTok{(}\StringTok{"openxlsx.hdpi"}\NormalTok{, }\AttributeTok{default =} \FunctionTok{getOption}\NormalTok{(}\StringTok{"openxlsx.dpi"}\NormalTok{, }\AttributeTok{default =} \DecValTok{300}\NormalTok{))}
\NormalTok{)}
\end{Highlighting}
\end{Shaded}

\begin{Shaded}
\begin{Highlighting}[]
\NormalTok{gridLines参数:表格中是否有网格线,在Excle『视图』选项卡下面的网格线去除打勾的效果一致}

\NormalTok{tabColour参数:输出表格sheet标签颜色}

\NormalTok{zoom:发大缩小,默认是100,可选范围10}\DecValTok{{-}400}

\NormalTok{header}\SpecialCharTok{:}\NormalTok{页眉 长度为3的字符向量,左、中、右三个位置,用Na可跳过一位置,以下页眉页脚相同。}

\NormalTok{footer}\SpecialCharTok{:}\NormalTok{ 页脚}

\NormalTok{evenHeader}\SpecialCharTok{:}\NormalTok{ 每页页眉}

\NormalTok{evenFooter}\SpecialCharTok{:}\NormalTok{ 每页页脚}

\NormalTok{firstHeader}\SpecialCharTok{:}\NormalTok{ 第一页页眉}

\NormalTok{firstFooter}\SpecialCharTok{:}\NormalTok{ 第一页页脚}

\NormalTok{visible}\SpecialCharTok{:}\NormalTok{sheet是否隐藏,如果为否sheet将被隐藏}

\NormalTok{paperSize}\SpecialCharTok{:}\NormalTok{页面大小,详见 ?pageSetup }

\NormalTok{orientation}\SpecialCharTok{:}\NormalTok{One of }\StringTok{"portrait"}\NormalTok{ or }\StringTok{"landscape"}\NormalTok{ 不清楚干嘛用}

\NormalTok{vdpi}\SpecialCharTok{:}\NormalTok{ 屏幕分辨率 默认值即可,不用调整}

\NormalTok{hdpi}\SpecialCharTok{:}\NormalTok{ 屏幕分辨率 默认值即可,不用调整}
\end{Highlighting}
\end{Shaded}

\begin{itemize}
\tightlist
\item
  writeDataTable
\end{itemize}

writeDataTable()函数将data.frame写入Excel。

wb:即createWorkbook()函数创建

\begin{itemize}
\tightlist
\item
  saveWorkbook
\end{itemize}

\begin{Shaded}
\begin{Highlighting}[]
\FunctionTok{saveWorkbook}\NormalTok{(wb, file, }\AttributeTok{overwrite =} \ConstantTok{FALSE}\NormalTok{, }\AttributeTok{returnValue =} \ConstantTok{FALSE}\NormalTok{)}
\end{Highlighting}
\end{Shaded}

参数较为简单,wb即上文中的workbook对象,file即输出的文件名,overwrite即如果存在是否覆盖,returnValue如果设置为TRUE,返回TRUE代表保存成功

\hypertarget{ux603bux7ed3}{%
\subsection{总结}\label{ux603bux7ed3}}

openxlsx包功能较为强大,更多详细用法大家可自行探索,或关注我的语雀笔记,笔记会不定期持续更新。

R包openxlsx学习笔记:\url{https://www.yuque.com/docs/share/7a768e6f-95e0-417c-a9b5-dfc8862dc6be?\#}

个人主页:\url{https://www.yuque.com/zyufei}

\hypertarget{readr:package}{%
\section{readr}\label{readr:package}}

readr提供了一种快速友好的方式读取矩形数据\footnote{矩形数据英文中表示为 rectangular data,矩形数据每一列都是变量(特征),而每一行都是案例或记录,关系数据库中的单表就是矩形数据的一种。}(如:csv,tsv,fwf),且当读取大型数据集时默认有进度条显示。

如果对readr包不熟悉,可以直接阅读包作者大神Hadley Wickham的书R for data science 中\href{https://r4ds.had.co.nz/data-import.html}{data import chapter}章节。

\hypertarget{readr:install}{%
\subsection{安装}\label{readr:install}}

直接安装tidyverse获取或单独安装readr。

\begin{Shaded}
\begin{Highlighting}[]
\CommentTok{\# The easiest way to get readr is to install the whole tidyverse:}
\FunctionTok{install.packages}\NormalTok{(}\StringTok{"tidyverse"}\NormalTok{)}

\CommentTok{\# Alternatively, install just readr:}
\FunctionTok{install.packages}\NormalTok{(}\StringTok{"readr"}\NormalTok{)}

\CommentTok{\# Or the the development version from GitHub:}
\CommentTok{\# install.packages("devtools")}
\NormalTok{devtools}\SpecialCharTok{::}\FunctionTok{install\_github}\NormalTok{(}\StringTok{"tidyverse/readr"}\NormalTok{)}
\end{Highlighting}
\end{Shaded}

\hypertarget{readr:usage}{%
\subsection{用法}\label{readr:usage}}

readr包是tidyverse系列的核心包,可以加载tidyverse使用。

\begin{Shaded}
\begin{Highlighting}[]
\FunctionTok{library}\NormalTok{(tidyverse)}
\end{Highlighting}
\end{Shaded}

\hypertarget{ux4e3bux8981ux51fdux6570}{%
\subsubsection{主要函数}\label{ux4e3bux8981ux51fdux6570}}

readr支持七种\texttt{read\_}功能的文件格式。

--- \texttt{read\_csv()}:逗号分隔符文件
- \texttt{read\_tsv()}:制表符分割文件
- \texttt{read\_delim()}:规定分隔符文件
- \texttt{read\_fwf()}:固定宽度文件
- \texttt{read\_table()}:表格文件,列间用空格隔开
- \texttt{read\_log()}:Web日志文件

在大多数情况下,我们常使用\texttt{read\_csv()},提供文件路径,将得到数据表。示例如下:

\begin{Shaded}
\begin{Highlighting}[]
\NormalTok{mtcars }\OtherTok{\textless{}{-}} \FunctionTok{read\_csv}\NormalTok{(}\FunctionTok{readr\_example}\NormalTok{(}\StringTok{"mtcars.csv"}\NormalTok{))}
\CommentTok{\#\textgreater{} }
\CommentTok{\#\textgreater{} {-}{-} Column specification {-}{-}{-}{-}{-}{-}{-}{-}{-}{-}{-}{-}{-}{-}{-}{-}{-}{-}{-}{-}{-}{-}{-}{-}{-}{-}{-}{-}{-}{-}{-}{-}{-}{-}{-}{-}{-}{-}{-}{-}{-}{-}{-}{-}{-}{-}{-}{-}{-}{-}{-}{-}{-}{-}{-}{-}}
\CommentTok{\#\textgreater{} cols(}
\CommentTok{\#\textgreater{}   mpg = col\_double(),}
\CommentTok{\#\textgreater{}   cyl = col\_double(),}
\CommentTok{\#\textgreater{}   disp = col\_double(),}
\CommentTok{\#\textgreater{}   hp = col\_double(),}
\CommentTok{\#\textgreater{}   drat = col\_double(),}
\CommentTok{\#\textgreater{}   wt = col\_double(),}
\CommentTok{\#\textgreater{}   qsec = col\_double(),}
\CommentTok{\#\textgreater{}   vs = col\_double(),}
\CommentTok{\#\textgreater{}   am = col\_double(),}
\CommentTok{\#\textgreater{}   gear = col\_double(),}
\CommentTok{\#\textgreater{}   carb = col\_double()}
\CommentTok{\#\textgreater{} )}
\end{Highlighting}
\end{Shaded}

通过上述输出反馈,我们可以知道读进去的数据集每列类型。如果发现不对可以通过\texttt{col\_types}参数修改。大多数情况下,我们并不需要指定列的类型,readr会自动猜测列类型。

\begin{Shaded}
\begin{Highlighting}[]
\NormalTok{mtcars }\OtherTok{\textless{}{-}} \FunctionTok{read\_csv}\NormalTok{(}\FunctionTok{readr\_example}\NormalTok{(}\StringTok{"mtcars.csv"}\NormalTok{), }\AttributeTok{col\_types =} 
  \FunctionTok{cols}\NormalTok{(}
    \AttributeTok{mpg =} \FunctionTok{col\_double}\NormalTok{(),}
    \AttributeTok{cyl =} \FunctionTok{col\_integer}\NormalTok{(),}
    \AttributeTok{disp =} \FunctionTok{col\_double}\NormalTok{(),}
    \AttributeTok{hp =} \FunctionTok{col\_integer}\NormalTok{(),}
    \AttributeTok{drat =} \FunctionTok{col\_double}\NormalTok{(),}
    \AttributeTok{vs =} \FunctionTok{col\_integer}\NormalTok{(),}
    \AttributeTok{wt =} \FunctionTok{col\_double}\NormalTok{(),}
    \AttributeTok{qsec =} \FunctionTok{col\_double}\NormalTok{(),}
    \AttributeTok{am =} \FunctionTok{col\_integer}\NormalTok{(),}
    \AttributeTok{gear =} \FunctionTok{col\_integer}\NormalTok{(),}
    \AttributeTok{carb =} \FunctionTok{col\_integer}\NormalTok{()}
\NormalTok{  )}
\NormalTok{)}
\end{Highlighting}
\end{Shaded}

\hypertarget{ux53c2ux6570}{%
\subsubsection{参数}\label{ux53c2ux6570}}

\texttt{read\_csv()}的参数如下:

\begin{Shaded}
\begin{Highlighting}[]
\FunctionTok{read\_csv}\NormalTok{(}
\NormalTok{  file,}
  \AttributeTok{col\_names =} \ConstantTok{TRUE}\NormalTok{,}
  \AttributeTok{col\_types =} \ConstantTok{NULL}\NormalTok{,}
  \AttributeTok{locale =} \FunctionTok{default\_locale}\NormalTok{(),}
  \AttributeTok{na =} \FunctionTok{c}\NormalTok{(}\StringTok{""}\NormalTok{, }\StringTok{"NA"}\NormalTok{),}
  \AttributeTok{quoted\_na =} \ConstantTok{TRUE}\NormalTok{,}
  \AttributeTok{quote =} \StringTok{"}\SpecialCharTok{\textbackslash{}"}\StringTok{"}\NormalTok{,}
  \AttributeTok{comment =} \StringTok{""}\NormalTok{,}
  \AttributeTok{trim\_ws =} \ConstantTok{TRUE}\NormalTok{,}
  \AttributeTok{skip =} \DecValTok{0}\NormalTok{,}
  \AttributeTok{n\_max =} \ConstantTok{Inf}\NormalTok{,}
  \AttributeTok{guess\_max =} \FunctionTok{min}\NormalTok{(}\DecValTok{1000}\NormalTok{, n\_max),}
  \AttributeTok{progress =} \FunctionTok{show\_progress}\NormalTok{(),}
  \AttributeTok{skip\_empty\_rows =} \ConstantTok{TRUE}
\NormalTok{)}
\end{Highlighting}
\end{Shaded}

col\_types :指定列类型,可用项如下所示(含简写):
c = character,i = integer,n = number,d = double,l = logical,f = factor,D = date,T = date time,t = time,默认值为guess.

locale:locale参数是readr中很重要的一个参数,指定日期使用的月和日的名称,时区,字符编码,日期格式,数字的小数和点位数和分隔符。

\texttt{locale()}的第一个参数是date\_names,控制月份和日期的名称,指定最简单的方式\href{https://en.wikipedia.org/wiki/List_of_ISO_639-1_codes}{ISO 639 language code}

\begin{Shaded}
\begin{Highlighting}[]
\FunctionTok{locale}\NormalTok{(}\StringTok{\textquotesingle{}zh\textquotesingle{}}\NormalTok{) }\CommentTok{\# 中文}
\CommentTok{\#\textgreater{} \textless{}locale\textgreater{}}
\CommentTok{\#\textgreater{} Numbers:  123,456.78}
\CommentTok{\#\textgreater{} Formats:  \%AD / \%AT}
\CommentTok{\#\textgreater{} Timezone: UTC}
\CommentTok{\#\textgreater{} Encoding: UTF{-}8}
\CommentTok{\#\textgreater{} \textless{}date\_names\textgreater{}}
\CommentTok{\#\textgreater{} Days:   星期日 (周日), 星期一 (周一), 星期二 (周二), 星期三 (周三), 星期四}
\CommentTok{\#\textgreater{}         (周四), 星期五 (周五), 星期六 (周六)}
\CommentTok{\#\textgreater{} Months: 一月 (1月), 二月 (2月), 三月 (3月), 四月 (4月), 五月 (5月), 六月 (6月),}
\CommentTok{\#\textgreater{}         七月 (7月), 八月 (8月), 九月 (9月), 十月 (10月), 十一月 (11月),}
\CommentTok{\#\textgreater{}         十二月 (12月)}
\CommentTok{\#\textgreater{} AM/PM:  上午/下午}
\FunctionTok{locale}\NormalTok{(}\StringTok{\textquotesingle{}ja\textquotesingle{}}\NormalTok{) }\CommentTok{\#日本}
\CommentTok{\#\textgreater{} \textless{}locale\textgreater{}}
\CommentTok{\#\textgreater{} Numbers:  123,456.78}
\CommentTok{\#\textgreater{} Formats:  \%AD / \%AT}
\CommentTok{\#\textgreater{} Timezone: UTC}
\CommentTok{\#\textgreater{} Encoding: UTF{-}8}
\CommentTok{\#\textgreater{} \textless{}date\_names\textgreater{}}
\CommentTok{\#\textgreater{} Days:   日曜日 (日), 月曜日 (月), 火曜日 (火), 水曜日 (水), 木曜日 (木), 金曜日}
\CommentTok{\#\textgreater{}         (金), 土曜日 (土)}
\CommentTok{\#\textgreater{} Months: 1月, 2月, 3月, 4月, 5月, 6月, 7月, 8月, 9月, 10月, 11月, 12月}
\CommentTok{\#\textgreater{} AM/PM:  午前/午後}
\FunctionTok{locale}\NormalTok{(}\StringTok{\textquotesingle{}ko\textquotesingle{}}\NormalTok{) }\CommentTok{\#韩国}
\CommentTok{\#\textgreater{} \textless{}locale\textgreater{}}
\CommentTok{\#\textgreater{} Numbers:  123,456.78}
\CommentTok{\#\textgreater{} Formats:  \%AD / \%AT}
\CommentTok{\#\textgreater{} Timezone: UTC}
\CommentTok{\#\textgreater{} Encoding: UTF{-}8}
\CommentTok{\#\textgreater{} \textless{}date\_names\textgreater{}}
\CommentTok{\#\textgreater{} Days:   \textless{}U+C77C\textgreater{}\textless{}U+C694\textgreater{}\textless{}U+C77C\textgreater{} (\textless{}U+C77C\textgreater{}), \textless{}U+C6D4\textgreater{}\textless{}U+C694\textgreater{}\textless{}U+C77C\textgreater{} (\textless{}U+C6D4\textgreater{}), \textless{}U+D654\textgreater{}\textless{}U+C694\textgreater{}\textless{}U+C77C\textgreater{} (\textless{}U+D654\textgreater{}), \textless{}U+C218\textgreater{}\textless{}U+C694\textgreater{}\textless{}U+C77C\textgreater{} (\textless{}U+C218\textgreater{}), \textless{}U+BAA9\textgreater{}\textless{}U+C694\textgreater{}\textless{}U+C77C\textgreater{} (\textless{}U+BAA9\textgreater{}), \textless{}U+AE08\textgreater{}\textless{}U+C694\textgreater{}\textless{}U+C77C\textgreater{}}
\CommentTok{\#\textgreater{}         (\textless{}U+AE08\textgreater{}), \textless{}U+D1A0\textgreater{}\textless{}U+C694\textgreater{}\textless{}U+C77C\textgreater{} (\textless{}U+D1A0\textgreater{})}
\CommentTok{\#\textgreater{} Months: 1\textless{}U+C6D4\textgreater{}, 2\textless{}U+C6D4\textgreater{}, 3\textless{}U+C6D4\textgreater{}, 4\textless{}U+C6D4\textgreater{}, 5\textless{}U+C6D4\textgreater{}, 6\textless{}U+C6D4\textgreater{}, 7\textless{}U+C6D4\textgreater{}, 8\textless{}U+C6D4\textgreater{}, 9\textless{}U+C6D4\textgreater{}, 10\textless{}U+C6D4\textgreater{}, 11\textless{}U+C6D4\textgreater{}, 12\textless{}U+C6D4\textgreater{}}
\CommentTok{\#\textgreater{} AM/PM:  \textless{}U+C624\textgreater{}\textless{}U+C804\textgreater{}/\textless{}U+C624\textgreater{}\textless{}U+D6C4\textgreater{}}
\end{Highlighting}
\end{Shaded}

编码和时区问题是我们常面临的问题。

\begin{Shaded}
\begin{Highlighting}[]
\FunctionTok{read\_csv}\NormalTok{(}\FunctionTok{readr\_example}\NormalTok{(}\StringTok{"mtcars.csv"}\NormalTok{),}\AttributeTok{locale =} \FunctionTok{locale}\NormalTok{(}\AttributeTok{encoding =} \StringTok{\textquotesingle{}UTF{-}8\textquotesingle{}}\NormalTok{,}\AttributeTok{tz =} \StringTok{\textquotesingle{}Asia/Shanghai\textquotesingle{}}\NormalTok{))}
\end{Highlighting}
\end{Shaded}

详细信息查看手册\texttt{vignette("locales")}。

\hypertarget{readr:write-function}{%
\subsection{导出功能}\label{readr:write-function}}

由于系统不同的缘故,在win系统下可能面临编码问题。我用readr导出数据时一般采用\texttt{write\_excel\_csv()}功能导出。

需要说明的是\texttt{write\_}系列函数可以将输出文件压缩。

\begin{Shaded}
\begin{Highlighting}[]
\FunctionTok{data}\NormalTok{(storms, }\AttributeTok{package =} \StringTok{"dplyr"}\NormalTok{)}
\FunctionTok{write\_csv}\NormalTok{(storms, }\StringTok{"storms.csv"}\NormalTok{)}
\FunctionTok{write\_csv}\NormalTok{(storms, }\StringTok{"storms.csv.gz"}\NormalTok{)}
\end{Highlighting}
\end{Shaded}

\hypertarget{ux603bux7ed3-1}{%
\subsection{总结}\label{ux603bux7ed3-1}}

大部分情况下,当数据整洁时且不涉及时间(日期不影响)时,采用默认参数读取数据即可。

当数据集前面行缺失值较多,readr自动猜数据列类型错误时,需要我们人为指定列类型;当编码时区等错误时,需指定\texttt{locale()};以上是可能遇到的问题以及解决办法。

\hypertarget{data:vroom}{%
\section{vroom}\label{data:vroom}}

vroom实现读取矩形数据到R中,如 comma separated(csv),tab separated(tsv), fixed width files(fwf)。该包的功能类似\texttt{readr::read\_csv()},\texttt{data.table::fread()}和\texttt{read.csv()},但是对于许多数据集来说,\texttt{vroom::vroom()}读取速度会快得多。

\href{https://vroom.r-lib.org/index.html}{vroom项目地址}

\hypertarget{ux5b89ux88c5}{%
\subsection{安装}\label{ux5b89ux88c5}}

\begin{Shaded}
\begin{Highlighting}[]
\CommentTok{\# 从cran安装}
\FunctionTok{install.packages}\NormalTok{(}\StringTok{"vroom"}\NormalTok{)}
\CommentTok{\# install.packages("devtools")}
\NormalTok{devtools}\SpecialCharTok{::}\FunctionTok{install\_dev}\NormalTok{(}\StringTok{"vroom"}\NormalTok{)}
\end{Highlighting}
\end{Shaded}

\hypertarget{ux7528ux6cd5}{%
\subsection{用法}\label{ux7528ux6cd5}}

\begin{enumerate}
\def\labelenumi{\arabic{enumi}.}
\tightlist
\item
  读取文件
\end{enumerate}

\begin{Shaded}
\begin{Highlighting}[]
\FunctionTok{library}\NormalTok{(vroom)}
\NormalTok{file }\OtherTok{\textless{}{-}} \FunctionTok{vroom\_example}\NormalTok{(}\StringTok{"mtcars.csv"}\NormalTok{)}
\NormalTok{file}
\CommentTok{\#\textgreater{} [1] "C:/R/R{-}4.1.0/library/vroom/extdata/mtcars.csv"}

\FunctionTok{vroom}\NormalTok{(file)}
\CommentTok{\#\textgreater{} Rows: 32}
\CommentTok{\#\textgreater{} Columns: 12}
\CommentTok{\#\textgreater{} Delimiter: ","}
\CommentTok{\#\textgreater{} chr [ 1]: model}
\CommentTok{\#\textgreater{} dbl [11]: mpg, cyl, disp, hp, drat, wt, qsec, vs, am, gear, carb}
\CommentTok{\#\textgreater{} }
\CommentTok{\#\textgreater{} Use \textasciigrave{}spec()\textasciigrave{} to retrieve the guessed column specification}
\CommentTok{\#\textgreater{} Pass a specification to the \textasciigrave{}col\_types\textasciigrave{} argument to quiet this message}
\CommentTok{\#\textgreater{} \# A tibble: 32 x 12}
\CommentTok{\#\textgreater{}   model          mpg   cyl  disp    hp  drat    wt  qsec    vs    am  gear  carb}
\CommentTok{\#\textgreater{}   \textless{}chr\textgreater{}        \textless{}dbl\textgreater{} \textless{}dbl\textgreater{} \textless{}dbl\textgreater{} \textless{}dbl\textgreater{} \textless{}dbl\textgreater{} \textless{}dbl\textgreater{} \textless{}dbl\textgreater{} \textless{}dbl\textgreater{} \textless{}dbl\textgreater{} \textless{}dbl\textgreater{} \textless{}dbl\textgreater{}}
\CommentTok{\#\textgreater{} 1 Mazda RX4     21       6   160   110  3.9   2.62  16.5     0     1     4     4}
\CommentTok{\#\textgreater{} 2 Mazda RX4 W\textasciitilde{}  21       6   160   110  3.9   2.88  17.0     0     1     4     4}
\CommentTok{\#\textgreater{} 3 Datsun 710    22.8     4   108    93  3.85  2.32  18.6     1     1     4     1}
\CommentTok{\#\textgreater{} 4 Hornet 4 Dr\textasciitilde{}  21.4     6   258   110  3.08  3.22  19.4     1     0     3     1}
\CommentTok{\#\textgreater{} 5 Hornet Spor\textasciitilde{}  18.7     8   360   175  3.15  3.44  17.0     0     0     3     2}
\CommentTok{\#\textgreater{} 6 Valiant       18.1     6   225   105  2.76  3.46  20.2     1     0     3     1}
\CommentTok{\#\textgreater{} \# ... with 26 more rows}
\FunctionTok{vroom}\NormalTok{(file, }\AttributeTok{delim =} \StringTok{","}\NormalTok{)}
\CommentTok{\#\textgreater{} Rows: 32}
\CommentTok{\#\textgreater{} Columns: 12}
\CommentTok{\#\textgreater{} Delimiter: ","}
\CommentTok{\#\textgreater{} chr [ 1]: model}
\CommentTok{\#\textgreater{} dbl [11]: mpg, cyl, disp, hp, drat, wt, qsec, vs, am, gear, carb}
\CommentTok{\#\textgreater{} }
\CommentTok{\#\textgreater{} Use \textasciigrave{}spec()\textasciigrave{} to retrieve the guessed column specification}
\CommentTok{\#\textgreater{} Pass a specification to the \textasciigrave{}col\_types\textasciigrave{} argument to quiet this message}
\CommentTok{\#\textgreater{} \# A tibble: 32 x 12}
\CommentTok{\#\textgreater{}   model          mpg   cyl  disp    hp  drat    wt  qsec    vs    am  gear  carb}
\CommentTok{\#\textgreater{}   \textless{}chr\textgreater{}        \textless{}dbl\textgreater{} \textless{}dbl\textgreater{} \textless{}dbl\textgreater{} \textless{}dbl\textgreater{} \textless{}dbl\textgreater{} \textless{}dbl\textgreater{} \textless{}dbl\textgreater{} \textless{}dbl\textgreater{} \textless{}dbl\textgreater{} \textless{}dbl\textgreater{} \textless{}dbl\textgreater{}}
\CommentTok{\#\textgreater{} 1 Mazda RX4     21       6   160   110  3.9   2.62  16.5     0     1     4     4}
\CommentTok{\#\textgreater{} 2 Mazda RX4 W\textasciitilde{}  21       6   160   110  3.9   2.88  17.0     0     1     4     4}
\CommentTok{\#\textgreater{} 3 Datsun 710    22.8     4   108    93  3.85  2.32  18.6     1     1     4     1}
\CommentTok{\#\textgreater{} 4 Hornet 4 Dr\textasciitilde{}  21.4     6   258   110  3.08  3.22  19.4     1     0     3     1}
\CommentTok{\#\textgreater{} 5 Hornet Spor\textasciitilde{}  18.7     8   360   175  3.15  3.44  17.0     0     0     3     2}
\CommentTok{\#\textgreater{} 6 Valiant       18.1     6   225   105  2.76  3.46  20.2     1     0     3     1}
\CommentTok{\#\textgreater{} \# ... with 26 more rows}
\end{Highlighting}
\end{Shaded}

\begin{enumerate}
\def\labelenumi{\arabic{enumi}.}
\setcounter{enumi}{1}
\tightlist
\item
  读取多文件
\end{enumerate}

即\texttt{vroom::vroom()}具备迭代效果,具体情况如下:

\begin{Shaded}
\begin{Highlighting}[]
\NormalTok{mt }\OtherTok{\textless{}{-}}\NormalTok{ tibble}\SpecialCharTok{::}\FunctionTok{rownames\_to\_column}\NormalTok{(mtcars, }\StringTok{"model"}\NormalTok{)}
\NormalTok{purrr}\SpecialCharTok{::}\FunctionTok{iwalk}\NormalTok{(}
  \FunctionTok{split}\NormalTok{(mt, mt}\SpecialCharTok{$}\NormalTok{cyl),}
  \SpecialCharTok{\textasciitilde{}} \FunctionTok{vroom\_write}\NormalTok{(.x, glue}\SpecialCharTok{::}\FunctionTok{glue}\NormalTok{(}\StringTok{"mtcars\_\{.y\}.csv"}\NormalTok{), }\StringTok{"}\SpecialCharTok{\textbackslash{}t}\StringTok{"}\NormalTok{)}
\NormalTok{)}

\NormalTok{files }\OtherTok{\textless{}{-}}\NormalTok{ fs}\SpecialCharTok{::}\FunctionTok{dir\_ls}\NormalTok{(}\AttributeTok{glob =} \StringTok{"mtcars*csv"}\NormalTok{)}
\NormalTok{files}

\CommentTok{\# read\_csv}

\NormalTok{purrr}\SpecialCharTok{::}\FunctionTok{map\_dfr}\NormalTok{(files,readr}\SpecialCharTok{::}\NormalTok{read\_delim,}\AttributeTok{delim=}\StringTok{"}\SpecialCharTok{\textbackslash{}t}\StringTok{"}\NormalTok{)}

\CommentTok{\# vroom same above}
\FunctionTok{vroom}\NormalTok{(files) }
\end{Highlighting}
\end{Shaded}

\begin{enumerate}
\def\labelenumi{\arabic{enumi}.}
\setcounter{enumi}{2}
\tightlist
\item
  读取压缩文件
\end{enumerate}

vroom支持zip,gz,bz2,xz等压缩文件,只需要将压缩文件名称传递给vroom即可。

\begin{Shaded}
\begin{Highlighting}[]
\NormalTok{file }\OtherTok{\textless{}{-}} \FunctionTok{vroom\_example}\NormalTok{(}\StringTok{"mtcars.csv.gz"}\NormalTok{)}

\FunctionTok{vroom}\NormalTok{(file)}
\end{Highlighting}
\end{Shaded}

\begin{enumerate}
\def\labelenumi{\arabic{enumi}.}
\setcounter{enumi}{3}
\tightlist
\item
  读取网络文件
\end{enumerate}

\begin{Shaded}
\begin{Highlighting}[]
\NormalTok{file }\OtherTok{\textless{}{-}} \StringTok{"https://raw.githubusercontent.com/r{-}lib/vroom/master/inst/extdata/mtcars.csv"}
\FunctionTok{vroom}\NormalTok{(file)}
\CommentTok{\#\textgreater{} Rows: 32}
\CommentTok{\#\textgreater{} Columns: 12}
\CommentTok{\#\textgreater{} Delimiter: ","}
\CommentTok{\#\textgreater{} chr [ 1]: model}
\CommentTok{\#\textgreater{} dbl [11]: mpg, cyl, disp, hp, drat, wt, qsec, vs, am, gear, carb}
\CommentTok{\#\textgreater{} }
\CommentTok{\#\textgreater{} Use \textasciigrave{}spec()\textasciigrave{} to retrieve the guessed column specification}
\CommentTok{\#\textgreater{} Pass a specification to the \textasciigrave{}col\_types\textasciigrave{} argument to quiet this message}
\CommentTok{\#\textgreater{} \# A tibble: 32 x 12}
\CommentTok{\#\textgreater{}   model          mpg   cyl  disp    hp  drat    wt  qsec    vs    am  gear  carb}
\CommentTok{\#\textgreater{}   \textless{}chr\textgreater{}        \textless{}dbl\textgreater{} \textless{}dbl\textgreater{} \textless{}dbl\textgreater{} \textless{}dbl\textgreater{} \textless{}dbl\textgreater{} \textless{}dbl\textgreater{} \textless{}dbl\textgreater{} \textless{}dbl\textgreater{} \textless{}dbl\textgreater{} \textless{}dbl\textgreater{} \textless{}dbl\textgreater{}}
\CommentTok{\#\textgreater{} 1 Mazda RX4     21       6   160   110  3.9   2.62  16.5     0     1     4     4}
\CommentTok{\#\textgreater{} 2 Mazda RX4 W\textasciitilde{}  21       6   160   110  3.9   2.88  17.0     0     1     4     4}
\CommentTok{\#\textgreater{} 3 Datsun 710    22.8     4   108    93  3.85  2.32  18.6     1     1     4     1}
\CommentTok{\#\textgreater{} 4 Hornet 4 Dr\textasciitilde{}  21.4     6   258   110  3.08  3.22  19.4     1     0     3     1}
\CommentTok{\#\textgreater{} 5 Hornet Spor\textasciitilde{}  18.7     8   360   175  3.15  3.44  17.0     0     0     3     2}
\CommentTok{\#\textgreater{} 6 Valiant       18.1     6   225   105  2.76  3.46  20.2     1     0     3     1}
\CommentTok{\#\textgreater{} \# ... with 26 more rows}
\end{Highlighting}
\end{Shaded}

\begin{enumerate}
\def\labelenumi{\arabic{enumi}.}
\setcounter{enumi}{4}
\tightlist
\item
  选择列读取
\end{enumerate}

room提供了与\texttt{dplyr::select()}相同的列选择和重命名接口

\begin{Shaded}
\begin{Highlighting}[]
\NormalTok{file }\OtherTok{\textless{}{-}} \FunctionTok{vroom\_example}\NormalTok{(}\StringTok{"mtcars.csv.gz"}\NormalTok{)}

\FunctionTok{vroom}\NormalTok{(file, }\AttributeTok{col\_select =} \FunctionTok{c}\NormalTok{(model, cyl, gear))}
\CommentTok{\#\textgreater{} \# A tibble: 32 x 3}
\CommentTok{\#\textgreater{}   model               cyl  gear}
\CommentTok{\#\textgreater{}   \textless{}chr\textgreater{}             \textless{}dbl\textgreater{} \textless{}dbl\textgreater{}}
\CommentTok{\#\textgreater{} 1 Mazda RX4             6     4}
\CommentTok{\#\textgreater{} 2 Mazda RX4 Wag         6     4}
\CommentTok{\#\textgreater{} 3 Datsun 710            4     4}
\CommentTok{\#\textgreater{} 4 Hornet 4 Drive        6     3}
\CommentTok{\#\textgreater{} 5 Hornet Sportabout     8     3}
\CommentTok{\#\textgreater{} 6 Valiant               6     3}
\CommentTok{\#\textgreater{} \# ... with 26 more rows}

\CommentTok{\# vroom(file, col\_select = c(1, 3, 11))}

\CommentTok{\# vroom(file, col\_select = list(car = model, everything()))}
\end{Highlighting}
\end{Shaded}

\hypertarget{data:rstudio-addins}{%
\section{Rstudio导入}\label{data:rstudio-addins}}

\hypertarget{ux5229ux7528rstudioux5de5ux5177ux680fux5bfcux5165}{%
\subsection{利用rstudio工具栏导入}\label{ux5229ux7528rstudioux5de5ux5177ux680fux5bfcux5165}}

本质也是调用\texttt{readr}和\texttt{readxl}包,如下所示:

\begin{figure}
\centering
\includegraphics{picture/read-write/Rstudio-load-data.png}
\caption{rstudio-load-data}
\end{figure}

\hypertarget{ux63d2ux4ef6ux5bfcux5165}{%
\subsection{插件导入}\label{ux63d2ux4ef6ux5bfcux5165}}

\href{https://github.com/milesmcbain/datapasta}{项目地址}

datapasta是一个addins插件,方便将数据复制到R。

1.安装

\begin{Shaded}
\begin{Highlighting}[]
\FunctionTok{install.packages}\NormalTok{(}\StringTok{"datapasta"}\NormalTok{)}
\end{Highlighting}
\end{Shaded}

2.使用

\begin{figure}
\centering
\includegraphics{picture/read-write/datapasta-copy.gif}
\caption{datapasta}
\end{figure}

\hypertarget{data:file-path}{%
\section{文件路径}\label{data:file-path}}

我们读取数据时都是读取某路径下的某文件,但是由于系统等原因,路径在不同系统下的表示方式不一致。

\hypertarget{ux6307ux5b9aux8defux5f84}{%
\subsection{指定路径}\label{ux6307ux5b9aux8defux5f84}}

\begin{itemize}
\tightlist
\item
  win 路径
\end{itemize}

winOS系统:\texttt{C:\textbackslash{}Users\textbackslash{}zhongyf\textbackslash{}Desktop\textbackslash{}Rbook},注意路径中是一个反斜杠()。

\begin{figure}
\centering
\includegraphics{picture/read-write/win-path.png}
\caption{win-path}
\end{figure}

在R中读取时需要用一个正斜杠或两个反斜杠。

\begin{Shaded}
\begin{Highlighting}[]
\NormalTok{readr}\SpecialCharTok{::}\FunctionTok{read\_csv}\NormalTok{(}\StringTok{\textquotesingle{}C:/Users/zhongyf/Desktop/Rbook/data/flights.csv\textquotesingle{}}\NormalTok{)}
\NormalTok{readr}\SpecialCharTok{::}\FunctionTok{read\_csv}\NormalTok{(}\StringTok{\textquotesingle{}C:}\SpecialCharTok{\textbackslash{}\textbackslash{}}\StringTok{Users}\SpecialCharTok{\textbackslash{}\textbackslash{}}\StringTok{zhongyf}\SpecialCharTok{\textbackslash{}\textbackslash{}}\StringTok{Desktop}\SpecialCharTok{\textbackslash{}\textbackslash{}}\StringTok{Rbook}\SpecialCharTok{\textbackslash{}\textbackslash{}}\StringTok{data}\SpecialCharTok{\textbackslash{}f}\StringTok{lights.csv\textquotesingle{}}\NormalTok{) }\CommentTok{\# same above}
\NormalTok{readr}\SpecialCharTok{:::}\FunctionTok{read\_csv}\NormalTok{(}\AttributeTok{file =}\NormalTok{ r}\StringTok{"(C:\textbackslash{}Users\textbackslash{}zhongyf\textbackslash{}Desktop\textbackslash{}Rbook\textbackslash{}data}\SpecialCharTok{\textbackslash{}f}\StringTok{lights.csv)"}\NormalTok{) }\CommentTok{\# same above}
\end{Highlighting}
\end{Shaded}

工作中,当需要读取或写入共享盘\footnote{共享盘的地址即某电脑(服务器)的地址,知道共享盘在局域网中的ip地址后在 开始-\textgreater 运行中输入\textbackslash192.168.1.247即可打开共享盘。}中文件时,路径表示方式为:

\begin{Shaded}
\begin{Highlighting}[]
\NormalTok{the\_shared\_disk }\OtherTok{\textless{}{-}}\NormalTok{ r}\StringTok{"(}\SpecialCharTok{\textbackslash{}\textbackslash{}}\StringTok{192.168.2.117\textbackslash{}公司A{-}新}\SpecialCharTok{\textbackslash{}01}\StringTok{事业部\textbackslash{})"}

\CommentTok{\# load data into R}

\NormalTok{readr}\SpecialCharTok{::}\FunctionTok{read\_csv}\NormalTok{(}\AttributeTok{file =} \StringTok{"}\SpecialCharTok{\textbackslash{}\textbackslash{}\textbackslash{}\textbackslash{}}\StringTok{192.168.2.117}\SpecialCharTok{\textbackslash{}\textbackslash{}}\StringTok{公司A{-}新}\SpecialCharTok{\textbackslash{}\textbackslash{}}\StringTok{01事业部}\SpecialCharTok{\textbackslash{}\textbackslash{}}\StringTok{flights.csv"}\NormalTok{)}
\NormalTok{readr}\SpecialCharTok{::}\FunctionTok{read\_csv}\NormalTok{(}\AttributeTok{file =}\NormalTok{ r}\StringTok{"(}\SpecialCharTok{\textbackslash{}\textbackslash{}}\StringTok{192.168.2.117\textbackslash{}公司A{-}新}\SpecialCharTok{\textbackslash{}01}\StringTok{事业部\textbackslash{})"}\NormalTok{)}
\end{Highlighting}
\end{Shaded}

\begin{quote}
r``()''用法是R-4.0-之后的特性。在win系统下表示路径特别有用
\end{quote}

\begin{itemize}
\tightlist
\item
  mac 路径
\end{itemize}

macOS系统: \texttt{/User/vega\_mac/Desktop/r},路径中是一个正斜杠。

\begin{figure}
\centering
\includegraphics{picture/read-write/mac-path.png}
\caption{mac-path}
\end{figure}

\begin{Shaded}
\begin{Highlighting}[]
\NormalTok{readr}\SpecialCharTok{::}\FunctionTok{read\_csv}\NormalTok{(}\StringTok{\textquotesingle{}/User/vega\_mac/Desktop/r/Rbook/data/flights.csv\textquotesingle{}}\NormalTok{)}
\end{Highlighting}
\end{Shaded}

\hypertarget{ux9ed8ux8ba4ux8defux5f84}{%
\subsection{默认路径}\label{ux9ed8ux8ba4ux8defux5f84}}

\texttt{getwd()}是查看当前工作目录的函数,在进行文件读写时的默认路径,也就是当没有明确指定路径时,读取导出的默认路径是\texttt{getwd()}。想要改变工作目录,通过设定\texttt{setwd()}即可。

\begin{Shaded}
\begin{Highlighting}[]
\FunctionTok{getwd}\NormalTok{()}
\CommentTok{\#\textgreater{} [1] "C:/Users/zhongyf/Desktop/Rbook"}
\end{Highlighting}
\end{Shaded}

\begin{Shaded}
\begin{Highlighting}[]
\CommentTok{\# not run}
\FunctionTok{setwd}\NormalTok{(}\StringTok{\textquotesingle{}C:/Users/zhongyf/Desktop/Rbook/data\textquotesingle{}}\NormalTok{)}
\FunctionTok{getwd}\NormalTok{()}
\end{Highlighting}
\end{Shaded}

\hypertarget{data:expand}{%
\section{拓展}\label{data:expand}}

\begin{enumerate}
\def\labelenumi{\arabic{enumi}.}
\item
  feather项目地址\url{https://github.com/wesm/feather}
\item
  qs提供接口,用于快速将R对象保存到磁盘以及从磁盘读取。该包的目标是替换R中的\texttt{saveRDS}和\texttt{readRDS}。项目地址\url{https://github.com/traversc/qs}
\item
  arrow是feather的接替项目,地址\url{https://arrow.apache.org/docs/r/}
\item
  其它统计学软件数据如spss,stata,SAs等可用\texttt{foreign}包读取
\end{enumerate}

\hypertarget{Data:Manipulation-dplyr}{%
\chapter{数据处理之-dplyr}\label{Data:Manipulation-dplyr}}

本章节主要目的是通过阐述dplyr动词用法,实现与\texttt{Excel透视表}或\texttt{sql}相同功能,从而达到不同的数据整理、聚合需求。

本章主要从以下方面阐述:

\begin{enumerate}
\def\labelenumi{\arabic{enumi}.}
\tightlist
\item
  行条件筛选
\item
  列筛选
\item
  字段重命名
\item
  列位置排序
\item
  行排序
\item
  新增计算字段
\item
  分组聚合
\item
  表关联
\item
  行列操作
\item
  使用dplyr编写自定义函数
\end{enumerate}

其中9,10行列操作和自定义函数有一定难度,大家可以先熟悉dplyr基本用法后再了解其用法。

与\texttt{sql}相比,用dplyr的优势:

\begin{itemize}
\item
  代码量极大减少
\item
  逻辑复杂时,dplyr动词可以按照顺序一步步实现,无需嵌套,实现过程简单
\item
  代码可读性好
\item
  配合\texttt{dbplyr}包使用,大部分情况下可以扔掉\texttt{sql}语法,从而实现不同数据库间语法并不完全一致时,代码可重复使用
\end{itemize}

\begin{quote}
本章节中部分案例照搬dplyr包的官方案例,
dplyr动词从数据库相关操作中抽象而来,从sql迁移成本低
\end{quote}

\hypertarget{dplyr:description}{%
\section{前言}\label{dplyr:description}}

\texttt{dplyr}包是\texttt{tidyverse}系列的核心包之一。dplyr是\textbf{A Grammar of Data Manipulation },即dplyr是数据处理的语法。数据操作在数据库中往往被增、改、删、查四字描述,加上表连接查询基本涵盖了大部分的数据操作。

\texttt{dplyr}包通过提供一组动词来解决最常见的数据处理问题:

\begin{itemize}
\item
  \texttt{mutate()} 添加新变量,现有变量的函数
\item
  \texttt{select()} 筛选列,根据现有变量名称选择变量
\item
  \texttt{filter()} 筛选行,根据条件筛选
\item
  \texttt{summarise()} 按照一定条件汇总聚合
\item
  \texttt{arrange()} 行排序
\end{itemize}

以上动词都可以和\texttt{group\_by()}结合,使我们可以按组执行以上任何操作。除了以上单个表操作的动词,dplyr中还有操作两表(表关联)的动词,可以通过\texttt{vignette("two-table")}查看学习。

\hypertarget{dplyr:install-package}{%
\subsection{安装}\label{dplyr:install-package}}

dplyr包可以直接安装。

\begin{Shaded}
\begin{Highlighting}[]
\DocumentationTok{\#\# 最简单是的方式就是安装tidyverse}
\FunctionTok{install.packages}\NormalTok{(}\StringTok{\textquotesingle{}tidyverse\textquotesingle{}}\NormalTok{)}

\DocumentationTok{\#\# 或者仅仅安装 tidyr:}
\FunctionTok{install.packages}\NormalTok{(}\StringTok{\textquotesingle{}dplyr\textquotesingle{}}\NormalTok{)}

\DocumentationTok{\#\# 或者从github 安装开发版本}
\DocumentationTok{\#\# install.packages("devtools")}
\NormalTok{devtools}\SpecialCharTok{::}\FunctionTok{install\_github}\NormalTok{(}\StringTok{"tidyverse/dplyr"}\NormalTok{)}
\end{Highlighting}
\end{Shaded}

在开始使用前,请确保自己dplyr版本是较新版本,因为1.0.0版本有较大更新。

\begin{Shaded}
\begin{Highlighting}[]
\FunctionTok{packageVersion}\NormalTok{(}\StringTok{\textquotesingle{}dplyr\textquotesingle{}}\NormalTok{)}
\CommentTok{\#\textgreater{} [1] \textquotesingle{}1.0.6\textquotesingle{}}
\end{Highlighting}
\end{Shaded}

\hypertarget{dplyr:difference-of-sql}{%
\subsection{Excel and Sql 类比}\label{dplyr:difference-of-sql}}

与Excel相比,dplyr使用\texttt{filter}实现筛选,\texttt{mutate}实现列新增计算,\texttt{summarise}配合\texttt{group\_by}实现数据透视表,\texttt{arrange}实现排序功能。
\texttt{dplyr::left\_join()}等表连接功能,实现Excel中的\texttt{vlookup},\texttt{xlookup}等函数效果。

请看案例:

\begin{quote}
案例中使用的数据集是R包\texttt{nycflights13}带的flights数据集。
\end{quote}

Excel实现

\begin{figure}
\centering
\includegraphics{./picture/data-table/01picture.png}
\caption{透视表截图}
\end{figure}

R实现:

\begin{Shaded}
\begin{Highlighting}[]
\FunctionTok{library}\NormalTok{(tidyverse,}\AttributeTok{warn.conflicts =} \ConstantTok{FALSE}\NormalTok{)}
\NormalTok{data }\OtherTok{\textless{}{-}}\NormalTok{ readr}\SpecialCharTok{::}\FunctionTok{read\_csv}\NormalTok{(}\StringTok{"./data/flights.csv"}\NormalTok{)}

\NormalTok{data }\SpecialCharTok{\%\textgreater{}\%} 
  \FunctionTok{filter}\NormalTok{(year}\SpecialCharTok{==}\DecValTok{2014}\NormalTok{,month}\SpecialCharTok{==}\DecValTok{6}\NormalTok{) }\SpecialCharTok{\%\textgreater{}\%} 
  \FunctionTok{group\_by}\NormalTok{(origin,dest) }\SpecialCharTok{\%\textgreater{}\%} 
  \FunctionTok{summarise}\NormalTok{(distance求和项 }\OtherTok{=} \FunctionTok{sum}\NormalTok{(distance))}
\CommentTok{\#\textgreater{} \# A tibble: 195 x 3}
\CommentTok{\#\textgreater{} \# Groups:   origin [3]}
\CommentTok{\#\textgreater{}   origin dest  distance求和项}
\CommentTok{\#\textgreater{}   \textless{}chr\textgreater{}  \textless{}chr\textgreater{}          \textless{}dbl\textgreater{}}
\CommentTok{\#\textgreater{} 1 EWR    ALB              715}
\CommentTok{\#\textgreater{} 2 EWR    ANC            13480}
\CommentTok{\#\textgreater{} 3 EWR    ATL           317050}
\CommentTok{\#\textgreater{} 4 EWR    AUS            88736}
\CommentTok{\#\textgreater{} 5 EWR    AVL            13409}
\CommentTok{\#\textgreater{} 6 EWR    BDL             8236}
\CommentTok{\#\textgreater{} \# ... with 189 more rows}
\end{Highlighting}
\end{Shaded}

Sql实现:

\begin{Shaded}
\begin{Highlighting}[]
\NormalTok{select origin,dest,}\FunctionTok{sum}\NormalTok{(distance) distance求和项 from flights where year }\OtherTok{=} \DecValTok{2014}\NormalTok{ and month }\OtherTok{=}\DecValTok{6}\NormalTok{ group by origin,dest}
\end{Highlighting}
\end{Shaded}

\hypertarget{ux5e38ux89c1ux95eeux9898}{%
\subsection{常见问题}\label{ux5e38ux89c1ux95eeux9898}}

\begin{enumerate}
\def\labelenumi{\arabic{enumi}.}
\tightlist
\item
  筛选订单表中的1-5月订单数据,按照城市汇总,求每个城市的销售额和门店数(去重)?
\end{enumerate}

\begin{Shaded}
\begin{Highlighting}[]
\NormalTok{data }\SpecialCharTok{\%\textgreater{}\%} 
  \FunctionTok{filter}\NormalTok{(}\FunctionTok{between}\NormalTok{(月,}\DecValTok{1}\NormalTok{,}\DecValTok{5}\NormalTok{)) }\SpecialCharTok{\%\textgreater{}\%} 
  \FunctionTok{group\_by}\NormalTok{(城市) }\SpecialCharTok{\%\textgreater{}\%} 
  \FunctionTok{summarise}\NormalTok{(金额 }\OtherTok{=} \FunctionTok{sum}\NormalTok{(金额),门店数 }\OtherTok{=} \FunctionTok{n\_distinct}\NormalTok{(门店编码))}
\end{Highlighting}
\end{Shaded}

\begin{enumerate}
\def\labelenumi{\arabic{enumi}.}
\setcounter{enumi}{1}
\tightlist
\item
  近30天商品销量排名?
\end{enumerate}

\begin{Shaded}
\begin{Highlighting}[]
\NormalTok{data }\SpecialCharTok{\%\textgreater{}\%} 
  \FunctionTok{filter}\NormalTok{(订单日期 }\SpecialCharTok{\textgreater{}=} \FunctionTok{Sys.Date}\NormalTok{()}\SpecialCharTok{{-}}\DecValTok{30}\NormalTok{) }\SpecialCharTok{\%\textgreater{}\%} 
  \FunctionTok{group\_by}\NormalTok{(分析大类,商品编码) }\SpecialCharTok{\%\textgreater{}\%} 
  \FunctionTok{summarise}\NormalTok{(商品销量 }\OtherTok{=} \FunctionTok{sum}\NormalTok{(数量)) }\SpecialCharTok{\%\textgreater{}\%} 
  \FunctionTok{group\_by}\NormalTok{(分析大类) }\SpecialCharTok{\%\textgreater{}\%} 
  \FunctionTok{mutate}\NormalTok{(商品排名 }\OtherTok{=} \FunctionTok{dense\_rank}\NormalTok{(}\FunctionTok{desc}\NormalTok{(商品销量))) }
\CommentTok{\# 注意用desc倒序,销量高排第一}
\end{Highlighting}
\end{Shaded}

\begin{enumerate}
\def\labelenumi{\arabic{enumi}.}
\setcounter{enumi}{2}
\tightlist
\item
  销售和库存形成笛卡尔积表,计算商品有货率、动销率?
\end{enumerate}

\textbf{Cheat Sheet}

手册搬运于dplyr\href{https://dplyr.tidyverse.org/}{官方介绍}

\begin{figure}
\centering
\includegraphics[width=1\textwidth,height=4.16667in]{./picture/dplyr/data-transformation.pdf}
\caption{dplyr-sheet}
\end{figure}

Rstudio其它手册:\url{https://www.rstudio.com/resources/cheatsheets/}

\hypertarget{dplyr:usage}{%
\section{基础用法}\label{dplyr:usage}}

基础用法部分,我们将从行筛选,重命名、列位置调整、新增计算列、排序、分组聚合几个方面阐述\texttt{dplyr}动词功能。

首先加载包,加载包时可能会有一些重名函数的提示,可以通过warn.conflict参数禁掉提示。如下所示:

\begin{Shaded}
\begin{Highlighting}[]
\CommentTok{\# 禁掉提示}
\FunctionTok{library}\NormalTok{(dplyr,}\AttributeTok{warn.conflicts =} \ConstantTok{FALSE}\NormalTok{)}
\end{Highlighting}
\end{Shaded}

\hypertarget{dplyr-filter}{%
\subsection{filter}\label{dplyr-filter}}

\texttt{filter}动词顾名思义即筛选功能,按照一定条件筛选data.frame;与Excel中的筛选功能和\texttt{SQL}中\texttt{where}条件一致。

filter条件筛选中可以分为单条件筛选和多条件筛选;多条件中间用\texttt{,}分隔。

\begin{itemize}
\tightlist
\item
  单条件
\end{itemize}

条件为\texttt{species\ ==\ "Droid"}时,如下所示:

\begin{Shaded}
\begin{Highlighting}[]
\NormalTok{starwars }\SpecialCharTok{\%\textgreater{}\%} 
  \FunctionTok{filter}\NormalTok{(species }\SpecialCharTok{==} \StringTok{"Droid"}\NormalTok{)}
\CommentTok{\#\textgreater{} \# A tibble: 6 x 14}
\CommentTok{\#\textgreater{}   name   height  mass hair\_color skin\_color  eye\_color birth\_year sex   gender  }
\CommentTok{\#\textgreater{}   \textless{}chr\textgreater{}   \textless{}int\textgreater{} \textless{}dbl\textgreater{} \textless{}chr\textgreater{}      \textless{}chr\textgreater{}       \textless{}chr\textgreater{}          \textless{}dbl\textgreater{} \textless{}chr\textgreater{} \textless{}chr\textgreater{}   }
\CommentTok{\#\textgreater{} 1 C{-}3PO     167    75 \textless{}NA\textgreater{}       gold        yellow           112 none  masculi\textasciitilde{}}
\CommentTok{\#\textgreater{} 2 R2{-}D2      96    32 \textless{}NA\textgreater{}       white, blue red               33 none  masculi\textasciitilde{}}
\CommentTok{\#\textgreater{} 3 R5{-}D4      97    32 \textless{}NA\textgreater{}       white, red  red               NA none  masculi\textasciitilde{}}
\CommentTok{\#\textgreater{} 4 IG{-}88     200   140 none       metal       red               15 none  masculi\textasciitilde{}}
\CommentTok{\#\textgreater{} 5 R4{-}P17     96    NA none       silver, red red, blue         NA none  feminine}
\CommentTok{\#\textgreater{} 6 BB8        NA    NA none       none        black             NA none  masculi\textasciitilde{}}
\CommentTok{\#\textgreater{} \# ... with 5 more variables: homeworld \textless{}chr\textgreater{}, species \textless{}chr\textgreater{}, films \textless{}list\textgreater{},}
\CommentTok{\#\textgreater{} \#   vehicles \textless{}list\textgreater{}, starships \textless{}list\textgreater{}}
\end{Highlighting}
\end{Shaded}

\begin{Shaded}
\begin{Highlighting}[]
\KeywordTok{select} \OperatorTok{*} \KeywordTok{from}\NormalTok{ starwars }\KeywordTok{where}\NormalTok{ species }\OperatorTok{=} \OtherTok{"Droid"} \CommentTok{{-}{-} 注意=与==的区别}
\end{Highlighting}
\end{Shaded}

\begin{itemize}
\tightlist
\item
  多条件
\end{itemize}

多条件筛选时,用英文逗号隔开多个条件。用``and''连接多个条件与用逗号隔开效果相同,``and''在R中用\&表示。

\begin{Shaded}
\begin{Highlighting}[]
\NormalTok{starwars }\SpecialCharTok{\%\textgreater{}\%} 
  \FunctionTok{filter}\NormalTok{(species }\SpecialCharTok{==} \StringTok{"Droid"}\NormalTok{,skin\_color }\SpecialCharTok{==} \StringTok{"gold"}\NormalTok{)}
\CommentTok{\#\textgreater{} \# A tibble: 1 x 14}
\CommentTok{\#\textgreater{}   name  height  mass hair\_color skin\_color eye\_color birth\_year sex   gender   }
\CommentTok{\#\textgreater{}   \textless{}chr\textgreater{}  \textless{}int\textgreater{} \textless{}dbl\textgreater{} \textless{}chr\textgreater{}      \textless{}chr\textgreater{}      \textless{}chr\textgreater{}          \textless{}dbl\textgreater{} \textless{}chr\textgreater{} \textless{}chr\textgreater{}    }
\CommentTok{\#\textgreater{} 1 C{-}3PO    167    75 \textless{}NA\textgreater{}       gold       yellow           112 none  masculine}
\CommentTok{\#\textgreater{} \# ... with 5 more variables: homeworld \textless{}chr\textgreater{}, species \textless{}chr\textgreater{}, films \textless{}list\textgreater{},}
\CommentTok{\#\textgreater{} \#   vehicles \textless{}list\textgreater{}, starships \textless{}list\textgreater{}}

\CommentTok{\# same above}
\CommentTok{\# starwars \%\textgreater{}\% }
\CommentTok{\#   filter(species == "Droid" \& skin\_color == "gold")}
\end{Highlighting}
\end{Shaded}

\begin{Shaded}
\begin{Highlighting}[]
\KeywordTok{select} \OperatorTok{*} \KeywordTok{from}\NormalTok{ starwars }\KeywordTok{where}\NormalTok{ species }\OperatorTok{=} \OtherTok{"Droid"} \KeywordTok{and}\NormalTok{ skin\_color }\OperatorTok{=} \OtherTok{"gold"} 
\end{Highlighting}
\end{Shaded}

\begin{itemize}
\tightlist
\item
  多情况筛选
\end{itemize}

类似\texttt{SQL}中 \texttt{in} 的用法,或Excel中筛选条件时``或''条件

\begin{Shaded}
\begin{Highlighting}[]
\NormalTok{starwars }\SpecialCharTok{\%\textgreater{}\%} 
  \FunctionTok{filter}\NormalTok{(species }\SpecialCharTok{\%in\%}  \FunctionTok{c}\NormalTok{(}\StringTok{"Droid"}\NormalTok{,}\StringTok{\textquotesingle{}Clawdite\textquotesingle{}}\NormalTok{))}
\CommentTok{\#\textgreater{} \# A tibble: 7 x 14}
\CommentTok{\#\textgreater{}   name    height  mass hair\_color skin\_color   eye\_color birth\_year sex   gender}
\CommentTok{\#\textgreater{}   \textless{}chr\textgreater{}    \textless{}int\textgreater{} \textless{}dbl\textgreater{} \textless{}chr\textgreater{}      \textless{}chr\textgreater{}        \textless{}chr\textgreater{}          \textless{}dbl\textgreater{} \textless{}chr\textgreater{} \textless{}chr\textgreater{} }
\CommentTok{\#\textgreater{} 1 C{-}3PO      167    75 \textless{}NA\textgreater{}       gold         yellow           112 none  mascu\textasciitilde{}}
\CommentTok{\#\textgreater{} 2 R2{-}D2       96    32 \textless{}NA\textgreater{}       white, blue  red               33 none  mascu\textasciitilde{}}
\CommentTok{\#\textgreater{} 3 R5{-}D4       97    32 \textless{}NA\textgreater{}       white, red   red               NA none  mascu\textasciitilde{}}
\CommentTok{\#\textgreater{} 4 IG{-}88      200   140 none       metal        red               15 none  mascu\textasciitilde{}}
\CommentTok{\#\textgreater{} 5 Zam We\textasciitilde{}    168    55 blonde     fair, green\textasciitilde{} yellow            NA fema\textasciitilde{} femin\textasciitilde{}}
\CommentTok{\#\textgreater{} 6 R4{-}P17      96    NA none       silver, red  red, blue         NA none  femin\textasciitilde{}}
\CommentTok{\#\textgreater{} \# ... with 1 more row, and 5 more variables: homeworld \textless{}chr\textgreater{}, species \textless{}chr\textgreater{},}
\CommentTok{\#\textgreater{} \#   films \textless{}list\textgreater{}, vehicles \textless{}list\textgreater{}, starships \textless{}list\textgreater{}}
\end{Highlighting}
\end{Shaded}

\begin{Shaded}
\begin{Highlighting}[]
\KeywordTok{select} \OperatorTok{*} \KeywordTok{from}\NormalTok{ starwars }\KeywordTok{where}\NormalTok{ species }\KeywordTok{in}\NormalTok{ (}\OtherTok{"Droid"}\NormalTok{,}\OtherTok{"Clawdite"}\NormalTok{) }\CommentTok{{-}{-}sql查询}
\end{Highlighting}
\end{Shaded}

\begin{itemize}
\tightlist
\item
  逻辑关系筛选
\end{itemize}

条件运算分为逻辑运算、关系运算。

关系运算符 \textgreater、\textless、==、!=、\textgreater=、\textless=分别代表大于、小于、等于、不等于、大于等于、小于等于。

逻辑运算符 \&、\textbar、!。 \texttt{\textbar{}}为 或, \texttt{\&} 为并、且条件,\texttt{!}为非。

\begin{Shaded}
\begin{Highlighting}[]
\FunctionTok{library}\NormalTok{(nycflights13)}
\FunctionTok{filter}\NormalTok{(flights, }\SpecialCharTok{!}\NormalTok{(arr\_delay }\SpecialCharTok{\textgreater{}} \DecValTok{120} \SpecialCharTok{|}\NormalTok{ dep\_delay }\SpecialCharTok{\textgreater{}} \DecValTok{120}\NormalTok{))}
\CommentTok{\#\textgreater{} \# A tibble: 316,050 x 19}
\CommentTok{\#\textgreater{}    year month   day dep\_time sched\_dep\_time dep\_delay arr\_time sched\_arr\_time}
\CommentTok{\#\textgreater{}   \textless{}int\textgreater{} \textless{}int\textgreater{} \textless{}int\textgreater{}    \textless{}int\textgreater{}          \textless{}int\textgreater{}     \textless{}dbl\textgreater{}    \textless{}int\textgreater{}          \textless{}int\textgreater{}}
\CommentTok{\#\textgreater{} 1  2013     1     1      517            515         2      830            819}
\CommentTok{\#\textgreater{} 2  2013     1     1      533            529         4      850            830}
\CommentTok{\#\textgreater{} 3  2013     1     1      542            540         2      923            850}
\CommentTok{\#\textgreater{} 4  2013     1     1      544            545        {-}1     1004           1022}
\CommentTok{\#\textgreater{} 5  2013     1     1      554            600        {-}6      812            837}
\CommentTok{\#\textgreater{} 6  2013     1     1      554            558        {-}4      740            728}
\CommentTok{\#\textgreater{} \# ... with 316,044 more rows, and 11 more variables: arr\_delay \textless{}dbl\textgreater{},}
\CommentTok{\#\textgreater{} \#   carrier \textless{}chr\textgreater{}, flight \textless{}int\textgreater{}, tailnum \textless{}chr\textgreater{}, origin \textless{}chr\textgreater{}, dest \textless{}chr\textgreater{},}
\CommentTok{\#\textgreater{} \#   air\_time \textless{}dbl\textgreater{}, distance \textless{}dbl\textgreater{}, hour \textless{}dbl\textgreater{}, minute \textless{}dbl\textgreater{}, time\_hour \textless{}dttm\textgreater{}}
\FunctionTok{filter}\NormalTok{(flights, arr\_delay }\SpecialCharTok{\textless{}=} \DecValTok{120}\NormalTok{, dep\_delay }\SpecialCharTok{\textless{}=} \DecValTok{120}\NormalTok{)}
\CommentTok{\#\textgreater{} \# A tibble: 316,050 x 19}
\CommentTok{\#\textgreater{}    year month   day dep\_time sched\_dep\_time dep\_delay arr\_time sched\_arr\_time}
\CommentTok{\#\textgreater{}   \textless{}int\textgreater{} \textless{}int\textgreater{} \textless{}int\textgreater{}    \textless{}int\textgreater{}          \textless{}int\textgreater{}     \textless{}dbl\textgreater{}    \textless{}int\textgreater{}          \textless{}int\textgreater{}}
\CommentTok{\#\textgreater{} 1  2013     1     1      517            515         2      830            819}
\CommentTok{\#\textgreater{} 2  2013     1     1      533            529         4      850            830}
\CommentTok{\#\textgreater{} 3  2013     1     1      542            540         2      923            850}
\CommentTok{\#\textgreater{} 4  2013     1     1      544            545        {-}1     1004           1022}
\CommentTok{\#\textgreater{} 5  2013     1     1      554            600        {-}6      812            837}
\CommentTok{\#\textgreater{} 6  2013     1     1      554            558        {-}4      740            728}
\CommentTok{\#\textgreater{} \# ... with 316,044 more rows, and 11 more variables: arr\_delay \textless{}dbl\textgreater{},}
\CommentTok{\#\textgreater{} \#   carrier \textless{}chr\textgreater{}, flight \textless{}int\textgreater{}, tailnum \textless{}chr\textgreater{}, origin \textless{}chr\textgreater{}, dest \textless{}chr\textgreater{},}
\CommentTok{\#\textgreater{} \#   air\_time \textless{}dbl\textgreater{}, distance \textless{}dbl\textgreater{}, hour \textless{}dbl\textgreater{}, minute \textless{}dbl\textgreater{}, time\_hour \textless{}dttm\textgreater{}}

\CommentTok{\# same above}
\FunctionTok{filter}\NormalTok{(flights, arr\_delay }\SpecialCharTok{\textless{}=} \DecValTok{120} \SpecialCharTok{\&}\NormalTok{ dep\_delay }\SpecialCharTok{\textless{}=} \DecValTok{120}\NormalTok{)}
\CommentTok{\#\textgreater{} \# A tibble: 316,050 x 19}
\CommentTok{\#\textgreater{}    year month   day dep\_time sched\_dep\_time dep\_delay arr\_time sched\_arr\_time}
\CommentTok{\#\textgreater{}   \textless{}int\textgreater{} \textless{}int\textgreater{} \textless{}int\textgreater{}    \textless{}int\textgreater{}          \textless{}int\textgreater{}     \textless{}dbl\textgreater{}    \textless{}int\textgreater{}          \textless{}int\textgreater{}}
\CommentTok{\#\textgreater{} 1  2013     1     1      517            515         2      830            819}
\CommentTok{\#\textgreater{} 2  2013     1     1      533            529         4      850            830}
\CommentTok{\#\textgreater{} 3  2013     1     1      542            540         2      923            850}
\CommentTok{\#\textgreater{} 4  2013     1     1      544            545        {-}1     1004           1022}
\CommentTok{\#\textgreater{} 5  2013     1     1      554            600        {-}6      812            837}
\CommentTok{\#\textgreater{} 6  2013     1     1      554            558        {-}4      740            728}
\CommentTok{\#\textgreater{} \# ... with 316,044 more rows, and 11 more variables: arr\_delay \textless{}dbl\textgreater{},}
\CommentTok{\#\textgreater{} \#   carrier \textless{}chr\textgreater{}, flight \textless{}int\textgreater{}, tailnum \textless{}chr\textgreater{}, origin \textless{}chr\textgreater{}, dest \textless{}chr\textgreater{},}
\CommentTok{\#\textgreater{} \#   air\_time \textless{}dbl\textgreater{}, distance \textless{}dbl\textgreater{}, hour \textless{}dbl\textgreater{}, minute \textless{}dbl\textgreater{}, time\_hour \textless{}dttm\textgreater{}}

\CommentTok{\# \%in\% 的反面}
\NormalTok{starwars }\SpecialCharTok{\%\textgreater{}\%} 
  \FunctionTok{filter}\NormalTok{(}\SpecialCharTok{!}\NormalTok{species }\SpecialCharTok{\%in\%}  \FunctionTok{c}\NormalTok{(}\StringTok{"Droid"}\NormalTok{,}\StringTok{\textquotesingle{}Clawdite\textquotesingle{}}\NormalTok{))}
\CommentTok{\#\textgreater{} \# A tibble: 80 x 14}
\CommentTok{\#\textgreater{}   name     height  mass hair\_color  skin\_color eye\_color birth\_year sex   gender}
\CommentTok{\#\textgreater{}   \textless{}chr\textgreater{}     \textless{}int\textgreater{} \textless{}dbl\textgreater{} \textless{}chr\textgreater{}       \textless{}chr\textgreater{}      \textless{}chr\textgreater{}          \textless{}dbl\textgreater{} \textless{}chr\textgreater{} \textless{}chr\textgreater{} }
\CommentTok{\#\textgreater{} 1 Luke Sk\textasciitilde{}    172    77 blond       fair       blue            19   male  mascu\textasciitilde{}}
\CommentTok{\#\textgreater{} 2 Darth V\textasciitilde{}    202   136 none        white      yellow          41.9 male  mascu\textasciitilde{}}
\CommentTok{\#\textgreater{} 3 Leia Or\textasciitilde{}    150    49 brown       light      brown           19   fema\textasciitilde{} femin\textasciitilde{}}
\CommentTok{\#\textgreater{} 4 Owen La\textasciitilde{}    178   120 brown, grey light      blue            52   male  mascu\textasciitilde{}}
\CommentTok{\#\textgreater{} 5 Beru Wh\textasciitilde{}    165    75 brown       light      blue            47   fema\textasciitilde{} femin\textasciitilde{}}
\CommentTok{\#\textgreater{} 6 Biggs D\textasciitilde{}    183    84 black       light      brown           24   male  mascu\textasciitilde{}}
\CommentTok{\#\textgreater{} \# ... with 74 more rows, and 5 more variables: homeworld \textless{}chr\textgreater{}, species \textless{}chr\textgreater{},}
\CommentTok{\#\textgreater{} \#   films \textless{}list\textgreater{}, vehicles \textless{}list\textgreater{}, starships \textless{}list\textgreater{}}
\end{Highlighting}
\end{Shaded}

\begin{quote}
!的运算级别相比 \%in\% 更高
\end{quote}

\hypertarget{dplyr-select}{%
\subsection{select}\label{dplyr-select}}

当完整数据集列较多时,我们某次分析可能并不需要那么多列,通过动词\texttt{select()}筛选列,剔除不需要的列。

\begin{itemize}
\tightlist
\item
  基础用法
\end{itemize}

通过指定列名称筛选,并指定列之间顺序

\begin{Shaded}
\begin{Highlighting}[]
\NormalTok{starwars }\SpecialCharTok{\%\textgreater{}\%} 
  \FunctionTok{select}\NormalTok{(name,height,mass,hair\_color,skin\_color,eye\_color)}
\CommentTok{\#\textgreater{} \# A tibble: 87 x 6}
\CommentTok{\#\textgreater{}   name           height  mass hair\_color  skin\_color  eye\_color}
\CommentTok{\#\textgreater{}   \textless{}chr\textgreater{}           \textless{}int\textgreater{} \textless{}dbl\textgreater{} \textless{}chr\textgreater{}       \textless{}chr\textgreater{}       \textless{}chr\textgreater{}    }
\CommentTok{\#\textgreater{} 1 Luke Skywalker    172    77 blond       fair        blue     }
\CommentTok{\#\textgreater{} 2 C{-}3PO             167    75 \textless{}NA\textgreater{}        gold        yellow   }
\CommentTok{\#\textgreater{} 3 R2{-}D2              96    32 \textless{}NA\textgreater{}        white, blue red      }
\CommentTok{\#\textgreater{} 4 Darth Vader       202   136 none        white       yellow   }
\CommentTok{\#\textgreater{} 5 Leia Organa       150    49 brown       light       brown    }
\CommentTok{\#\textgreater{} 6 Owen Lars         178   120 brown, grey light       blue     }
\CommentTok{\#\textgreater{} \# ... with 81 more rows}
\end{Highlighting}
\end{Shaded}

\begin{itemize}
\tightlist
\item
  列索引
\end{itemize}

通过列名或数字向量索引,但是不建议用数字索引,避免原始数据列顺序变化后导致报错。

\begin{Shaded}
\begin{Highlighting}[]
\NormalTok{starwars }\SpecialCharTok{\%\textgreater{}\%} 
  \FunctionTok{select}\NormalTok{(name }\SpecialCharTok{:}\NormalTok{ eye\_color)}
\CommentTok{\#\textgreater{} \# A tibble: 87 x 6}
\CommentTok{\#\textgreater{}   name           height  mass hair\_color  skin\_color  eye\_color}
\CommentTok{\#\textgreater{}   \textless{}chr\textgreater{}           \textless{}int\textgreater{} \textless{}dbl\textgreater{} \textless{}chr\textgreater{}       \textless{}chr\textgreater{}       \textless{}chr\textgreater{}    }
\CommentTok{\#\textgreater{} 1 Luke Skywalker    172    77 blond       fair        blue     }
\CommentTok{\#\textgreater{} 2 C{-}3PO             167    75 \textless{}NA\textgreater{}        gold        yellow   }
\CommentTok{\#\textgreater{} 3 R2{-}D2              96    32 \textless{}NA\textgreater{}        white, blue red      }
\CommentTok{\#\textgreater{} 4 Darth Vader       202   136 none        white       yellow   }
\CommentTok{\#\textgreater{} 5 Leia Organa       150    49 brown       light       brown    }
\CommentTok{\#\textgreater{} 6 Owen Lars         178   120 brown, grey light       blue     }
\CommentTok{\#\textgreater{} \# ... with 81 more rows}

\CommentTok{\#same above}
\NormalTok{starwars }\SpecialCharTok{\%\textgreater{}\%} 
  \FunctionTok{select}\NormalTok{(}\DecValTok{1}\SpecialCharTok{:}\DecValTok{6}\NormalTok{)}
\CommentTok{\#\textgreater{} \# A tibble: 87 x 6}
\CommentTok{\#\textgreater{}   name           height  mass hair\_color  skin\_color  eye\_color}
\CommentTok{\#\textgreater{}   \textless{}chr\textgreater{}           \textless{}int\textgreater{} \textless{}dbl\textgreater{} \textless{}chr\textgreater{}       \textless{}chr\textgreater{}       \textless{}chr\textgreater{}    }
\CommentTok{\#\textgreater{} 1 Luke Skywalker    172    77 blond       fair        blue     }
\CommentTok{\#\textgreater{} 2 C{-}3PO             167    75 \textless{}NA\textgreater{}        gold        yellow   }
\CommentTok{\#\textgreater{} 3 R2{-}D2              96    32 \textless{}NA\textgreater{}        white, blue red      }
\CommentTok{\#\textgreater{} 4 Darth Vader       202   136 none        white       yellow   }
\CommentTok{\#\textgreater{} 5 Leia Organa       150    49 brown       light       brown    }
\CommentTok{\#\textgreater{} 6 Owen Lars         178   120 brown, grey light       blue     }
\CommentTok{\#\textgreater{} \# ... with 81 more rows}

\CommentTok{\# starwars \%\textgreater{}\% select(c(1,2,4,5,7))}
\end{Highlighting}
\end{Shaded}

\begin{itemize}
\tightlist
\item
  新增列筛选方式
\end{itemize}

\begin{Shaded}
\begin{Highlighting}[]
\CommentTok{\# starwars \%\textgreater{}\% select(!(name:mass))}
\CommentTok{\# iris \%\textgreater{}\% select(!ends\_with("Width"))}
\CommentTok{\# iris \%\textgreater{}\% select(starts\_with("Petal") \& ends\_with("Width"))}
\CommentTok{\# iris \%\textgreater{}\% select(starts\_with("Petal") | ends\_with("Width"))}
\end{Highlighting}
\end{Shaded}

\hypertarget{dplyr-rename}{%
\subsection{rename}\label{dplyr-rename}}

列重命名使用\texttt{rename()}函数,新名称写前面,如下所示:

\begin{Shaded}
\begin{Highlighting}[]
\NormalTok{starwars }\SpecialCharTok{\%\textgreater{}\%} \FunctionTok{rename}\NormalTok{(}\AttributeTok{home\_world =}\NormalTok{ homeworld)}
\CommentTok{\#\textgreater{} \# A tibble: 87 x 14}
\CommentTok{\#\textgreater{}   name     height  mass hair\_color  skin\_color eye\_color birth\_year sex   gender}
\CommentTok{\#\textgreater{}   \textless{}chr\textgreater{}     \textless{}int\textgreater{} \textless{}dbl\textgreater{} \textless{}chr\textgreater{}       \textless{}chr\textgreater{}      \textless{}chr\textgreater{}          \textless{}dbl\textgreater{} \textless{}chr\textgreater{} \textless{}chr\textgreater{} }
\CommentTok{\#\textgreater{} 1 Luke Sk\textasciitilde{}    172    77 blond       fair       blue            19   male  mascu\textasciitilde{}}
\CommentTok{\#\textgreater{} 2 C{-}3PO       167    75 \textless{}NA\textgreater{}        gold       yellow         112   none  mascu\textasciitilde{}}
\CommentTok{\#\textgreater{} 3 R2{-}D2        96    32 \textless{}NA\textgreater{}        white, bl\textasciitilde{} red             33   none  mascu\textasciitilde{}}
\CommentTok{\#\textgreater{} 4 Darth V\textasciitilde{}    202   136 none        white      yellow          41.9 male  mascu\textasciitilde{}}
\CommentTok{\#\textgreater{} 5 Leia Or\textasciitilde{}    150    49 brown       light      brown           19   fema\textasciitilde{} femin\textasciitilde{}}
\CommentTok{\#\textgreater{} 6 Owen La\textasciitilde{}    178   120 brown, grey light      blue            52   male  mascu\textasciitilde{}}
\CommentTok{\#\textgreater{} \# ... with 81 more rows, and 5 more variables: home\_world \textless{}chr\textgreater{}, species \textless{}chr\textgreater{},}
\CommentTok{\#\textgreater{} \#   films \textless{}list\textgreater{}, vehicles \textless{}list\textgreater{}, starships \textless{}list\textgreater{}}
\CommentTok{\# 多列同换}
\NormalTok{starwars }\SpecialCharTok{\%\textgreater{}\%} \FunctionTok{rename}\NormalTok{(}\AttributeTok{home\_world =}\NormalTok{ homeworld,}\AttributeTok{skincolor =}\NormalTok{ skin\_color)}
\CommentTok{\#\textgreater{} \# A tibble: 87 x 14}
\CommentTok{\#\textgreater{}   name     height  mass hair\_color  skincolor  eye\_color birth\_year sex   gender}
\CommentTok{\#\textgreater{}   \textless{}chr\textgreater{}     \textless{}int\textgreater{} \textless{}dbl\textgreater{} \textless{}chr\textgreater{}       \textless{}chr\textgreater{}      \textless{}chr\textgreater{}          \textless{}dbl\textgreater{} \textless{}chr\textgreater{} \textless{}chr\textgreater{} }
\CommentTok{\#\textgreater{} 1 Luke Sk\textasciitilde{}    172    77 blond       fair       blue            19   male  mascu\textasciitilde{}}
\CommentTok{\#\textgreater{} 2 C{-}3PO       167    75 \textless{}NA\textgreater{}        gold       yellow         112   none  mascu\textasciitilde{}}
\CommentTok{\#\textgreater{} 3 R2{-}D2        96    32 \textless{}NA\textgreater{}        white, bl\textasciitilde{} red             33   none  mascu\textasciitilde{}}
\CommentTok{\#\textgreater{} 4 Darth V\textasciitilde{}    202   136 none        white      yellow          41.9 male  mascu\textasciitilde{}}
\CommentTok{\#\textgreater{} 5 Leia Or\textasciitilde{}    150    49 brown       light      brown           19   fema\textasciitilde{} femin\textasciitilde{}}
\CommentTok{\#\textgreater{} 6 Owen La\textasciitilde{}    178   120 brown, grey light      blue            52   male  mascu\textasciitilde{}}
\CommentTok{\#\textgreater{} \# ... with 81 more rows, and 5 more variables: home\_world \textless{}chr\textgreater{}, species \textless{}chr\textgreater{},}
\CommentTok{\#\textgreater{} \#   films \textless{}list\textgreater{}, vehicles \textless{}list\textgreater{}, starships \textless{}list\textgreater{}}
\end{Highlighting}
\end{Shaded}

\begin{Shaded}
\begin{Highlighting}[]
\KeywordTok{select} \OperatorTok{*}\NormalTok{ ,homeworld }\KeywordTok{as}\NormalTok{ home\_word }\KeywordTok{from}\NormalTok{ starwars }
\KeywordTok{select} \OperatorTok{*}\NormalTok{ ,homeworld  home\_word }\KeywordTok{from}\NormalTok{ starwars }
\end{Highlighting}
\end{Shaded}

\begin{quote}
as 可以省略,但中间有一个以上空格。与R的差异是新增home\_word列,原始列继续存在,R中是替换列名。
\end{quote}

\hypertarget{dplyr-relocate}{%
\subsection{relocate}\label{dplyr-relocate}}

更改列顺序,与使用\texttt{select()}动词指定列顺序功能相似。

relocate参数如下:

\begin{Shaded}
\begin{Highlighting}[]
\FunctionTok{relocate}\NormalTok{(.data, ..., }\AttributeTok{.before =} \ConstantTok{NULL}\NormalTok{, }\AttributeTok{.after =} \ConstantTok{NULL}\NormalTok{)}
\end{Highlighting}
\end{Shaded}

\begin{Shaded}
\begin{Highlighting}[]
\CommentTok{\# sex:homeworld列在height列前面}
\NormalTok{starwars }\SpecialCharTok{\%\textgreater{}\%} \FunctionTok{relocate}\NormalTok{(sex}\SpecialCharTok{:}\NormalTok{homeworld, }\AttributeTok{.before =}\NormalTok{ height)}
\CommentTok{\#\textgreater{} \# A tibble: 87 x 14}
\CommentTok{\#\textgreater{}   name     sex    gender  homeworld height  mass hair\_color skin\_color eye\_color}
\CommentTok{\#\textgreater{}   \textless{}chr\textgreater{}    \textless{}chr\textgreater{}  \textless{}chr\textgreater{}   \textless{}chr\textgreater{}      \textless{}int\textgreater{} \textless{}dbl\textgreater{} \textless{}chr\textgreater{}      \textless{}chr\textgreater{}      \textless{}chr\textgreater{}    }
\CommentTok{\#\textgreater{} 1 Luke Sk\textasciitilde{} male   mascul\textasciitilde{} Tatooine     172    77 blond      fair       blue     }
\CommentTok{\#\textgreater{} 2 C{-}3PO    none   mascul\textasciitilde{} Tatooine     167    75 \textless{}NA\textgreater{}       gold       yellow   }
\CommentTok{\#\textgreater{} 3 R2{-}D2    none   mascul\textasciitilde{} Naboo         96    32 \textless{}NA\textgreater{}       white, bl\textasciitilde{} red      }
\CommentTok{\#\textgreater{} 4 Darth V\textasciitilde{} male   mascul\textasciitilde{} Tatooine     202   136 none       white      yellow   }
\CommentTok{\#\textgreater{} 5 Leia Or\textasciitilde{} female femini\textasciitilde{} Alderaan     150    49 brown      light      brown    }
\CommentTok{\#\textgreater{} 6 Owen La\textasciitilde{} male   mascul\textasciitilde{} Tatooine     178   120 brown, gr\textasciitilde{} light      blue     }
\CommentTok{\#\textgreater{} \# ... with 81 more rows, and 5 more variables: birth\_year \textless{}dbl\textgreater{}, species \textless{}chr\textgreater{},}
\CommentTok{\#\textgreater{} \#   films \textless{}list\textgreater{}, vehicles \textless{}list\textgreater{}, starships \textless{}list\textgreater{}}
\end{Highlighting}
\end{Shaded}

\hypertarget{dplyr-mutate}{%
\subsection{mutate}\label{dplyr-mutate}}

动词\texttt{mutate}可以新增计算列,删除列,更新已有列,列之间的计算都可以通过mutate实现。

\begin{itemize}
\tightlist
\item
  新增计算列
\end{itemize}

\begin{Shaded}
\begin{Highlighting}[]
\NormalTok{starwars }\SpecialCharTok{\%\textgreater{}\%} 
  \FunctionTok{mutate}\NormalTok{(}\AttributeTok{bmi =}\NormalTok{ mass }\SpecialCharTok{/}\NormalTok{ ((height }\SpecialCharTok{/} \DecValTok{100}\NormalTok{)  }\SpecialCharTok{\^{}} \DecValTok{2}\NormalTok{)) }\SpecialCharTok{\%\textgreater{}\%} 
  \FunctionTok{select}\NormalTok{(name}\SpecialCharTok{:}\NormalTok{mass,bmi)}
\CommentTok{\#\textgreater{} \# A tibble: 87 x 4}
\CommentTok{\#\textgreater{}   name           height  mass   bmi}
\CommentTok{\#\textgreater{}   \textless{}chr\textgreater{}           \textless{}int\textgreater{} \textless{}dbl\textgreater{} \textless{}dbl\textgreater{}}
\CommentTok{\#\textgreater{} 1 Luke Skywalker    172    77  26.0}
\CommentTok{\#\textgreater{} 2 C{-}3PO             167    75  26.9}
\CommentTok{\#\textgreater{} 3 R2{-}D2              96    32  34.7}
\CommentTok{\#\textgreater{} 4 Darth Vader       202   136  33.3}
\CommentTok{\#\textgreater{} 5 Leia Organa       150    49  21.8}
\CommentTok{\#\textgreater{} 6 Owen Lars         178   120  37.9}
\CommentTok{\#\textgreater{} \# ... with 81 more rows}
\end{Highlighting}
\end{Shaded}

\begin{itemize}
\tightlist
\item
  新增计算列基础上新增列
\end{itemize}

\begin{Shaded}
\begin{Highlighting}[]
\NormalTok{starwars }\SpecialCharTok{\%\textgreater{}\%} 
  \FunctionTok{mutate}\NormalTok{(}\AttributeTok{bmi =}\NormalTok{ mass }\SpecialCharTok{/}\NormalTok{ ((height }\SpecialCharTok{/} \DecValTok{100}\NormalTok{)  }\SpecialCharTok{\^{}} \DecValTok{2}\NormalTok{),}\AttributeTok{newbmi =}\NormalTok{ bmi }\SpecialCharTok{*}\DecValTok{2}\NormalTok{) }\SpecialCharTok{\%\textgreater{}\%} 
  \FunctionTok{select}\NormalTok{(name}\SpecialCharTok{:}\NormalTok{mass,bmi,newbmi)}
\CommentTok{\#\textgreater{} \# A tibble: 87 x 5}
\CommentTok{\#\textgreater{}   name           height  mass   bmi newbmi}
\CommentTok{\#\textgreater{}   \textless{}chr\textgreater{}           \textless{}int\textgreater{} \textless{}dbl\textgreater{} \textless{}dbl\textgreater{}  \textless{}dbl\textgreater{}}
\CommentTok{\#\textgreater{} 1 Luke Skywalker    172    77  26.0   52.1}
\CommentTok{\#\textgreater{} 2 C{-}3PO             167    75  26.9   53.8}
\CommentTok{\#\textgreater{} 3 R2{-}D2              96    32  34.7   69.4}
\CommentTok{\#\textgreater{} 4 Darth Vader       202   136  33.3   66.7}
\CommentTok{\#\textgreater{} 5 Leia Organa       150    49  21.8   43.6}
\CommentTok{\#\textgreater{} 6 Owen Lars         178   120  37.9   75.7}
\CommentTok{\#\textgreater{} \# ... with 81 more rows}
\end{Highlighting}
\end{Shaded}

\begin{itemize}
\tightlist
\item
  删除列
\end{itemize}

\begin{Shaded}
\begin{Highlighting}[]
\NormalTok{starwars }\SpecialCharTok{\%\textgreater{}\%} \FunctionTok{mutate}\NormalTok{(}\AttributeTok{height =} \ConstantTok{NULL}\NormalTok{)}
\CommentTok{\#\textgreater{} \# A tibble: 87 x 13}
\CommentTok{\#\textgreater{}   name    mass hair\_color skin\_color eye\_color birth\_year sex   gender homeworld}
\CommentTok{\#\textgreater{}   \textless{}chr\textgreater{}  \textless{}dbl\textgreater{} \textless{}chr\textgreater{}      \textless{}chr\textgreater{}      \textless{}chr\textgreater{}          \textless{}dbl\textgreater{} \textless{}chr\textgreater{} \textless{}chr\textgreater{}  \textless{}chr\textgreater{}    }
\CommentTok{\#\textgreater{} 1 Luke \textasciitilde{}    77 blond      fair       blue            19   male  mascu\textasciitilde{} Tatooine }
\CommentTok{\#\textgreater{} 2 C{-}3PO     75 \textless{}NA\textgreater{}       gold       yellow         112   none  mascu\textasciitilde{} Tatooine }
\CommentTok{\#\textgreater{} 3 R2{-}D2     32 \textless{}NA\textgreater{}       white, bl\textasciitilde{} red             33   none  mascu\textasciitilde{} Naboo    }
\CommentTok{\#\textgreater{} 4 Darth\textasciitilde{}   136 none       white      yellow          41.9 male  mascu\textasciitilde{} Tatooine }
\CommentTok{\#\textgreater{} 5 Leia \textasciitilde{}    49 brown      light      brown           19   fema\textasciitilde{} femin\textasciitilde{} Alderaan }
\CommentTok{\#\textgreater{} 6 Owen \textasciitilde{}   120 brown, gr\textasciitilde{} light      blue            52   male  mascu\textasciitilde{} Tatooine }
\CommentTok{\#\textgreater{} \# ... with 81 more rows, and 4 more variables: species \textless{}chr\textgreater{}, films \textless{}list\textgreater{},}
\CommentTok{\#\textgreater{} \#   vehicles \textless{}list\textgreater{}, starships \textless{}list\textgreater{}}
\end{Highlighting}
\end{Shaded}

\hypertarget{dplyr-arrange}{%
\subsection{arrange}\label{dplyr-arrange}}

\begin{itemize}
\tightlist
\item
  单列排序,默认升序,通过\texttt{desc()}降序排列
\end{itemize}

\begin{Shaded}
\begin{Highlighting}[]
\NormalTok{starwars }\SpecialCharTok{\%\textgreater{}\%} 
  \FunctionTok{arrange}\NormalTok{(}\FunctionTok{desc}\NormalTok{(mass))}
\CommentTok{\#\textgreater{} \# A tibble: 87 x 14}
\CommentTok{\#\textgreater{}   name    height  mass hair\_color  skin\_color  eye\_color birth\_year sex   gender}
\CommentTok{\#\textgreater{}   \textless{}chr\textgreater{}    \textless{}int\textgreater{} \textless{}dbl\textgreater{} \textless{}chr\textgreater{}       \textless{}chr\textgreater{}       \textless{}chr\textgreater{}          \textless{}dbl\textgreater{} \textless{}chr\textgreater{} \textless{}chr\textgreater{} }
\CommentTok{\#\textgreater{} 1 Jabba \textasciitilde{}    175  1358 \textless{}NA\textgreater{}        green{-}tan,\textasciitilde{} orange         600   herm\textasciitilde{} mascu\textasciitilde{}}
\CommentTok{\#\textgreater{} 2 Grievo\textasciitilde{}    216   159 none        brown, whi\textasciitilde{} green, y\textasciitilde{}       NA   male  mascu\textasciitilde{}}
\CommentTok{\#\textgreater{} 3 IG{-}88      200   140 none        metal       red             15   none  mascu\textasciitilde{}}
\CommentTok{\#\textgreater{} 4 Darth \textasciitilde{}    202   136 none        white       yellow          41.9 male  mascu\textasciitilde{}}
\CommentTok{\#\textgreater{} 5 Tarfful    234   136 brown       brown       blue            NA   male  mascu\textasciitilde{}}
\CommentTok{\#\textgreater{} 6 Owen L\textasciitilde{}    178   120 brown, grey light       blue            52   male  mascu\textasciitilde{}}
\CommentTok{\#\textgreater{} \# ... with 81 more rows, and 5 more variables: homeworld \textless{}chr\textgreater{}, species \textless{}chr\textgreater{},}
\CommentTok{\#\textgreater{} \#   films \textless{}list\textgreater{}, vehicles \textless{}list\textgreater{}, starships \textless{}list\textgreater{}}
\end{Highlighting}
\end{Shaded}

\begin{itemize}
\tightlist
\item
  多列排序
\end{itemize}

\begin{Shaded}
\begin{Highlighting}[]
\NormalTok{starwars }\SpecialCharTok{\%\textgreater{}\%} 
  \FunctionTok{arrange}\NormalTok{(height,}\FunctionTok{desc}\NormalTok{(mass))}
\CommentTok{\#\textgreater{} \# A tibble: 87 x 14}
\CommentTok{\#\textgreater{}   name      height  mass hair\_color skin\_color eye\_color birth\_year sex   gender}
\CommentTok{\#\textgreater{}   \textless{}chr\textgreater{}      \textless{}int\textgreater{} \textless{}dbl\textgreater{} \textless{}chr\textgreater{}      \textless{}chr\textgreater{}      \textless{}chr\textgreater{}          \textless{}dbl\textgreater{} \textless{}chr\textgreater{} \textless{}chr\textgreater{} }
\CommentTok{\#\textgreater{} 1 Yoda          66    17 white      green      brown            896 male  mascu\textasciitilde{}}
\CommentTok{\#\textgreater{} 2 Ratts Ty\textasciitilde{}     79    15 none       grey, blue unknown           NA male  mascu\textasciitilde{}}
\CommentTok{\#\textgreater{} 3 Wicket S\textasciitilde{}     88    20 brown      brown      brown              8 male  mascu\textasciitilde{}}
\CommentTok{\#\textgreater{} 4 Dud Bolt      94    45 none       blue, grey yellow            NA male  mascu\textasciitilde{}}
\CommentTok{\#\textgreater{} 5 R2{-}D2         96    32 \textless{}NA\textgreater{}       white, bl\textasciitilde{} red               33 none  mascu\textasciitilde{}}
\CommentTok{\#\textgreater{} 6 R4{-}P17        96    NA none       silver, r\textasciitilde{} red, blue         NA none  femin\textasciitilde{}}
\CommentTok{\#\textgreater{} \# ... with 81 more rows, and 5 more variables: homeworld \textless{}chr\textgreater{}, species \textless{}chr\textgreater{},}
\CommentTok{\#\textgreater{} \#   films \textless{}list\textgreater{}, vehicles \textless{}list\textgreater{}, starships \textless{}list\textgreater{}}
\end{Highlighting}
\end{Shaded}

\begin{Shaded}
\begin{Highlighting}[]
\KeywordTok{select} \OperatorTok{*} \KeywordTok{from}\NormalTok{ starwars }\KeywordTok{order} \KeywordTok{by}\NormalTok{ height,mass }\KeywordTok{desc}
\end{Highlighting}
\end{Shaded}

\hypertarget{dplyr-summarise}{%
\subsection{summarise}\label{dplyr-summarise}}

\texttt{summarise}常与\texttt{group\_by}结合使用。

\begin{Shaded}
\begin{Highlighting}[]
\NormalTok{mtcars }\SpecialCharTok{\%\textgreater{}\%}
  \FunctionTok{summarise}\NormalTok{(}\AttributeTok{mean =} \FunctionTok{mean}\NormalTok{(disp), }\AttributeTok{n =} \FunctionTok{n}\NormalTok{())}
\CommentTok{\#\textgreater{} \# A tibble: 1 x 2}
\CommentTok{\#\textgreater{}    mean     n}
\CommentTok{\#\textgreater{}   \textless{}dbl\textgreater{} \textless{}int\textgreater{}}
\CommentTok{\#\textgreater{} 1  231.    32}
\end{Highlighting}
\end{Shaded}

\begin{quote}
n()是dplyr包中的计算当前组的大小,用在summarise()和mutate()中。通常可用来组计算。
\end{quote}

\hypertarget{dplyr-groupby}{%
\subsection{group\_by}\label{dplyr-groupby}}

聚合前一般都需要分组,\texttt{group\_by()}动词实现该功能,与\texttt{SQL}中\texttt{group\ by\ ···}类似。

\begin{Shaded}
\begin{Highlighting}[]
\NormalTok{starwars }\SpecialCharTok{\%\textgreater{}\%}
  \FunctionTok{group\_by}\NormalTok{(species) }\SpecialCharTok{\%\textgreater{}\%}
  \FunctionTok{summarise}\NormalTok{(}
    \AttributeTok{n =} \FunctionTok{n}\NormalTok{(),}
    \AttributeTok{mass =} \FunctionTok{mean}\NormalTok{(mass, }\AttributeTok{na.rm =} \ConstantTok{TRUE}\NormalTok{)}
\NormalTok{  )}
\CommentTok{\#\textgreater{} \# A tibble: 38 x 3}
\CommentTok{\#\textgreater{}   species      n  mass}
\CommentTok{\#\textgreater{}   \textless{}chr\textgreater{}    \textless{}int\textgreater{} \textless{}dbl\textgreater{}}
\CommentTok{\#\textgreater{} 1 Aleena       1  15  }
\CommentTok{\#\textgreater{} 2 Besalisk     1 102  }
\CommentTok{\#\textgreater{} 3 Cerean       1  82  }
\CommentTok{\#\textgreater{} 4 Chagrian     1 NaN  }
\CommentTok{\#\textgreater{} 5 Clawdite     1  55  }
\CommentTok{\#\textgreater{} 6 Droid        6  69.8}
\CommentTok{\#\textgreater{} \# ... with 32 more rows}
\end{Highlighting}
\end{Shaded}

\begin{Shaded}
\begin{Highlighting}[]
\KeywordTok{SELECT}\NormalTok{ species,}
  \FunctionTok{count}\NormalTok{(species) n,}
  \FunctionTok{AVG}\NormalTok{(mass) mass}
\KeywordTok{FROM}\NormalTok{ [spb].[dbo].[starwars]}
\KeywordTok{GROUP} \KeywordTok{BY}\NormalTok{  species}
\end{Highlighting}
\end{Shaded}

\hypertarget{dplyr:merge-table}{%
\section{表关联}\label{dplyr:merge-table}}

表关联指像\texttt{sql}中的\texttt{left\ join},\texttt{inner\ join}等表格之间的操作,或者是Excel中\texttt{Power\ Piovt}建模的建立关系,从而实现不同表格间的关联。

\hypertarget{ux57faux7840}{%
\subsection{基础}\label{ux57faux7840}}

\texttt{left\_join()},\texttt{full\_join},\texttt{inner\_join()}等动词关联两个表。详情请查看:\texttt{vignette("two-table")}

\texttt{left\_join()}实现类似Excel中\texttt{VLOOKUP}函数功能或数据库中\texttt{left\ join}功能,将``右表''的字段依据``主键''关联到``左表''上。

\begin{itemize}
\tightlist
\item
  基础用法
\end{itemize}

\texttt{left\_join()},\texttt{right\_join()},\texttt{full\_join()},\texttt{inner\_join}(),第一个以左表为主,第二个右表为主,第三个全连接,第四个内连接(只返回两表中都有的记录),和数据库中连接方式一致。

默认会自动寻找两表中相同的字段名作为关联的条件

\begin{Shaded}
\begin{Highlighting}[]
\FunctionTok{library}\NormalTok{(}\StringTok{"nycflights13"}\NormalTok{)}
\CommentTok{\# Drop unimportant variables so it\textquotesingle{}s easier to understand the join results.}
\NormalTok{flights2 }\OtherTok{\textless{}{-}}\NormalTok{ flights }\SpecialCharTok{\%\textgreater{}\%} \FunctionTok{select}\NormalTok{(year}\SpecialCharTok{:}\NormalTok{day, hour, origin, dest, tailnum, carrier)}

\NormalTok{flights2 }\SpecialCharTok{\%\textgreater{}\%} 
  \FunctionTok{left\_join}\NormalTok{(airlines)}
\CommentTok{\#\textgreater{} Joining, by = "carrier"}
\CommentTok{\#\textgreater{} \# A tibble: 336,776 x 9}
\CommentTok{\#\textgreater{}    year month   day  hour origin dest  tailnum carrier name                  }
\CommentTok{\#\textgreater{}   \textless{}int\textgreater{} \textless{}int\textgreater{} \textless{}int\textgreater{} \textless{}dbl\textgreater{} \textless{}chr\textgreater{}  \textless{}chr\textgreater{} \textless{}chr\textgreater{}   \textless{}chr\textgreater{}   \textless{}chr\textgreater{}                 }
\CommentTok{\#\textgreater{} 1  2013     1     1     5 EWR    IAH   N14228  UA      United Air Lines Inc. }
\CommentTok{\#\textgreater{} 2  2013     1     1     5 LGA    IAH   N24211  UA      United Air Lines Inc. }
\CommentTok{\#\textgreater{} 3  2013     1     1     5 JFK    MIA   N619AA  AA      American Airlines Inc.}
\CommentTok{\#\textgreater{} 4  2013     1     1     5 JFK    BQN   N804JB  B6      JetBlue Airways       }
\CommentTok{\#\textgreater{} 5  2013     1     1     6 LGA    ATL   N668DN  DL      Delta Air Lines Inc.  }
\CommentTok{\#\textgreater{} 6  2013     1     1     5 EWR    ORD   N39463  UA      United Air Lines Inc. }
\CommentTok{\#\textgreater{} \# ... with 336,770 more rows}
\end{Highlighting}
\end{Shaded}

指定关联条件列,类似数据库中\texttt{on\ a.column\ =\ b.column}。

\begin{itemize}
\tightlist
\item
  R实现
\end{itemize}

\begin{Shaded}
\begin{Highlighting}[]
\NormalTok{flights2 }\SpecialCharTok{\%\textgreater{}\%} \FunctionTok{left\_join}\NormalTok{(planes, }\AttributeTok{by =} \StringTok{"tailnum"}\NormalTok{)}
\CommentTok{\#\textgreater{} \# A tibble: 336,776 x 16}
\CommentTok{\#\textgreater{}   year.x month   day  hour origin dest  tailnum carrier year.y type             }
\CommentTok{\#\textgreater{}    \textless{}int\textgreater{} \textless{}int\textgreater{} \textless{}int\textgreater{} \textless{}dbl\textgreater{} \textless{}chr\textgreater{}  \textless{}chr\textgreater{} \textless{}chr\textgreater{}   \textless{}chr\textgreater{}    \textless{}int\textgreater{} \textless{}chr\textgreater{}            }
\CommentTok{\#\textgreater{} 1   2013     1     1     5 EWR    IAH   N14228  UA        1999 Fixed wing multi\textasciitilde{}}
\CommentTok{\#\textgreater{} 2   2013     1     1     5 LGA    IAH   N24211  UA        1998 Fixed wing multi\textasciitilde{}}
\CommentTok{\#\textgreater{} 3   2013     1     1     5 JFK    MIA   N619AA  AA        1990 Fixed wing multi\textasciitilde{}}
\CommentTok{\#\textgreater{} 4   2013     1     1     5 JFK    BQN   N804JB  B6        2012 Fixed wing multi\textasciitilde{}}
\CommentTok{\#\textgreater{} 5   2013     1     1     6 LGA    ATL   N668DN  DL        1991 Fixed wing multi\textasciitilde{}}
\CommentTok{\#\textgreater{} 6   2013     1     1     5 EWR    ORD   N39463  UA        2012 Fixed wing multi\textasciitilde{}}
\CommentTok{\#\textgreater{} \# ... with 336,770 more rows, and 6 more variables: manufacturer \textless{}chr\textgreater{},}
\CommentTok{\#\textgreater{} \#   model \textless{}chr\textgreater{}, engines \textless{}int\textgreater{}, seats \textless{}int\textgreater{}, speed \textless{}int\textgreater{}, engine \textless{}chr\textgreater{}}
\end{Highlighting}
\end{Shaded}

\begin{itemize}
\tightlist
\item
  Sql实现
\end{itemize}

\begin{Shaded}
\begin{Highlighting}[]
\KeywordTok{select} \OperatorTok{*} \KeywordTok{from}\NormalTok{ flights2 a }\KeywordTok{left} \KeywordTok{join}\NormalTok{ planes b }\KeywordTok{on}\NormalTok{ a.tailnum }\OperatorTok{=}\NormalTok{ b.tailnum }
\end{Highlighting}
\end{Shaded}

\begin{itemize}
\tightlist
\item
  不同名称列关联
\end{itemize}

\texttt{left\_join(x,y,by\ =\ c("a"\ =\ "b",\ "c"\ =\ "d"))} 将会匹配 \texttt{x\$a} to \texttt{y\$b} 和 \texttt{x\$c} to \texttt{y\$d} 作为关联条件

\begin{Shaded}
\begin{Highlighting}[]
\CommentTok{\#出发机场和目的机场信息}
\NormalTok{flights2 }\SpecialCharTok{\%\textgreater{}\%} \FunctionTok{left\_join}\NormalTok{(airports, }\AttributeTok{by =} \FunctionTok{c}\NormalTok{(}\StringTok{"dest"} \OtherTok{=} \StringTok{"faa"}\NormalTok{))}
\CommentTok{\#\textgreater{} \# A tibble: 336,776 x 15}
\CommentTok{\#\textgreater{}    year month   day  hour origin dest  tailnum carrier name      lat   lon   alt}
\CommentTok{\#\textgreater{}   \textless{}int\textgreater{} \textless{}int\textgreater{} \textless{}int\textgreater{} \textless{}dbl\textgreater{} \textless{}chr\textgreater{}  \textless{}chr\textgreater{} \textless{}chr\textgreater{}   \textless{}chr\textgreater{}   \textless{}chr\textgreater{}   \textless{}dbl\textgreater{} \textless{}dbl\textgreater{} \textless{}dbl\textgreater{}}
\CommentTok{\#\textgreater{} 1  2013     1     1     5 EWR    IAH   N14228  UA      George\textasciitilde{}  30.0 {-}95.3    97}
\CommentTok{\#\textgreater{} 2  2013     1     1     5 LGA    IAH   N24211  UA      George\textasciitilde{}  30.0 {-}95.3    97}
\CommentTok{\#\textgreater{} 3  2013     1     1     5 JFK    MIA   N619AA  AA      Miami \textasciitilde{}  25.8 {-}80.3     8}
\CommentTok{\#\textgreater{} 4  2013     1     1     5 JFK    BQN   N804JB  B6      \textless{}NA\textgreater{}     NA    NA      NA}
\CommentTok{\#\textgreater{} 5  2013     1     1     6 LGA    ATL   N668DN  DL      Hartsf\textasciitilde{}  33.6 {-}84.4  1026}
\CommentTok{\#\textgreater{} 6  2013     1     1     5 EWR    ORD   N39463  UA      Chicag\textasciitilde{}  42.0 {-}87.9   668}
\CommentTok{\#\textgreater{} \# ... with 336,770 more rows, and 3 more variables: tz \textless{}dbl\textgreater{}, dst \textless{}chr\textgreater{},}
\CommentTok{\#\textgreater{} \#   tzone \textless{}chr\textgreater{}}
\CommentTok{\#flights2 \%\textgreater{}\% left\_join(airports, c("origin" = "faa"))}
\CommentTok{\# 组合条件 多条件时用向量包裹即可c("dest" = "faa","cola" = "colb"))}
\end{Highlighting}
\end{Shaded}

\begin{itemize}
\tightlist
\item
  筛选连接
\end{itemize}

\texttt{anti\_join()} 删除所有左表中在右表中匹配到的行

\texttt{semi\_join()}保留所有左表在右表中匹配到的行

\begin{Shaded}
\begin{Highlighting}[]
\NormalTok{df1 }\OtherTok{\textless{}{-}} \FunctionTok{tibble}\NormalTok{(}\AttributeTok{a=}\NormalTok{letters[}\DecValTok{1}\SpecialCharTok{:}\DecValTok{20}\NormalTok{],}\AttributeTok{b=}\DecValTok{1}\SpecialCharTok{:}\DecValTok{20}\NormalTok{)}
\NormalTok{df2 }\OtherTok{\textless{}{-}} \FunctionTok{tibble}\NormalTok{(}\AttributeTok{a=}\NormalTok{letters,}\AttributeTok{b=}\DecValTok{1}\SpecialCharTok{:}\DecValTok{26}\NormalTok{)}

\NormalTok{df1 }\SpecialCharTok{\%\textgreater{}\%} \FunctionTok{semi\_join}\NormalTok{(df2)}
\CommentTok{\#\textgreater{} Joining, by = c("a", "b")}
\CommentTok{\#\textgreater{} \# A tibble: 20 x 2}
\CommentTok{\#\textgreater{}   a         b}
\CommentTok{\#\textgreater{}   \textless{}chr\textgreater{} \textless{}int\textgreater{}}
\CommentTok{\#\textgreater{} 1 a         1}
\CommentTok{\#\textgreater{} 2 b         2}
\CommentTok{\#\textgreater{} 3 c         3}
\CommentTok{\#\textgreater{} 4 d         4}
\CommentTok{\#\textgreater{} 5 e         5}
\CommentTok{\#\textgreater{} 6 f         6}
\CommentTok{\#\textgreater{} \# ... with 14 more rows}
\NormalTok{df2 }\SpecialCharTok{\%\textgreater{}\%} \FunctionTok{anti\_join}\NormalTok{(df1)}
\CommentTok{\#\textgreater{} Joining, by = c("a", "b")}
\CommentTok{\#\textgreater{} \# A tibble: 6 x 2}
\CommentTok{\#\textgreater{}   a         b}
\CommentTok{\#\textgreater{}   \textless{}chr\textgreater{} \textless{}int\textgreater{}}
\CommentTok{\#\textgreater{} 1 u        21}
\CommentTok{\#\textgreater{} 2 v        22}
\CommentTok{\#\textgreater{} 3 w        23}
\CommentTok{\#\textgreater{} 4 x        24}
\CommentTok{\#\textgreater{} 5 y        25}
\CommentTok{\#\textgreater{} 6 z        26}
\end{Highlighting}
\end{Shaded}

\begin{itemize}
\tightlist
\item
  集合操作
\end{itemize}

\begin{enumerate}
\def\labelenumi{\arabic{enumi}.}
\item
  \texttt{intersect(x,y)}返回x,y交集
\item
  \texttt{union(x,y)}返回x,y中唯一的值
\item
  \texttt{setdiff(x,y)}返回存在x中但是不存在y中的记录
\end{enumerate}

\begin{Shaded}
\begin{Highlighting}[]
\NormalTok{(df1 }\OtherTok{\textless{}{-}} \FunctionTok{tibble}\NormalTok{(}\AttributeTok{x =} \DecValTok{1}\SpecialCharTok{:}\DecValTok{2}\NormalTok{, }\AttributeTok{y =} \FunctionTok{c}\NormalTok{(1L, 1L)))}
\CommentTok{\#\textgreater{} \# A tibble: 2 x 2}
\CommentTok{\#\textgreater{}       x     y}
\CommentTok{\#\textgreater{}   \textless{}int\textgreater{} \textless{}int\textgreater{}}
\CommentTok{\#\textgreater{} 1     1     1}
\CommentTok{\#\textgreater{} 2     2     1}
\NormalTok{(df2 }\OtherTok{\textless{}{-}} \FunctionTok{tibble}\NormalTok{(}\AttributeTok{x =} \DecValTok{1}\SpecialCharTok{:}\DecValTok{2}\NormalTok{, }\AttributeTok{y =} \DecValTok{1}\SpecialCharTok{:}\DecValTok{2}\NormalTok{))}
\CommentTok{\#\textgreater{} \# A tibble: 2 x 2}
\CommentTok{\#\textgreater{}       x     y}
\CommentTok{\#\textgreater{}   \textless{}int\textgreater{} \textless{}int\textgreater{}}
\CommentTok{\#\textgreater{} 1     1     1}
\CommentTok{\#\textgreater{} 2     2     2}
\FunctionTok{intersect}\NormalTok{(df1, df2)}
\CommentTok{\#\textgreater{} \# A tibble: 1 x 2}
\CommentTok{\#\textgreater{}       x     y}
\CommentTok{\#\textgreater{}   \textless{}int\textgreater{} \textless{}int\textgreater{}}
\CommentTok{\#\textgreater{} 1     1     1}
\FunctionTok{union}\NormalTok{(df1, df2)}
\CommentTok{\#\textgreater{} \# A tibble: 3 x 2}
\CommentTok{\#\textgreater{}       x     y}
\CommentTok{\#\textgreater{}   \textless{}int\textgreater{} \textless{}int\textgreater{}}
\CommentTok{\#\textgreater{} 1     1     1}
\CommentTok{\#\textgreater{} 2     2     1}
\CommentTok{\#\textgreater{} 3     2     2}
\FunctionTok{setdiff}\NormalTok{(df1, df2)}
\CommentTok{\#\textgreater{} \# A tibble: 1 x 2}
\CommentTok{\#\textgreater{}       x     y}
\CommentTok{\#\textgreater{}   \textless{}int\textgreater{} \textless{}int\textgreater{}}
\CommentTok{\#\textgreater{} 1     2     1}
\FunctionTok{setdiff}\NormalTok{(df2, df1)}
\CommentTok{\#\textgreater{} \# A tibble: 1 x 2}
\CommentTok{\#\textgreater{}       x     y}
\CommentTok{\#\textgreater{}   \textless{}int\textgreater{} \textless{}int\textgreater{}}
\CommentTok{\#\textgreater{} 1     2     2}
\end{Highlighting}
\end{Shaded}

\hypertarget{ux591aux8868ux64cdux4f5c}{%
\subsection{多表操作}\label{ux591aux8868ux64cdux4f5c}}

多表操作请使用\texttt{purrr::reduce()},当需要合并多个表格时,可用以下方式减少合并代码量。

\begin{Shaded}
\begin{Highlighting}[]
\NormalTok{dt1 }\OtherTok{\textless{}{-}} \FunctionTok{data.frame}\NormalTok{(}\AttributeTok{x =}\NormalTok{ letters)}
\NormalTok{dt2 }\OtherTok{\textless{}{-}} \FunctionTok{data.frame}\NormalTok{(}\AttributeTok{x =}\NormalTok{ letters,}\AttributeTok{cola =} \DecValTok{1}\SpecialCharTok{:}\DecValTok{26}\NormalTok{)}
\NormalTok{dt3 }\OtherTok{\textless{}{-}} \FunctionTok{data.frame}\NormalTok{(}\AttributeTok{x =}\NormalTok{ letters,}\AttributeTok{colb =} \DecValTok{1}\SpecialCharTok{:}\DecValTok{26}\NormalTok{)}
\NormalTok{dt4 }\OtherTok{\textless{}{-}} \FunctionTok{data.frame}\NormalTok{(}\AttributeTok{x =}\NormalTok{ letters,}\AttributeTok{cold =} \DecValTok{1}\SpecialCharTok{:}\DecValTok{26}\NormalTok{)}
\NormalTok{dt5 }\OtherTok{\textless{}{-}} \FunctionTok{data.frame}\NormalTok{(}\AttributeTok{x =}\NormalTok{ letters,}\AttributeTok{cole =} \DecValTok{1}\SpecialCharTok{:}\DecValTok{26}\NormalTok{)}

\NormalTok{dtlist }\OtherTok{\textless{}{-}} \FunctionTok{list}\NormalTok{(dt1,dt2,dt3,dt4,dt5)}
\NormalTok{purrr}\SpecialCharTok{::}\FunctionTok{reduce}\NormalTok{(dtlist,left\_join,}\AttributeTok{by=}\StringTok{\textquotesingle{}x\textquotesingle{}}\NormalTok{)}
\CommentTok{\#\textgreater{}    x cola colb cold cole}
\CommentTok{\#\textgreater{} 1  a    1    1    1    1}
\CommentTok{\#\textgreater{} 2  b    2    2    2    2}
\CommentTok{\#\textgreater{} 3  c    3    3    3    3}
\CommentTok{\#\textgreater{} 4  d    4    4    4    4}
\CommentTok{\#\textgreater{} 5  e    5    5    5    5}
\CommentTok{\#\textgreater{} 6  f    6    6    6    6}
\CommentTok{\#\textgreater{} 7  g    7    7    7    7}
\CommentTok{\#\textgreater{} 8  h    8    8    8    8}
\CommentTok{\#\textgreater{} 9  i    9    9    9    9}
\CommentTok{\#\textgreater{} 10 j   10   10   10   10}
\CommentTok{\#\textgreater{} 11 k   11   11   11   11}
\CommentTok{\#\textgreater{} 12 l   12   12   12   12}
\CommentTok{\#\textgreater{} 13 m   13   13   13   13}
\CommentTok{\#\textgreater{} 14 n   14   14   14   14}
\CommentTok{\#\textgreater{} 15 o   15   15   15   15}
\CommentTok{\#\textgreater{} 16 p   16   16   16   16}
\CommentTok{\#\textgreater{} 17 q   17   17   17   17}
\CommentTok{\#\textgreater{} 18 r   18   18   18   18}
\CommentTok{\#\textgreater{} 19 s   19   19   19   19}
\CommentTok{\#\textgreater{} 20 t   20   20   20   20}
\CommentTok{\#\textgreater{} 21 u   21   21   21   21}
\CommentTok{\#\textgreater{} 22 v   22   22   22   22}
\CommentTok{\#\textgreater{} 23 w   23   23   23   23}
\CommentTok{\#\textgreater{} 24 x   24   24   24   24}
\CommentTok{\#\textgreater{} 25 y   25   25   25   25}
\CommentTok{\#\textgreater{} 26 z   26   26   26   26}
\end{Highlighting}
\end{Shaded}

\hypertarget{dplyr-column-manipulation}{%
\section{列操作}\label{dplyr-column-manipulation}}

在多列上执行相同的操作是常用的操作,但是通过复制和粘贴代码,麻烦不说还容易错:

\begin{Shaded}
\begin{Highlighting}[]
\NormalTok{df }\SpecialCharTok{\%\textgreater{}\%} 
  \FunctionTok{group\_by}\NormalTok{(g1, g2) }\SpecialCharTok{\%\textgreater{}\%} 
  \FunctionTok{summarise}\NormalTok{(}\AttributeTok{a =} \FunctionTok{mean}\NormalTok{(a), }\AttributeTok{b =} \FunctionTok{mean}\NormalTok{(b), }\AttributeTok{c =} \FunctionTok{mean}\NormalTok{(c), }\AttributeTok{d =} \FunctionTok{mean}\NormalTok{(d))}
\end{Highlighting}
\end{Shaded}

通过\texttt{across()}函数可以更简洁地重写上面代码:

\begin{Shaded}
\begin{Highlighting}[]
\NormalTok{df }\SpecialCharTok{\%\textgreater{}\%} 
  \FunctionTok{group\_by}\NormalTok{(g1, g2) }\SpecialCharTok{\%\textgreater{}\%} 
  \FunctionTok{summarise}\NormalTok{(}\FunctionTok{across}\NormalTok{(a}\SpecialCharTok{:}\NormalTok{d, mean))}
\end{Highlighting}
\end{Shaded}

\hypertarget{ux57faux672cux64cdux4f5c}{%
\subsection{基本操作}\label{ux57faux672cux64cdux4f5c}}

across() 有两个主要参数:

\begin{itemize}
\item
  第一个参数,.cols选择要操作的列。它使用\texttt{tidyr}的方式选择(例如select()),因此您可以按位置,名称和类型选择变量。
\item
  第二个参数,.fns是要应用于每一列的一个函数或函数列表。这也可以是purrr样式的公式(或公式列表),例如\textasciitilde{} .x / 2。
\end{itemize}

\begin{Shaded}
\begin{Highlighting}[]
\NormalTok{starwars }\SpecialCharTok{\%\textgreater{}\%} 
  \FunctionTok{summarise}\NormalTok{(}\FunctionTok{across}\NormalTok{(}\FunctionTok{where}\NormalTok{(is.character), }\SpecialCharTok{\textasciitilde{}} \FunctionTok{length}\NormalTok{(}\FunctionTok{unique}\NormalTok{(.x))))}
\CommentTok{\#\textgreater{} \# A tibble: 1 x 8}
\CommentTok{\#\textgreater{}    name hair\_color skin\_color eye\_color   sex gender homeworld species}
\CommentTok{\#\textgreater{}   \textless{}int\textgreater{}      \textless{}int\textgreater{}      \textless{}int\textgreater{}     \textless{}int\textgreater{} \textless{}int\textgreater{}  \textless{}int\textgreater{}     \textless{}int\textgreater{}   \textless{}int\textgreater{}}
\CommentTok{\#\textgreater{} 1    87         13         31        15     5      3        49      38}

\CommentTok{\# 列属性是字符的列求唯一值数}
\CommentTok{\# starwars \%\textgreater{}\% }
\CommentTok{\#   summarise(length(unique(name)))}
\CommentTok{\# starwars \%\textgreater{}\% }
\CommentTok{\#   summarise(length(unique(hair\_color)))}

\NormalTok{starwars }\SpecialCharTok{\%\textgreater{}\%} 
  \FunctionTok{group\_by}\NormalTok{(species) }\SpecialCharTok{\%\textgreater{}\%} 
  \FunctionTok{filter}\NormalTok{(}\FunctionTok{n}\NormalTok{() }\SpecialCharTok{\textgreater{}} \DecValTok{1}\NormalTok{) }\SpecialCharTok{\%\textgreater{}\%} 
  \FunctionTok{summarise}\NormalTok{(}\FunctionTok{across}\NormalTok{(}\FunctionTok{c}\NormalTok{(sex, gender, homeworld), }\SpecialCharTok{\textasciitilde{}} \FunctionTok{length}\NormalTok{(}\FunctionTok{unique}\NormalTok{(.x))))}
\CommentTok{\#\textgreater{} \# A tibble: 9 x 4}
\CommentTok{\#\textgreater{}   species    sex gender homeworld}
\CommentTok{\#\textgreater{}   \textless{}chr\textgreater{}    \textless{}int\textgreater{}  \textless{}int\textgreater{}     \textless{}int\textgreater{}}
\CommentTok{\#\textgreater{} 1 Droid        1      2         3}
\CommentTok{\#\textgreater{} 2 Gungan       1      1         1}
\CommentTok{\#\textgreater{} 3 Human        2      2        16}
\CommentTok{\#\textgreater{} 4 Kaminoan     2      2         1}
\CommentTok{\#\textgreater{} 5 Mirialan     1      1         1}
\CommentTok{\#\textgreater{} 6 Twi\textquotesingle{}lek      2      2         1}
\CommentTok{\#\textgreater{} \# ... with 3 more rows}

\NormalTok{starwars }\SpecialCharTok{\%\textgreater{}\%} 
  \FunctionTok{group\_by}\NormalTok{(homeworld) }\SpecialCharTok{\%\textgreater{}\%} 
  \FunctionTok{filter}\NormalTok{(}\FunctionTok{n}\NormalTok{() }\SpecialCharTok{\textgreater{}} \DecValTok{1}\NormalTok{) }\SpecialCharTok{\%\textgreater{}\%} 
  \FunctionTok{summarise}\NormalTok{(}\FunctionTok{across}\NormalTok{(}\FunctionTok{where}\NormalTok{(is.numeric), }\SpecialCharTok{\textasciitilde{}} \FunctionTok{mean}\NormalTok{(.x, }\AttributeTok{na.rm =} \ConstantTok{TRUE}\NormalTok{)))}
\CommentTok{\#\textgreater{} \# A tibble: 10 x 4}
\CommentTok{\#\textgreater{}   homeworld height  mass birth\_year}
\CommentTok{\#\textgreater{}   \textless{}chr\textgreater{}      \textless{}dbl\textgreater{} \textless{}dbl\textgreater{}      \textless{}dbl\textgreater{}}
\CommentTok{\#\textgreater{} 1 Alderaan    176.  64         43  }
\CommentTok{\#\textgreater{} 2 Corellia    175   78.5       25  }
\CommentTok{\#\textgreater{} 3 Coruscant   174.  50         91  }
\CommentTok{\#\textgreater{} 4 Kamino      208.  83.1       31.5}
\CommentTok{\#\textgreater{} 5 Kashyyyk    231  124        200  }
\CommentTok{\#\textgreater{} 6 Mirial      168   53.1       49  }
\CommentTok{\#\textgreater{} \# ... with 4 more rows}
\end{Highlighting}
\end{Shaded}

\texttt{across()} 不会选择分组变量:

\begin{Shaded}
\begin{Highlighting}[]
\NormalTok{df }\OtherTok{\textless{}{-}} \FunctionTok{data.frame}\NormalTok{(}\AttributeTok{g =} \FunctionTok{c}\NormalTok{(}\DecValTok{1}\NormalTok{, }\DecValTok{1}\NormalTok{, }\DecValTok{2}\NormalTok{), }\AttributeTok{x =} \FunctionTok{c}\NormalTok{(}\SpecialCharTok{{-}}\DecValTok{1}\NormalTok{, }\DecValTok{1}\NormalTok{, }\DecValTok{3}\NormalTok{), }\AttributeTok{y =} \FunctionTok{c}\NormalTok{(}\SpecialCharTok{{-}}\DecValTok{1}\NormalTok{, }\SpecialCharTok{{-}}\DecValTok{4}\NormalTok{, }\SpecialCharTok{{-}}\DecValTok{9}\NormalTok{))}
\NormalTok{df }\SpecialCharTok{\%\textgreater{}\%} 
  \FunctionTok{group\_by}\NormalTok{(g) }\SpecialCharTok{\%\textgreater{}\%} 
  \FunctionTok{summarise}\NormalTok{(}\FunctionTok{across}\NormalTok{(}\FunctionTok{where}\NormalTok{(is.numeric), sum))}
\CommentTok{\#\textgreater{} \# A tibble: 2 x 3}
\CommentTok{\#\textgreater{}       g     x     y}
\CommentTok{\#\textgreater{}   \textless{}dbl\textgreater{} \textless{}dbl\textgreater{} \textless{}dbl\textgreater{}}
\CommentTok{\#\textgreater{} 1     1     0    {-}5}
\CommentTok{\#\textgreater{} 2     2     3    {-}9}
\end{Highlighting}
\end{Shaded}

\hypertarget{ux591aux79cdux51fdux6570ux529fux80fd}{%
\subsection{多种函数功能}\label{ux591aux79cdux51fdux6570ux529fux80fd}}

通过在第二个参数提供函数或lambda函数的命名列表,可是使用多个函数转换每个变量:

\begin{Shaded}
\begin{Highlighting}[]
\NormalTok{min\_max }\OtherTok{\textless{}{-}} \FunctionTok{list}\NormalTok{(}
  \AttributeTok{min =} \SpecialCharTok{\textasciitilde{}}\FunctionTok{min}\NormalTok{(.x, }\AttributeTok{na.rm =} \ConstantTok{TRUE}\NormalTok{), }
  \AttributeTok{max =} \SpecialCharTok{\textasciitilde{}}\FunctionTok{max}\NormalTok{(.x, }\AttributeTok{na.rm =} \ConstantTok{TRUE}\NormalTok{)}
\NormalTok{)}
\NormalTok{starwars }\SpecialCharTok{\%\textgreater{}\%} \FunctionTok{summarise}\NormalTok{(}\FunctionTok{across}\NormalTok{(}\FunctionTok{where}\NormalTok{(is.numeric), min\_max))}
\CommentTok{\#\textgreater{} \# A tibble: 1 x 6}
\CommentTok{\#\textgreater{}   height\_min height\_max mass\_min mass\_max birth\_year\_min birth\_year\_max}
\CommentTok{\#\textgreater{}        \textless{}int\textgreater{}      \textless{}int\textgreater{}    \textless{}dbl\textgreater{}    \textless{}dbl\textgreater{}          \textless{}dbl\textgreater{}          \textless{}dbl\textgreater{}}
\CommentTok{\#\textgreater{} 1         66        264       15     1358              8            896}
\end{Highlighting}
\end{Shaded}

通过\texttt{.names}参数控制名称:

NB:该参数的机制没有特别理解,需多练习体会。主要是运用到匿名函数时

以下是官方案例,但是报错(目前已修复):

\begin{Shaded}
\begin{Highlighting}[]
\NormalTok{starwars }\SpecialCharTok{\%\textgreater{}\%} \FunctionTok{summarise}\NormalTok{(}\FunctionTok{across}\NormalTok{(}\FunctionTok{where}\NormalTok{(is.numeric), min\_max, }\AttributeTok{.names =} \StringTok{"\{.fn\}.\{.col\}"}\NormalTok{))}
\end{Highlighting}
\end{Shaded}

修改后正常运行:

\begin{Shaded}
\begin{Highlighting}[]
\NormalTok{starwars }\SpecialCharTok{\%\textgreater{}\%} \FunctionTok{summarise}\NormalTok{(}\FunctionTok{across}\NormalTok{(}\FunctionTok{where}\NormalTok{(is.numeric), min\_max, }\AttributeTok{.names =} \StringTok{"\{fn\}.\{col\}"}\NormalTok{))}
\CommentTok{\#\textgreater{} \# A tibble: 1 x 6}
\CommentTok{\#\textgreater{}   min.height max.height min.mass max.mass min.birth\_year max.birth\_year}
\CommentTok{\#\textgreater{}        \textless{}int\textgreater{}      \textless{}int\textgreater{}    \textless{}dbl\textgreater{}    \textless{}dbl\textgreater{}          \textless{}dbl\textgreater{}          \textless{}dbl\textgreater{}}
\CommentTok{\#\textgreater{} 1         66        264       15     1358              8            896}
\end{Highlighting}
\end{Shaded}

区别主要是\texttt{.names}参数的使用方式问题,\texttt{.}加不加的问题。

\begin{Shaded}
\begin{Highlighting}[]

\NormalTok{starwars }\SpecialCharTok{\%\textgreater{}\%} \FunctionTok{summarise}\NormalTok{(}\FunctionTok{across}\NormalTok{(}\FunctionTok{where}\NormalTok{(is.numeric), min\_max, }\AttributeTok{.names =} \StringTok{"\{fn\}——\{col\}"}\NormalTok{))}
\end{Highlighting}
\end{Shaded}

\hypertarget{ux5f53ux524dux5217}{%
\subsection{当前列}\label{ux5f53ux524dux5217}}

如果需要,可以通过调用访问内部的``当前''列的名称\texttt{cur\_column()}。

该函数不是特别容易理解,需要多尝试使用加深认识。

\begin{Shaded}
\begin{Highlighting}[]
\NormalTok{df }\OtherTok{\textless{}{-}} \FunctionTok{tibble}\NormalTok{(}\AttributeTok{x =} \DecValTok{1}\SpecialCharTok{:}\DecValTok{3}\NormalTok{, }\AttributeTok{y =} \DecValTok{3}\SpecialCharTok{:}\DecValTok{5}\NormalTok{, }\AttributeTok{z =} \DecValTok{5}\SpecialCharTok{:}\DecValTok{7}\NormalTok{)}
\NormalTok{mult }\OtherTok{\textless{}{-}} \FunctionTok{list}\NormalTok{(}\AttributeTok{x =} \DecValTok{1}\NormalTok{, }\AttributeTok{y =} \DecValTok{10}\NormalTok{, }\AttributeTok{z =} \DecValTok{100}\NormalTok{)}

\NormalTok{df }\SpecialCharTok{\%\textgreater{}\%} \FunctionTok{mutate}\NormalTok{(}\FunctionTok{across}\NormalTok{(}\FunctionTok{all\_of}\NormalTok{(}\FunctionTok{names}\NormalTok{(mult)), }\SpecialCharTok{\textasciitilde{}}\NormalTok{ .x }\SpecialCharTok{*}\NormalTok{ mult[[}\FunctionTok{cur\_column}\NormalTok{()]]))}
\CommentTok{\#\textgreater{} \# A tibble: 3 x 3}
\CommentTok{\#\textgreater{}       x     y     z}
\CommentTok{\#\textgreater{}   \textless{}dbl\textgreater{} \textless{}dbl\textgreater{} \textless{}dbl\textgreater{}}
\CommentTok{\#\textgreater{} 1     1    30   500}
\CommentTok{\#\textgreater{} 2     2    40   600}
\CommentTok{\#\textgreater{} 3     3    50   700}
\end{Highlighting}
\end{Shaded}

\hypertarget{dplyr-row-manipulation}{%
\section{行操作}\label{dplyr-row-manipulation}}

行操作指不同字段间的计算,如\texttt{Excle}的列与列之间计算,\texttt{Excle}中的函数对行列不敏感,没有明显区别,但是\texttt{R}中\texttt{tidyverse}里列计算简单,行间计算依赖\texttt{rowwise()}函数实现

\hypertarget{ux6bd4ux8f83ux5deeux5f02}{%
\subsection{比较差异}\label{ux6bd4ux8f83ux5deeux5f02}}

\begin{Shaded}
\begin{Highlighting}[]
\NormalTok{df }\OtherTok{\textless{}{-}} \FunctionTok{tibble}\NormalTok{(}\AttributeTok{x =} \DecValTok{1}\SpecialCharTok{:}\DecValTok{2}\NormalTok{, }\AttributeTok{y =} \DecValTok{3}\SpecialCharTok{:}\DecValTok{4}\NormalTok{, }\AttributeTok{z =} \DecValTok{5}\SpecialCharTok{:}\DecValTok{6}\NormalTok{)}
\NormalTok{df }\SpecialCharTok{\%\textgreater{}\%} \FunctionTok{rowwise}\NormalTok{()}
\CommentTok{\#\textgreater{} \# A tibble: 2 x 3}
\CommentTok{\#\textgreater{} \# Rowwise: }
\CommentTok{\#\textgreater{}       x     y     z}
\CommentTok{\#\textgreater{}   \textless{}int\textgreater{} \textless{}int\textgreater{} \textless{}int\textgreater{}}
\CommentTok{\#\textgreater{} 1     1     3     5}
\CommentTok{\#\textgreater{} 2     2     4     6}
\end{Highlighting}
\end{Shaded}

像\texttt{group\_by()},\texttt{rowwise()}并没有做任何事情,它的作用是改变其他动词的工作方式。

注意以下代码返回结果不同

\begin{Shaded}
\begin{Highlighting}[]
\NormalTok{df }\SpecialCharTok{\%\textgreater{}\%} \FunctionTok{mutate}\NormalTok{(}\AttributeTok{m =} \FunctionTok{mean}\NormalTok{(}\FunctionTok{c}\NormalTok{(x, y, z)))}
\CommentTok{\#\textgreater{} \# A tibble: 2 x 4}
\CommentTok{\#\textgreater{}       x     y     z     m}
\CommentTok{\#\textgreater{}   \textless{}int\textgreater{} \textless{}int\textgreater{} \textless{}int\textgreater{} \textless{}dbl\textgreater{}}
\CommentTok{\#\textgreater{} 1     1     3     5   3.5}
\CommentTok{\#\textgreater{} 2     2     4     6   3.5}
\NormalTok{df }\SpecialCharTok{\%\textgreater{}\%} \FunctionTok{rowwise}\NormalTok{() }\SpecialCharTok{\%\textgreater{}\%} \FunctionTok{mutate}\NormalTok{(}\AttributeTok{m =} \FunctionTok{mean}\NormalTok{(}\FunctionTok{c}\NormalTok{(x, y, z)))}
\CommentTok{\#\textgreater{} \# A tibble: 2 x 4}
\CommentTok{\#\textgreater{} \# Rowwise: }
\CommentTok{\#\textgreater{}       x     y     z     m}
\CommentTok{\#\textgreater{}   \textless{}int\textgreater{} \textless{}int\textgreater{} \textless{}int\textgreater{} \textless{}dbl\textgreater{}}
\CommentTok{\#\textgreater{} 1     1     3     5     3}
\CommentTok{\#\textgreater{} 2     2     4     6     4}
\end{Highlighting}
\end{Shaded}

\texttt{df\ \%\textgreater{}\%\ mutate(m\ =\ mean(c(x,\ y,\ z)))}返回的结果是x,y,z散列全部数据的均值;\texttt{df\ \%\textgreater{}\%\ rowwise()\ \%\textgreater{}\%\ mutate(m\ =\ mean(c(x,\ y,\ z)))}通过rowwise改变了mean的作为范围,返回的某行x,y,z列3个数字的均值。两种动词的作用的范围因为rowwise完全改变。

您可以选择在调用中提供``标识符''变量\texttt{rowwise()}。这些变量在您调用时被保留\texttt{summarise()},因此它们的行为与传递给的分组变量有些相似\texttt{group\_by()}:

\begin{Shaded}
\begin{Highlighting}[]
\NormalTok{df }\OtherTok{\textless{}{-}} \FunctionTok{tibble}\NormalTok{(}\AttributeTok{name =} \FunctionTok{c}\NormalTok{(}\StringTok{"Mara"}\NormalTok{, }\StringTok{"Hadley"}\NormalTok{), }\AttributeTok{x =} \DecValTok{1}\SpecialCharTok{:}\DecValTok{2}\NormalTok{, }\AttributeTok{y =} \DecValTok{3}\SpecialCharTok{:}\DecValTok{4}\NormalTok{, }\AttributeTok{z =} \DecValTok{5}\SpecialCharTok{:}\DecValTok{6}\NormalTok{)}

\NormalTok{df }\SpecialCharTok{\%\textgreater{}\%} 
  \FunctionTok{rowwise}\NormalTok{() }\SpecialCharTok{\%\textgreater{}\%} 
  \FunctionTok{summarise}\NormalTok{(}\AttributeTok{m =} \FunctionTok{mean}\NormalTok{(}\FunctionTok{c}\NormalTok{(x, y, z)))}
\CommentTok{\#\textgreater{} \textasciigrave{}summarise()\textasciigrave{} has ungrouped output. You can override using the \textasciigrave{}.groups\textasciigrave{} argument.}
\CommentTok{\#\textgreater{} \# A tibble: 2 x 1}
\CommentTok{\#\textgreater{}       m}
\CommentTok{\#\textgreater{}   \textless{}dbl\textgreater{}}
\CommentTok{\#\textgreater{} 1     3}
\CommentTok{\#\textgreater{} 2     4}

\NormalTok{df }\SpecialCharTok{\%\textgreater{}\%} 
  \FunctionTok{rowwise}\NormalTok{(name) }\SpecialCharTok{\%\textgreater{}\%} 
  \FunctionTok{summarise}\NormalTok{(}\AttributeTok{m =} \FunctionTok{mean}\NormalTok{(}\FunctionTok{c}\NormalTok{(x, y, z)))}
\CommentTok{\#\textgreater{} \textasciigrave{}summarise()\textasciigrave{} has grouped output by \textquotesingle{}name\textquotesingle{}. You can override using the \textasciigrave{}.groups\textasciigrave{} argument.}
\CommentTok{\#\textgreater{} \# A tibble: 2 x 2}
\CommentTok{\#\textgreater{} \# Groups:   name [2]}
\CommentTok{\#\textgreater{}   name       m}
\CommentTok{\#\textgreater{}   \textless{}chr\textgreater{}  \textless{}dbl\textgreater{}}
\CommentTok{\#\textgreater{} 1 Mara       3}
\CommentTok{\#\textgreater{} 2 Hadley     4}
\end{Highlighting}
\end{Shaded}

\hypertarget{ux884cux6c47ux603b}{%
\subsection{行汇总}\label{ux884cux6c47ux603b}}

\texttt{dplyr::summarise()}使得聚合一列的值非常容易。当\texttt{summarise()}与\texttt{rowwise()}结合使用时,可以轻松聚合一行各列的值:

\begin{Shaded}
\begin{Highlighting}[]
\NormalTok{df }\OtherTok{\textless{}{-}} \FunctionTok{tibble}\NormalTok{(}\AttributeTok{id =} \DecValTok{1}\SpecialCharTok{:}\DecValTok{6}\NormalTok{, }\AttributeTok{w =} \DecValTok{10}\SpecialCharTok{:}\DecValTok{15}\NormalTok{, }\AttributeTok{x =} \DecValTok{20}\SpecialCharTok{:}\DecValTok{25}\NormalTok{, }\AttributeTok{y =} \DecValTok{30}\SpecialCharTok{:}\DecValTok{35}\NormalTok{, }\AttributeTok{z =} \DecValTok{40}\SpecialCharTok{:}\DecValTok{45}\NormalTok{)}
\NormalTok{rf }\OtherTok{\textless{}{-}}\NormalTok{ df }\SpecialCharTok{\%\textgreater{}\%} \FunctionTok{rowwise}\NormalTok{(id)}
\NormalTok{rf }\SpecialCharTok{\%\textgreater{}\%} \FunctionTok{mutate}\NormalTok{(}\AttributeTok{total =} \FunctionTok{sum}\NormalTok{(}\FunctionTok{c}\NormalTok{(w, x, y, z)))}
\CommentTok{\#\textgreater{} \# A tibble: 6 x 6}
\CommentTok{\#\textgreater{} \# Rowwise:  id}
\CommentTok{\#\textgreater{}      id     w     x     y     z total}
\CommentTok{\#\textgreater{}   \textless{}int\textgreater{} \textless{}int\textgreater{} \textless{}int\textgreater{} \textless{}int\textgreater{} \textless{}int\textgreater{} \textless{}int\textgreater{}}
\CommentTok{\#\textgreater{} 1     1    10    20    30    40   100}
\CommentTok{\#\textgreater{} 2     2    11    21    31    41   104}
\CommentTok{\#\textgreater{} 3     3    12    22    32    42   108}
\CommentTok{\#\textgreater{} 4     4    13    23    33    43   112}
\CommentTok{\#\textgreater{} 5     5    14    24    34    44   116}
\CommentTok{\#\textgreater{} 6     6    15    25    35    45   120}
\NormalTok{rf }\SpecialCharTok{\%\textgreater{}\%} \FunctionTok{summarise}\NormalTok{(}\AttributeTok{total =} \FunctionTok{sum}\NormalTok{(}\FunctionTok{c}\NormalTok{(w, x, y, z)))}
\CommentTok{\#\textgreater{} \textasciigrave{}summarise()\textasciigrave{} has grouped output by \textquotesingle{}id\textquotesingle{}. You can override using the \textasciigrave{}.groups\textasciigrave{} argument.}
\CommentTok{\#\textgreater{} \# A tibble: 6 x 2}
\CommentTok{\#\textgreater{} \# Groups:   id [6]}
\CommentTok{\#\textgreater{}      id total}
\CommentTok{\#\textgreater{}   \textless{}int\textgreater{} \textless{}int\textgreater{}}
\CommentTok{\#\textgreater{} 1     1   100}
\CommentTok{\#\textgreater{} 2     2   104}
\CommentTok{\#\textgreater{} 3     3   108}
\CommentTok{\#\textgreater{} 4     4   112}
\CommentTok{\#\textgreater{} 5     5   116}
\CommentTok{\#\textgreater{} 6     6   120}
\end{Highlighting}
\end{Shaded}

键入每个变量名称很繁琐,通过\texttt{c\_across()}使更简单。

\begin{quote}
详情可见vignette(``rowwise'')。
\end{quote}

\begin{Shaded}
\begin{Highlighting}[]
\NormalTok{rf }\SpecialCharTok{\%\textgreater{}\%} \FunctionTok{mutate}\NormalTok{(}\AttributeTok{total =} \FunctionTok{sum}\NormalTok{(}\FunctionTok{c\_across}\NormalTok{(w}\SpecialCharTok{:}\NormalTok{z)))}
\CommentTok{\#\textgreater{} \# A tibble: 6 x 6}
\CommentTok{\#\textgreater{} \# Rowwise:  id}
\CommentTok{\#\textgreater{}      id     w     x     y     z total}
\CommentTok{\#\textgreater{}   \textless{}int\textgreater{} \textless{}int\textgreater{} \textless{}int\textgreater{} \textless{}int\textgreater{} \textless{}int\textgreater{} \textless{}int\textgreater{}}
\CommentTok{\#\textgreater{} 1     1    10    20    30    40   100}
\CommentTok{\#\textgreater{} 2     2    11    21    31    41   104}
\CommentTok{\#\textgreater{} 3     3    12    22    32    42   108}
\CommentTok{\#\textgreater{} 4     4    13    23    33    43   112}
\CommentTok{\#\textgreater{} 5     5    14    24    34    44   116}
\CommentTok{\#\textgreater{} 6     6    15    25    35    45   120}
\NormalTok{rf }\SpecialCharTok{\%\textgreater{}\%} \FunctionTok{mutate}\NormalTok{(}\AttributeTok{total =} \FunctionTok{sum}\NormalTok{(}\FunctionTok{c\_across}\NormalTok{(}\FunctionTok{where}\NormalTok{(is.numeric))))}
\CommentTok{\#\textgreater{} \# A tibble: 6 x 6}
\CommentTok{\#\textgreater{} \# Rowwise:  id}
\CommentTok{\#\textgreater{}      id     w     x     y     z total}
\CommentTok{\#\textgreater{}   \textless{}int\textgreater{} \textless{}int\textgreater{} \textless{}int\textgreater{} \textless{}int\textgreater{} \textless{}int\textgreater{} \textless{}int\textgreater{}}
\CommentTok{\#\textgreater{} 1     1    10    20    30    40   100}
\CommentTok{\#\textgreater{} 2     2    11    21    31    41   104}
\CommentTok{\#\textgreater{} 3     3    12    22    32    42   108}
\CommentTok{\#\textgreater{} 4     4    13    23    33    43   112}
\CommentTok{\#\textgreater{} 5     5    14    24    34    44   116}
\CommentTok{\#\textgreater{} 6     6    15    25    35    45   120}

\NormalTok{rf }\SpecialCharTok{\%\textgreater{}\%} 
  \FunctionTok{mutate}\NormalTok{(}\AttributeTok{total =} \FunctionTok{sum}\NormalTok{(}\FunctionTok{c\_across}\NormalTok{(w}\SpecialCharTok{:}\NormalTok{z))) }\SpecialCharTok{\%\textgreater{}\%} 
  \FunctionTok{ungroup}\NormalTok{() }\SpecialCharTok{\%\textgreater{}\%} 
  \FunctionTok{mutate}\NormalTok{(}\FunctionTok{across}\NormalTok{(w}\SpecialCharTok{:}\NormalTok{z, }\SpecialCharTok{\textasciitilde{}}\NormalTok{ . }\SpecialCharTok{/}\NormalTok{ total))}
\CommentTok{\#\textgreater{} \# A tibble: 6 x 6}
\CommentTok{\#\textgreater{}      id     w     x     y     z total}
\CommentTok{\#\textgreater{}   \textless{}int\textgreater{} \textless{}dbl\textgreater{} \textless{}dbl\textgreater{} \textless{}dbl\textgreater{} \textless{}dbl\textgreater{} \textless{}int\textgreater{}}
\CommentTok{\#\textgreater{} 1     1 0.1   0.2   0.3   0.4     100}
\CommentTok{\#\textgreater{} 2     2 0.106 0.202 0.298 0.394   104}
\CommentTok{\#\textgreater{} 3     3 0.111 0.204 0.296 0.389   108}
\CommentTok{\#\textgreater{} 4     4 0.116 0.205 0.295 0.384   112}
\CommentTok{\#\textgreater{} 5     5 0.121 0.207 0.293 0.379   116}
\CommentTok{\#\textgreater{} 6     6 0.125 0.208 0.292 0.375   120}
\end{Highlighting}
\end{Shaded}

\hypertarget{dplyr-groupby-manipulation}{%
\section{分组操作}\label{dplyr-groupby-manipulation}}

详情: \url{https://cloud.r-project.org/web/packages/dplyr/vignettes/grouping.html}

\texttt{group\_by()}最重要的分组动词,需要一个数据框和一个或多个变量进行分组:

\hypertarget{ux6dfbux52a0ux5206ux7ec4}{%
\subsection{添加分组}\label{ux6dfbux52a0ux5206ux7ec4}}

\begin{Shaded}
\begin{Highlighting}[]
\NormalTok{by\_species }\OtherTok{\textless{}{-}}\NormalTok{ starwars }\SpecialCharTok{\%\textgreater{}\%} \FunctionTok{group\_by}\NormalTok{(species)}
\NormalTok{by\_sex\_gender }\OtherTok{\textless{}{-}}\NormalTok{ starwars }\SpecialCharTok{\%\textgreater{}\%} \FunctionTok{group\_by}\NormalTok{(sex, gender)}
\end{Highlighting}
\end{Shaded}

除了按照现有变量分组外,还可以按照函数处理后的变量分组,等效在\texttt{mutate()}之后执行\texttt{group\_by}:

\begin{Shaded}
\begin{Highlighting}[]
\NormalTok{bmi\_breaks }\OtherTok{\textless{}{-}} \FunctionTok{c}\NormalTok{(}\DecValTok{0}\NormalTok{, }\FloatTok{18.5}\NormalTok{, }\DecValTok{25}\NormalTok{, }\DecValTok{30}\NormalTok{, }\ConstantTok{Inf}\NormalTok{)}
\NormalTok{starwars }\SpecialCharTok{\%\textgreater{}\%}
  \FunctionTok{group\_by}\NormalTok{(}\AttributeTok{bmi\_cat =} \FunctionTok{cut}\NormalTok{(mass}\SpecialCharTok{/}\NormalTok{(height}\SpecialCharTok{/}\DecValTok{100}\NormalTok{)}\SpecialCharTok{\^{}}\DecValTok{2}\NormalTok{, }\AttributeTok{breaks=}\NormalTok{bmi\_breaks)) }\SpecialCharTok{\%\textgreater{}\%}
  \FunctionTok{tally}\NormalTok{()}
\CommentTok{\#\textgreater{} \# A tibble: 5 x 2}
\CommentTok{\#\textgreater{}   bmi\_cat       n}
\CommentTok{\#\textgreater{}   \textless{}fct\textgreater{}     \textless{}int\textgreater{}}
\CommentTok{\#\textgreater{} 1 (0,18.5]     10}
\CommentTok{\#\textgreater{} 2 (18.5,25]    24}
\CommentTok{\#\textgreater{} 3 (25,30]      13}
\CommentTok{\#\textgreater{} 4 (30,Inf]     12}
\CommentTok{\#\textgreater{} 5 \textless{}NA\textgreater{}         28}
\end{Highlighting}
\end{Shaded}

\hypertarget{ux5220ux9664ux5206ux7ec4ux53d8ux91cf}{%
\subsection{删除分组变量}\label{ux5220ux9664ux5206ux7ec4ux53d8ux91cf}}

要删除所有分组变量,使用\texttt{ungroup()}:

\begin{Shaded}
\begin{Highlighting}[]
\NormalTok{by\_species }\SpecialCharTok{\%\textgreater{}\%}
  \FunctionTok{ungroup}\NormalTok{() }\SpecialCharTok{\%\textgreater{}\%}
  \FunctionTok{tally}\NormalTok{()}
\CommentTok{\#\textgreater{} \# A tibble: 1 x 1}
\CommentTok{\#\textgreater{}       n}
\CommentTok{\#\textgreater{}   \textless{}int\textgreater{}}
\CommentTok{\#\textgreater{} 1    87}
\end{Highlighting}
\end{Shaded}

\hypertarget{ux52a8ux8bcd}{%
\subsection{动词}\label{ux52a8ux8bcd}}

\texttt{summarise()} 计算每个组的汇总,表示从\texttt{group\_keys}开始右侧添加汇总变量

\begin{Shaded}
\begin{Highlighting}[]
\NormalTok{by\_species }\SpecialCharTok{\%\textgreater{}\%}
  \FunctionTok{summarise}\NormalTok{(}
    \AttributeTok{n =} \FunctionTok{n}\NormalTok{(),}
    \AttributeTok{height =} \FunctionTok{mean}\NormalTok{(height, }\AttributeTok{na.rm =} \ConstantTok{TRUE}\NormalTok{)}
\NormalTok{  )}
\CommentTok{\#\textgreater{} \# A tibble: 38 x 3}
\CommentTok{\#\textgreater{}   species      n height}
\CommentTok{\#\textgreater{}   \textless{}chr\textgreater{}    \textless{}int\textgreater{}  \textless{}dbl\textgreater{}}
\CommentTok{\#\textgreater{} 1 Aleena       1    79 }
\CommentTok{\#\textgreater{} 2 Besalisk     1   198 }
\CommentTok{\#\textgreater{} 3 Cerean       1   198 }
\CommentTok{\#\textgreater{} 4 Chagrian     1   196 }
\CommentTok{\#\textgreater{} 5 Clawdite     1   168 }
\CommentTok{\#\textgreater{} 6 Droid        6   131.}
\CommentTok{\#\textgreater{} \# ... with 32 more rows}
\end{Highlighting}
\end{Shaded}

该\texttt{.groups=}参数控制输出的分组结构。删除右侧分组变量的历史行为对应于\texttt{.groups\ =} ``drop\_last''没有消息或.groups = NULL有消息(默认值)。

从1.0.0版开始,分组信息可以保留\texttt{(.groups\ =\ "keep")}或删除 \texttt{(.groups\ =\ \textquotesingle{}drop)}

\begin{Shaded}
\begin{Highlighting}[]
\NormalTok{a }\OtherTok{\textless{}{-}}\NormalTok{ by\_species }\SpecialCharTok{\%\textgreater{}\%}
  \FunctionTok{summarise}\NormalTok{(}
    \AttributeTok{n =} \FunctionTok{n}\NormalTok{(),}
    \AttributeTok{height =} \FunctionTok{mean}\NormalTok{(height, }\AttributeTok{na.rm =} \ConstantTok{TRUE}\NormalTok{),}\AttributeTok{.groups=}\StringTok{\textquotesingle{}drop\textquotesingle{}}\NormalTok{) }\SpecialCharTok{\%\textgreater{}\%} 
  \FunctionTok{group\_vars}\NormalTok{()}

\NormalTok{b }\OtherTok{\textless{}{-}}\NormalTok{ by\_species }\SpecialCharTok{\%\textgreater{}\%}
  \FunctionTok{summarise}\NormalTok{(}
    \AttributeTok{n =} \FunctionTok{n}\NormalTok{(),}
    \AttributeTok{height =} \FunctionTok{mean}\NormalTok{(height, }\AttributeTok{na.rm =} \ConstantTok{TRUE}\NormalTok{),}\AttributeTok{.groups=}\StringTok{\textquotesingle{}keep\textquotesingle{}}\NormalTok{) }\SpecialCharTok{\%\textgreater{}\%} 
  \FunctionTok{group\_vars}\NormalTok{()}

\FunctionTok{object.size}\NormalTok{(a)}
\CommentTok{\#\textgreater{} 48 bytes}
\FunctionTok{object.size}\NormalTok{(b)}
\CommentTok{\#\textgreater{} 112 bytes}
\end{Highlighting}
\end{Shaded}

在实际使用中,当数据较大时需要删掉分组信息。以上可以看到保留分组信息的比没保留的大了两倍多。

\hypertarget{dplyr-functions}{%
\section{常用函数}\label{dplyr-functions}}

\hypertarget{ux6761ux4ef6ux5224ux65ad}{%
\subsection{条件判断}\label{ux6761ux4ef6ux5224ux65ad}}

相比于\texttt{base::ifelse},\texttt{if\_else}更为严格,无论\texttt{TRUE}或\texttt{FALSE}输出类型一致,这样速度更快。与\texttt{data.table::fifelse()}功能相似。

\begin{Shaded}
\begin{Highlighting}[]
\FunctionTok{if\_else}\NormalTok{(condition, true, false, }\AttributeTok{missing =} \ConstantTok{NULL}\NormalTok{)}
\end{Highlighting}
\end{Shaded}

与\texttt{ifelse}不同的是,\texttt{if\_else}保留类型

\begin{Shaded}
\begin{Highlighting}[]
\NormalTok{x }\OtherTok{\textless{}{-}} \FunctionTok{factor}\NormalTok{(}\FunctionTok{sample}\NormalTok{(letters[}\DecValTok{1}\SpecialCharTok{:}\DecValTok{5}\NormalTok{], }\DecValTok{10}\NormalTok{, }\AttributeTok{replace =} \ConstantTok{TRUE}\NormalTok{))}
\FunctionTok{ifelse}\NormalTok{(x }\SpecialCharTok{\%in\%} \FunctionTok{c}\NormalTok{(}\StringTok{"a"}\NormalTok{, }\StringTok{"b"}\NormalTok{, }\StringTok{"c"}\NormalTok{), x, }\FunctionTok{factor}\NormalTok{(}\ConstantTok{NA}\NormalTok{))}
\CommentTok{\#\textgreater{}  [1]  1  2 NA  2  1 NA  1  2  3  1}
\FunctionTok{if\_else}\NormalTok{(x }\SpecialCharTok{\%in\%} \FunctionTok{c}\NormalTok{(}\StringTok{"a"}\NormalTok{, }\StringTok{"b"}\NormalTok{, }\StringTok{"c"}\NormalTok{), x, }\FunctionTok{factor}\NormalTok{(}\ConstantTok{NA}\NormalTok{))}
\CommentTok{\#\textgreater{}  [1] a    b    \textless{}NA\textgreater{} b    a    \textless{}NA\textgreater{} a    b    c    a   }
\CommentTok{\#\textgreater{} Levels: a b c d e}
\end{Highlighting}
\end{Shaded}

\hypertarget{case_when}{%
\subsection{case\_when}\label{case_when}}

当条件嵌套条件较多时,使用\texttt{case\_when},使代码可读并且不易出错。与sql 中的case when 等价。

\begin{Shaded}
\begin{Highlighting}[]
\NormalTok{Dates }\OtherTok{\textless{}{-}} \FunctionTok{as.Date}\NormalTok{(}\FunctionTok{c}\NormalTok{(}\StringTok{\textquotesingle{}2018{-}10{-}01\textquotesingle{}}\NormalTok{, }\StringTok{\textquotesingle{}2018{-}10{-}02\textquotesingle{}}\NormalTok{, }\StringTok{\textquotesingle{}2018{-}10{-}03\textquotesingle{}}\NormalTok{))}
\FunctionTok{case\_when}\NormalTok{(}
\NormalTok{  Dates }\SpecialCharTok{==} \StringTok{\textquotesingle{}2018{-}10{-}01\textquotesingle{}} \SpecialCharTok{\textasciitilde{}}\NormalTok{ Dates }\SpecialCharTok{{-}} \DecValTok{1}\NormalTok{,}
\NormalTok{  Dates }\SpecialCharTok{==} \StringTok{\textquotesingle{}2018{-}10{-}02\textquotesingle{}} \SpecialCharTok{\textasciitilde{}}\NormalTok{ Dates }\SpecialCharTok{+} \DecValTok{1}\NormalTok{,}
\NormalTok{  Dates }\SpecialCharTok{==} \StringTok{\textquotesingle{}2018{-}10{-}03\textquotesingle{}} \SpecialCharTok{\textasciitilde{}}\NormalTok{ Dates }\SpecialCharTok{+} \DecValTok{2}\NormalTok{,}
  \ConstantTok{TRUE} \SpecialCharTok{\textasciitilde{}}\NormalTok{ Dates}
\NormalTok{)}
\CommentTok{\#\textgreater{} [1] "2018{-}09{-}30" "2018{-}10{-}03" "2018{-}10{-}05"}
\end{Highlighting}
\end{Shaded}

\hypertarget{ux8ba1ux6570ux51fdux6570}{%
\subsection{计数函数}\label{ux8ba1ux6570ux51fdux6570}}

\begin{itemize}
\tightlist
\item
  计数
\end{itemize}

\texttt{count()}函数用来计数。下面两种表达方式等价。

\begin{Shaded}
\begin{Highlighting}[]
\NormalTok{df }\SpecialCharTok{\%\textgreater{}\%} \FunctionTok{count}\NormalTok{(a, b)}
\CommentTok{\# same above}
\NormalTok{df }\SpecialCharTok{\%\textgreater{}\%} \FunctionTok{group\_by}\NormalTok{(a, b) }\SpecialCharTok{\%\textgreater{}\%} \FunctionTok{summarise}\NormalTok{(}\AttributeTok{n =} \FunctionTok{n}\NormalTok{())}
\end{Highlighting}
\end{Shaded}

\begin{Shaded}
\begin{Highlighting}[]
\NormalTok{starwars }\SpecialCharTok{\%\textgreater{}\%} \FunctionTok{count}\NormalTok{(species)}
\CommentTok{\#\textgreater{} \# A tibble: 38 x 2}
\CommentTok{\#\textgreater{}   species      n}
\CommentTok{\#\textgreater{}   \textless{}chr\textgreater{}    \textless{}int\textgreater{}}
\CommentTok{\#\textgreater{} 1 Aleena       1}
\CommentTok{\#\textgreater{} 2 Besalisk     1}
\CommentTok{\#\textgreater{} 3 Cerean       1}
\CommentTok{\#\textgreater{} 4 Chagrian     1}
\CommentTok{\#\textgreater{} 5 Clawdite     1}
\CommentTok{\#\textgreater{} 6 Droid        6}
\CommentTok{\#\textgreater{} \# ... with 32 more rows}
\CommentTok{\# same above 等价}
\NormalTok{starwars }\SpecialCharTok{\%\textgreater{}\%} \FunctionTok{group\_by}\NormalTok{(species) }\SpecialCharTok{\%\textgreater{}\%} \FunctionTok{summarise}\NormalTok{(}\AttributeTok{n =} \FunctionTok{n}\NormalTok{())}
\CommentTok{\#\textgreater{} \# A tibble: 38 x 2}
\CommentTok{\#\textgreater{}   species      n}
\CommentTok{\#\textgreater{}   \textless{}chr\textgreater{}    \textless{}int\textgreater{}}
\CommentTok{\#\textgreater{} 1 Aleena       1}
\CommentTok{\#\textgreater{} 2 Besalisk     1}
\CommentTok{\#\textgreater{} 3 Cerean       1}
\CommentTok{\#\textgreater{} 4 Chagrian     1}
\CommentTok{\#\textgreater{} 5 Clawdite     1}
\CommentTok{\#\textgreater{} 6 Droid        6}
\CommentTok{\#\textgreater{} \# ... with 32 more rows}
\end{Highlighting}
\end{Shaded}

\begin{itemize}
\tightlist
\item
  非重复计数
\end{itemize}

\texttt{n\_distinct()}与\texttt{length(unique(x))}等价,但是更快更简洁。当我们需要给门店或订单之类数据需要去重计算时采用该函数。

\begin{Shaded}
\begin{Highlighting}[]
\NormalTok{x }\OtherTok{\textless{}{-}} \FunctionTok{sample}\NormalTok{(}\DecValTok{1}\SpecialCharTok{:}\DecValTok{10}\NormalTok{, }\FloatTok{1e5}\NormalTok{, }\AttributeTok{rep =} \ConstantTok{TRUE}\NormalTok{)}
\FunctionTok{length}\NormalTok{(}\FunctionTok{unique}\NormalTok{(x))}
\CommentTok{\#\textgreater{} [1] 10}
\FunctionTok{n\_distinct}\NormalTok{(x)}
\CommentTok{\#\textgreater{} [1] 10}
\end{Highlighting}
\end{Shaded}

\hypertarget{ux6392ux5e8fux51fdux6570}{%
\subsection{排序函数}\label{ux6392ux5e8fux51fdux6570}}

\texttt{dplyr}共六种排序函数,模仿SQL2003中的排名函数。

\begin{itemize}
\tightlist
\item
  row\_number():等于 rank(ties.method = ``first'')
\item
  min\_rank(): 等于 rank(ties.method = ``min'')
\item
  dense\_rank(): 与min\_rank()相似,但是没有间隔
\item
  percent\_rank():返回0,1之间,通过min\_rank()返回值缩放至{[}0,1{]}
\end{itemize}

\begin{Shaded}
\begin{Highlighting}[]
\NormalTok{x }\OtherTok{\textless{}{-}} \FunctionTok{c}\NormalTok{(}\DecValTok{5}\NormalTok{, }\DecValTok{1}\NormalTok{, }\DecValTok{3}\NormalTok{, }\DecValTok{2}\NormalTok{, }\DecValTok{2}\NormalTok{, }\ConstantTok{NA}\NormalTok{)}
\FunctionTok{row\_number}\NormalTok{(x)}
\CommentTok{\#\textgreater{} [1]  5  1  4  2  3 NA}
\FunctionTok{min\_rank}\NormalTok{(x)}
\CommentTok{\#\textgreater{} [1]  5  1  4  2  2 NA}
\FunctionTok{dense\_rank}\NormalTok{(x)}
\CommentTok{\#\textgreater{} [1]  4  1  3  2  2 NA}
\FunctionTok{percent\_rank}\NormalTok{(x)}
\CommentTok{\#\textgreater{} [1] 1.00 0.00 0.75 0.25 0.25   NA}
\FunctionTok{cume\_dist}\NormalTok{(x)}
\CommentTok{\#\textgreater{} [1] 1.0 0.2 0.8 0.6 0.6  NA}
\end{Highlighting}
\end{Shaded}

\hypertarget{ux63d0ux53d6ux5411ux91cf}{%
\subsection{提取向量}\label{ux63d0ux53d6ux5411ux91cf}}

该系列函数是对\texttt{{[}{[}}的包装。

\begin{Shaded}
\begin{Highlighting}[]
\FunctionTok{nth}\NormalTok{(x, n, }\AttributeTok{order\_by =} \ConstantTok{NULL}\NormalTok{, }\AttributeTok{default =} \FunctionTok{default\_missing}\NormalTok{(x))}
\FunctionTok{first}\NormalTok{(x, }\AttributeTok{order\_by =} \ConstantTok{NULL}\NormalTok{, }\AttributeTok{default =} \FunctionTok{default\_missing}\NormalTok{(x))}
\FunctionTok{last}\NormalTok{(x, }\AttributeTok{order\_by =} \ConstantTok{NULL}\NormalTok{, }\AttributeTok{default =} \FunctionTok{default\_missing}\NormalTok{(x))}
\end{Highlighting}
\end{Shaded}

\begin{Shaded}
\begin{Highlighting}[]
\NormalTok{x }\OtherTok{\textless{}{-}} \DecValTok{1}\SpecialCharTok{:}\DecValTok{10}
\NormalTok{y }\OtherTok{\textless{}{-}} \DecValTok{10}\SpecialCharTok{:}\DecValTok{1}
\FunctionTok{first}\NormalTok{(x)}
\CommentTok{\#\textgreater{} [1] 1}
\FunctionTok{last}\NormalTok{(y)}
\CommentTok{\#\textgreater{} [1] 1}
\FunctionTok{nth}\NormalTok{(x, }\DecValTok{1}\NormalTok{)}
\CommentTok{\#\textgreater{} [1] 1}
\FunctionTok{nth}\NormalTok{(x, }\DecValTok{5}\NormalTok{)}
\CommentTok{\#\textgreater{} [1] 5}
\end{Highlighting}
\end{Shaded}

\hypertarget{group-ux7cfbux5217}{%
\subsection{group 系列}\label{group-ux7cfbux5217}}

group\_by(),group\_map(), group\_nest(), group\_split(), group\_trim()等一系列函数。

其中我常用group\_by(),group\_split()两个函数。group\_by()是大部分数据操作中的分组操作,按照group\_by()的指定分组条件。

\begin{itemize}
\tightlist
\item
  group\_by()
\end{itemize}

\begin{Shaded}
\begin{Highlighting}[]
\CommentTok{\#group\_by()不会改变数据框}
\NormalTok{by\_cyl }\OtherTok{\textless{}{-}}\NormalTok{ mtcars }\SpecialCharTok{\%\textgreater{}\%} \FunctionTok{group\_by}\NormalTok{(cyl)}
\NormalTok{by\_cyl}
\CommentTok{\#\textgreater{} \# A tibble: 32 x 11}
\CommentTok{\#\textgreater{} \# Groups:   cyl [3]}
\CommentTok{\#\textgreater{}     mpg   cyl  disp    hp  drat    wt  qsec    vs    am  gear  carb}
\CommentTok{\#\textgreater{}   \textless{}dbl\textgreater{} \textless{}int\textgreater{} \textless{}dbl\textgreater{} \textless{}int\textgreater{} \textless{}dbl\textgreater{} \textless{}dbl\textgreater{} \textless{}dbl\textgreater{} \textless{}int\textgreater{} \textless{}int\textgreater{} \textless{}int\textgreater{} \textless{}int\textgreater{}}
\CommentTok{\#\textgreater{} 1  21       6   160   110  3.9   2.62  16.5     0     1     4     4}
\CommentTok{\#\textgreater{} 2  21       6   160   110  3.9   2.88  17.0     0     1     4     4}
\CommentTok{\#\textgreater{} 3  22.8     4   108    93  3.85  2.32  18.6     1     1     4     1}
\CommentTok{\#\textgreater{} 4  21.4     6   258   110  3.08  3.22  19.4     1     0     3     1}
\CommentTok{\#\textgreater{} 5  18.7     8   360   175  3.15  3.44  17.0     0     0     3     2}
\CommentTok{\#\textgreater{} 6  18.1     6   225   105  2.76  3.46  20.2     1     0     3     1}
\CommentTok{\#\textgreater{} \# ... with 26 more rows}
\CommentTok{\# It changes how it acts with the other dplyr verbs:}
\NormalTok{by\_cyl }\SpecialCharTok{\%\textgreater{}\%} \FunctionTok{summarise}\NormalTok{(}
  \AttributeTok{disp =} \FunctionTok{mean}\NormalTok{(disp),}
  \AttributeTok{hp =} \FunctionTok{mean}\NormalTok{(hp)}
\NormalTok{)}
\CommentTok{\#\textgreater{} \# A tibble: 3 x 3}
\CommentTok{\#\textgreater{}     cyl  disp    hp}
\CommentTok{\#\textgreater{}   \textless{}int\textgreater{} \textless{}dbl\textgreater{} \textless{}dbl\textgreater{}}
\CommentTok{\#\textgreater{} 1     4  105.  82.6}
\CommentTok{\#\textgreater{} 2     6  183. 122. }
\CommentTok{\#\textgreater{} 3     8  353. 209.}
\CommentTok{\# group\_by中可以添加计算字段 即mutate操作}
\NormalTok{mtcars }\SpecialCharTok{\%\textgreater{}\%} \FunctionTok{group\_by}\NormalTok{(}\AttributeTok{vsam =}\NormalTok{ vs }\SpecialCharTok{+}\NormalTok{ am) }\SpecialCharTok{\%\textgreater{}\%}
  \FunctionTok{group\_vars}\NormalTok{()}
\CommentTok{\#\textgreater{} [1] "vsam"}
\end{Highlighting}
\end{Shaded}

\begin{itemize}
\tightlist
\item
  group\_map()
\end{itemize}

group\_map,group\_modify,group\_walk等三个函数是purrr类具有迭代风格的函数。简单关系数据库的数据清洗一般不涉及,常用在建模等方面。

但是目前三个函数是实验性的,未来可能会发生变化。

\begin{Shaded}
\begin{Highlighting}[]
\CommentTok{\# return a list}
\CommentTok{\# 返回列表}
\NormalTok{mtcars }\SpecialCharTok{\%\textgreater{}\%}
  \FunctionTok{group\_by}\NormalTok{(cyl) }\SpecialCharTok{\%\textgreater{}\%}
  \FunctionTok{group\_map}\NormalTok{(}\SpecialCharTok{\textasciitilde{}} \FunctionTok{head}\NormalTok{(.x, 2L))}
\CommentTok{\#\textgreater{} [[1]]}
\CommentTok{\#\textgreater{} \# A tibble: 2 x 10}
\CommentTok{\#\textgreater{}     mpg  disp    hp  drat    wt  qsec    vs    am  gear  carb}
\CommentTok{\#\textgreater{}   \textless{}dbl\textgreater{} \textless{}dbl\textgreater{} \textless{}int\textgreater{} \textless{}dbl\textgreater{} \textless{}dbl\textgreater{} \textless{}dbl\textgreater{} \textless{}int\textgreater{} \textless{}int\textgreater{} \textless{}int\textgreater{} \textless{}int\textgreater{}}
\CommentTok{\#\textgreater{} 1  22.8  108     93  3.85  2.32  18.6     1     1     4     1}
\CommentTok{\#\textgreater{} 2  24.4  147.    62  3.69  3.19  20       1     0     4     2}
\CommentTok{\#\textgreater{} }
\CommentTok{\#\textgreater{} [[2]]}
\CommentTok{\#\textgreater{} \# A tibble: 2 x 10}
\CommentTok{\#\textgreater{}     mpg  disp    hp  drat    wt  qsec    vs    am  gear  carb}
\CommentTok{\#\textgreater{}   \textless{}dbl\textgreater{} \textless{}dbl\textgreater{} \textless{}int\textgreater{} \textless{}dbl\textgreater{} \textless{}dbl\textgreater{} \textless{}dbl\textgreater{} \textless{}int\textgreater{} \textless{}int\textgreater{} \textless{}int\textgreater{} \textless{}int\textgreater{}}
\CommentTok{\#\textgreater{} 1    21   160   110   3.9  2.62  16.5     0     1     4     4}
\CommentTok{\#\textgreater{} 2    21   160   110   3.9  2.88  17.0     0     1     4     4}
\CommentTok{\#\textgreater{} }
\CommentTok{\#\textgreater{} [[3]]}
\CommentTok{\#\textgreater{} \# A tibble: 2 x 10}
\CommentTok{\#\textgreater{}     mpg  disp    hp  drat    wt  qsec    vs    am  gear  carb}
\CommentTok{\#\textgreater{}   \textless{}dbl\textgreater{} \textless{}dbl\textgreater{} \textless{}int\textgreater{} \textless{}dbl\textgreater{} \textless{}dbl\textgreater{} \textless{}dbl\textgreater{} \textless{}int\textgreater{} \textless{}int\textgreater{} \textless{}int\textgreater{} \textless{}int\textgreater{}}
\CommentTok{\#\textgreater{} 1  18.7   360   175  3.15  3.44  17.0     0     0     3     2}
\CommentTok{\#\textgreater{} 2  14.3   360   245  3.21  3.57  15.8     0     0     3     4}
\end{Highlighting}
\end{Shaded}

\begin{Shaded}
\begin{Highlighting}[]
\NormalTok{iris }\SpecialCharTok{\%\textgreater{}\%}
  \FunctionTok{group\_by}\NormalTok{(Species) }\SpecialCharTok{\%\textgreater{}\%}
  \FunctionTok{group\_modify}\NormalTok{(}\SpecialCharTok{\textasciitilde{}}\NormalTok{ \{}
\NormalTok{    .x }\SpecialCharTok{\%\textgreater{}\%}
\NormalTok{      purrr}\SpecialCharTok{::}\FunctionTok{map\_dfc}\NormalTok{(fivenum) }\SpecialCharTok{\%\textgreater{}\%}
      \FunctionTok{mutate}\NormalTok{(}\AttributeTok{nms =} \FunctionTok{c}\NormalTok{(}\StringTok{"min"}\NormalTok{, }\StringTok{"Q1"}\NormalTok{, }\StringTok{"median"}\NormalTok{, }\StringTok{"Q3"}\NormalTok{, }\StringTok{"max"}\NormalTok{))}
\NormalTok{  \})}
\CommentTok{\#\textgreater{} \# A tibble: 15 x 6}
\CommentTok{\#\textgreater{} \# Groups:   Species [3]}
\CommentTok{\#\textgreater{}   Species    Sepal.Length Sepal.Width Petal.Length Petal.Width nms   }
\CommentTok{\#\textgreater{}   \textless{}fct\textgreater{}             \textless{}dbl\textgreater{}       \textless{}dbl\textgreater{}        \textless{}dbl\textgreater{}       \textless{}dbl\textgreater{} \textless{}chr\textgreater{} }
\CommentTok{\#\textgreater{} 1 setosa              4.3         2.3          1           0.1 min   }
\CommentTok{\#\textgreater{} 2 setosa              4.8         3.2          1.4         0.2 Q1    }
\CommentTok{\#\textgreater{} 3 setosa              5           3.4          1.5         0.2 median}
\CommentTok{\#\textgreater{} 4 setosa              5.2         3.7          1.6         0.3 Q3    }
\CommentTok{\#\textgreater{} 5 setosa              5.8         4.4          1.9         0.6 max   }
\CommentTok{\#\textgreater{} 6 versicolor          4.9         2            3           1   min   }
\CommentTok{\#\textgreater{} \# ... with 9 more rows}
\end{Highlighting}
\end{Shaded}

\begin{Shaded}
\begin{Highlighting}[]
\CommentTok{\# group\_walk}
\FunctionTok{dir.create}\NormalTok{(temp }\OtherTok{\textless{}{-}} \FunctionTok{tempfile}\NormalTok{())}
\NormalTok{iris }\SpecialCharTok{\%\textgreater{}\%}
  \FunctionTok{group\_by}\NormalTok{(Species) }\SpecialCharTok{\%\textgreater{}\%}
  \FunctionTok{group\_walk}\NormalTok{(}\SpecialCharTok{\textasciitilde{}} \FunctionTok{write.csv}\NormalTok{(.x, }\AttributeTok{file =} \FunctionTok{file.path}\NormalTok{(temp, }\FunctionTok{paste0}\NormalTok{(.y}\SpecialCharTok{$}\NormalTok{Species, }\StringTok{".csv"}\NormalTok{))))}
\FunctionTok{list.files}\NormalTok{(temp, }\AttributeTok{pattern =} \StringTok{"csv$"}\NormalTok{)}
\FunctionTok{unlink}\NormalTok{(temp, }\AttributeTok{recursive =} \ConstantTok{TRUE}\NormalTok{)}
\end{Highlighting}
\end{Shaded}

\begin{itemize}
\tightlist
\item
  group\_cols()
\end{itemize}

选择分组变量

\begin{Shaded}
\begin{Highlighting}[]
\NormalTok{gdf }\OtherTok{\textless{}{-}}\NormalTok{ iris }\SpecialCharTok{\%\textgreater{}\%} \FunctionTok{group\_by}\NormalTok{(Species)}
\NormalTok{gdf }\SpecialCharTok{\%\textgreater{}\%} \FunctionTok{select}\NormalTok{(}\FunctionTok{group\_cols}\NormalTok{())}
\CommentTok{\#\textgreater{} \# A tibble: 150 x 1}
\CommentTok{\#\textgreater{} \# Groups:   Species [3]}
\CommentTok{\#\textgreater{}   Species}
\CommentTok{\#\textgreater{}   \textless{}fct\textgreater{}  }
\CommentTok{\#\textgreater{} 1 setosa }
\CommentTok{\#\textgreater{} 2 setosa }
\CommentTok{\#\textgreater{} 3 setosa }
\CommentTok{\#\textgreater{} 4 setosa }
\CommentTok{\#\textgreater{} 5 setosa }
\CommentTok{\#\textgreater{} 6 setosa }
\CommentTok{\#\textgreater{} \# ... with 144 more rows}
\end{Highlighting}
\end{Shaded}

\hypertarget{ux5176ux5b83ux51fdux6570}{%
\subsection{其它函数}\label{ux5176ux5b83ux51fdux6570}}

\begin{itemize}
\item
  between
\item
  cummean cumsum cumall cumany
\end{itemize}

累计系列函数

\begin{Shaded}
\begin{Highlighting}[]
\NormalTok{x }\OtherTok{\textless{}{-}} \FunctionTok{c}\NormalTok{(}\DecValTok{1}\NormalTok{, }\DecValTok{3}\NormalTok{, }\DecValTok{5}\NormalTok{, }\DecValTok{2}\NormalTok{, }\DecValTok{2}\NormalTok{)}
\FunctionTok{cummean}\NormalTok{(x)}
\CommentTok{\#\textgreater{} [1] 1.00 2.00 3.00 2.75 2.60}
\FunctionTok{cumsum}\NormalTok{(x) }\SpecialCharTok{/} \FunctionTok{seq\_along}\NormalTok{(x)}
\CommentTok{\#\textgreater{} [1] 1.00 2.00 3.00 2.75 2.60}

\FunctionTok{cumall}\NormalTok{(x }\SpecialCharTok{\textless{}} \DecValTok{5}\NormalTok{)}
\CommentTok{\#\textgreater{} [1]  TRUE  TRUE FALSE FALSE FALSE}
\FunctionTok{cumany}\NormalTok{(x }\SpecialCharTok{==} \DecValTok{3}\NormalTok{)}
\CommentTok{\#\textgreater{} [1] FALSE  TRUE  TRUE  TRUE  TRUE}
\end{Highlighting}
\end{Shaded}

\begin{itemize}
\tightlist
\item
  distinct
\end{itemize}

\begin{Shaded}
\begin{Highlighting}[]
\NormalTok{df }\OtherTok{\textless{}{-}} \FunctionTok{tibble}\NormalTok{(}
  \AttributeTok{x =} \FunctionTok{sample}\NormalTok{(}\DecValTok{10}\NormalTok{, }\DecValTok{100}\NormalTok{, }\AttributeTok{rep =} \ConstantTok{TRUE}\NormalTok{),}
  \AttributeTok{y =} \FunctionTok{sample}\NormalTok{(}\DecValTok{10}\NormalTok{, }\DecValTok{100}\NormalTok{, }\AttributeTok{rep =} \ConstantTok{TRUE}\NormalTok{)}
\NormalTok{)}

\FunctionTok{distinct}\NormalTok{(df, x)}
\FunctionTok{distinct}\NormalTok{(df, x, }\AttributeTok{.keep\_all =} \ConstantTok{TRUE}\NormalTok{)}
\FunctionTok{distinct}\NormalTok{(df, }\AttributeTok{diff =} \FunctionTok{abs}\NormalTok{(x }\SpecialCharTok{{-}}\NormalTok{ y))}
\end{Highlighting}
\end{Shaded}

\hypertarget{dplyr-programming}{%
\section{\texorpdfstring{\texttt{dplyr}编程}{dplyr编程}}\label{dplyr-programming}}

\href{https://cloud.r-project.org/web/packages/dplyr/vignettes/programming.html}{Programming with dplyr}

本节概念性东西较多且复杂不易理解,先尝试会使用,概念再慢慢消化理解。虽然复杂,但是比较实用,尤其是当我们需要定义一些通用功能函数时。

以下是对原文引用

两种情况:

\begin{itemize}
\tightlist
\item
  When you have the data-variable in a function argument (i.e.~an env-variable that holds a promise2), you need to ** embrace ** the argument by surrounding it in doubled braces, like \texttt{filter(df,\ \{\{\ var\ \}\})}.
\end{itemize}

The following function uses embracing to create a wrapper around \texttt{summarise()} that computes the minimum and maximum values of a variable, as well as the number of observations that were summarised:

\begin{Shaded}
\begin{Highlighting}[]
\NormalTok{var\_summary }\OtherTok{\textless{}{-}} \ControlFlowTok{function}\NormalTok{(data, var) \{}
\NormalTok{  data }\SpecialCharTok{\%\textgreater{}\%}
    \FunctionTok{summarise}\NormalTok{(}\AttributeTok{n =} \FunctionTok{n}\NormalTok{(), }\AttributeTok{min =} \FunctionTok{min}\NormalTok{(\{\{ var \}\}), }\AttributeTok{max =} \FunctionTok{max}\NormalTok{(\{\{ var \}\}))}
\NormalTok{\}}
\NormalTok{mtcars }\SpecialCharTok{\%\textgreater{}\%} 
  \FunctionTok{group\_by}\NormalTok{(cyl) }\SpecialCharTok{\%\textgreater{}\%} 
  \FunctionTok{var\_summary}\NormalTok{(mpg)}
\end{Highlighting}
\end{Shaded}

\begin{itemize}
\tightlist
\item
  When you have an env-variable that is a character vector, you need to index into the .data pronoun with {[}{[}, like summarise(df, mean = mean(.data{[}{[}var{]}{]})).
\end{itemize}

The following example uses .data to count the number of unique values in each variable of mtcars:

\begin{Shaded}
\begin{Highlighting}[]
\ControlFlowTok{for}\NormalTok{ (var }\ControlFlowTok{in} \FunctionTok{names}\NormalTok{(mtcars)) \{}
\NormalTok{  mtcars }\SpecialCharTok{\%\textgreater{}\%} \FunctionTok{count}\NormalTok{(.data[[var]]) }\SpecialCharTok{\%\textgreater{}\%} \FunctionTok{print}\NormalTok{()}
\NormalTok{\}}
\end{Highlighting}
\end{Shaded}

Note that .data is not a data frame; it's a special construct, a pronoun, that allows you to access the current variables either directly, with \texttt{.data\$x} or indirectly with \texttt{.data{[}{[}var{]}{]}}. Don't expect other functions to work with it.

\hypertarget{ux6848ux4f8b}{%
\subsection{案例}\label{ux6848ux4f8b}}

当我们不知道接下来会用哪个变量汇总时:

\begin{Shaded}
\begin{Highlighting}[]
\NormalTok{my\_summarise }\OtherTok{\textless{}{-}} \ControlFlowTok{function}\NormalTok{(data, group\_var) \{}
\NormalTok{  data }\SpecialCharTok{\%\textgreater{}\%}
    \FunctionTok{group\_by}\NormalTok{(\{\{ group\_var \}\}) }\SpecialCharTok{\%\textgreater{}\%}
    \FunctionTok{summarise}\NormalTok{(}\AttributeTok{mean =} \FunctionTok{mean}\NormalTok{(mass))}
\NormalTok{\}}
\end{Highlighting}
\end{Shaded}

如果在多个位置使用:

\begin{Shaded}
\begin{Highlighting}[]
\NormalTok{my\_summarise2 }\OtherTok{\textless{}{-}} \ControlFlowTok{function}\NormalTok{(data, expr) \{}
\NormalTok{  data }\SpecialCharTok{\%\textgreater{}\%} \FunctionTok{summarise}\NormalTok{(}
    \AttributeTok{mean =} \FunctionTok{mean}\NormalTok{(\{\{ expr \}\}),}
    \AttributeTok{sum =} \FunctionTok{sum}\NormalTok{(\{\{ expr \}\}),}
    \AttributeTok{n =} \FunctionTok{n}\NormalTok{()}
\NormalTok{  )}
\NormalTok{\}}
\end{Highlighting}
\end{Shaded}

当多个表达式时:

\begin{Shaded}
\begin{Highlighting}[]
\NormalTok{my\_summarise3 }\OtherTok{\textless{}{-}} \ControlFlowTok{function}\NormalTok{(data, mean\_var, sd\_var) \{}
\NormalTok{  data }\SpecialCharTok{\%\textgreater{}\%} 
    \FunctionTok{summarise}\NormalTok{(}\AttributeTok{mean =} \FunctionTok{mean}\NormalTok{(\{\{ mean\_var \}\}), }\AttributeTok{sd =} \FunctionTok{mean}\NormalTok{(\{\{ sd\_var \}\}))}
\NormalTok{\}}
\end{Highlighting}
\end{Shaded}

如果要输出变量名时:

\begin{Shaded}
\begin{Highlighting}[]
\NormalTok{my\_summarise4 }\OtherTok{\textless{}{-}} \ControlFlowTok{function}\NormalTok{(data, expr) \{}
\NormalTok{  data }\SpecialCharTok{\%\textgreater{}\%} \FunctionTok{summarise}\NormalTok{(}
    \StringTok{"mean\_\{\{expr\}\}"} \SpecialCharTok{:}\ErrorTok{=} \FunctionTok{mean}\NormalTok{(\{\{ expr \}\}),}
    \StringTok{"sum\_\{\{expr\}\}"} \SpecialCharTok{:}\ErrorTok{=} \FunctionTok{sum}\NormalTok{(\{\{ expr \}\}),}
    \StringTok{"n\_\{\{expr\}\}"} \SpecialCharTok{:}\ErrorTok{=} \FunctionTok{n}\NormalTok{()}
\NormalTok{  )}
\NormalTok{\}}
\NormalTok{my\_summarise5 }\OtherTok{\textless{}{-}} \ControlFlowTok{function}\NormalTok{(data, mean\_var, sd\_var) \{}
\NormalTok{  data }\SpecialCharTok{\%\textgreater{}\%} 
    \FunctionTok{summarise}\NormalTok{(}
      \StringTok{"mean\_\{\{mean\_var\}\}"} \SpecialCharTok{:}\ErrorTok{=} \FunctionTok{mean}\NormalTok{(\{\{ mean\_var \}\}), }
      \StringTok{"sd\_\{\{sd\_var\}\}"} \SpecialCharTok{:}\ErrorTok{=} \FunctionTok{mean}\NormalTok{(\{\{ sd\_var \}\})}
\NormalTok{    )}
\NormalTok{\}}
\end{Highlighting}
\end{Shaded}

任意个表达式:

这种使用场景更多

\begin{Shaded}
\begin{Highlighting}[]
\NormalTok{my\_summarise }\OtherTok{\textless{}{-}} \ControlFlowTok{function}\NormalTok{(.data, ...) \{}
\NormalTok{  .data }\SpecialCharTok{\%\textgreater{}\%}
    \FunctionTok{group\_by}\NormalTok{(...) }\SpecialCharTok{\%\textgreater{}\%}
    \FunctionTok{summarise}\NormalTok{(}\AttributeTok{mass =} \FunctionTok{mean}\NormalTok{(mass, }\AttributeTok{na.rm =} \ConstantTok{TRUE}\NormalTok{), }\AttributeTok{height =} \FunctionTok{mean}\NormalTok{(height, }\AttributeTok{na.rm =} \ConstantTok{TRUE}\NormalTok{))}
\NormalTok{\}}
\NormalTok{starwars }\SpecialCharTok{\%\textgreater{}\%} \FunctionTok{my\_summarise}\NormalTok{(homeworld)}
\CommentTok{\#\textgreater{} \# A tibble: 49 x 3}
\CommentTok{\#\textgreater{}   homeworld       mass height}
\CommentTok{\#\textgreater{}   \textless{}chr\textgreater{}          \textless{}dbl\textgreater{}  \textless{}dbl\textgreater{}}
\CommentTok{\#\textgreater{} 1 Alderaan          64   176.}
\CommentTok{\#\textgreater{} 2 Aleen Minor       15    79 }
\CommentTok{\#\textgreater{} 3 Bespin            79   175 }
\CommentTok{\#\textgreater{} 4 Bestine IV       110   180 }
\CommentTok{\#\textgreater{} 5 Cato Neimoidia    90   191 }
\CommentTok{\#\textgreater{} 6 Cerea             82   198 }
\CommentTok{\#\textgreater{} \# ... with 43 more rows}
\NormalTok{starwars }\SpecialCharTok{\%\textgreater{}\%} \FunctionTok{my\_summarise}\NormalTok{(sex, gender)}
\CommentTok{\#\textgreater{} \textasciigrave{}summarise()\textasciigrave{} has grouped output by \textquotesingle{}sex\textquotesingle{}. You can override using the \textasciigrave{}.groups\textasciigrave{} argument.}
\CommentTok{\#\textgreater{} \# A tibble: 6 x 4}
\CommentTok{\#\textgreater{} \# Groups:   sex [5]}
\CommentTok{\#\textgreater{}   sex            gender      mass height}
\CommentTok{\#\textgreater{}   \textless{}chr\textgreater{}          \textless{}chr\textgreater{}      \textless{}dbl\textgreater{}  \textless{}dbl\textgreater{}}
\CommentTok{\#\textgreater{} 1 female         feminine    54.7   169.}
\CommentTok{\#\textgreater{} 2 hermaphroditic masculine 1358     175 }
\CommentTok{\#\textgreater{} 3 male           masculine   81.0   179.}
\CommentTok{\#\textgreater{} 4 none           feminine   NaN      96 }
\CommentTok{\#\textgreater{} 5 none           masculine   69.8   140 }
\CommentTok{\#\textgreater{} 6 \textless{}NA\textgreater{}           \textless{}NA\textgreater{}        48     181.}
\end{Highlighting}
\end{Shaded}

\hypertarget{ux53c2ux8003ux8d44ux6599}{%
\section{参考资料}\label{ux53c2ux8003ux8d44ux6599}}

1.programming \url{https://dplyr.tidyverse.org/dev/articles/programming.html}
2.https://cloud.r-project.org/web/packages/dplyr/vignettes/programming.html

\hypertarget{tidyr}{%
\chapter{tidyr}\label{tidyr}}

在实际工作中,我们数据分析工作者80\%的时间可能贡献在数据准备和数据清晰上。另外发现新问题时,可能又要重复数据准备、数据清晰的过程。如果采用不能完全复现的方式做数据准备清洗的工作,那将是一场灾难。

数据工作者最常用的工具可能是Excel,但是Excel并不具备很强的数据清洗能力,即使Excel有POwer query 、Dax等两大利器。工作中,实际面临原始的数据是脏乱无须的,业务系统仅仅只是记录了历史过程数据。当我们需要分析某一现象时,需要按照自己的需求重新采集数据,清洗为``标准''的数据格式。

\begin{quote}
标准数据:达到工作需求的数据,可以直接用Excel,power bi ,tableau等BI工具直接使用的程度。
\end{quote}

\texttt{R}中的tidyverse系列构建了一种一致的数据结构,当我们用tidyverse软件包提供的``数据整洁工具''整洁数据时,我们将花费更少的时间将数据从一种形式迁移到另外一种形式。从而,我们拥有更多的时间专注在具体的业务问题上。

\hypertarget{ux5b89ux88c5-1}{%
\section{安装}\label{ux5b89ux88c5-1}}

本章节,我们重点关注\texttt{tidyr}包,这个软件包提供了许多的功能函数整理混乱的数据。tidyr是tidyverse的核心成员包

\begin{Shaded}
\begin{Highlighting}[]
\DocumentationTok{\#\# 最简单是的方式就是安装tidyverse}
\FunctionTok{install.packages}\NormalTok{(}\StringTok{\textquotesingle{}tidyverse\textquotesingle{}}\NormalTok{)}

\DocumentationTok{\#\# 或者仅仅安装 tidyr:}
\FunctionTok{install.packages}\NormalTok{(}\StringTok{\textquotesingle{}tidyr\textquotesingle{}}\NormalTok{)}

\DocumentationTok{\#\# 或者从github 安装开发版本}
\DocumentationTok{\#\# install.packages("devtools")}
\NormalTok{devtools}\SpecialCharTok{::}\FunctionTok{install\_github}\NormalTok{(}\StringTok{"tidyverse/tidyr"}\NormalTok{)}

\CommentTok{\# CTEST CODE}
\end{Highlighting}
\end{Shaded}

\hypertarget{ux4e3bux8981ux529fux80fd}{%
\section{主要功能}\label{ux4e3bux8981ux529fux80fd}}

整洁的数据表现为:

\begin{enumerate}
\def\labelenumi{\arabic{enumi}.}
\tightlist
\item
  每个变量是单独的一列
\item
  每一个观察的值都在自己的行
\item
  每一个值都是独立的单元格
\end{enumerate}

大部分的数据集都是用行和列构成的\texttt{data.frame}。用Excel的单元格来表示,即每列代表不同意义的字段,每行是某个情形下的一系列字段;单元格则是独立的值,属于某个变量的观察值,这样构建的二维数据结构则是``整洁数据''。

\begin{Shaded}
\begin{Highlighting}[]
\FunctionTok{library}\NormalTok{(tidyr)}
\end{Highlighting}
\end{Shaded}

\texttt{tidyr}包中的函数可以分为5个主要大类

\begin{itemize}
\item
  \texttt{pivot\_longer()} 和 \texttt{pivot\_wider()} 宽转长以及长转宽
\item
  \texttt{unnest\_longer()} 和 \texttt{unnest\_wider()},\texttt{hoist()} 将列表嵌套转化为整洁数据
\item
  \texttt{nest()} 数据嵌套
\item
  \texttt{separate()},\texttt{extract()}拆分列,提取新列
\item
  \texttt{replace\_na()} 缺失值处理
\end{itemize}

\hypertarget{ux5bbdux8f6cux957f}{%
\subsection{宽转长}\label{ux5bbdux8f6cux957f}}

详情查看\texttt{vignette("pivot")},以下是照搬该图册中的内容

\hypertarget{ux57faux7840-1}{%
\subsubsection{基础}\label{ux57faux7840-1}}

长数据与宽数据之间的转换,类似我们常用的EXcel中的透视表功能。接下来用\texttt{tidyr}包自带的插图案例记录相关函数用法

在Excel中有时候方便我们肉眼观察,可能一个数据集会有很多列,如下所示:

\begin{longtable}[]{@{}lllllll@{}}
\toprule
col1 & col2 & col3 & col4 & col5 & col6 & col7 \\
\midrule
\endhead
v1 & v2 & v3 & v4 & v5 & v6 & v7 \\
vb1 & vb2 & vb3 & vb4 & vb5 & vb6 & vb7 \\
\bottomrule
\end{longtable}

方便观察,但是不方便统计分析,这是我们需要把数据做处理,从``宽数据变成长数据''即宽转长。

\begin{Shaded}
\begin{Highlighting}[]
\FunctionTok{library}\NormalTok{(tidyr)}
\FunctionTok{library}\NormalTok{(dplyr)}
\FunctionTok{library}\NormalTok{(readr)}
\end{Highlighting}
\end{Shaded}

\begin{Shaded}
\begin{Highlighting}[]
\NormalTok{relig\_income }\SpecialCharTok{\%\textgreater{}\%} 
  \FunctionTok{pivot\_longer}\NormalTok{(}\AttributeTok{cols =} \SpecialCharTok{!}\NormalTok{religion,}\AttributeTok{names\_to =} \StringTok{\textquotesingle{}income\textquotesingle{}}\NormalTok{,}\AttributeTok{values\_to =} \StringTok{"count"}\NormalTok{)}
\CommentTok{\#\textgreater{} \# A tibble: 180 x 3}
\CommentTok{\#\textgreater{}   religion income  count}
\CommentTok{\#\textgreater{}   \textless{}chr\textgreater{}    \textless{}chr\textgreater{}   \textless{}dbl\textgreater{}}
\CommentTok{\#\textgreater{} 1 Agnostic \textless{}$10k      27}
\CommentTok{\#\textgreater{} 2 Agnostic $10{-}20k    34}
\CommentTok{\#\textgreater{} 3 Agnostic $20{-}30k    60}
\CommentTok{\#\textgreater{} 4 Agnostic $30{-}40k    81}
\CommentTok{\#\textgreater{} 5 Agnostic $40{-}50k    76}
\CommentTok{\#\textgreater{} 6 Agnostic $50{-}75k   137}
\CommentTok{\#\textgreater{} \# ... with 174 more rows}
\end{Highlighting}
\end{Shaded}

\begin{itemize}
\tightlist
\item
  第一个参数是数据集
\item
  第二个参数是那些列需要重塑,在该例中除了\texttt{religion}的其他全部列
\item
  \texttt{names\_to}这个参数是新增的列名
\item
  \texttt{values\_to}是新增的存储之前数据集中数据的列名
\end{itemize}

\hypertarget{ux5217ux540dux5e26ux6570ux5b57}{%
\subsubsection{列名带数字}\label{ux5217ux540dux5e26ux6570ux5b57}}

\begin{Shaded}
\begin{Highlighting}[]
\NormalTok{billboard }\SpecialCharTok{\%\textgreater{}\%} 
  \FunctionTok{pivot\_longer}\NormalTok{(}
    \AttributeTok{cols =} \FunctionTok{starts\_with}\NormalTok{(}\StringTok{"wk"}\NormalTok{), }
    \AttributeTok{names\_to =} \StringTok{"week"}\NormalTok{, }
    \AttributeTok{values\_to =} \StringTok{"rank"}\NormalTok{,}
    \AttributeTok{values\_drop\_na =} \ConstantTok{TRUE}
\NormalTok{  )}
\CommentTok{\#\textgreater{} \# A tibble: 5,307 x 5}
\CommentTok{\#\textgreater{}   artist track                   date.entered week   rank}
\CommentTok{\#\textgreater{}   \textless{}chr\textgreater{}  \textless{}chr\textgreater{}                   \textless{}date\textgreater{}       \textless{}chr\textgreater{} \textless{}dbl\textgreater{}}
\CommentTok{\#\textgreater{} 1 2 Pac  Baby Don\textquotesingle{}t Cry (Keep... 2000{-}02{-}26   wk1      87}
\CommentTok{\#\textgreater{} 2 2 Pac  Baby Don\textquotesingle{}t Cry (Keep... 2000{-}02{-}26   wk2      82}
\CommentTok{\#\textgreater{} 3 2 Pac  Baby Don\textquotesingle{}t Cry (Keep... 2000{-}02{-}26   wk3      72}
\CommentTok{\#\textgreater{} 4 2 Pac  Baby Don\textquotesingle{}t Cry (Keep... 2000{-}02{-}26   wk4      77}
\CommentTok{\#\textgreater{} 5 2 Pac  Baby Don\textquotesingle{}t Cry (Keep... 2000{-}02{-}26   wk5      87}
\CommentTok{\#\textgreater{} 6 2 Pac  Baby Don\textquotesingle{}t Cry (Keep... 2000{-}02{-}26   wk6      94}
\CommentTok{\#\textgreater{} \# ... with 5,301 more rows}
\end{Highlighting}
\end{Shaded}

\texttt{names\_prefix} 调整内容前缀,配合\texttt{names\_transform}参数使用

\begin{Shaded}
\begin{Highlighting}[]
\NormalTok{billboard }\SpecialCharTok{\%\textgreater{}\%} 
  \FunctionTok{pivot\_longer}\NormalTok{(}
    \AttributeTok{cols =} \FunctionTok{starts\_with}\NormalTok{(}\StringTok{"wk"}\NormalTok{), }
    \AttributeTok{names\_to =} \StringTok{"week"}\NormalTok{, }
    \AttributeTok{names\_prefix =} \StringTok{"wk"}\NormalTok{,}
    \AttributeTok{names\_transform =} \FunctionTok{list}\NormalTok{(}\AttributeTok{week =}\NormalTok{ as.integer),}
    \AttributeTok{values\_to =} \StringTok{"rank"}\NormalTok{,}
    \AttributeTok{values\_drop\_na =} \ConstantTok{TRUE}\NormalTok{,}
\NormalTok{  )}
\CommentTok{\#\textgreater{} \# A tibble: 5,307 x 5}
\CommentTok{\#\textgreater{}   artist track                   date.entered  week  rank}
\CommentTok{\#\textgreater{}   \textless{}chr\textgreater{}  \textless{}chr\textgreater{}                   \textless{}date\textgreater{}       \textless{}int\textgreater{} \textless{}dbl\textgreater{}}
\CommentTok{\#\textgreater{} 1 2 Pac  Baby Don\textquotesingle{}t Cry (Keep... 2000{-}02{-}26       1    87}
\CommentTok{\#\textgreater{} 2 2 Pac  Baby Don\textquotesingle{}t Cry (Keep... 2000{-}02{-}26       2    82}
\CommentTok{\#\textgreater{} 3 2 Pac  Baby Don\textquotesingle{}t Cry (Keep... 2000{-}02{-}26       3    72}
\CommentTok{\#\textgreater{} 4 2 Pac  Baby Don\textquotesingle{}t Cry (Keep... 2000{-}02{-}26       4    77}
\CommentTok{\#\textgreater{} 5 2 Pac  Baby Don\textquotesingle{}t Cry (Keep... 2000{-}02{-}26       5    87}
\CommentTok{\#\textgreater{} 6 2 Pac  Baby Don\textquotesingle{}t Cry (Keep... 2000{-}02{-}26       6    94}
\CommentTok{\#\textgreater{} \# ... with 5,301 more rows}
\end{Highlighting}
\end{Shaded}

经过以上转换\texttt{week}列属性变成了整数,当然达到以上效果有其他的途径,如下:

\begin{Shaded}
\begin{Highlighting}[]
\FunctionTok{library}\NormalTok{(tidyverse,}\AttributeTok{warn.conflicts =} \ConstantTok{TRUE}\NormalTok{)}

\CommentTok{\# method 1}
\NormalTok{billboard }\SpecialCharTok{\%\textgreater{}\%} 
  \FunctionTok{pivot\_longer}\NormalTok{(}
    \AttributeTok{cols =} \FunctionTok{starts\_with}\NormalTok{(}\StringTok{"wk"}\NormalTok{), }
    \AttributeTok{names\_to =} \StringTok{"week"}\NormalTok{, }
    \AttributeTok{names\_transform =} \FunctionTok{list}\NormalTok{(}\AttributeTok{week =}\NormalTok{ readr}\SpecialCharTok{::}\NormalTok{parse\_number),}
    \AttributeTok{values\_to =} \StringTok{"rank"}\NormalTok{,}
    \AttributeTok{values\_drop\_na =} \ConstantTok{TRUE}\NormalTok{,}
\NormalTok{)}

\CommentTok{\# method 2}
\NormalTok{billboard }\SpecialCharTok{\%\textgreater{}\%}
  \FunctionTok{pivot\_longer}\NormalTok{(}
    \AttributeTok{cols =} \FunctionTok{starts\_with}\NormalTok{(}\StringTok{"wk"}\NormalTok{),}
    \AttributeTok{names\_to =} \StringTok{"week"}\NormalTok{,}
    \AttributeTok{values\_to =} \StringTok{"rank"}\NormalTok{,}
    \AttributeTok{values\_drop\_na =} \ConstantTok{TRUE}\NormalTok{,}
\NormalTok{  ) }\SpecialCharTok{\%\textgreater{}\%}
  \FunctionTok{mutate}\NormalTok{(}\AttributeTok{week =} \FunctionTok{str\_remove}\NormalTok{(week, }\StringTok{"wk"}\NormalTok{) }\SpecialCharTok{\%\textgreater{}\%} \FunctionTok{as.integer}\NormalTok{())}
\end{Highlighting}
\end{Shaded}

\hypertarget{ux591aux53d8ux91cfux5217ux540d}{%
\subsubsection{多变量列名}\label{ux591aux53d8ux91cfux5217ux540d}}

该案列设计比较复杂的正则表达式,\texttt{new\_?(.*)\_(.)(.*)}需要一定正则表达式基础。
\texttt{new\_?}表示匹配\texttt{new}或\texttt{new\_},\texttt{(.*)}匹配任意0次或多次任意字符。

\href{https://www.runoob.com/regexp/regexp-syntax.html}{正则表达式介绍}

\begin{Shaded}
\begin{Highlighting}[]
\NormalTok{who }\SpecialCharTok{\%\textgreater{}\%} \FunctionTok{pivot\_longer}\NormalTok{(}
  \AttributeTok{cols =}\NormalTok{ new\_sp\_m014}\SpecialCharTok{:}\NormalTok{newrel\_f65,}
  \AttributeTok{names\_to =} \FunctionTok{c}\NormalTok{(}\StringTok{"diagnosis"}\NormalTok{, }\StringTok{"gender"}\NormalTok{, }\StringTok{"age"}\NormalTok{), }
  \AttributeTok{names\_pattern =} \StringTok{"new\_?(.*)\_(.)(.*)"}\NormalTok{,}
  \AttributeTok{values\_to =} \StringTok{"count"}
\NormalTok{)}
\CommentTok{\#\textgreater{} \# A tibble: 405,440 x 8}
\CommentTok{\#\textgreater{}   country     iso2  iso3   year diagnosis gender age   count}
\CommentTok{\#\textgreater{}   \textless{}chr\textgreater{}       \textless{}chr\textgreater{} \textless{}chr\textgreater{} \textless{}int\textgreater{} \textless{}chr\textgreater{}     \textless{}chr\textgreater{}  \textless{}chr\textgreater{} \textless{}int\textgreater{}}
\CommentTok{\#\textgreater{} 1 Afghanistan AF    AFG    1980 sp        m      014      NA}
\CommentTok{\#\textgreater{} 2 Afghanistan AF    AFG    1980 sp        m      1524     NA}
\CommentTok{\#\textgreater{} 3 Afghanistan AF    AFG    1980 sp        m      2534     NA}
\CommentTok{\#\textgreater{} 4 Afghanistan AF    AFG    1980 sp        m      3544     NA}
\CommentTok{\#\textgreater{} 5 Afghanistan AF    AFG    1980 sp        m      4554     NA}
\CommentTok{\#\textgreater{} 6 Afghanistan AF    AFG    1980 sp        m      5564     NA}
\CommentTok{\#\textgreater{} \# ... with 405,434 more rows}
\end{Highlighting}
\end{Shaded}

进一步处理列\texttt{gender},\texttt{age} 。

\begin{Shaded}
\begin{Highlighting}[]
\NormalTok{who }\SpecialCharTok{\%\textgreater{}\%} \FunctionTok{pivot\_longer}\NormalTok{(}
  \AttributeTok{cols =}\NormalTok{ new\_sp\_m014}\SpecialCharTok{:}\NormalTok{newrel\_f65,}
  \AttributeTok{names\_to =} \FunctionTok{c}\NormalTok{(}\StringTok{"diagnosis"}\NormalTok{, }\StringTok{"gender"}\NormalTok{, }\StringTok{"age"}\NormalTok{), }
  \AttributeTok{names\_pattern =} \StringTok{"new\_?(.*)\_(.)(.*)"}\NormalTok{,}
  \AttributeTok{names\_transform =} \FunctionTok{list}\NormalTok{(}
    \AttributeTok{gender =} \SpecialCharTok{\textasciitilde{}}\NormalTok{ readr}\SpecialCharTok{::}\FunctionTok{parse\_factor}\NormalTok{(.x, }\AttributeTok{levels =} \FunctionTok{c}\NormalTok{(}\StringTok{"f"}\NormalTok{, }\StringTok{"m"}\NormalTok{)),}
    \AttributeTok{age =} \SpecialCharTok{\textasciitilde{}}\NormalTok{ readr}\SpecialCharTok{::}\FunctionTok{parse\_factor}\NormalTok{(}
\NormalTok{      .x,}
      \AttributeTok{levels =} \FunctionTok{c}\NormalTok{(}\StringTok{"014"}\NormalTok{, }\StringTok{"1524"}\NormalTok{, }\StringTok{"2534"}\NormalTok{, }\StringTok{"3544"}\NormalTok{, }\StringTok{"4554"}\NormalTok{, }\StringTok{"5564"}\NormalTok{, }\StringTok{"65"}\NormalTok{), }
      \AttributeTok{ordered =} \ConstantTok{TRUE}
\NormalTok{    )}
\NormalTok{  ),}
  \AttributeTok{values\_to =} \StringTok{"count"}\NormalTok{,}
\NormalTok{)}
\CommentTok{\#\textgreater{} \# A tibble: 405,440 x 8}
\CommentTok{\#\textgreater{}   country     iso2  iso3   year diagnosis gender age   count}
\CommentTok{\#\textgreater{}   \textless{}chr\textgreater{}       \textless{}chr\textgreater{} \textless{}chr\textgreater{} \textless{}int\textgreater{} \textless{}chr\textgreater{}     \textless{}fct\textgreater{}  \textless{}ord\textgreater{} \textless{}int\textgreater{}}
\CommentTok{\#\textgreater{} 1 Afghanistan AF    AFG    1980 sp        m      014      NA}
\CommentTok{\#\textgreater{} 2 Afghanistan AF    AFG    1980 sp        m      1524     NA}
\CommentTok{\#\textgreater{} 3 Afghanistan AF    AFG    1980 sp        m      2534     NA}
\CommentTok{\#\textgreater{} 4 Afghanistan AF    AFG    1980 sp        m      3544     NA}
\CommentTok{\#\textgreater{} 5 Afghanistan AF    AFG    1980 sp        m      4554     NA}
\CommentTok{\#\textgreater{} 6 Afghanistan AF    AFG    1980 sp        m      5564     NA}
\CommentTok{\#\textgreater{} \# ... with 405,434 more rows}
\end{Highlighting}
\end{Shaded}

\hypertarget{ux4e00ux884cux591aux89c2ux6d4bux503c}{%
\subsubsection{一行多观测值}\label{ux4e00ux884cux591aux89c2ux6d4bux503c}}

\begin{Shaded}
\begin{Highlighting}[]
\NormalTok{family }\OtherTok{\textless{}{-}} \FunctionTok{tribble}\NormalTok{(}
  \SpecialCharTok{\textasciitilde{}}\NormalTok{family, }\SpecialCharTok{\textasciitilde{}}\NormalTok{dob\_child1, }\SpecialCharTok{\textasciitilde{}}\NormalTok{dob\_child2, }\SpecialCharTok{\textasciitilde{}}\NormalTok{gender\_child1, }\SpecialCharTok{\textasciitilde{}}\NormalTok{gender\_child2,}
\NormalTok{  1L, }\StringTok{"1998{-}11{-}26"}\NormalTok{, }\StringTok{"2000{-}01{-}29"}\NormalTok{, 1L, 2L,}
\NormalTok{  2L, }\StringTok{"1996{-}06{-}22"}\NormalTok{, }\ConstantTok{NA}\NormalTok{, 2L, }\ConstantTok{NA}\NormalTok{,}
\NormalTok{  3L, }\StringTok{"2002{-}07{-}11"}\NormalTok{, }\StringTok{"2004{-}04{-}05"}\NormalTok{, 2L, 2L,}
\NormalTok{  4L, }\StringTok{"2004{-}10{-}10"}\NormalTok{, }\StringTok{"2009{-}08{-}27"}\NormalTok{, 1L, 1L,}
\NormalTok{  5L, }\StringTok{"2000{-}12{-}05"}\NormalTok{, }\StringTok{"2005{-}02{-}28"}\NormalTok{, 2L, 1L,}
\NormalTok{)}
\NormalTok{family }\OtherTok{\textless{}{-}}\NormalTok{ family }\SpecialCharTok{\%\textgreater{}\%} \FunctionTok{mutate\_at}\NormalTok{(}\FunctionTok{vars}\NormalTok{(}\FunctionTok{starts\_with}\NormalTok{(}\StringTok{"dob"}\NormalTok{)), parse\_date)}
\NormalTok{family}
\CommentTok{\#\textgreater{} \# A tibble: 5 x 5}
\CommentTok{\#\textgreater{}   family dob\_child1 dob\_child2 gender\_child1 gender\_child2}
\CommentTok{\#\textgreater{}    \textless{}int\textgreater{} \textless{}date\textgreater{}     \textless{}date\textgreater{}             \textless{}int\textgreater{}         \textless{}int\textgreater{}}
\CommentTok{\#\textgreater{} 1      1 1998{-}11{-}26 2000{-}01{-}29             1             2}
\CommentTok{\#\textgreater{} 2      2 1996{-}06{-}22 NA                     2            NA}
\CommentTok{\#\textgreater{} 3      3 2002{-}07{-}11 2004{-}04{-}05             2             2}
\CommentTok{\#\textgreater{} 4      4 2004{-}10{-}10 2009{-}08{-}27             1             1}
\CommentTok{\#\textgreater{} 5      5 2000{-}12{-}05 2005{-}02{-}28             2             1}
\end{Highlighting}
\end{Shaded}

\begin{Shaded}
\begin{Highlighting}[]

\NormalTok{family }\SpecialCharTok{\%\textgreater{}\%} 
  \FunctionTok{pivot\_longer}\NormalTok{(}
    \SpecialCharTok{!}\NormalTok{family, }
    \AttributeTok{names\_to =} \FunctionTok{c}\NormalTok{(}\StringTok{".value"}\NormalTok{, }\StringTok{"child"}\NormalTok{), }
    \AttributeTok{names\_sep =} \StringTok{"\_"}\NormalTok{, }
    \AttributeTok{values\_drop\_na =} \ConstantTok{TRUE}
\NormalTok{  )}
\CommentTok{\#\textgreater{} \# A tibble: 9 x 4}
\CommentTok{\#\textgreater{}   family child  dob        gender}
\CommentTok{\#\textgreater{}    \textless{}int\textgreater{} \textless{}chr\textgreater{}  \textless{}date\textgreater{}      \textless{}int\textgreater{}}
\CommentTok{\#\textgreater{} 1      1 child1 1998{-}11{-}26      1}
\CommentTok{\#\textgreater{} 2      1 child2 2000{-}01{-}29      2}
\CommentTok{\#\textgreater{} 3      2 child1 1996{-}06{-}22      2}
\CommentTok{\#\textgreater{} 4      3 child1 2002{-}07{-}11      2}
\CommentTok{\#\textgreater{} 5      3 child2 2004{-}04{-}05      2}
\CommentTok{\#\textgreater{} 6      4 child1 2004{-}10{-}10      1}
\CommentTok{\#\textgreater{} \# ... with 3 more rows}
\end{Highlighting}
\end{Shaded}

\begin{Shaded}
\begin{Highlighting}[]
\NormalTok{anscombe }\SpecialCharTok{\%\textgreater{}\%} 
  \FunctionTok{pivot\_longer}\NormalTok{(}\FunctionTok{everything}\NormalTok{(), }
    \AttributeTok{names\_to =} \FunctionTok{c}\NormalTok{(}\StringTok{".value"}\NormalTok{, }\StringTok{"set"}\NormalTok{), }
    \AttributeTok{names\_pattern =} \StringTok{"(.)(.)"}
\NormalTok{  ) }\SpecialCharTok{\%\textgreater{}\%} 
  \FunctionTok{arrange}\NormalTok{(set)}
\CommentTok{\#\textgreater{} \# A tibble: 44 x 3}
\CommentTok{\#\textgreater{}   set       x     y}
\CommentTok{\#\textgreater{}   \textless{}chr\textgreater{} \textless{}dbl\textgreater{} \textless{}dbl\textgreater{}}
\CommentTok{\#\textgreater{} 1 1        10  8.04}
\CommentTok{\#\textgreater{} 2 1         8  6.95}
\CommentTok{\#\textgreater{} 3 1        13  7.58}
\CommentTok{\#\textgreater{} 4 1         9  8.81}
\CommentTok{\#\textgreater{} 5 1        11  8.33}
\CommentTok{\#\textgreater{} 6 1        14  9.96}
\CommentTok{\#\textgreater{} \# ... with 38 more rows}
\end{Highlighting}
\end{Shaded}

\begin{Shaded}
\begin{Highlighting}[]
\NormalTok{pnl }\OtherTok{\textless{}{-}} \FunctionTok{tibble}\NormalTok{(}
  \AttributeTok{x =} \DecValTok{1}\SpecialCharTok{:}\DecValTok{4}\NormalTok{,}
  \AttributeTok{a =} \FunctionTok{c}\NormalTok{(}\DecValTok{1}\NormalTok{, }\DecValTok{1}\NormalTok{,}\DecValTok{0}\NormalTok{, }\DecValTok{0}\NormalTok{),}
  \AttributeTok{b =} \FunctionTok{c}\NormalTok{(}\DecValTok{0}\NormalTok{, }\DecValTok{1}\NormalTok{, }\DecValTok{1}\NormalTok{, }\DecValTok{1}\NormalTok{),}
  \AttributeTok{y1 =} \FunctionTok{rnorm}\NormalTok{(}\DecValTok{4}\NormalTok{),}
  \AttributeTok{y2 =} \FunctionTok{rnorm}\NormalTok{(}\DecValTok{4}\NormalTok{),}
  \AttributeTok{z1 =} \FunctionTok{rep}\NormalTok{(}\DecValTok{3}\NormalTok{, }\DecValTok{4}\NormalTok{),}
  \AttributeTok{z2 =} \FunctionTok{rep}\NormalTok{(}\SpecialCharTok{{-}}\DecValTok{2}\NormalTok{, }\DecValTok{4}\NormalTok{),}
\NormalTok{)}

\NormalTok{pnl }\SpecialCharTok{\%\textgreater{}\%} 
  \FunctionTok{pivot\_longer}\NormalTok{(}
    \SpecialCharTok{!}\FunctionTok{c}\NormalTok{(x, a, b), }
    \AttributeTok{names\_to =} \FunctionTok{c}\NormalTok{(}\StringTok{".value"}\NormalTok{, }\StringTok{"time"}\NormalTok{), }
    \AttributeTok{names\_pattern =} \StringTok{"(.)(.)"}
\NormalTok{  )}
\CommentTok{\#\textgreater{} \# A tibble: 8 x 6}
\CommentTok{\#\textgreater{}       x     a     b time       y     z}
\CommentTok{\#\textgreater{}   \textless{}int\textgreater{} \textless{}dbl\textgreater{} \textless{}dbl\textgreater{} \textless{}chr\textgreater{}  \textless{}dbl\textgreater{} \textless{}dbl\textgreater{}}
\CommentTok{\#\textgreater{} 1     1     1     0 1     {-}1.40      3}
\CommentTok{\#\textgreater{} 2     1     1     0 2      0.622    {-}2}
\CommentTok{\#\textgreater{} 3     2     1     1 1      0.255     3}
\CommentTok{\#\textgreater{} 4     2     1     1 2      1.15     {-}2}
\CommentTok{\#\textgreater{} 5     3     0     1 1     {-}2.44      3}
\CommentTok{\#\textgreater{} 6     3     0     1 2     {-}1.82     {-}2}
\CommentTok{\#\textgreater{} \# ... with 2 more rows}
\end{Highlighting}
\end{Shaded}

\hypertarget{ux91cdux590dux5217ux540d}{%
\subsubsection{重复列名}\label{ux91cdux590dux5217ux540d}}

\begin{Shaded}
\begin{Highlighting}[]
\NormalTok{df }\OtherTok{\textless{}{-}} \FunctionTok{tibble}\NormalTok{(}\AttributeTok{id =} \DecValTok{1}\SpecialCharTok{:}\DecValTok{3}\NormalTok{, }\AttributeTok{y =} \DecValTok{4}\SpecialCharTok{:}\DecValTok{6}\NormalTok{, }\AttributeTok{y =} \DecValTok{5}\SpecialCharTok{:}\DecValTok{7}\NormalTok{, }\AttributeTok{y =} \DecValTok{7}\SpecialCharTok{:}\DecValTok{9}\NormalTok{, }\AttributeTok{.name\_repair =} \StringTok{"minimal"}\NormalTok{)}
\NormalTok{df }\SpecialCharTok{\%\textgreater{}\%} \FunctionTok{pivot\_longer}\NormalTok{(}\SpecialCharTok{!}\NormalTok{id, }\AttributeTok{names\_to =} \StringTok{"name"}\NormalTok{, }\AttributeTok{values\_to =} \StringTok{"value"}\NormalTok{)}
\CommentTok{\#\textgreater{} \# A tibble: 9 x 3}
\CommentTok{\#\textgreater{}      id name  value}
\CommentTok{\#\textgreater{}   \textless{}int\textgreater{} \textless{}chr\textgreater{} \textless{}int\textgreater{}}
\CommentTok{\#\textgreater{} 1     1 y         4}
\CommentTok{\#\textgreater{} 2     1 y         5}
\CommentTok{\#\textgreater{} 3     1 y         7}
\CommentTok{\#\textgreater{} 4     2 y         5}
\CommentTok{\#\textgreater{} 5     2 y         6}
\CommentTok{\#\textgreater{} 6     2 y         8}
\CommentTok{\#\textgreater{} \# ... with 3 more rows}
\end{Highlighting}
\end{Shaded}

\hypertarget{ux957fux8f6cux5bbd}{%
\subsection{长转宽}\label{ux957fux8f6cux5bbd}}

\texttt{pivot\_wider()}功能与\texttt{pivot\_longer()}相反。通过增加列数减少行数使数据集变得更宽,通常我们在汇总时候使用,达到类似Excel透视表结果。

\hypertarget{ux57faux7840-2}{%
\subsubsection{基础}\label{ux57faux7840-2}}

\begin{Shaded}
\begin{Highlighting}[]
\NormalTok{fish\_encounters }\SpecialCharTok{\%\textgreater{}\%} \FunctionTok{pivot\_wider}\NormalTok{(}\AttributeTok{names\_from =}\NormalTok{ station, }\AttributeTok{values\_from =}\NormalTok{ seen)}
\CommentTok{\#\textgreater{} \# A tibble: 19 x 12}
\CommentTok{\#\textgreater{}   fish  Release I80\_1 Lisbon  Rstr Base\_TD   BCE   BCW  BCE2  BCW2   MAE   MAW}
\CommentTok{\#\textgreater{}   \textless{}fct\textgreater{}   \textless{}int\textgreater{} \textless{}int\textgreater{}  \textless{}int\textgreater{} \textless{}int\textgreater{}   \textless{}int\textgreater{} \textless{}int\textgreater{} \textless{}int\textgreater{} \textless{}int\textgreater{} \textless{}int\textgreater{} \textless{}int\textgreater{} \textless{}int\textgreater{}}
\CommentTok{\#\textgreater{} 1 4842        1     1      1     1       1     1     1     1     1     1     1}
\CommentTok{\#\textgreater{} 2 4843        1     1      1     1       1     1     1     1     1     1     1}
\CommentTok{\#\textgreater{} 3 4844        1     1      1     1       1     1     1     1     1     1     1}
\CommentTok{\#\textgreater{} 4 4845        1     1      1     1       1    NA    NA    NA    NA    NA    NA}
\CommentTok{\#\textgreater{} 5 4847        1     1      1    NA      NA    NA    NA    NA    NA    NA    NA}
\CommentTok{\#\textgreater{} 6 4848        1     1      1     1      NA    NA    NA    NA    NA    NA    NA}
\CommentTok{\#\textgreater{} \# ... with 13 more rows}
\end{Highlighting}
\end{Shaded}

缺失值填充

\begin{Shaded}
\begin{Highlighting}[]
\NormalTok{fish\_encounters }\SpecialCharTok{\%\textgreater{}\%} \FunctionTok{pivot\_wider}\NormalTok{(}
  \AttributeTok{names\_from =}\NormalTok{ station, }
  \AttributeTok{values\_from =}\NormalTok{ seen,}
  \AttributeTok{values\_fill =} \DecValTok{0}
\NormalTok{)}
\CommentTok{\#\textgreater{} \# A tibble: 19 x 12}
\CommentTok{\#\textgreater{}   fish  Release I80\_1 Lisbon  Rstr Base\_TD   BCE   BCW  BCE2  BCW2   MAE   MAW}
\CommentTok{\#\textgreater{}   \textless{}fct\textgreater{}   \textless{}int\textgreater{} \textless{}int\textgreater{}  \textless{}int\textgreater{} \textless{}int\textgreater{}   \textless{}int\textgreater{} \textless{}int\textgreater{} \textless{}int\textgreater{} \textless{}int\textgreater{} \textless{}int\textgreater{} \textless{}int\textgreater{} \textless{}int\textgreater{}}
\CommentTok{\#\textgreater{} 1 4842        1     1      1     1       1     1     1     1     1     1     1}
\CommentTok{\#\textgreater{} 2 4843        1     1      1     1       1     1     1     1     1     1     1}
\CommentTok{\#\textgreater{} 3 4844        1     1      1     1       1     1     1     1     1     1     1}
\CommentTok{\#\textgreater{} 4 4845        1     1      1     1       1     0     0     0     0     0     0}
\CommentTok{\#\textgreater{} 5 4847        1     1      1     0       0     0     0     0     0     0     0}
\CommentTok{\#\textgreater{} 6 4848        1     1      1     1       0     0     0     0     0     0     0}
\CommentTok{\#\textgreater{} \# ... with 13 more rows}
\end{Highlighting}
\end{Shaded}

\hypertarget{ux805aux5408}{%
\subsubsection{聚合}\label{ux805aux5408}}

\begin{Shaded}
\begin{Highlighting}[]
\NormalTok{warpbreaks }\OtherTok{\textless{}{-}}\NormalTok{ warpbreaks }\SpecialCharTok{\%\textgreater{}\%} \FunctionTok{as\_tibble}\NormalTok{() }
\NormalTok{warpbreaks }\SpecialCharTok{\%\textgreater{}\%} \FunctionTok{count}\NormalTok{(wool, tension)}
\CommentTok{\#\textgreater{} \# A tibble: 6 x 3}
\CommentTok{\#\textgreater{}   wool  tension     n}
\CommentTok{\#\textgreater{}   \textless{}fct\textgreater{} \textless{}fct\textgreater{}   \textless{}int\textgreater{}}
\CommentTok{\#\textgreater{} 1 A     L           9}
\CommentTok{\#\textgreater{} 2 A     M           9}
\CommentTok{\#\textgreater{} 3 A     H           9}
\CommentTok{\#\textgreater{} 4 B     L           9}
\CommentTok{\#\textgreater{} 5 B     M           9}
\CommentTok{\#\textgreater{} 6 B     H           9}
\end{Highlighting}
\end{Shaded}

需要通过\texttt{values\_fn}指定聚合方式

\begin{Shaded}
\begin{Highlighting}[]
\NormalTok{warpbreaks }\SpecialCharTok{\%\textgreater{}\%} \FunctionTok{pivot\_wider}\NormalTok{(}\AttributeTok{names\_from =}\NormalTok{ wool, }\AttributeTok{values\_from =}\NormalTok{ breaks,}\AttributeTok{values\_fn=} \FunctionTok{list}\NormalTok{(}\AttributeTok{breaks =}\NormalTok{ sum))}
\CommentTok{\#\textgreater{} \# A tibble: 3 x 3}
\CommentTok{\#\textgreater{}   tension     A     B}
\CommentTok{\#\textgreater{}   \textless{}fct\textgreater{}   \textless{}dbl\textgreater{} \textless{}dbl\textgreater{}}
\CommentTok{\#\textgreater{} 1 L         401   254}
\CommentTok{\#\textgreater{} 2 M         216   259}
\CommentTok{\#\textgreater{} 3 H         221   169}
\end{Highlighting}
\end{Shaded}

\hypertarget{ux4eceux591aux4e2aux53d8ux91cfux751fux6210ux65b0ux5217ux540d}{%
\subsubsection{从多个变量生成新列名}\label{ux4eceux591aux4e2aux53d8ux91cfux751fux6210ux65b0ux5217ux540d}}

\begin{Shaded}
\begin{Highlighting}[]
\NormalTok{production }\OtherTok{\textless{}{-}} \FunctionTok{expand\_grid}\NormalTok{(}
    \AttributeTok{product =} \FunctionTok{c}\NormalTok{(}\StringTok{"A"}\NormalTok{, }\StringTok{"B"}\NormalTok{), }
    \AttributeTok{country =} \FunctionTok{c}\NormalTok{(}\StringTok{"AI"}\NormalTok{, }\StringTok{"EI"}\NormalTok{), }
    \AttributeTok{year =} \DecValTok{2000}\SpecialCharTok{:}\DecValTok{2014}
\NormalTok{  ) }\SpecialCharTok{\%\textgreater{}\%}
  \FunctionTok{filter}\NormalTok{((product }\SpecialCharTok{==} \StringTok{"A"} \SpecialCharTok{\&}\NormalTok{ country }\SpecialCharTok{==} \StringTok{"AI"}\NormalTok{) }\SpecialCharTok{|}\NormalTok{ product }\SpecialCharTok{==} \StringTok{"B"}\NormalTok{) }\SpecialCharTok{\%\textgreater{}\%} 
  \FunctionTok{mutate}\NormalTok{(}\AttributeTok{production =} \FunctionTok{rnorm}\NormalTok{(}\FunctionTok{nrow}\NormalTok{(.)))}
\NormalTok{production}
\CommentTok{\#\textgreater{} \# A tibble: 45 x 4}
\CommentTok{\#\textgreater{}   product country  year production}
\CommentTok{\#\textgreater{}   \textless{}chr\textgreater{}   \textless{}chr\textgreater{}   \textless{}int\textgreater{}      \textless{}dbl\textgreater{}}
\CommentTok{\#\textgreater{} 1 A       AI       2000     {-}0.244}
\CommentTok{\#\textgreater{} 2 A       AI       2001     {-}0.283}
\CommentTok{\#\textgreater{} 3 A       AI       2002     {-}0.554}
\CommentTok{\#\textgreater{} 4 A       AI       2003      0.629}
\CommentTok{\#\textgreater{} 5 A       AI       2004      2.07 }
\CommentTok{\#\textgreater{} 6 A       AI       2005     {-}1.63 }
\CommentTok{\#\textgreater{} \# ... with 39 more rows}
\end{Highlighting}
\end{Shaded}

\begin{Shaded}
\begin{Highlighting}[]
\NormalTok{production }\SpecialCharTok{\%\textgreater{}\%} \FunctionTok{pivot\_wider}\NormalTok{(}
  \AttributeTok{names\_from =} \FunctionTok{c}\NormalTok{(product, country), }
  \AttributeTok{values\_from =}\NormalTok{ production}
\NormalTok{)}
\CommentTok{\#\textgreater{} \# A tibble: 15 x 4}
\CommentTok{\#\textgreater{}    year   A\_AI    B\_AI    B\_EI}
\CommentTok{\#\textgreater{}   \textless{}int\textgreater{}  \textless{}dbl\textgreater{}   \textless{}dbl\textgreater{}   \textless{}dbl\textgreater{}}
\CommentTok{\#\textgreater{} 1  2000 {-}0.244  0.738  {-}0.313 }
\CommentTok{\#\textgreater{} 2  2001 {-}0.283  1.89    1.07  }
\CommentTok{\#\textgreater{} 3  2002 {-}0.554 {-}0.0974  0.0700}
\CommentTok{\#\textgreater{} 4  2003  0.629 {-}0.936  {-}0.639 }
\CommentTok{\#\textgreater{} 5  2004  2.07  {-}0.0160 {-}0.0500}
\CommentTok{\#\textgreater{} 6  2005 {-}1.63  {-}0.827  {-}0.251 }
\CommentTok{\#\textgreater{} \# ... with 9 more rows}
\end{Highlighting}
\end{Shaded}

通过\texttt{names\_sep}和\texttt{names\_prefix}参数控制新的列名,或通过\texttt{names\_glue}

\begin{Shaded}
\begin{Highlighting}[]
\NormalTok{production }\SpecialCharTok{\%\textgreater{}\%} \FunctionTok{pivot\_wider}\NormalTok{(}
  \AttributeTok{names\_from =} \FunctionTok{c}\NormalTok{(product, country), }
  \AttributeTok{values\_from =}\NormalTok{ production,}
  \AttributeTok{names\_sep =} \StringTok{"."}\NormalTok{,}
  \AttributeTok{names\_prefix =} \StringTok{"prod."}
\NormalTok{)}
\CommentTok{\#\textgreater{} \# A tibble: 15 x 4}
\CommentTok{\#\textgreater{}    year prod.A.AI prod.B.AI prod.B.EI}
\CommentTok{\#\textgreater{}   \textless{}int\textgreater{}     \textless{}dbl\textgreater{}     \textless{}dbl\textgreater{}     \textless{}dbl\textgreater{}}
\CommentTok{\#\textgreater{} 1  2000    {-}0.244    0.738    {-}0.313 }
\CommentTok{\#\textgreater{} 2  2001    {-}0.283    1.89      1.07  }
\CommentTok{\#\textgreater{} 3  2002    {-}0.554   {-}0.0974    0.0700}
\CommentTok{\#\textgreater{} 4  2003     0.629   {-}0.936    {-}0.639 }
\CommentTok{\#\textgreater{} 5  2004     2.07    {-}0.0160   {-}0.0500}
\CommentTok{\#\textgreater{} 6  2005    {-}1.63    {-}0.827    {-}0.251 }
\CommentTok{\#\textgreater{} \# ... with 9 more rows}
\end{Highlighting}
\end{Shaded}

\begin{Shaded}
\begin{Highlighting}[]
\NormalTok{production }\SpecialCharTok{\%\textgreater{}\%} \FunctionTok{pivot\_wider}\NormalTok{(}
  \AttributeTok{names\_from =} \FunctionTok{c}\NormalTok{(product, country), }
  \AttributeTok{values\_from =}\NormalTok{ production,}
  \AttributeTok{names\_glue =} \StringTok{"prod\_\{product\}\_\{country\}"}
\NormalTok{)}
\CommentTok{\#\textgreater{} \# A tibble: 15 x 4}
\CommentTok{\#\textgreater{}    year prod\_A\_AI prod\_B\_AI prod\_B\_EI}
\CommentTok{\#\textgreater{}   \textless{}int\textgreater{}     \textless{}dbl\textgreater{}     \textless{}dbl\textgreater{}     \textless{}dbl\textgreater{}}
\CommentTok{\#\textgreater{} 1  2000    {-}0.244    0.738    {-}0.313 }
\CommentTok{\#\textgreater{} 2  2001    {-}0.283    1.89      1.07  }
\CommentTok{\#\textgreater{} 3  2002    {-}0.554   {-}0.0974    0.0700}
\CommentTok{\#\textgreater{} 4  2003     0.629   {-}0.936    {-}0.639 }
\CommentTok{\#\textgreater{} 5  2004     2.07    {-}0.0160   {-}0.0500}
\CommentTok{\#\textgreater{} 6  2005    {-}1.63    {-}0.827    {-}0.251 }
\CommentTok{\#\textgreater{} \# ... with 9 more rows}
\end{Highlighting}
\end{Shaded}

\hypertarget{ux591aux503cux53d8ux5bbd}{%
\subsubsection{多值变宽}\label{ux591aux503cux53d8ux5bbd}}

\begin{Shaded}
\begin{Highlighting}[]
\NormalTok{us\_rent\_income }\SpecialCharTok{\%\textgreater{}\%} 
  \FunctionTok{pivot\_wider}\NormalTok{(}\AttributeTok{names\_from =}\NormalTok{ variable, }\AttributeTok{values\_from =} \FunctionTok{c}\NormalTok{(estimate, moe))}
\CommentTok{\#\textgreater{} \# A tibble: 52 x 6}
\CommentTok{\#\textgreater{}   GEOID NAME       estimate\_income estimate\_rent moe\_income moe\_rent}
\CommentTok{\#\textgreater{}   \textless{}chr\textgreater{} \textless{}chr\textgreater{}                \textless{}dbl\textgreater{}         \textless{}dbl\textgreater{}      \textless{}dbl\textgreater{}    \textless{}dbl\textgreater{}}
\CommentTok{\#\textgreater{} 1 01    Alabama              24476           747        136        3}
\CommentTok{\#\textgreater{} 2 02    Alaska               32940          1200        508       13}
\CommentTok{\#\textgreater{} 3 04    Arizona              27517           972        148        4}
\CommentTok{\#\textgreater{} 4 05    Arkansas             23789           709        165        5}
\CommentTok{\#\textgreater{} 5 06    California           29454          1358        109        3}
\CommentTok{\#\textgreater{} 6 08    Colorado             32401          1125        109        5}
\CommentTok{\#\textgreater{} \# ... with 46 more rows}
\end{Highlighting}
\end{Shaded}

\hypertarget{ux5904ux7406jsonhtmlux7684ux6570ux636e}{%
\subsection{处理json,html的数据}\label{ux5904ux7406jsonhtmlux7684ux6570ux636e}}

实际工作中不是经常使用,需要使用的时候往往会用相关的包处理:\texttt{jsonlite}

可通过\texttt{vignette("rectangle")}自行学习

\begin{Shaded}
\begin{Highlighting}[]
\FunctionTok{library}\NormalTok{(tidyr)}
\FunctionTok{library}\NormalTok{(dplyr)}
\FunctionTok{library}\NormalTok{(repurrrsive)}
\end{Highlighting}
\end{Shaded}

\begin{Shaded}
\begin{Highlighting}[]
\NormalTok{users }\OtherTok{\textless{}{-}} \FunctionTok{tibble}\NormalTok{(}\AttributeTok{user =}\NormalTok{ gh\_users)}
\NormalTok{users}
\CommentTok{\#\textgreater{} \# A tibble: 6 x 1}
\CommentTok{\#\textgreater{}   user             }
\CommentTok{\#\textgreater{}   \textless{}list\textgreater{}           }
\CommentTok{\#\textgreater{} 1 \textless{}named list [30]\textgreater{}}
\CommentTok{\#\textgreater{} 2 \textless{}named list [30]\textgreater{}}
\CommentTok{\#\textgreater{} 3 \textless{}named list [30]\textgreater{}}
\CommentTok{\#\textgreater{} 4 \textless{}named list [30]\textgreater{}}
\CommentTok{\#\textgreater{} 5 \textless{}named list [30]\textgreater{}}
\CommentTok{\#\textgreater{} 6 \textless{}named list [30]\textgreater{}}
\NormalTok{users }\SpecialCharTok{\%\textgreater{}\%} \FunctionTok{unnest\_wider}\NormalTok{(user)}
\CommentTok{\#\textgreater{} \# A tibble: 6 x 30}
\CommentTok{\#\textgreater{}   login     id avatar\_url gravatar\_id url   html\_url followers\_url following\_url}
\CommentTok{\#\textgreater{}   \textless{}chr\textgreater{}  \textless{}int\textgreater{} \textless{}chr\textgreater{}      \textless{}chr\textgreater{}       \textless{}chr\textgreater{} \textless{}chr\textgreater{}    \textless{}chr\textgreater{}         \textless{}chr\textgreater{}        }
\CommentTok{\#\textgreater{} 1 gabo\textasciitilde{} 6.60e5 https://a\textasciitilde{} ""          http\textasciitilde{} https:/\textasciitilde{} https://api.\textasciitilde{} https://api.\textasciitilde{}}
\CommentTok{\#\textgreater{} 2 jenn\textasciitilde{} 5.99e5 https://a\textasciitilde{} ""          http\textasciitilde{} https:/\textasciitilde{} https://api.\textasciitilde{} https://api.\textasciitilde{}}
\CommentTok{\#\textgreater{} 3 jtle\textasciitilde{} 1.57e6 https://a\textasciitilde{} ""          http\textasciitilde{} https:/\textasciitilde{} https://api.\textasciitilde{} https://api.\textasciitilde{}}
\CommentTok{\#\textgreater{} 4 juli\textasciitilde{} 1.25e7 https://a\textasciitilde{} ""          http\textasciitilde{} https:/\textasciitilde{} https://api.\textasciitilde{} https://api.\textasciitilde{}}
\CommentTok{\#\textgreater{} 5 leep\textasciitilde{} 3.51e6 https://a\textasciitilde{} ""          http\textasciitilde{} https:/\textasciitilde{} https://api.\textasciitilde{} https://api.\textasciitilde{}}
\CommentTok{\#\textgreater{} 6 masa\textasciitilde{} 8.36e6 https://a\textasciitilde{} ""          http\textasciitilde{} https:/\textasciitilde{} https://api.\textasciitilde{} https://api.\textasciitilde{}}
\CommentTok{\#\textgreater{} \# ... with 22 more variables: gists\_url \textless{}chr\textgreater{}, starred\_url \textless{}chr\textgreater{},}
\CommentTok{\#\textgreater{} \#   subscriptions\_url \textless{}chr\textgreater{}, organizations\_url \textless{}chr\textgreater{}, repos\_url \textless{}chr\textgreater{},}
\CommentTok{\#\textgreater{} \#   events\_url \textless{}chr\textgreater{}, received\_events\_url \textless{}chr\textgreater{}, type \textless{}chr\textgreater{}, site\_admin \textless{}lgl\textgreater{},}
\CommentTok{\#\textgreater{} \#   name \textless{}chr\textgreater{}, company \textless{}chr\textgreater{}, blog \textless{}chr\textgreater{}, location \textless{}chr\textgreater{}, email \textless{}chr\textgreater{},}
\CommentTok{\#\textgreater{} \#   public\_repos \textless{}int\textgreater{}, public\_gists \textless{}int\textgreater{}, followers \textless{}int\textgreater{}, following \textless{}int\textgreater{},}
\CommentTok{\#\textgreater{} \#   created\_at \textless{}chr\textgreater{}, updated\_at \textless{}chr\textgreater{}, bio \textless{}chr\textgreater{}, hireable \textless{}lgl\textgreater{}}
\end{Highlighting}
\end{Shaded}

\hypertarget{ux5d4cux5957ux6570ux636e}{%
\subsection{嵌套数据}\label{ux5d4cux5957ux6570ux636e}}

\begin{Shaded}
\begin{Highlighting}[]
\FunctionTok{library}\NormalTok{(tidyr)}
\FunctionTok{library}\NormalTok{(dplyr)}
\FunctionTok{library}\NormalTok{(purrr)}
\end{Highlighting}
\end{Shaded}

\hypertarget{ux57faux7840-3}{%
\subsubsection{基础}\label{ux57faux7840-3}}

嵌套数据即:数据框中嵌套数据框,如下所示:

\begin{Shaded}
\begin{Highlighting}[]
\NormalTok{df1 }\OtherTok{\textless{}{-}} \FunctionTok{tibble}\NormalTok{(}
  \AttributeTok{g =} \FunctionTok{c}\NormalTok{(}\DecValTok{1}\NormalTok{, }\DecValTok{2}\NormalTok{, }\DecValTok{3}\NormalTok{),}
  \AttributeTok{data =} \FunctionTok{list}\NormalTok{(}
    \FunctionTok{tibble}\NormalTok{(}\AttributeTok{x =} \DecValTok{1}\NormalTok{, }\AttributeTok{y =} \DecValTok{2}\NormalTok{),}
    \FunctionTok{tibble}\NormalTok{(}\AttributeTok{x =} \DecValTok{4}\SpecialCharTok{:}\DecValTok{5}\NormalTok{, }\AttributeTok{y =} \DecValTok{6}\SpecialCharTok{:}\DecValTok{7}\NormalTok{),}
    \FunctionTok{tibble}\NormalTok{(}\AttributeTok{x =} \DecValTok{10}\NormalTok{)}
\NormalTok{  )}
\NormalTok{)}
\NormalTok{df1}
\CommentTok{\#\textgreater{} \# A tibble: 3 x 2}
\CommentTok{\#\textgreater{}       g data            }
\CommentTok{\#\textgreater{}   \textless{}dbl\textgreater{} \textless{}list\textgreater{}          }
\CommentTok{\#\textgreater{} 1     1 \textless{}tibble [1 x 2]\textgreater{}}
\CommentTok{\#\textgreater{} 2     2 \textless{}tibble [2 x 2]\textgreater{}}
\CommentTok{\#\textgreater{} 3     3 \textless{}tibble [1 x 1]\textgreater{}}
\end{Highlighting}
\end{Shaded}

因为\texttt{data.frame()}的列特性【每列都是列表】【不确定理解对不对】:可以做如下操作:

\begin{Shaded}
\begin{Highlighting}[]
\NormalTok{df2 }\OtherTok{\textless{}{-}} \FunctionTok{tribble}\NormalTok{(}
  \SpecialCharTok{\textasciitilde{}}\NormalTok{g, }\SpecialCharTok{\textasciitilde{}}\NormalTok{x, }\SpecialCharTok{\textasciitilde{}}\NormalTok{y,}
   \DecValTok{1}\NormalTok{,  }\DecValTok{1}\NormalTok{,  }\DecValTok{2}\NormalTok{,}
   \DecValTok{2}\NormalTok{,  }\DecValTok{4}\NormalTok{,  }\DecValTok{6}\NormalTok{,}
   \DecValTok{2}\NormalTok{,  }\DecValTok{5}\NormalTok{,  }\DecValTok{7}\NormalTok{,}
   \DecValTok{3}\NormalTok{, }\DecValTok{10}\NormalTok{,  }\ConstantTok{NA}
\NormalTok{)}
\NormalTok{df2 }\SpecialCharTok{\%\textgreater{}\%} \FunctionTok{nest}\NormalTok{(}\AttributeTok{data =} \FunctionTok{c}\NormalTok{(x, y))}
\CommentTok{\#\textgreater{} \# A tibble: 3 x 2}
\CommentTok{\#\textgreater{}       g data            }
\CommentTok{\#\textgreater{}   \textless{}dbl\textgreater{} \textless{}list\textgreater{}          }
\CommentTok{\#\textgreater{} 1     1 \textless{}tibble [1 x 2]\textgreater{}}
\CommentTok{\#\textgreater{} 2     2 \textless{}tibble [2 x 2]\textgreater{}}
\CommentTok{\#\textgreater{} 3     3 \textless{}tibble [1 x 2]\textgreater{}}

\CommentTok{\#sample above}
\CommentTok{\#df2 \%\textgreater{}\% group\_by(g) \%\textgreater{}\% nest()}
\end{Highlighting}
\end{Shaded}

nest的反面 unnest

\begin{Shaded}
\begin{Highlighting}[]
\NormalTok{df1 }\SpecialCharTok{\%\textgreater{}\%} \FunctionTok{unnest}\NormalTok{(data)}
\CommentTok{\#\textgreater{} \# A tibble: 4 x 3}
\CommentTok{\#\textgreater{}       g     x     y}
\CommentTok{\#\textgreater{}   \textless{}dbl\textgreater{} \textless{}dbl\textgreater{} \textless{}dbl\textgreater{}}
\CommentTok{\#\textgreater{} 1     1     1     2}
\CommentTok{\#\textgreater{} 2     2     4     6}
\CommentTok{\#\textgreater{} 3     2     5     7}
\CommentTok{\#\textgreater{} 4     3    10    NA}
\end{Highlighting}
\end{Shaded}

\hypertarget{ux5d4cux5957ux6570ux636eux548cux6a21ux578b}{%
\subsection{嵌套数据和模型}\label{ux5d4cux5957ux6570ux636eux548cux6a21ux578b}}

\begin{Shaded}
\begin{Highlighting}[]
\NormalTok{mtcars\_nested }\OtherTok{\textless{}{-}}\NormalTok{ mtcars }\SpecialCharTok{\%\textgreater{}\%} 
  \FunctionTok{group\_by}\NormalTok{(cyl) }\SpecialCharTok{\%\textgreater{}\%} 
  \FunctionTok{nest}\NormalTok{()}

\NormalTok{mtcars\_nested}
\CommentTok{\#\textgreater{} \# A tibble: 3 x 2}
\CommentTok{\#\textgreater{} \# Groups:   cyl [3]}
\CommentTok{\#\textgreater{}     cyl data              }
\CommentTok{\#\textgreater{}   \textless{}int\textgreater{} \textless{}list\textgreater{}            }
\CommentTok{\#\textgreater{} 1     6 \textless{}tibble [7 x 10]\textgreater{} }
\CommentTok{\#\textgreater{} 2     4 \textless{}tibble [11 x 10]\textgreater{}}
\CommentTok{\#\textgreater{} 3     8 \textless{}tibble [14 x 10]\textgreater{}}
\end{Highlighting}
\end{Shaded}

\begin{Shaded}
\begin{Highlighting}[]
\NormalTok{mtcars\_nested }\OtherTok{\textless{}{-}}\NormalTok{ mtcars\_nested }\SpecialCharTok{\%\textgreater{}\%} 
  \FunctionTok{mutate}\NormalTok{(}\AttributeTok{model =} \FunctionTok{map}\NormalTok{(data, }\ControlFlowTok{function}\NormalTok{(df) }\FunctionTok{lm}\NormalTok{(mpg }\SpecialCharTok{\textasciitilde{}}\NormalTok{ wt, }\AttributeTok{data =}\NormalTok{ df)))}
\NormalTok{mtcars\_nested}
\CommentTok{\#\textgreater{} \# A tibble: 3 x 3}
\CommentTok{\#\textgreater{} \# Groups:   cyl [3]}
\CommentTok{\#\textgreater{}     cyl data               model }
\CommentTok{\#\textgreater{}   \textless{}int\textgreater{} \textless{}list\textgreater{}             \textless{}list\textgreater{}}
\CommentTok{\#\textgreater{} 1     6 \textless{}tibble [7 x 10]\textgreater{}  \textless{}lm\textgreater{}  }
\CommentTok{\#\textgreater{} 2     4 \textless{}tibble [11 x 10]\textgreater{} \textless{}lm\textgreater{}  }
\CommentTok{\#\textgreater{} 3     8 \textless{}tibble [14 x 10]\textgreater{} \textless{}lm\textgreater{}}
\end{Highlighting}
\end{Shaded}

\begin{Shaded}
\begin{Highlighting}[]
\NormalTok{mtcars\_nested }\OtherTok{\textless{}{-}}\NormalTok{ mtcars\_nested }\SpecialCharTok{\%\textgreater{}\%} 
  \FunctionTok{mutate}\NormalTok{(}\AttributeTok{model =} \FunctionTok{map}\NormalTok{(model, predict))}
\NormalTok{mtcars\_nested  }
\CommentTok{\#\textgreater{} \# A tibble: 3 x 3}
\CommentTok{\#\textgreater{} \# Groups:   cyl [3]}
\CommentTok{\#\textgreater{}     cyl data               model     }
\CommentTok{\#\textgreater{}   \textless{}int\textgreater{} \textless{}list\textgreater{}             \textless{}list\textgreater{}    }
\CommentTok{\#\textgreater{} 1     6 \textless{}tibble [7 x 10]\textgreater{}  \textless{}dbl [7]\textgreater{} }
\CommentTok{\#\textgreater{} 2     4 \textless{}tibble [11 x 10]\textgreater{} \textless{}dbl [11]\textgreater{}}
\CommentTok{\#\textgreater{} 3     8 \textless{}tibble [14 x 10]\textgreater{} \textless{}dbl [14]\textgreater{}}
\end{Highlighting}
\end{Shaded}

\hypertarget{ux62c6ux5206ux548cux5408ux5e76}{%
\subsection{拆分和合并}\label{ux62c6ux5206ux548cux5408ux5e76}}

\hypertarget{ux62c6ux5206}{%
\subsubsection{拆分}\label{ux62c6ux5206}}

有时我们需要将一列拆分为多列:

\begin{Shaded}
\begin{Highlighting}[]
\FunctionTok{library}\NormalTok{(tidyr)}
\NormalTok{df }\OtherTok{\textless{}{-}} \FunctionTok{data.frame}\NormalTok{(}\AttributeTok{x =} \FunctionTok{c}\NormalTok{(}\ConstantTok{NA}\NormalTok{, }\StringTok{"a.b"}\NormalTok{, }\StringTok{"a.d"}\NormalTok{, }\StringTok{"b.c"}\NormalTok{))}
\NormalTok{df }\SpecialCharTok{\%\textgreater{}\%} \FunctionTok{separate}\NormalTok{(x, }\FunctionTok{c}\NormalTok{(}\StringTok{"A"}\NormalTok{, }\StringTok{"B"}\NormalTok{))}
\CommentTok{\#\textgreater{}      A    B}
\CommentTok{\#\textgreater{} 1 \textless{}NA\textgreater{} \textless{}NA\textgreater{}}
\CommentTok{\#\textgreater{} 2    a    b}
\CommentTok{\#\textgreater{} 3    a    d}
\CommentTok{\#\textgreater{} 4    b    c}
\end{Highlighting}
\end{Shaded}

拆分数多列或少列时用\texttt{NA}补齐:

\begin{Shaded}
\begin{Highlighting}[]
\NormalTok{df }\OtherTok{\textless{}{-}} \FunctionTok{data.frame}\NormalTok{(}\AttributeTok{x =} \FunctionTok{c}\NormalTok{(}\StringTok{"a"}\NormalTok{, }\StringTok{"a b"}\NormalTok{, }\StringTok{"a b c"}\NormalTok{, }\ConstantTok{NA}\NormalTok{))}
\NormalTok{df }\SpecialCharTok{\%\textgreater{}\%} \FunctionTok{separate}\NormalTok{(x, }\FunctionTok{c}\NormalTok{(}\StringTok{"a"}\NormalTok{, }\StringTok{"b"}\NormalTok{))}
\CommentTok{\#\textgreater{} Warning: Expected 2 pieces. Additional pieces discarded in 1 rows [3].}
\CommentTok{\#\textgreater{} Warning: Expected 2 pieces. Missing pieces filled with \textasciigrave{}NA\textasciigrave{} in 1 rows [1].}
\CommentTok{\#\textgreater{}      a    b}
\CommentTok{\#\textgreater{} 1    a \textless{}NA\textgreater{}}
\CommentTok{\#\textgreater{} 2    a    b}
\CommentTok{\#\textgreater{} 3    a    b}
\CommentTok{\#\textgreater{} 4 \textless{}NA\textgreater{} \textless{}NA\textgreater{}}
\end{Highlighting}
\end{Shaded}

多余的部分舍弃,缺失填充在左边还是右边:

\begin{Shaded}
\begin{Highlighting}[]
\CommentTok{\# The same behaviour as previous, but drops the c without warnings:}
\NormalTok{df }\SpecialCharTok{\%\textgreater{}\%} \FunctionTok{separate}\NormalTok{(x, }\FunctionTok{c}\NormalTok{(}\StringTok{"a"}\NormalTok{, }\StringTok{"b"}\NormalTok{), }\AttributeTok{extra =} \StringTok{"drop"}\NormalTok{, }\AttributeTok{fill =} \StringTok{"right"}\NormalTok{)}
\CommentTok{\#\textgreater{}      a    b}
\CommentTok{\#\textgreater{} 1    a \textless{}NA\textgreater{}}
\CommentTok{\#\textgreater{} 2    a    b}
\CommentTok{\#\textgreater{} 3    a    b}
\CommentTok{\#\textgreater{} 4 \textless{}NA\textgreater{} \textless{}NA\textgreater{}}
\end{Highlighting}
\end{Shaded}

多余部分合并,缺失填充在左边

\begin{Shaded}
\begin{Highlighting}[]
\NormalTok{df }\SpecialCharTok{\%\textgreater{}\%} \FunctionTok{separate}\NormalTok{(x, }\FunctionTok{c}\NormalTok{(}\StringTok{"a"}\NormalTok{, }\StringTok{"b"}\NormalTok{), }\AttributeTok{extra =} \StringTok{"merge"}\NormalTok{, }\AttributeTok{fill =} \StringTok{"left"}\NormalTok{)}
\CommentTok{\#\textgreater{}      a    b}
\CommentTok{\#\textgreater{} 1 \textless{}NA\textgreater{}    a}
\CommentTok{\#\textgreater{} 2    a    b}
\CommentTok{\#\textgreater{} 3    a  b c}
\CommentTok{\#\textgreater{} 4 \textless{}NA\textgreater{} \textless{}NA\textgreater{}}
\end{Highlighting}
\end{Shaded}

或者全部保留

\begin{Shaded}
\begin{Highlighting}[]
\NormalTok{df }\SpecialCharTok{\%\textgreater{}\%} \FunctionTok{separate}\NormalTok{(x, }\FunctionTok{c}\NormalTok{(}\StringTok{"a"}\NormalTok{, }\StringTok{"b"}\NormalTok{, }\StringTok{"c"}\NormalTok{))}
\CommentTok{\#\textgreater{} Warning: Expected 3 pieces. Missing pieces filled with \textasciigrave{}NA\textasciigrave{} in 2 rows [1, 2].}
\CommentTok{\#\textgreater{}      a    b    c}
\CommentTok{\#\textgreater{} 1    a \textless{}NA\textgreater{} \textless{}NA\textgreater{}}
\CommentTok{\#\textgreater{} 2    a    b \textless{}NA\textgreater{}}
\CommentTok{\#\textgreater{} 3    a    b    c}
\CommentTok{\#\textgreater{} 4 \textless{}NA\textgreater{} \textless{}NA\textgreater{} \textless{}NA\textgreater{}}
\end{Highlighting}
\end{Shaded}

指定分隔符

\begin{Shaded}
\begin{Highlighting}[]
\NormalTok{df }\SpecialCharTok{\%\textgreater{}\%} \FunctionTok{separate}\NormalTok{(x, }\FunctionTok{c}\NormalTok{(}\StringTok{"key"}\NormalTok{, }\StringTok{"value"}\NormalTok{), }\AttributeTok{sep =} \StringTok{": "}\NormalTok{, }\AttributeTok{extra =} \StringTok{"merge"}\NormalTok{)}
\CommentTok{\#\textgreater{} Warning: Expected 2 pieces. Missing pieces filled with \textasciigrave{}NA\textasciigrave{} in 3 rows [1, 2, 3].}
\CommentTok{\#\textgreater{}     key value}
\CommentTok{\#\textgreater{} 1     a  \textless{}NA\textgreater{}}
\CommentTok{\#\textgreater{} 2   a b  \textless{}NA\textgreater{}}
\CommentTok{\#\textgreater{} 3 a b c  \textless{}NA\textgreater{}}
\CommentTok{\#\textgreater{} 4  \textless{}NA\textgreater{}  \textless{}NA\textgreater{}}
\end{Highlighting}
\end{Shaded}

使用正则表达式

\begin{Shaded}
\begin{Highlighting}[]
\CommentTok{\# Use regular expressions to separate on multiple characters:}
\NormalTok{df }\OtherTok{\textless{}{-}} \FunctionTok{data.frame}\NormalTok{(}\AttributeTok{x =} \FunctionTok{c}\NormalTok{(}\ConstantTok{NA}\NormalTok{, }\StringTok{"a?b"}\NormalTok{, }\StringTok{"a.d"}\NormalTok{, }\StringTok{"b:c"}\NormalTok{))}
\NormalTok{df }\SpecialCharTok{\%\textgreater{}\%} \FunctionTok{separate}\NormalTok{(x, }\FunctionTok{c}\NormalTok{(}\StringTok{"A"}\NormalTok{,}\StringTok{"B"}\NormalTok{), }\AttributeTok{sep =} \StringTok{"([.?:])"}\NormalTok{)}
\CommentTok{\#\textgreater{}      A    B}
\CommentTok{\#\textgreater{} 1 \textless{}NA\textgreater{} \textless{}NA\textgreater{}}
\CommentTok{\#\textgreater{} 2    a    b}
\CommentTok{\#\textgreater{} 3    a    d}
\CommentTok{\#\textgreater{} 4    b    c}
\end{Highlighting}
\end{Shaded}

\hypertarget{ux65b0ux5217ux63d0ux53d6}{%
\subsubsection{新列提取}\label{ux65b0ux5217ux63d0ux53d6}}

\begin{Shaded}
\begin{Highlighting}[]
\NormalTok{df }\OtherTok{\textless{}{-}} \FunctionTok{data.frame}\NormalTok{(}\AttributeTok{x =} \FunctionTok{c}\NormalTok{(}\ConstantTok{NA}\NormalTok{, }\StringTok{"a{-}b"}\NormalTok{, }\StringTok{"a{-}d"}\NormalTok{, }\StringTok{"b{-}c"}\NormalTok{, }\StringTok{"d{-}e"}\NormalTok{))}
\NormalTok{df }\SpecialCharTok{\%\textgreater{}\%} \FunctionTok{extract}\NormalTok{(x, }\StringTok{"A"}\NormalTok{)}
\CommentTok{\#\textgreater{}      A}
\CommentTok{\#\textgreater{} 1 \textless{}NA\textgreater{}}
\CommentTok{\#\textgreater{} 2    a}
\CommentTok{\#\textgreater{} 3    a}
\CommentTok{\#\textgreater{} 4    b}
\CommentTok{\#\textgreater{} 5    d}
\NormalTok{df }\SpecialCharTok{\%\textgreater{}\%} \FunctionTok{extract}\NormalTok{(x, }\FunctionTok{c}\NormalTok{(}\StringTok{"A"}\NormalTok{, }\StringTok{"B"}\NormalTok{), }\StringTok{"([[:alnum:]]+){-}([[:alnum:]]+)"}\NormalTok{)}
\CommentTok{\#\textgreater{}      A    B}
\CommentTok{\#\textgreater{} 1 \textless{}NA\textgreater{} \textless{}NA\textgreater{}}
\CommentTok{\#\textgreater{} 2    a    b}
\CommentTok{\#\textgreater{} 3    a    d}
\CommentTok{\#\textgreater{} 4    b    c}
\CommentTok{\#\textgreater{} 5    d    e}
\CommentTok{\# [:alnum:] 匹配字母和数字}
\end{Highlighting}
\end{Shaded}

以上本质是字符处理,\href{http://baiy.cn/utils/_regex_doc/index.htm}{正则表达式}

\hypertarget{ux5408ux5e76}{%
\subsubsection{合并}\label{ux5408ux5e76}}

\begin{Shaded}
\begin{Highlighting}[]
\NormalTok{df }\OtherTok{\textless{}{-}} \FunctionTok{expand\_grid}\NormalTok{(}\AttributeTok{x =} \FunctionTok{c}\NormalTok{(}\StringTok{"a"}\NormalTok{, }\ConstantTok{NA}\NormalTok{), }\AttributeTok{y =} \FunctionTok{c}\NormalTok{(}\StringTok{"b"}\NormalTok{, }\ConstantTok{NA}\NormalTok{))}
\NormalTok{df}
\CommentTok{\#\textgreater{} \# A tibble: 4 x 2}
\CommentTok{\#\textgreater{}   x     y    }
\CommentTok{\#\textgreater{}   \textless{}chr\textgreater{} \textless{}chr\textgreater{}}
\CommentTok{\#\textgreater{} 1 a     b    }
\CommentTok{\#\textgreater{} 2 a     \textless{}NA\textgreater{} }
\CommentTok{\#\textgreater{} 3 \textless{}NA\textgreater{}  b    }
\CommentTok{\#\textgreater{} 4 \textless{}NA\textgreater{}  \textless{}NA\textgreater{}}
\NormalTok{df }\SpecialCharTok{\%\textgreater{}\%} \FunctionTok{unite}\NormalTok{(}\StringTok{"z"}\NormalTok{, x}\SpecialCharTok{:}\NormalTok{y, }\AttributeTok{remove =} \ConstantTok{FALSE}\NormalTok{)}
\CommentTok{\#\textgreater{} \# A tibble: 4 x 3}
\CommentTok{\#\textgreater{}   z     x     y    }
\CommentTok{\#\textgreater{}   \textless{}chr\textgreater{} \textless{}chr\textgreater{} \textless{}chr\textgreater{}}
\CommentTok{\#\textgreater{} 1 a\_b   a     b    }
\CommentTok{\#\textgreater{} 2 a\_NA  a     \textless{}NA\textgreater{} }
\CommentTok{\#\textgreater{} 3 NA\_b  \textless{}NA\textgreater{}  b    }
\CommentTok{\#\textgreater{} 4 NA\_NA \textless{}NA\textgreater{}  \textless{}NA\textgreater{}}
\CommentTok{\# expand\_grid 类似笛卡尔积功能}
\end{Highlighting}
\end{Shaded}

移除缺失值

\begin{Shaded}
\begin{Highlighting}[]
\NormalTok{df }\SpecialCharTok{\%\textgreater{}\%} \FunctionTok{unite}\NormalTok{(}\StringTok{"z"}\NormalTok{, x}\SpecialCharTok{:}\NormalTok{y, }\AttributeTok{na.rm =} \ConstantTok{TRUE}\NormalTok{, }\AttributeTok{remove =} \ConstantTok{FALSE}\NormalTok{)}
\CommentTok{\#\textgreater{} \# A tibble: 4 x 3}
\CommentTok{\#\textgreater{}   z     x     y    }
\CommentTok{\#\textgreater{}   \textless{}chr\textgreater{} \textless{}chr\textgreater{} \textless{}chr\textgreater{}}
\CommentTok{\#\textgreater{} 1 "a\_b" a     b    }
\CommentTok{\#\textgreater{} 2 "a"   a     \textless{}NA\textgreater{} }
\CommentTok{\#\textgreater{} 3 "b"   \textless{}NA\textgreater{}  b    }
\CommentTok{\#\textgreater{} 4 ""    \textless{}NA\textgreater{}  \textless{}NA\textgreater{}}
\end{Highlighting}
\end{Shaded}

合并后再拆分

\begin{Shaded}
\begin{Highlighting}[]
\NormalTok{df }\SpecialCharTok{\%\textgreater{}\%}
  \FunctionTok{unite}\NormalTok{(}\StringTok{"xy"}\NormalTok{, x}\SpecialCharTok{:}\NormalTok{y) }\SpecialCharTok{\%\textgreater{}\%}
  \FunctionTok{separate}\NormalTok{(xy, }\FunctionTok{c}\NormalTok{(}\StringTok{"x"}\NormalTok{, }\StringTok{"y"}\NormalTok{))}
\CommentTok{\#\textgreater{} \# A tibble: 4 x 2}
\CommentTok{\#\textgreater{}   x     y    }
\CommentTok{\#\textgreater{}   \textless{}chr\textgreater{} \textless{}chr\textgreater{}}
\CommentTok{\#\textgreater{} 1 a     b    }
\CommentTok{\#\textgreater{} 2 a     NA   }
\CommentTok{\#\textgreater{} 3 NA    b    }
\CommentTok{\#\textgreater{} 4 NA    NA}
\end{Highlighting}
\end{Shaded}

\hypertarget{ux7f3aux5931ux503cux5904ux7406}{%
\subsection{缺失值处理}\label{ux7f3aux5931ux503cux5904ux7406}}

\texttt{replace\_na()}用特定值替换缺失值。

\begin{Shaded}
\begin{Highlighting}[]
\NormalTok{df }\OtherTok{\textless{}{-}} \FunctionTok{tibble}\NormalTok{(}\AttributeTok{x =} \FunctionTok{c}\NormalTok{(}\DecValTok{1}\NormalTok{, }\DecValTok{2}\NormalTok{, }\ConstantTok{NA}\NormalTok{), }\AttributeTok{y =} \FunctionTok{c}\NormalTok{(}\StringTok{"a"}\NormalTok{, }\ConstantTok{NA}\NormalTok{, }\StringTok{"b"}\NormalTok{))}
\NormalTok{df }\SpecialCharTok{\%\textgreater{}\%} \FunctionTok{replace\_na}\NormalTok{(}\FunctionTok{list}\NormalTok{(}\AttributeTok{x =} \DecValTok{0}\NormalTok{, }\AttributeTok{y =} \StringTok{"unknown"}\NormalTok{))}
\CommentTok{\#\textgreater{} \# A tibble: 3 x 2}
\CommentTok{\#\textgreater{}       x y      }
\CommentTok{\#\textgreater{}   \textless{}dbl\textgreater{} \textless{}chr\textgreater{}  }
\CommentTok{\#\textgreater{} 1     1 a      }
\CommentTok{\#\textgreater{} 2     2 unknown}
\CommentTok{\#\textgreater{} 3     0 b}
\end{Highlighting}
\end{Shaded}

\begin{Shaded}
\begin{Highlighting}[]
\NormalTok{df }\SpecialCharTok{\%\textgreater{}\%}\NormalTok{ dplyr}\SpecialCharTok{::}\FunctionTok{mutate}\NormalTok{(}\AttributeTok{x =} \FunctionTok{replace\_na}\NormalTok{(x, }\DecValTok{0}\NormalTok{))}
\CommentTok{\#\textgreater{} \# A tibble: 3 x 2}
\CommentTok{\#\textgreater{}       x y    }
\CommentTok{\#\textgreater{}   \textless{}dbl\textgreater{} \textless{}chr\textgreater{}}
\CommentTok{\#\textgreater{} 1     1 a    }
\CommentTok{\#\textgreater{} 2     2 \textless{}NA\textgreater{} }
\CommentTok{\#\textgreater{} 3     0 b}
\end{Highlighting}
\end{Shaded}

\hypertarget{character:handling}{%
\chapter{字符处理}\label{character:handling}}

实际数据分析工作中,经常需要处理字符串以便让数据整洁,符合分析需求。在我们常用的Excel或sql中也有处理字符串的经验,大部分时候截断、替换等基础实现就能满足我们的大部分字符处理需求。

Excel中自带的字符串函数\footnote{Excel中支持的{[}TEXT functions{]} (\url{https://support.microsoft.com/zh-cn/office/\%e6\%96\%87\%e6\%9c\%ac\%e5\%87\%bd\%e6\%95\%b0\%ef\%bc\%88\%e5\%8f\%82\%e8\%80\%83\%ef\%bc\%89-cccd86ad-547d-4ea9-a065-7bb697c2a56e?ui=zh-CN\&rs=zh-CN\&ad=CN})},如:\texttt{left},\texttt{len},\texttt{mid},\texttt{find},\texttt{Proper},\texttt{rept},\texttt{trim},\texttt{upper},\texttt{substitute},\texttt{concatenate},以及\texttt{Excle}2019新出的\texttt{concat},\texttt{TEXTJOIN}等字符函数,\texttt{TEXTJOIN}函数我个人比较喜欢用。在学习R的字符处理时候可以自行尝试实现以上相对应功能。

但是Excel中字符处理功能存在一定局限性,没有直接可用的正则表达式\footnote{正则表达式(regular expression)描述了一种字符串匹配的模式(pattern),查看帮助\texttt{?regex}。}函数,在正则表达式本身就很困难的情况下,在VBA中实现较复杂的函数将会难上加难。

字符处理本人觉得本质上就是将字符定位后去实现不同的操作。所以觉得难点在于字符串中字符的定位,而实现这个功能就需要用到正则表达式,所以字符处理真正的难点在于正则表达式的编写,但是在我看来正则表达式想要掌握,难度过高,我们秉着随用随查的态度对待即可。

因为大部分的数据分析工作者并不会面临太多复杂的字符处理工作,对大部分常规商业数据分析工作者面对的数据而言,字符处理可能仅仅只是合并、剔除、删除空格、倒序等基础操作。面对舆情监控,购物评价等纯文本情感分析工作,个人觉得对普通数据分析岗位有点超纲,所以本章节的字符处理仅仅是常规的字符处理。

\hypertarget{character:base-R}{%
\section{base R}\label{character:base-R}}

本部分简述base R中关于字符处理的常用函数。

\hypertarget{ux5355ux53ccux5f15ux53f7}{%
\subsection{单双引号}\label{ux5355ux53ccux5f15ux53f7}}

\texttt{R}语言中字符串输入时,可以使用单引号,也可以使用双引号,详情请看\texttt{?Quotes}。

\begin{itemize}
\item
  单双引号用法和意义没有差别
\item
  R中推荐使用双引号分隔符,打印、显示时都是用双引号
\item
  单引号字符串通常用在字符串内包含双引号时,如用R执行sql字符串代码时
\item
  R-4.0之后引入的R字符{[}newfeatures{]},让单双引号基本没区别
\end{itemize}

R中的字符用单双引号都可创建,如下所示:

\begin{Shaded}
\begin{Highlighting}[]
\NormalTok{x }\OtherTok{\textless{}{-}} \StringTok{"R语言"}
\NormalTok{x}
\CommentTok{\#\textgreater{} [1] "R语言"}
\NormalTok{x }\OtherTok{\textless{}{-}} \StringTok{\textquotesingle{}R语言\textquotesingle{}}
\NormalTok{x}
\CommentTok{\#\textgreater{} [1] "R语言"}
\end{Highlighting}
\end{Shaded}

\hypertarget{ux8f6cux4e49}{%
\subsection{转义}\label{ux8f6cux4e49}}

要在字符串中包含单引号或双引号,需用~转义它,即遇到特殊符号时需要转义,如果不正确使用转义,可能会报错,如下所示:

\begin{Shaded}
\begin{Highlighting}[]
\FunctionTok{paste}\NormalTok{(}\StringTok{""","}\NormalTok{abc}\StringTok{",sep="")}
\StringTok{\#\textgreater{} 错误: unexpected string constant in "}\FunctionTok{paste}\NormalTok{(}\StringTok{""",""}
\end{Highlighting}
\end{Shaded}

R语言中使用``"把特定的字符转义为特殊字符,例如``\t”是制表符,``\n''是换行符,或者是``\r\n''(系统差异)。想要正确显示``''',需使用转义,如下所示:

\begin{Shaded}
\begin{Highlighting}[]
\NormalTok{char }\OtherTok{\textless{}{-}} \StringTok{"我是一名}\SpecialCharTok{\textbackslash{}\textquotesingle{}}\StringTok{小学生}\SpecialCharTok{\textbackslash{}\textquotesingle{}}\StringTok{"} 
\NormalTok{char}
\CommentTok{\#\textgreater{} [1] "我是一名\textquotesingle{}小学生\textquotesingle{}"}
\end{Highlighting}
\end{Shaded}

打印会显示转义符,主要是因为R语言内数据存储和打印是不一样的,运行结果为存储的数据形式,不是打印形式。

要查看字符串的原始内容,可使用writeLines()或cat()

\begin{Shaded}
\begin{Highlighting}[]
\NormalTok{x }\OtherTok{\textless{}{-}} \FunctionTok{c}\NormalTok{(}\StringTok{"}\SpecialCharTok{\textbackslash{}"}\StringTok{"}\NormalTok{, }\StringTok{"}\SpecialCharTok{\textbackslash{}\textbackslash{}}\StringTok{"}\NormalTok{)}
\NormalTok{x}
\CommentTok{\#\textgreater{} [1] "\textbackslash{}"" "\textbackslash{}\textbackslash{}"}

\FunctionTok{writeLines}\NormalTok{(x)}
\CommentTok{\#\textgreater{} "}
\CommentTok{\#\textgreater{} \textbackslash{}}
\FunctionTok{writeLines}\NormalTok{(char)}
\CommentTok{\#\textgreater{} 我是一名\textquotesingle{}小学生\textquotesingle{}}
\FunctionTok{cat}\NormalTok{(char)}
\CommentTok{\#\textgreater{} 我是一名\textquotesingle{}小学生\textquotesingle{}}
\end{Highlighting}
\end{Shaded}

\hypertarget{ux5e38ux7528ux51fdux6570}{%
\subsection{常用函数}\label{ux5e38ux7528ux51fdux6570}}

\begin{itemize}
\tightlist
\item
  字符数量
\end{itemize}

\begin{Shaded}
\begin{Highlighting}[]
\NormalTok{s }\OtherTok{\textless{}{-}} \StringTok{\textquotesingle{}abcdefg\textquotesingle{}}
\FunctionTok{nchar}\NormalTok{(s)}
\CommentTok{\#\textgreater{} [1] 7}
\end{Highlighting}
\end{Shaded}

\begin{itemize}
\tightlist
\item
  大小写
\end{itemize}

\begin{Shaded}
\begin{Highlighting}[]
\FunctionTok{toupper}\NormalTok{(}\StringTok{\textquotesingle{}abc\textquotesingle{}}\NormalTok{)}
\CommentTok{\#\textgreater{} [1] "ABC"}
\FunctionTok{tolower}\NormalTok{(}\StringTok{\textquotesingle{}ABC\textquotesingle{}}\NormalTok{)}
\CommentTok{\#\textgreater{} [1] "abc"}
\end{Highlighting}
\end{Shaded}

\begin{itemize}
\tightlist
\item
  拼接
\end{itemize}

函数\texttt{paste()}将不同的字符向量拼接组合起来,返回的数据类型是字符向量。其中分割参数sep默认值是" "(空格)。collapse参数是使用指定的字符把各元素拼接在一次成一个单独的字符串。

\begin{Shaded}
\begin{Highlighting}[]
\FunctionTok{paste}\NormalTok{(}\StringTok{\textquotesingle{}a\textquotesingle{}}\NormalTok{,}\StringTok{\textquotesingle{}b\textquotesingle{}}\NormalTok{,}\StringTok{\textquotesingle{}d\textquotesingle{}}\NormalTok{)}
\CommentTok{\#\textgreater{} [1] "a b d"}
\FunctionTok{paste}\NormalTok{(}\StringTok{\textquotesingle{}a\textquotesingle{}}\NormalTok{,}\StringTok{\textquotesingle{}b\textquotesingle{}}\NormalTok{,}\StringTok{\textquotesingle{}d\textquotesingle{}}\NormalTok{,}\AttributeTok{sep =} \StringTok{\textquotesingle{}\textquotesingle{}}\NormalTok{)}
\CommentTok{\#\textgreater{} [1] "abd"}
\FunctionTok{paste}\NormalTok{(}\StringTok{\textquotesingle{}a\textquotesingle{}}\NormalTok{,}\StringTok{\textquotesingle{}b\textquotesingle{}}\NormalTok{,}\StringTok{\textquotesingle{}d\textquotesingle{}}\NormalTok{,}\AttributeTok{sep =} \StringTok{\textquotesingle{}\_\textquotesingle{}}\NormalTok{)}
\CommentTok{\#\textgreater{} [1] "a\_b\_d"}
\FunctionTok{paste}\NormalTok{(}\FunctionTok{c}\NormalTok{(}\StringTok{\textquotesingle{}a\textquotesingle{}}\NormalTok{,}\StringTok{\textquotesingle{}b\textquotesingle{}}\NormalTok{),}\FunctionTok{c}\NormalTok{(}\StringTok{\textquotesingle{}d\textquotesingle{}}\NormalTok{,}\StringTok{\textquotesingle{}e\textquotesingle{}}\NormalTok{),}\AttributeTok{collapse =} \StringTok{\textquotesingle{}\_\textquotesingle{}}\NormalTok{)}
\CommentTok{\#\textgreater{} [1] "a d\_b e"}
\end{Highlighting}
\end{Shaded}

大家可以自行了解\texttt{paste}和\texttt{paste0}两个函数的差异。

\begin{itemize}
\tightlist
\item
  截取
\end{itemize}

\texttt{substr}函数用于字符串截取子字符串,start,stop参数是整数。

\begin{Shaded}
\begin{Highlighting}[]
\CommentTok{\# 空格占据一个长度}
\FunctionTok{substr}\NormalTok{(}\StringTok{"R is free software"}\NormalTok{ ,}\AttributeTok{start =} \DecValTok{1}\NormalTok{,}\AttributeTok{stop =} \DecValTok{6}\NormalTok{)}
\CommentTok{\#\textgreater{} [1] "R is f"}
\end{Highlighting}
\end{Shaded}

\begin{itemize}
\tightlist
\item
  分割
\end{itemize}

函数\texttt{strsplit}按照指定的字符把字符分割成子字符。参数x被分割的字符,split是用于分割的字符标准,fixed是否完全匹配分隔符,默认是TRUE,完全匹配模式,当fixed参数为FALSE时,表名split参数是正则表达式,使用正则匹配。

\begin{Shaded}
\begin{Highlighting}[]
\FunctionTok{strsplit}\NormalTok{(x,split,fixed,perl,useBytes)}
\end{Highlighting}
\end{Shaded}

strsplit函数返回的结果是列表,大部分时候需要向量化后使用。

\begin{Shaded}
\begin{Highlighting}[]
\FunctionTok{strsplit}\NormalTok{(}\StringTok{\textquotesingle{}广东省{-}深圳市{-}宝安区\textquotesingle{}}\NormalTok{,}\AttributeTok{split=}\StringTok{\textquotesingle{}{-}\textquotesingle{}}\NormalTok{)}
\CommentTok{\#\textgreater{} [[1]]}
\CommentTok{\#\textgreater{} [1] "广东省" "深圳市" "宝安区"}
\CommentTok{\# 向量化}
\CommentTok{\# unlist(strsplit(\textquotesingle{}广东省{-}深圳市{-}宝安区\textquotesingle{},split=\textquotesingle{}{-}\textquotesingle{}))}
\end{Highlighting}
\end{Shaded}

官方手册中提供一个字符倒叙的自定义编写的函数:

\begin{Shaded}
\begin{Highlighting}[]
\NormalTok{strReverse }\OtherTok{\textless{}{-}} \ControlFlowTok{function}\NormalTok{(x) }\FunctionTok{sapply}\NormalTok{(}\FunctionTok{lapply}\NormalTok{(}\FunctionTok{strsplit}\NormalTok{(x, }\ConstantTok{NULL}\NormalTok{), rev), paste, }\AttributeTok{collapse =} \StringTok{""}\NormalTok{)}
\FunctionTok{strReverse}\NormalTok{(}\FunctionTok{c}\NormalTok{(}\StringTok{"abc"}\NormalTok{, }\StringTok{"Statistics"}\NormalTok{))}
\CommentTok{\#\textgreater{} [1] "cba"        "scitsitatS"}
\end{Highlighting}
\end{Shaded}

\hypertarget{newfeatures}{%
\subsection{新特性}\label{newfeatures}}

该特性让反斜杠或单引号和双引号书写变得容易。用法r``(\ldots)'',括号中可以是任意字符,详情请看\texttt{?Quotes}。

\begin{Shaded}
\begin{Highlighting}[]
\CommentTok{\# windows下路径 ,不用转义路径复制直接可用}
\NormalTok{char }\OtherTok{\textless{}{-}}\NormalTok{ r}\StringTok{"(C:\textbackslash{}Users\textbackslash{}zhongyf\textbackslash{}Desktop\textbackslash{}Rbook)"} 
\NormalTok{char}
\CommentTok{\#\textgreater{} [1] "C:\textbackslash{}\textbackslash{}Users\textbackslash{}\textbackslash{}zhongyf\textbackslash{}\textbackslash{}Desktop\textbackslash{}\textbackslash{}Rbook"}
\end{Highlighting}
\end{Shaded}

\begin{Shaded}
\begin{Highlighting}[]
\NormalTok{char }\OtherTok{\textless{}{-}} \StringTok{"我是一名}\SpecialCharTok{\textbackslash{}\textquotesingle{}}\StringTok{小学生}\SpecialCharTok{\textbackslash{}\textquotesingle{}}\StringTok{"} 
\FunctionTok{cat}\NormalTok{(char)}
\CommentTok{\#\textgreater{} 我是一名\textquotesingle{}小学生\textquotesingle{}}

\NormalTok{char }\OtherTok{\textless{}{-}}\NormalTok{ r}\StringTok{"(我是一名\textquotesingle{}R语言\textquotesingle{}学习者)"}
\FunctionTok{cat}\NormalTok{(char)}
\CommentTok{\#\textgreater{} 我是一名\textquotesingle{}R语言\textquotesingle{}学习者}
\end{Highlighting}
\end{Shaded}

\textbf{注意该特性需要在R-4.0.0之后的版本中使用}

\hypertarget{character:stringr-packages}{%
\section{stringr}\label{character:stringr-packages}}

本小节介绍R包\texttt{stringr},stringr处理字符相对简单,并且是tidyverse系列的一部分,是很成熟的R包,API功能稳定。stringr是基于\texttt{stringi}之上构建的,stringr包集合了常见字符功能函数,如果发现stringr缺少某些功能可以查看\texttt{stringi}包。

如上文所说,字符串处理的难点,个人觉得在于正则表达式的掌握程度。对大部分常规商业数据分析工作者的面对的表格数据而言,字符处理可能仅仅只是合并、剔除、删除空格、倒叙等基础操作,所以stringr包基本满足字符处理需求。

\texttt{stringr}项目地址:\url{https://github.com/tidyverse/stringr/}

如果不熟悉R中的字符串,可以从\href{https://r4ds.had.co.nz/strings.html}{R for Data Science}的字符串部分开始学习,

本小节的部分案例照搬\href{https://r4ds.had.co.nz/strings.html}{R for Data Science}。

\hypertarget{stringr-install}{%
\subsection{安装}\label{stringr-install}}

\begin{Shaded}
\begin{Highlighting}[]
\CommentTok{\# Install the released version from CRAN:}
\FunctionTok{install.packages}\NormalTok{(}\StringTok{"stringr"}\NormalTok{)}

\CommentTok{\# Install the cutting edge development version from GitHub:}
\CommentTok{\# install.packages("devtools")}
\NormalTok{devtools}\SpecialCharTok{::}\FunctionTok{install\_github}\NormalTok{(}\StringTok{"tidyverse/stringr"}\NormalTok{)}
\end{Highlighting}
\end{Shaded}

\hypertarget{stringr-usage}{%
\subsection{基本使用}\label{stringr-usage}}

stringr包中所有的函数都已\texttt{str\_}开头,让字符做第一个参数。

\begin{itemize}
\tightlist
\item
  字符串长度
\end{itemize}

\begin{Shaded}
\begin{Highlighting}[]
\FunctionTok{library}\NormalTok{(stringr)}
\NormalTok{char }\OtherTok{\textless{}{-}} \StringTok{"我是R语言学习者"}
\FunctionTok{str\_length}\NormalTok{(char)}
\CommentTok{\#\textgreater{} [1] 8}
\CommentTok{\# 向量化}
\FunctionTok{str\_length}\NormalTok{(}\FunctionTok{c}\NormalTok{(}\StringTok{"a"}\NormalTok{, }\StringTok{"R for data science"}\NormalTok{, }\ConstantTok{NA}\NormalTok{))}
\CommentTok{\#\textgreater{} [1]  1 18 NA}
\end{Highlighting}
\end{Shaded}

\begin{itemize}
\tightlist
\item
  连接字符串
\end{itemize}

R中字符串不像python中可以用加号连接字符串,如下所示:

R 版本

\begin{Shaded}
\begin{Highlighting}[]
\CommentTok{\#base R}
\FunctionTok{paste0}\NormalTok{(}\StringTok{\textquotesingle{}a\textquotesingle{}}\NormalTok{,}\StringTok{\textquotesingle{}b\textquotesingle{}}\NormalTok{)}
\CommentTok{\#\textgreater{} [1] "ab"}

\CommentTok{\#stringr}
\FunctionTok{str\_c}\NormalTok{(}\StringTok{"a"}\NormalTok{,}\StringTok{"b"}\NormalTok{)}
\CommentTok{\#\textgreater{} [1] "ab"}
\FunctionTok{str\_c}\NormalTok{(}\StringTok{"a"}\NormalTok{, }\StringTok{"b"}\NormalTok{, }\AttributeTok{sep =} \StringTok{", "}\NormalTok{) }\CommentTok{\#sep 参数控制分隔符}
\CommentTok{\#\textgreater{} [1] "a, b"}
\end{Highlighting}
\end{Shaded}

Python 版本

\begin{Shaded}
\begin{Highlighting}[]
\CommentTok{\textquotesingle{}a\textquotesingle{}} \OperatorTok{+} \StringTok{\textquotesingle{}b\textquotesingle{}}
\CommentTok{\#\textgreater{} \textquotesingle{}ab\textquotesingle{}}
\end{Highlighting}
\end{Shaded}

多个字符串合并为一个字符,\texttt{stringr}中的函数都是向量化的,合并一个和多个字符都是同样道理。

\begin{Shaded}
\begin{Highlighting}[]
\CommentTok{\#base R}
\FunctionTok{paste0}\NormalTok{(}\FunctionTok{c}\NormalTok{(}\StringTok{\textquotesingle{}a\textquotesingle{}}\NormalTok{,}\StringTok{\textquotesingle{}b\textquotesingle{}}\NormalTok{,}\StringTok{\textquotesingle{}d\textquotesingle{}}\NormalTok{,}\StringTok{\textquotesingle{}e\textquotesingle{}}\NormalTok{),}\AttributeTok{collapse =} \StringTok{\textquotesingle{},\textquotesingle{}}\NormalTok{)}
\CommentTok{\#\textgreater{} [1] "a,b,d,e"}
\CommentTok{\#stringr}
\FunctionTok{str\_c}\NormalTok{(}\FunctionTok{c}\NormalTok{(}\StringTok{\textquotesingle{}a\textquotesingle{}}\NormalTok{,}\StringTok{\textquotesingle{}b\textquotesingle{}}\NormalTok{,}\StringTok{\textquotesingle{}d\textquotesingle{}}\NormalTok{,}\StringTok{\textquotesingle{}e\textquotesingle{}}\NormalTok{),}\AttributeTok{collapse =} \StringTok{\textquotesingle{},\textquotesingle{}}\NormalTok{)  }\CommentTok{\#collapse 参数控制}
\CommentTok{\#\textgreater{} [1] "a,b,d,e"}
\end{Highlighting}
\end{Shaded}

\begin{itemize}
\tightlist
\item
  移除
\end{itemize}

在正则表达式中~有特殊含义,有时需要两个~,多体会下面这段,代码实现移除``\textbar\textbar{}''的功能。

\begin{Shaded}
\begin{Highlighting}[]
\FunctionTok{library}\NormalTok{(stringr)}
\FunctionTok{str\_remove}\NormalTok{(}\AttributeTok{string =} \StringTok{\textquotesingle{}a||b\textquotesingle{}}\NormalTok{,}\AttributeTok{pattern =} \StringTok{"}\SpecialCharTok{\textbackslash{}\textbackslash{}}\StringTok{|}\SpecialCharTok{\textbackslash{}\textbackslash{}}\StringTok{|"}\NormalTok{)}
\CommentTok{\#\textgreater{} [1] "ab"}
\end{Highlighting}
\end{Shaded}

另外常见的\textbackslash n, \textbackslash t需要被转义处理,在字符清洗,如小说语义分析,网页爬虫后整理等数据清洗过程中经常用到.

\hypertarget{stringr-functions}{%
\subsection{常用函数}\label{stringr-functions}}

\hypertarget{ux622aux53d6ux5b57ux7b26}{%
\subsubsection{截取字符}\label{ux622aux53d6ux5b57ux7b26}}

与\texttt{Excle}中\texttt{left},\texttt{mid},\texttt{right}函数功能类似

str\_sub() 函数 三个参数:

string:需要被截取的字符串

start: 默认1L,即从最开始截取

end:默认-1L,即截取到最后

\begin{Shaded}
\begin{Highlighting}[]
\CommentTok{\#注意end 3 和 {-}3的区别}
\FunctionTok{str\_sub}\NormalTok{(}\AttributeTok{string =} \StringTok{\textquotesingle{}我是R语言学习者\textquotesingle{}}\NormalTok{,}\AttributeTok{start =} \DecValTok{2}\NormalTok{,}\AttributeTok{end =} \DecValTok{3}\NormalTok{)}
\CommentTok{\#\textgreater{} [1] "是R"}
\FunctionTok{str\_sub}\NormalTok{(}\AttributeTok{string =} \StringTok{\textquotesingle{}我是R语言学习者\textquotesingle{}}\NormalTok{,}\AttributeTok{start =} \DecValTok{2}\NormalTok{,}\AttributeTok{end =} \SpecialCharTok{{-}}\DecValTok{3}\NormalTok{)}
\CommentTok{\#\textgreater{} [1] "是R语言学"}
\end{Highlighting}
\end{Shaded}

\hypertarget{ux5339ux914dux5b57ux7b26}{%
\subsubsection{匹配字符}\label{ux5339ux914dux5b57ux7b26}}

查看函数帮助文档,str\_match()按照指定pattern(正则表达式)查找字符。困难点在于正则表达式的编写。

\begin{Shaded}
\begin{Highlighting}[]
\NormalTok{?}\FunctionTok{str\_match}\NormalTok{()}
\NormalTok{?}\FunctionTok{str\_match\_all}\NormalTok{()}
\NormalTok{?}\FunctionTok{str\_extract}\NormalTok{()}
\NormalTok{?}\FunctionTok{str\_extract\_all}\NormalTok{()}
\end{Highlighting}
\end{Shaded}

str\_extract()函数返回向量,str\_match()函数返回矩阵.

\begin{Shaded}
\begin{Highlighting}[]
\CommentTok{\#原文来源烽火戏诸侯的\textless{}剑来\textgreater{}}
\NormalTok{strings }\OtherTok{\textless{}{-}} \FunctionTok{c}\NormalTok{(}\StringTok{\textquotesingle{}陈平安放下新折的那根桃枝,吹灭蜡烛,走出屋子后,坐在台阶上,仰头望去,星空璀璨.\textquotesingle{}}\NormalTok{) }
\FunctionTok{str\_extract}\NormalTok{(strings,}\StringTok{\textquotesingle{}陈平安\textquotesingle{}}\NormalTok{)}
\CommentTok{\#\textgreater{} [1] "陈平安"}
\FunctionTok{str\_match}\NormalTok{(strings,}\StringTok{\textquotesingle{}陈平安\textquotesingle{}}\NormalTok{)}
\CommentTok{\#\textgreater{}      [,1]    }
\CommentTok{\#\textgreater{} [1,] "陈平安"}
\end{Highlighting}
\end{Shaded}

\begin{itemize}
\tightlist
\item
  匹配中文
\end{itemize}

匹配中文的正则表达式\[\u4e00-\u9fa5\]

\begin{Shaded}
\begin{Highlighting}[]
\FunctionTok{str\_extract\_all}\NormalTok{(strings,}\StringTok{\textquotesingle{}[\textbackslash{}u4e00{-}\textbackslash{}u9fa5]\textquotesingle{}}\NormalTok{) }\CommentTok{\#返回list}
\CommentTok{\#\textgreater{} [[1]]}
\CommentTok{\#\textgreater{}  [1] "陈" "平" "安" "放" "下" "新" "折" "的" "那" "根" "桃" "枝" "吹" "灭" "蜡"}
\CommentTok{\#\textgreater{} [16] "烛" "走" "出" "屋" "子" "后" "坐" "在" "台" "阶" "上" "仰" "头" "望" "去"}
\CommentTok{\#\textgreater{} [31] "星" "空" "璀" "璨"}
\end{Highlighting}
\end{Shaded}

\begin{itemize}
\tightlist
\item
  匹配数字或英文
\end{itemize}

查找数字的正则表达式{[}0-9{]};查找英文的正则表达式:{[}a-zA-Z{]}

\begin{Shaded}
\begin{Highlighting}[]
\NormalTok{strings }\OtherTok{\textless{}{-}} \FunctionTok{c}\NormalTok{(}\StringTok{\textquotesingle{}00123545\textquotesingle{}}\NormalTok{,}\StringTok{\textquotesingle{}LOL league of legends\textquotesingle{}}\NormalTok{)}
\FunctionTok{str\_extract\_all}\NormalTok{(strings,}\StringTok{\textquotesingle{}[0{-}9]\textquotesingle{}}\NormalTok{)}
\CommentTok{\#\textgreater{} [[1]]}
\CommentTok{\#\textgreater{} [1] "0" "0" "1" "2" "3" "5" "4" "5"}
\CommentTok{\#\textgreater{} }
\CommentTok{\#\textgreater{} [[2]]}
\CommentTok{\#\textgreater{} character(0)}
\FunctionTok{str\_extract\_all}\NormalTok{(strings,}\StringTok{\textquotesingle{}[a{-}zA{-}Z]\textquotesingle{}}\NormalTok{) }
\CommentTok{\#\textgreater{} [[1]]}
\CommentTok{\#\textgreater{} character(0)}
\CommentTok{\#\textgreater{} }
\CommentTok{\#\textgreater{} [[2]]}
\CommentTok{\#\textgreater{}  [1] "L" "O" "L" "l" "e" "a" "g" "u" "e" "o" "f" "l" "e" "g" "e" "n" "d" "s"}
\end{Highlighting}
\end{Shaded}

\hypertarget{ux6dfbux52a0ux5b57ux7b26}{%
\subsubsection{添加字符}\label{ux6dfbux52a0ux5b57ux7b26}}

str\_pad() 函数向字符串添加字符

像工作中处理月份的时候,1,2,3,4,5,6,7,8,9,10,11,12变成01,02,03,04,05,06,07,08,09,10,11,12.按照日期时间输出文件名称,如下所示:

\begin{Shaded}
\begin{Highlighting}[]
\FunctionTok{str\_pad}\NormalTok{(}\AttributeTok{string =} \DecValTok{1}\SpecialCharTok{:}\DecValTok{12}\NormalTok{,}\AttributeTok{width =} \DecValTok{2}\NormalTok{,}\AttributeTok{side =} \StringTok{\textquotesingle{}left\textquotesingle{}}\NormalTok{,}\AttributeTok{pad =} \StringTok{\textquotesingle{}0\textquotesingle{}}\NormalTok{)}
\CommentTok{\#\textgreater{}  [1] "01" "02" "03" "04" "05" "06" "07" "08" "09" "10" "11" "12"}
\end{Highlighting}
\end{Shaded}

\hypertarget{ux53bbux9664ux7a7aux683c}{%
\subsubsection{去除空格}\label{ux53bbux9664ux7a7aux683c}}

与\texttt{excel}中\texttt{trim}函数功能类似,剔除字符中的空格,但是不可以剔除字符中的空格

\begin{Shaded}
\begin{Highlighting}[]
\CommentTok{\# side 可选 both  left right}
\FunctionTok{str\_trim}\NormalTok{(}\StringTok{\textquotesingle{} ab af \textquotesingle{}}\NormalTok{,}\AttributeTok{side =} \StringTok{\textquotesingle{}both\textquotesingle{}}\NormalTok{)}
\CommentTok{\#\textgreater{} [1] "ab af"}
\end{Highlighting}
\end{Shaded}

\hypertarget{ux5206ux5272ux5b57ux7b26}{%
\subsubsection{分割字符}\label{ux5206ux5272ux5b57ux7b26}}

\texttt{str\_split()}处理后的结果是列表

\begin{Shaded}
\begin{Highlighting}[]
\CommentTok{\# 得到列表,需要向量化}
\FunctionTok{str\_split}\NormalTok{(}\StringTok{"a,b,d,e"}\NormalTok{,}\AttributeTok{pattern =} \StringTok{\textquotesingle{},\textquotesingle{}}\NormalTok{)}
\CommentTok{\#\textgreater{} [[1]]}
\CommentTok{\#\textgreater{} [1] "a" "b" "d" "e"}

\FunctionTok{str\_split}\NormalTok{(}\StringTok{\textquotesingle{}ab||cd\textquotesingle{}}\NormalTok{,}\StringTok{\textquotesingle{}}\SpecialCharTok{\textbackslash{}\textbackslash{}}\StringTok{|}\SpecialCharTok{\textbackslash{}\textbackslash{}}\StringTok{|\textquotesingle{}}\NormalTok{) }\SpecialCharTok{\%\textgreater{}\%} \FunctionTok{unlist}\NormalTok{()}
\CommentTok{\#\textgreater{} [1] "ab" "cd"}
\CommentTok{\# same above}
\CommentTok{\#str\_split(\textquotesingle{}ab||cd\textquotesingle{},\textquotesingle{}\textbackslash{}\textbackslash{}|\textbackslash{}\textbackslash{}|\textquotesingle{}) \%\textgreater{}\% purrr::as\_vector()}
\end{Highlighting}
\end{Shaded}

当待处理的字符串是字符串向量时,得到的列表长度与向量长度一致

\begin{Shaded}
\begin{Highlighting}[]
\NormalTok{fruits }\OtherTok{\textless{}{-}} \FunctionTok{c}\NormalTok{(}
  \StringTok{"apples and oranges and pears and bananas"}\NormalTok{,}
  \StringTok{"pineapples and mangos and guavas"}
\NormalTok{)}

\FunctionTok{str\_split}\NormalTok{(fruits, }\StringTok{" and "}\NormalTok{)}
\CommentTok{\#\textgreater{} [[1]]}
\CommentTok{\#\textgreater{} [1] "apples"  "oranges" "pears"   "bananas"}
\CommentTok{\#\textgreater{} }
\CommentTok{\#\textgreater{} [[2]]}
\CommentTok{\#\textgreater{} [1] "pineapples" "mangos"     "guavas"}
\end{Highlighting}
\end{Shaded}

\hypertarget{ux66ffux6362ux5b57ux7b26}{%
\subsubsection{替换字符}\label{ux66ffux6362ux5b57ux7b26}}

\texttt{str\_replace()},\texttt{str\_replace\_all()}函数用来替换字符

\begin{Shaded}
\begin{Highlighting}[]
\NormalTok{fruits }\OtherTok{\textless{}{-}} \FunctionTok{c}\NormalTok{(}\StringTok{"one apple"}\NormalTok{, }\StringTok{"two pears"}\NormalTok{, }\StringTok{"three bananas"}\NormalTok{)}
\FunctionTok{str\_replace}\NormalTok{(fruits, }\StringTok{"[aeiou]"}\NormalTok{, }\StringTok{"{-}"}\NormalTok{)}
\CommentTok{\#\textgreater{} [1] "{-}ne apple"     "tw{-} pears"     "thr{-}e bananas"}
\FunctionTok{str\_replace\_all}\NormalTok{(fruits, }\StringTok{"[aeiou]"}\NormalTok{, }\StringTok{"{-}"}\NormalTok{)}
\CommentTok{\#\textgreater{} [1] "{-}n{-} {-}ppl{-}"     "tw{-} p{-}{-}rs"     "thr{-}{-} b{-}n{-}n{-}s"}
\end{Highlighting}
\end{Shaded}

\hypertarget{ux79fbux9664ux5b57ux7b26}{%
\subsubsection{移除字符}\label{ux79fbux9664ux5b57ux7b26}}

\texttt{str\_remove()},\texttt{str\_remove\_all()}移除字符。本人常用该函数剔除文本中的空格。

\begin{Shaded}
\begin{Highlighting}[]
\NormalTok{fruits }\OtherTok{\textless{}{-}} \FunctionTok{c}\NormalTok{(}\StringTok{"one apple"}\NormalTok{, }\StringTok{"two pears"}\NormalTok{, }\StringTok{"three bananas"}\NormalTok{)}
\FunctionTok{str\_remove}\NormalTok{(fruits, }\StringTok{"[aeiou]"}\NormalTok{)}
\CommentTok{\#\textgreater{} [1] "ne apple"     "tw pears"     "thre bananas"}
\FunctionTok{str\_remove\_all}\NormalTok{(fruits, }\StringTok{"[aeiou]"}\NormalTok{)}
\CommentTok{\#\textgreater{} [1] "n ppl"    "tw prs"   "thr bnns"}
\end{Highlighting}
\end{Shaded}

移除文本中空格

\begin{Shaded}
\begin{Highlighting}[]
\FunctionTok{str\_replace\_all}\NormalTok{(}\AttributeTok{string =} \StringTok{\textquotesingle{} d a  b \textquotesingle{}}\NormalTok{,}\AttributeTok{pattern =} \StringTok{\textquotesingle{} \textquotesingle{}}\NormalTok{,}\AttributeTok{replacement =} \StringTok{\textquotesingle{}\textquotesingle{}}\NormalTok{)}
\CommentTok{\#\textgreater{} [1] "dab"}
\end{Highlighting}
\end{Shaded}

\hypertarget{ux5b57ux7b26ux6392ux5e8f}{%
\subsubsection{字符排序}\label{ux5b57ux7b26ux6392ux5e8f}}

numeric参数决定是否按照数字排序。

\begin{Shaded}
\begin{Highlighting}[]
\FunctionTok{str\_order}\NormalTok{(x, }\AttributeTok{decreasing =} \ConstantTok{FALSE}\NormalTok{, }\AttributeTok{na\_last =} \ConstantTok{TRUE}\NormalTok{, }\AttributeTok{locale =} \StringTok{"en"}\NormalTok{,}
  \AttributeTok{numeric =} \ConstantTok{FALSE}\NormalTok{, ...)}

\FunctionTok{str\_sort}\NormalTok{(x, }\AttributeTok{decreasing =} \ConstantTok{FALSE}\NormalTok{, }\AttributeTok{na\_last =} \ConstantTok{TRUE}\NormalTok{, }\AttributeTok{locale =} \StringTok{"en"}\NormalTok{,}
  \AttributeTok{numeric =} \ConstantTok{FALSE}\NormalTok{, ...)}
\end{Highlighting}
\end{Shaded}

\begin{Shaded}
\begin{Highlighting}[]
\FunctionTok{str\_order}\NormalTok{(letters)}
\CommentTok{\#\textgreater{}  [1]  1  2  3  4  5  6  7  8  9 10 11 12 13 14 15 16 17 18 19 20 21 22 23 24 25}
\CommentTok{\#\textgreater{} [26] 26}
\FunctionTok{str\_sort}\NormalTok{(letters)}
\CommentTok{\#\textgreater{}  [1] "a" "b" "c" "d" "e" "f" "g" "h" "i" "j" "k" "l" "m" "n" "o" "p" "q" "r" "s"}
\CommentTok{\#\textgreater{} [20] "t" "u" "v" "w" "x" "y" "z"}
\end{Highlighting}
\end{Shaded}

numeric参数

\begin{Shaded}
\begin{Highlighting}[]
\NormalTok{x }\OtherTok{\textless{}{-}} \FunctionTok{c}\NormalTok{(}\StringTok{"100a10"}\NormalTok{, }\StringTok{"100a5"}\NormalTok{, }\StringTok{"2b"}\NormalTok{, }\StringTok{"2a"}\NormalTok{)}
\FunctionTok{str\_sort}\NormalTok{(x)}
\CommentTok{\#\textgreater{} [1] "100a10" "100a5"  "2a"     "2b"}
\FunctionTok{str\_sort}\NormalTok{(x, }\AttributeTok{numeric =} \ConstantTok{TRUE}\NormalTok{)}
\CommentTok{\#\textgreater{} [1] "2a"     "2b"     "100a5"  "100a10"}
\end{Highlighting}
\end{Shaded}

\hypertarget{ux63d0ux53d6ux5355ux8bcd}{%
\subsubsection{提取单词}\label{ux63d0ux53d6ux5355ux8bcd}}

从句子中提取单词。

\begin{itemize}
\tightlist
\item
  参数
\end{itemize}

\begin{Shaded}
\begin{Highlighting}[]
\FunctionTok{word}\NormalTok{(string, }\AttributeTok{start =}\NormalTok{ 1L, }\AttributeTok{end =}\NormalTok{ start, }\AttributeTok{sep =} \FunctionTok{fixed}\NormalTok{(}\StringTok{" "}\NormalTok{))}
\end{Highlighting}
\end{Shaded}

\begin{itemize}
\tightlist
\item
  案例
\end{itemize}

\begin{Shaded}
\begin{Highlighting}[]
\NormalTok{sentences }\OtherTok{\textless{}{-}} \FunctionTok{c}\NormalTok{(}\StringTok{"Jane saw a cat"}\NormalTok{, }\StringTok{"Jane sat down"}\NormalTok{)}
\FunctionTok{word}\NormalTok{(sentences, }\DecValTok{2}\NormalTok{, }\SpecialCharTok{{-}}\DecValTok{1}\NormalTok{)}
\CommentTok{\#\textgreater{} [1] "saw a cat" "sat down"}
\FunctionTok{word}\NormalTok{(sentences[}\DecValTok{1}\NormalTok{], }\DecValTok{1}\SpecialCharTok{:}\DecValTok{3}\NormalTok{, }\SpecialCharTok{{-}}\DecValTok{1}\NormalTok{)}
\CommentTok{\#\textgreater{} [1] "Jane saw a cat" "saw a cat"      "a cat"}
\end{Highlighting}
\end{Shaded}

指定分隔符

\begin{Shaded}
\begin{Highlighting}[]
\CommentTok{\# Can define words by other separators}
\NormalTok{str }\OtherTok{\textless{}{-}} \StringTok{\textquotesingle{}abc.def..123.4568.999\textquotesingle{}}
\FunctionTok{word}\NormalTok{(str, }\DecValTok{1}\NormalTok{, }\AttributeTok{sep =} \FunctionTok{fixed}\NormalTok{(}\StringTok{\textquotesingle{}..\textquotesingle{}}\NormalTok{))}
\CommentTok{\#\textgreater{} [1] "abc.def"}
\FunctionTok{word}\NormalTok{(str, }\DecValTok{2}\NormalTok{, }\AttributeTok{sep =} \FunctionTok{fixed}\NormalTok{(}\StringTok{\textquotesingle{}..\textquotesingle{}}\NormalTok{))}
\CommentTok{\#\textgreater{} [1] "123.4568.999"}
\end{Highlighting}
\end{Shaded}

\hypertarget{ux5176ux4ed6ux51fdux6570}{%
\subsubsection{其他函数}\label{ux5176ux4ed6ux51fdux6570}}

\begin{itemize}
\tightlist
\item
  str\_subset() str\_which()
\end{itemize}

匹配字符串本身行筛选时候能用

\begin{Shaded}
\begin{Highlighting}[]

\NormalTok{fruit }\OtherTok{\textless{}{-}} \FunctionTok{c}\NormalTok{(}\StringTok{"apple"}\NormalTok{, }\StringTok{"banana"}\NormalTok{, }\StringTok{"pear"}\NormalTok{, }\StringTok{"pinapple"}\NormalTok{)}
\FunctionTok{str\_subset}\NormalTok{(fruit, }\StringTok{"a"}\NormalTok{)}
\CommentTok{\#\textgreater{} [1] "apple"    "banana"   "pear"     "pinapple"}
\FunctionTok{str\_which}\NormalTok{(fruit, }\StringTok{"a"}\NormalTok{) }\CommentTok{\# 匹配字符首次出现的位置}
\CommentTok{\#\textgreater{} [1] 1 2 3 4}
\end{Highlighting}
\end{Shaded}

\begin{Shaded}
\begin{Highlighting}[]
\CommentTok{\#str\_which 是which(str\_detect(x,pattern))的包装}
\CommentTok{\#str\_which()}

\CommentTok{\#str\_subset是对x[str\_detect(x,pattern)]的包装}
\CommentTok{\#str\_subset()}

\CommentTok{\#筛选出字母行}
\FunctionTok{set.seed}\NormalTok{(}\DecValTok{24}\NormalTok{)}
\NormalTok{dt }\OtherTok{\textless{}{-}}\NormalTok{ data.table}\SpecialCharTok{::}\FunctionTok{data.table}\NormalTok{(}\AttributeTok{col=}\FunctionTok{sample}\NormalTok{(}\FunctionTok{c}\NormalTok{(letters,}\DecValTok{1}\SpecialCharTok{:}\DecValTok{10}\NormalTok{),}\DecValTok{100}\NormalTok{,}\AttributeTok{replace =}\NormalTok{ T))}
\FunctionTok{head}\NormalTok{(dt[}\FunctionTok{str\_which}\NormalTok{(col,}\AttributeTok{pattern =} \StringTok{\textquotesingle{}[a{-}z]\textquotesingle{}}\NormalTok{)])}
\end{Highlighting}
\end{Shaded}

\begin{itemize}
\tightlist
\item
  str\_dup()
\end{itemize}

复制字符串

\begin{Shaded}
\begin{Highlighting}[]
\NormalTok{fruit }\OtherTok{\textless{}{-}} \FunctionTok{c}\NormalTok{(}\StringTok{"apple"}\NormalTok{, }\StringTok{"pear"}\NormalTok{, }\StringTok{"banana"}\NormalTok{)}
\FunctionTok{str\_dup}\NormalTok{(fruit, }\DecValTok{2}\NormalTok{)}
\FunctionTok{str\_dup}\NormalTok{(fruit, }\DecValTok{1}\SpecialCharTok{:}\DecValTok{3}\NormalTok{)}
\FunctionTok{str\_c}\NormalTok{(}\StringTok{"ba"}\NormalTok{, }\FunctionTok{str\_dup}\NormalTok{(}\StringTok{"na"}\NormalTok{, }\DecValTok{0}\SpecialCharTok{:}\DecValTok{5}\NormalTok{))}
\end{Highlighting}
\end{Shaded}

\begin{itemize}
\tightlist
\item
  str\_starts() str\_ends()
\end{itemize}

从str\_detect()包装得到.

\begin{Shaded}
\begin{Highlighting}[]
\FunctionTok{str\_starts}\NormalTok{(}\StringTok{\textquotesingle{}abd\textquotesingle{}}\NormalTok{,}\StringTok{\textquotesingle{}a\textquotesingle{}}\NormalTok{)}
\CommentTok{\#\textgreater{} [1] TRUE}
\FunctionTok{str\_detect}\NormalTok{(}\StringTok{\textquotesingle{}abd\textquotesingle{}}\NormalTok{,}\StringTok{\textquotesingle{}\^{}a\textquotesingle{}}\NormalTok{)}
\CommentTok{\#\textgreater{} [1] TRUE}

\FunctionTok{str\_ends}\NormalTok{(}\StringTok{\textquotesingle{}abd\textquotesingle{}}\NormalTok{,}\StringTok{\textquotesingle{}d\textquotesingle{}}\NormalTok{)}
\CommentTok{\#\textgreater{} [1] TRUE}
\FunctionTok{str\_detect}\NormalTok{(}\StringTok{\textquotesingle{}abd\textquotesingle{}}\NormalTok{,}\StringTok{\textquotesingle{}a$\textquotesingle{}}\NormalTok{)}
\CommentTok{\#\textgreater{} [1] FALSE}
\end{Highlighting}
\end{Shaded}

\begin{itemize}
\tightlist
\item
  大小写转换
\end{itemize}

\begin{Shaded}
\begin{Highlighting}[]
\NormalTok{dog }\OtherTok{\textless{}{-}} \StringTok{"The quick brown dog"}
\FunctionTok{str\_to\_upper}\NormalTok{(dog)}
\CommentTok{\#\textgreater{} [1] "THE QUICK BROWN DOG"}
\FunctionTok{str\_to\_lower}\NormalTok{(dog)}
\CommentTok{\#\textgreater{} [1] "the quick brown dog"}
\FunctionTok{str\_to\_title}\NormalTok{(dog)}
\CommentTok{\#\textgreater{} [1] "The Quick Brown Dog"}
\FunctionTok{str\_to\_sentence}\NormalTok{(}\StringTok{"the quick brown dog"}\NormalTok{)}
\CommentTok{\#\textgreater{} [1] "The quick brown dog"}
\end{Highlighting}
\end{Shaded}

\hypertarget{character:application}{%
\section{综合运用}\label{character:application}}

\hypertarget{ux5b9eux73b0excelux51fdux6570}{%
\subsection{实现excel函数}\label{ux5b9eux73b0excelux51fdux6570}}

以下函数实现,仅仅只是从\texttt{stringr}包的函数上修改,并且没有完善,没有报错提示等的简陋版本,如果感兴趣的可以尝试利用\texttt{Rcpp}写出高性能版本的同功能函数。

\begin{itemize}
\tightlist
\item
  left
\end{itemize}

\begin{Shaded}
\begin{Highlighting}[]
\NormalTok{r\_left }\OtherTok{\textless{}{-}} \ControlFlowTok{function}\NormalTok{(str,num)\{}
  \FunctionTok{str\_sub}\NormalTok{(}\AttributeTok{string =}\NormalTok{ str,}\AttributeTok{start =} \DecValTok{1}\NormalTok{,}\AttributeTok{end =}\NormalTok{ num)}
\NormalTok{\}}
\FunctionTok{r\_left}\NormalTok{(}\StringTok{\textquotesingle{}我是R语言学习者\textquotesingle{}}\NormalTok{,}\DecValTok{3}\NormalTok{)}
\CommentTok{\#\textgreater{} [1] "我是R"}
\end{Highlighting}
\end{Shaded}

\begin{itemize}
\tightlist
\item
  right
\end{itemize}

\begin{Shaded}
\begin{Highlighting}[]
\NormalTok{r\_right }\OtherTok{\textless{}{-}} \ControlFlowTok{function}\NormalTok{(str,num)\{}
  \FunctionTok{str\_sub}\NormalTok{(}\AttributeTok{string =}\NormalTok{ str,}\AttributeTok{start =} \FunctionTok{str\_length}\NormalTok{(str) }\SpecialCharTok{{-}}\NormalTok{ num }\SpecialCharTok{+} \DecValTok{1}\NormalTok{)}
\NormalTok{\}}
\FunctionTok{r\_right}\NormalTok{(}\StringTok{\textquotesingle{}我是R语言学习者\textquotesingle{}}\NormalTok{,}\DecValTok{3}\NormalTok{)}
\CommentTok{\#\textgreater{} [1] "学习者"}
\end{Highlighting}
\end{Shaded}

\begin{itemize}
\tightlist
\item
  mid
\end{itemize}

\begin{Shaded}
\begin{Highlighting}[]
\NormalTok{r\_mid }\OtherTok{\textless{}{-}} \ControlFlowTok{function}\NormalTok{(str,start,num)\{}
  \FunctionTok{str\_sub}\NormalTok{(}\AttributeTok{string =}\NormalTok{ str,}\AttributeTok{start =}\NormalTok{ start,}\AttributeTok{end =}\NormalTok{ start }\SpecialCharTok{+}\NormalTok{ num }\SpecialCharTok{{-}}\DecValTok{1}\NormalTok{)}
\NormalTok{\}}
\FunctionTok{r\_mid}\NormalTok{(}\StringTok{\textquotesingle{}我是R语言学习者\textquotesingle{}}\NormalTok{,}\DecValTok{3}\NormalTok{,}\DecValTok{3}\NormalTok{)}
\CommentTok{\#\textgreater{} [1] "R语言"}
\end{Highlighting}
\end{Shaded}

其余函数可以尝试自行实现。

\hypertarget{ux4f7fux7528ux6848ux4f8b}{%
\subsection{使用案例}\label{ux4f7fux7528ux6848ux4f8b}}

实际运用案例

\begin{itemize}
\tightlist
\item
  合并
\end{itemize}

\begin{Shaded}
\begin{Highlighting}[]
\FunctionTok{library}\NormalTok{(data.table)}
\CommentTok{\#\textgreater{} }
\CommentTok{\#\textgreater{} 载入程辑包:\textquotesingle{}data.table\textquotesingle{}}
\CommentTok{\#\textgreater{} The following objects are masked from \textquotesingle{}package:dplyr\textquotesingle{}:}
\CommentTok{\#\textgreater{} }
\CommentTok{\#\textgreater{}     between, first, last}
\CommentTok{\#\textgreater{} The following object is masked from \textquotesingle{}package:purrr\textquotesingle{}:}
\CommentTok{\#\textgreater{} }
\CommentTok{\#\textgreater{}     transpose}
\NormalTok{dt }\OtherTok{\textless{}{-}} \FunctionTok{data.table}\NormalTok{(}\AttributeTok{col=}\FunctionTok{rep}\NormalTok{(}\StringTok{\textquotesingle{}a\textquotesingle{}}\NormalTok{,}\DecValTok{10}\NormalTok{),}\AttributeTok{letters=}\NormalTok{letters[}\DecValTok{1}\SpecialCharTok{:}\DecValTok{10}\NormalTok{])}
\NormalTok{dt[,newcol}\SpecialCharTok{:}\ErrorTok{=}\FunctionTok{str\_c}\NormalTok{(letters,}\AttributeTok{collapse =} \StringTok{\textquotesingle{}|\textquotesingle{}}\NormalTok{),by}\OtherTok{=}\NormalTok{.(col)][]}
\CommentTok{\#\textgreater{}     col letters              newcol}
\CommentTok{\#\textgreater{}  1:   a       a a|b|c|d|e|f|g|h|i|j}
\CommentTok{\#\textgreater{}  2:   a       b a|b|c|d|e|f|g|h|i|j}
\CommentTok{\#\textgreater{}  3:   a       c a|b|c|d|e|f|g|h|i|j}
\CommentTok{\#\textgreater{}  4:   a       d a|b|c|d|e|f|g|h|i|j}
\CommentTok{\#\textgreater{}  5:   a       e a|b|c|d|e|f|g|h|i|j}
\CommentTok{\#\textgreater{}  6:   a       f a|b|c|d|e|f|g|h|i|j}
\CommentTok{\#\textgreater{}  7:   a       g a|b|c|d|e|f|g|h|i|j}
\CommentTok{\#\textgreater{}  8:   a       h a|b|c|d|e|f|g|h|i|j}
\CommentTok{\#\textgreater{}  9:   a       i a|b|c|d|e|f|g|h|i|j}
\CommentTok{\#\textgreater{} 10:   a       j a|b|c|d|e|f|g|h|i|j}
\end{Highlighting}
\end{Shaded}

\begin{itemize}
\tightlist
\item
  拆解
\end{itemize}

\begin{Shaded}
\begin{Highlighting}[]

\CommentTok{\#工作中路径需要拆解 类似商品品类路径 进口水果{-}热带水果{-}生鲜,用户行为路径等}
\NormalTok{dt }\OtherTok{\textless{}{-}} \FunctionTok{data.table}\NormalTok{(}\AttributeTok{col=}\StringTok{\textquotesingle{}a\textquotesingle{}}\NormalTok{,}\AttributeTok{letters=}\FunctionTok{str\_c}\NormalTok{(letters[}\DecValTok{1}\SpecialCharTok{:}\DecValTok{10}\NormalTok{],}\AttributeTok{collapse =} \StringTok{\textquotesingle{}|\textquotesingle{}}\NormalTok{))}

\NormalTok{my\_str\_split }\OtherTok{\textless{}{-}} \ControlFlowTok{function}\NormalTok{(x)\{}
  
  \FunctionTok{str\_split}\NormalTok{(x,}\AttributeTok{pattern =} \StringTok{"}\SpecialCharTok{\textbackslash{}\textbackslash{}}\StringTok{|"}\NormalTok{) }\SpecialCharTok{\%\textgreater{}\%} \FunctionTok{unlist}\NormalTok{()  }\CommentTok{\#str\_split 拆解出来是列表 需要向量化}
\NormalTok{\}}

\NormalTok{dt[,}\FunctionTok{list}\NormalTok{(}\AttributeTok{newcol=}\FunctionTok{my\_str\_split}\NormalTok{(letters)),by}\OtherTok{=}\NormalTok{.(col)]}
\CommentTok{\#\textgreater{}     col newcol}
\CommentTok{\#\textgreater{}  1:   a      a}
\CommentTok{\#\textgreater{}  2:   a      b}
\CommentTok{\#\textgreater{}  3:   a      c}
\CommentTok{\#\textgreater{}  4:   a      d}
\CommentTok{\#\textgreater{}  5:   a      e}
\CommentTok{\#\textgreater{}  6:   a      f}
\CommentTok{\#\textgreater{}  7:   a      g}
\CommentTok{\#\textgreater{}  8:   a      h}
\CommentTok{\#\textgreater{}  9:   a      i}
\CommentTok{\#\textgreater{} 10:   a      j}
\end{Highlighting}
\end{Shaded}

\hypertarget{character:the-difference-stringr-and-base}{%
\section{base和stringr}\label{character:the-difference-stringr-and-base}}

以下表格数据对比,主要是base R 和 stringr中的相应字符处理功能函数对比。

表格数据来源\href{https://stringr.tidyverse.org/articles/from-base.html}{stringr and base differences}。表格数据可用以下代码获取(注意网络):

\begin{Shaded}
\begin{Highlighting}[]
\FunctionTok{library}\NormalTok{(tidyverse)}
\FunctionTok{library}\NormalTok{(rvest)}
\NormalTok{dt }\OtherTok{\textless{}{-}} \FunctionTok{read\_html}\NormalTok{(}\StringTok{\textquotesingle{}https://stringr.tidyverse.org/articles/from{-}base.html\textquotesingle{}}\NormalTok{) }\SpecialCharTok{\%\textgreater{}\%} 
   \FunctionTok{html\_table}\NormalTok{() }\SpecialCharTok{\%\textgreater{}\%} \StringTok{\textasciigrave{}}\AttributeTok{[[}\StringTok{\textasciigrave{}}\NormalTok{(}\DecValTok{1}\NormalTok{)}
\end{Highlighting}
\end{Shaded}

\begin{longtable}[]{@{}ll@{}}
\toprule
base & stringr \\
\midrule
\endhead
gregexpr(pattern, x) & str\_locate\_all(x, pattern) \\
grep(pattern, x, value = TRUE) & str\_subset(x, pattern) \\
grep(pattern, x) & str\_which(x, pattern) \\
grepl(pattern, x) & str\_detect(x, pattern) \\
gsub(pattern, replacement, x) & str\_replace\_all(x, pattern, replacement) \\
nchar(x) & str\_length(x) \\
order(x) & str\_order(x) \\
regexec(pattern, x) + regmatches() & str\_match(x, pattern) \\
regexpr(pattern, x) + regmatches() & str\_extract(x, pattern) \\
regexpr(pattern, x) & str\_locate(x, pattern) \\
sort(x) & str\_sort(x) \\
strrep(x, n) & str\_dup(x, n) \\
strsplit(x, pattern) & str\_split(x, pattern) \\
strwrap(x) & str\_wrap(x) \\
sub(pattern, replacement, x) & str\_replace(x, pattern, replacement) \\
substr(x, start, end) & str\_sub(x, start, end) \\
tolower(x) & str\_to\_lower(x) \\
tools::toTitleCase(x) & str\_to\_title(x) \\
toupper(x) & str\_to\_upper(x) \\
trimws(x) & str\_trim(x) \\
\bottomrule
\end{longtable}

通过以上对比,方便我们从Base R 切换到stringr包的使用。

\hypertarget{character:reference-material}{%
\section{参考资料}\label{character:reference-material}}

\begin{enumerate}
\def\labelenumi{\arabic{enumi}.}
\tightlist
\item
  tidyverse-stringr:\url{https://stringr.tidyverse.org/articles/from-base.html}
\item
  stringr vignettes:\url{https://cran.r-project.org/web/packages/stringr/vignettes/stringr.html}
\item
  R new feature:\url{https://www.r-bloggers.com/4-for-4-0-0-four-useful-new-features-in-r-4-0-0/}
\item
  R-4.0.0 NEW features:\url{https://cran.r-project.org/doc/manuals/r-devel/NEWS.html}
\end{enumerate}

\hypertarget{datetime}{%
\chapter{时间处理}\label{datetime}}

时间处理看起来是一件简单的事情,因为我们每天都在使用,但事实上是一件复杂的事情。闰年导致每年的天数并不一致,每天也并不是24小时。可以自行搜索``夏令时'',``为什么一天是24小时''或``Why do we have 24 hours in a day''等问题了解关于时间的概念。

但是我们做数据分析,可能仅需要简单的计算时间,并不是必须了解``时间''。我们大部分时候能处理同环比,间隔天数等常规问题即可。

由于能力有限以及处理的日期数据类型有限,本章节仅就常规商业数据分析中时间处理提供一种解决办法。

本章主要分为三大部分:

\begin{itemize}
\item
  Base R中时间处理函数
\item
  lubridate包提供的日期时间处理方法
\item
  常规运用以及和Excel的时间系统对比
\end{itemize}

\hypertarget{datetime:base-R}{%
\section{base R}\label{datetime:base-R}}

R中内置Date,POSIXct和POSIXlt三个关于日期和时间的类\footnote{类是面向对象编程的一个术语,一个对象通常有0个1个或多个类。在R中用\texttt{class()}函数查看所属对象的类。}。

\hypertarget{the-date-class}{%
\subsection{Date}\label{the-date-class}}

如果我们的数据结构中只有日期,没有时间,我们仅需要使用Date类。

\begin{Shaded}
\begin{Highlighting}[]
\FunctionTok{class}\NormalTok{(}\FunctionTok{Sys.Date}\NormalTok{())}
\CommentTok{\#\textgreater{} [1] "Date"}
\end{Highlighting}
\end{Shaded}

1.创建日期

\begin{Shaded}
\begin{Highlighting}[]
\NormalTok{date1 }\OtherTok{\textless{}{-}} \FunctionTok{as.Date}\NormalTok{(}\StringTok{\textquotesingle{}2021{-}05{-}18\textquotesingle{}}\NormalTok{)}
\CommentTok{\# as.Date(32768, origin = "1900{-}01{-}01")}
\CommentTok{\# date1 \textless{}{-} as.Date(\textquotesingle{}2021{-}05{-}18\textquotesingle{},origin = "1900{-}01{-}01")}
\end{Highlighting}
\end{Shaded}

当日期字符不规则时必须指定format参数

\begin{Shaded}
\begin{Highlighting}[]
\NormalTok{date2 }\OtherTok{\textless{}{-}} \FunctionTok{as.Date}\NormalTok{(}\StringTok{"5/14/2021"}\NormalTok{,}\AttributeTok{format=}\StringTok{"\%m/\%d/\%Y"}\NormalTok{)}
\end{Highlighting}
\end{Shaded}

想想如何才将将``2021年5月8日''转换成日期:

\begin{Shaded}
\begin{Highlighting}[]
\FunctionTok{as.Date}\NormalTok{(}\StringTok{"2021年5月18日"}\NormalTok{,}\AttributeTok{format=}\StringTok{"\%Y年\%m月\%d日"}\NormalTok{)}
\CommentTok{\#\textgreater{} [1] "2021{-}05{-}18"}
\end{Highlighting}
\end{Shaded}

重点是时间的format,详情可以通过\texttt{?strptime()}查看。

2.日期计算

两日期之间间隔

\begin{Shaded}
\begin{Highlighting}[]
\NormalTok{date1 }\SpecialCharTok{{-}}\NormalTok{ date2}
\CommentTok{\#\textgreater{} Time difference of 4 days}
\FunctionTok{difftime}\NormalTok{(date1,date2,}\AttributeTok{units =} \StringTok{\textquotesingle{}days\textquotesingle{}}\NormalTok{)}
\CommentTok{\#\textgreater{} Time difference of 4 days}
\end{Highlighting}
\end{Shaded}

日期加减天数

\begin{Shaded}
\begin{Highlighting}[]
\NormalTok{date1 }\SpecialCharTok{{-}} \DecValTok{4}
\CommentTok{\#\textgreater{} [1] "2021{-}05{-}14"}

\NormalTok{date2 }\SpecialCharTok{+} \DecValTok{4}
\CommentTok{\#\textgreater{} [1] "2021{-}05{-}18"}
\end{Highlighting}
\end{Shaded}

向量化计算

\begin{Shaded}
\begin{Highlighting}[]
\NormalTok{three\_date }\OtherTok{\textless{}{-}} \FunctionTok{as.Date}\NormalTok{(}\FunctionTok{c}\NormalTok{(}\StringTok{\textquotesingle{}2021{-}05{-}01\textquotesingle{}}\NormalTok{,}\StringTok{\textquotesingle{}2021{-}05{-}05\textquotesingle{}}\NormalTok{,}\StringTok{\textquotesingle{}2021{-}05{-}10\textquotesingle{}}\NormalTok{))}
\FunctionTok{diff}\NormalTok{(three\_date)}
\CommentTok{\#\textgreater{} Time differences in days}
\CommentTok{\#\textgreater{} [1] 4 5}
\end{Highlighting}
\end{Shaded}

在计算顾客购物间隔天数时比较有用。

3.创建日期向量

\begin{Shaded}
\begin{Highlighting}[]
\NormalTok{date3 }\OtherTok{\textless{}{-}} \FunctionTok{as.Date}\NormalTok{(}\StringTok{\textquotesingle{}2020{-}01{-}01\textquotesingle{}}\NormalTok{)}
\NormalTok{date4 }\OtherTok{\textless{}{-}} \FunctionTok{as.Date}\NormalTok{(}\StringTok{\textquotesingle{}2021{-}01{-}01\textquotesingle{}}\NormalTok{)}
\NormalTok{date\_col }\OtherTok{\textless{}{-}}\NormalTok{ date3}\SpecialCharTok{:}\NormalTok{date4}
\FunctionTok{head}\NormalTok{(date\_col)}
\CommentTok{\#\textgreater{} [1] 18262 18263 18264 18265 18266 18267}
\end{Highlighting}
\end{Shaded}

以上方式创建日期向量会数字化,正确方式如下所示:

\begin{Shaded}
\begin{Highlighting}[]
\CommentTok{\# seq(date3,date4)}
\FunctionTok{seq}\NormalTok{(date3,date4,}\AttributeTok{by=}\StringTok{"30 days"}\NormalTok{)}
\CommentTok{\#\textgreater{}  [1] "2020{-}01{-}01" "2020{-}01{-}31" "2020{-}03{-}01" "2020{-}03{-}31" "2020{-}04{-}30"}
\CommentTok{\#\textgreater{}  [6] "2020{-}05{-}30" "2020{-}06{-}29" "2020{-}07{-}29" "2020{-}08{-}28" "2020{-}09{-}27"}
\CommentTok{\#\textgreater{} [11] "2020{-}10{-}27" "2020{-}11{-}26" "2020{-}12{-}26"}
\FunctionTok{seq}\NormalTok{(date3,date4,}\AttributeTok{by=}\StringTok{"8 weeks"}\NormalTok{)}
\CommentTok{\#\textgreater{} [1] "2020{-}01{-}01" "2020{-}02{-}26" "2020{-}04{-}22" "2020{-}06{-}17" "2020{-}08{-}12"}
\CommentTok{\#\textgreater{} [6] "2020{-}10{-}07" "2020{-}12{-}02"}
\end{Highlighting}
\end{Shaded}

\hypertarget{the-POSIXct-class}{%
\subsection{POSIXct}\label{the-POSIXct-class}}

如果在数据中有时间,最好使用该类;

\begin{Shaded}
\begin{Highlighting}[]
\FunctionTok{Sys.time}\NormalTok{() }\CommentTok{\#获取当前时间}
\CommentTok{\#\textgreater{} [1] "2021{-}05{-}21 18:37:50 CST"}
\FunctionTok{class}\NormalTok{(}\FunctionTok{Sys.time}\NormalTok{())}
\CommentTok{\#\textgreater{} [1] "POSIXct" "POSIXt"}
\end{Highlighting}
\end{Shaded}

1.创建POSIXct类

\begin{Shaded}
\begin{Highlighting}[]
\NormalTok{tm1 }\OtherTok{\textless{}{-}} \FunctionTok{as.POSIXct}\NormalTok{(}\StringTok{"2021{-}5{-}19 16:05:45"}\NormalTok{)}
\NormalTok{tm1}
\CommentTok{\#\textgreater{} [1] "2021{-}05{-}19 16:05:45 CST"}
\NormalTok{tm2 }\OtherTok{\textless{}{-}} \FunctionTok{as.POSIXct}\NormalTok{(}\StringTok{"19052021 16:05:45"}\NormalTok{,}\AttributeTok{format =} \StringTok{"\%d\%m\%Y \%H:\%M:\%S"}\NormalTok{)}
\NormalTok{tm2}
\CommentTok{\#\textgreater{} [1] "2021{-}05{-}19 16:05:45 CST"}
\CommentTok{\# 比较是否相同}
\FunctionTok{identical}\NormalTok{(tm1,tm2)}
\CommentTok{\#\textgreater{} [1] TRUE}
\end{Highlighting}
\end{Shaded}

2.时区

时区如果不正确指定,将导致我们在做时间计算时可能出现错误,一般相差8小时,因为我们在东八区。

默认时区

\begin{Shaded}
\begin{Highlighting}[]
\FunctionTok{Sys.timezone}\NormalTok{()}
\CommentTok{\#\textgreater{} [1] "Asia/Taipei"}
\end{Highlighting}
\end{Shaded}

\begin{Shaded}
\begin{Highlighting}[]
\FunctionTok{as.POSIXct}\NormalTok{(}\StringTok{"2021{-}5{-}19 16:05:45"}\NormalTok{,}\AttributeTok{tz =} \StringTok{\textquotesingle{}CST6CDT\textquotesingle{}}\NormalTok{) }\CommentTok{\#不知道什么原因 CST需要变成CST6CDT不会报错}
\CommentTok{\#\textgreater{} [1] "2021{-}05{-}19 16:05:45 CDT"}
\end{Highlighting}
\end{Shaded}

\begin{Shaded}
\begin{Highlighting}[]
\FunctionTok{as.POSIXct}\NormalTok{(}\StringTok{"2021{-}5{-}19 16:05:45"}\NormalTok{,}\AttributeTok{tz =} \StringTok{\textquotesingle{}GMT\textquotesingle{}}\NormalTok{) }\SpecialCharTok{{-}} \FunctionTok{as.POSIXct}\NormalTok{(}\StringTok{"2021{-}5{-}19 16:05:45"}\NormalTok{,}\AttributeTok{tz =} \StringTok{\textquotesingle{}CST6CDT\textquotesingle{}}\NormalTok{)}
\CommentTok{\#\textgreater{} Time difference of {-}5 hours}
\FunctionTok{as.POSIXct}\NormalTok{(}\StringTok{"2021{-}5{-}19 16:05:45"}\NormalTok{,}\AttributeTok{tz =} \StringTok{\textquotesingle{}UTC\textquotesingle{}}\NormalTok{) }\SpecialCharTok{{-}} \FunctionTok{as.POSIXct}\NormalTok{(}\StringTok{"2021{-}5{-}19 16:05:45"}\NormalTok{,}\AttributeTok{tz =} \StringTok{\textquotesingle{}CST6CDT\textquotesingle{}}\NormalTok{)}
\CommentTok{\#\textgreater{} Time difference of {-}5 hours}
\end{Highlighting}
\end{Shaded}

3.计算

比较

\begin{Shaded}
\begin{Highlighting}[]
\NormalTok{tm1 }\OtherTok{\textless{}{-}} \FunctionTok{as.POSIXct}\NormalTok{(}\StringTok{"2021{-}5{-}19 16:05:45"}\NormalTok{) }
\NormalTok{tm2 }\OtherTok{\textless{}{-}} \FunctionTok{as.POSIXct}\NormalTok{(}\StringTok{"2021{-}5{-}19 16:15:45"}\NormalTok{) }
\NormalTok{tm2 }\SpecialCharTok{\textgreater{}}\NormalTok{ tm1}
\CommentTok{\#\textgreater{} [1] TRUE}
\end{Highlighting}
\end{Shaded}

加减计算,默认单位秒

\begin{Shaded}
\begin{Highlighting}[]
\NormalTok{tm1 }\SpecialCharTok{+} \DecValTok{300}
\CommentTok{\#\textgreater{} [1] "2021{-}05{-}19 16:10:45 CST"}
\NormalTok{tm2 }\SpecialCharTok{{-}} \DecValTok{300}
\CommentTok{\#\textgreater{} [1] "2021{-}05{-}19 16:10:45 CST"}
\end{Highlighting}
\end{Shaded}

\begin{Shaded}
\begin{Highlighting}[]
\NormalTok{tm2 }\SpecialCharTok{{-}}\NormalTok{ tm1}
\CommentTok{\#\textgreater{} Time difference of 10 mins}
\end{Highlighting}
\end{Shaded}

\hypertarget{the-POSIXlt-class}{%
\subsection{POSIXlt}\label{the-POSIXlt-class}}

通过此类,我们可以很便捷提取时间中的特定成分。其中``ct''代表日历时间,``it''代表本地时间,该类对象是list(列表)。

创建时间

\begin{Shaded}
\begin{Highlighting}[]
\NormalTok{t1 }\OtherTok{\textless{}{-}} \FunctionTok{as.POSIXlt}\NormalTok{(}\StringTok{\textquotesingle{}2021{-}5{-}19 16:05:45\textquotesingle{}}\NormalTok{)}
\NormalTok{t1}
\CommentTok{\#\textgreater{} [1] "2021{-}05{-}19 16:05:45 CST"}
\FunctionTok{unclass}\NormalTok{(t1)}
\CommentTok{\#\textgreater{} $sec}
\CommentTok{\#\textgreater{} [1] 45}
\CommentTok{\#\textgreater{} }
\CommentTok{\#\textgreater{} $min}
\CommentTok{\#\textgreater{} [1] 5}
\CommentTok{\#\textgreater{} }
\CommentTok{\#\textgreater{} $hour}
\CommentTok{\#\textgreater{} [1] 16}
\CommentTok{\#\textgreater{} }
\CommentTok{\#\textgreater{} $mday}
\CommentTok{\#\textgreater{} [1] 19}
\CommentTok{\#\textgreater{} }
\CommentTok{\#\textgreater{} $mon}
\CommentTok{\#\textgreater{} [1] 4}
\CommentTok{\#\textgreater{} }
\CommentTok{\#\textgreater{} $year}
\CommentTok{\#\textgreater{} [1] 121}
\CommentTok{\#\textgreater{} }
\CommentTok{\#\textgreater{} $wday}
\CommentTok{\#\textgreater{} [1] 3}
\CommentTok{\#\textgreater{} }
\CommentTok{\#\textgreater{} $yday}
\CommentTok{\#\textgreater{} [1] 138}
\CommentTok{\#\textgreater{} }
\CommentTok{\#\textgreater{} $isdst}
\CommentTok{\#\textgreater{} [1] 0}
\CommentTok{\#\textgreater{} }
\CommentTok{\#\textgreater{} $zone}
\CommentTok{\#\textgreater{} [1] "CST"}
\CommentTok{\#\textgreater{} }
\CommentTok{\#\textgreater{} $gmtoff}
\CommentTok{\#\textgreater{} [1] NA}
\end{Highlighting}
\end{Shaded}

提取

\begin{Shaded}
\begin{Highlighting}[]
\NormalTok{t1}\SpecialCharTok{$}\NormalTok{mday}
\CommentTok{\#\textgreater{} [1] 19}
\NormalTok{t1}\SpecialCharTok{$}\NormalTok{wday}
\CommentTok{\#\textgreater{} [1] 3}
\end{Highlighting}
\end{Shaded}

截断

\begin{Shaded}
\begin{Highlighting}[]
\FunctionTok{trunc}\NormalTok{(t1,}\StringTok{\textquotesingle{}day\textquotesingle{}}\NormalTok{)}
\CommentTok{\#\textgreater{} [1] "2021{-}05{-}19 CST"}
\FunctionTok{trunc}\NormalTok{(t1,}\StringTok{\textquotesingle{}min\textquotesingle{}}\NormalTok{)}
\CommentTok{\#\textgreater{} [1] "2021{-}05{-}19 16:05:00 CST"}
\end{Highlighting}
\end{Shaded}

\hypertarget{lubridate}{%
\section{lubridate}\label{lubridate}}

\texttt{lubridate}包是对Base R中POSIXct类的封装。所以无论从函数名还是功能等方面,lubridate包中的函数功能更加清晰明了。从获取当前日期、时间,解析时间日期中的年、月、日、星期,计算年月间隔天数等常用的时间日期功能,\texttt{lubridate}包中都有相对应的功能函数。

在处理日期时间数据时,我常用\texttt{lubridate}解决,本节将介绍包中部分函数用法。

\hypertarget{lubridate-install}{%
\subsection{安装包}\label{lubridate-install}}

\begin{Shaded}
\begin{Highlighting}[]
\FunctionTok{install.packages}\NormalTok{(}\StringTok{"tidyverse"}\NormalTok{)}
\CommentTok{\# 仅仅只安装lubridate}
\FunctionTok{install.packages}\NormalTok{(}\StringTok{\textquotesingle{}lubridate\textquotesingle{}}\NormalTok{)}
\CommentTok{\# 开发版}
\NormalTok{devtools}\SpecialCharTok{::}\FunctionTok{install\_github}\NormalTok{(}\StringTok{"tidyverse/lubridate"}\NormalTok{)}
\end{Highlighting}
\end{Shaded}

\begin{Shaded}
\begin{Highlighting}[]
\CommentTok{\# 加载包}
\FunctionTok{library}\NormalTok{(lubridate,}\AttributeTok{warn.conflicts =} \ConstantTok{FALSE}\NormalTok{)}
\end{Highlighting}
\end{Shaded}

\hypertarget{get-current-datetime}{%
\subsection{当前时间日期}\label{get-current-datetime}}

\begin{itemize}
\tightlist
\item
  \texttt{now}函数
\end{itemize}

now()函数是当前时间,只有一个参数tzone,默认为系统的timezone。

\begin{Shaded}
\begin{Highlighting}[]
\FunctionTok{now}\NormalTok{()}
\CommentTok{\#\textgreater{} [1] "2021{-}05{-}21 18:37:50 CST"}
\CommentTok{\# now(tzone = \textquotesingle{}Asia/Shanghai\textquotesingle{})}
\CommentTok{\# base R}
\NormalTok{base}\SpecialCharTok{::}\FunctionTok{Sys.time}\NormalTok{()}
\CommentTok{\#\textgreater{} [1] "2021{-}05{-}21 18:37:50 CST"}
\end{Highlighting}
\end{Shaded}

\begin{itemize}
\tightlist
\item
  \texttt{today}函数
\end{itemize}

时区同样默认为系统的timezone。

\begin{Shaded}
\begin{Highlighting}[]
\FunctionTok{today}\NormalTok{(}\AttributeTok{tzone =} \StringTok{\textquotesingle{}Asia/Shanghai\textquotesingle{}}\NormalTok{)}
\CommentTok{\#\textgreater{} [1] "2021{-}05{-}21"}
\CommentTok{\#base R}
\NormalTok{base}\SpecialCharTok{::}\FunctionTok{Sys.Date}\NormalTok{()}
\CommentTok{\#\textgreater{} [1] "2021{-}05{-}21"}
\end{Highlighting}
\end{Shaded}

\hypertarget{make-datetime}{%
\subsection{构造日期时间}\label{make-datetime}}

使用数值直接创建日期时间。

函数\texttt{make\_date()}和\texttt{make\_datetime()}函数默认时区\footnote{lubridate包中大部分函数默认时区为``UTC'',在涉及时间处理时需要注意时区。}为``UTC''。

\begin{Shaded}
\begin{Highlighting}[]
\FunctionTok{make\_date}\NormalTok{(}\AttributeTok{year =} \DecValTok{2021}\NormalTok{, }\AttributeTok{month =} \DecValTok{5}\NormalTok{, }\AttributeTok{day =} \DecValTok{1}\NormalTok{, }\AttributeTok{tz =} \StringTok{"Asia/Shanghai"}\NormalTok{)}

\FunctionTok{make\_datetime}\NormalTok{(}
  \AttributeTok{year =}\NormalTok{ 1970L,}
  \AttributeTok{month =}\NormalTok{ 1L,}
  \AttributeTok{day =}\NormalTok{ 1L,}
  \AttributeTok{hour =}\NormalTok{ 0L,}
  \AttributeTok{min =}\NormalTok{ 0L,}
  \AttributeTok{sec =} \DecValTok{0}\NormalTok{,}
  \AttributeTok{tz =} \StringTok{"Asia/Shanghai"}
\NormalTok{)}
\end{Highlighting}
\end{Shaded}

\begin{itemize}
\tightlist
\item
  make\_datetime
\end{itemize}

\begin{Shaded}
\begin{Highlighting}[]
\FunctionTok{make\_datetime}\NormalTok{(}
  \AttributeTok{year =} \FunctionTok{year}\NormalTok{(}\FunctionTok{today}\NormalTok{()),}
  \AttributeTok{month =} \FunctionTok{month}\NormalTok{(}\FunctionTok{today}\NormalTok{()),}
  \AttributeTok{day =} \FunctionTok{day}\NormalTok{(}\FunctionTok{today}\NormalTok{()),}
  \AttributeTok{hour =} \FunctionTok{hour}\NormalTok{(}\FunctionTok{now}\NormalTok{()),}
  \AttributeTok{min =} \FunctionTok{minute}\NormalTok{(}\FunctionTok{now}\NormalTok{()),}
  \AttributeTok{sec =} \FunctionTok{second}\NormalTok{(}\FunctionTok{now}\NormalTok{()),}
  \AttributeTok{tz =} \StringTok{"asia/shanghai"}
\NormalTok{)}
\CommentTok{\#\textgreater{} [1] "2021{-}05{-}21 18:37:50 CST"}
\end{Highlighting}
\end{Shaded}

\begin{itemize}
\tightlist
\item
  as\_datetime
\end{itemize}

\begin{Shaded}
\begin{Highlighting}[]
\FunctionTok{as\_datetime}\NormalTok{(}\StringTok{\textquotesingle{}2020{-}01{-}09 09:15:40\textquotesingle{}}\NormalTok{,}\AttributeTok{tz=}\StringTok{\textquotesingle{}asia/shanghai\textquotesingle{}}\NormalTok{)}
\CommentTok{\#\textgreater{} [1] "2020{-}01{-}09 09:15:40 CST"}
\FunctionTok{as\_date}\NormalTok{(}\StringTok{\textquotesingle{}2020{-}01{-}09\textquotesingle{}}\NormalTok{) }\CommentTok{\#ymd格式}
\CommentTok{\#\textgreater{} [1] "2020{-}01{-}09"}
\CommentTok{\# same above}
\CommentTok{\#as\_date(\textquotesingle{}2020/01/09\textquotesingle{})}
\CommentTok{\#as\_date(\textquotesingle{}20200109\textquotesingle{})}
\end{Highlighting}
\end{Shaded}

\hypertarget{parse-datetime}{%
\subsection{解析日期时间}\label{parse-datetime}}

数据源中日期列可能是各种的字符形式,需要转换为时间格式方便进行日期计算。商业环境中的数据是混乱的,生产库可能是不同的数据库系统,导致时间日期格式混乱,如果公司没有统一的用户层数据源,我们就需要自己清洗数据,将不同形式的日期格式转化为标准格式。

\begin{itemize}
\tightlist
\item
  解析日期
\end{itemize}

\begin{Shaded}
\begin{Highlighting}[]
\CommentTok{\# 整数和字符都可以}
\FunctionTok{ymd}\NormalTok{(}\DecValTok{20200604}\NormalTok{) }
\CommentTok{\#\textgreater{} [1] "2020{-}06{-}04"}
\FunctionTok{ymd}\NormalTok{(}\StringTok{\textquotesingle{}20200604\textquotesingle{}}\NormalTok{)}
\CommentTok{\#\textgreater{} [1] "2020{-}06{-}04"}
\FunctionTok{mdy}\NormalTok{(}\DecValTok{06042020}\NormalTok{)}
\CommentTok{\#\textgreater{} [1] "2020{-}06{-}04"}
\FunctionTok{dmy}\NormalTok{(}\DecValTok{04062020}\NormalTok{)}
\CommentTok{\#\textgreater{} [1] "2020{-}06{-}04"}
\end{Highlighting}
\end{Shaded}

\begin{itemize}
\tightlist
\item
  解析时间
\end{itemize}

\begin{Shaded}
\begin{Highlighting}[]
\FunctionTok{ymd\_hm}\NormalTok{(}\StringTok{"20100201 07{-}01"}\NormalTok{, }\StringTok{"20100201 07{-}1"}\NormalTok{, }\StringTok{"20100201 7{-}01"}\NormalTok{)}
\CommentTok{\#\textgreater{} [1] "2010{-}02{-}01 07:01:00 UTC" "2010{-}02{-}01 07:01:00 UTC"}
\CommentTok{\#\textgreater{} [3] "2010{-}02{-}01 07:01:00 UTC"}
\FunctionTok{ymd\_hms}\NormalTok{(}\StringTok{"2013{-}01{-}24 19:39:07"}\NormalTok{)}
\CommentTok{\#\textgreater{} [1] "2013{-}01{-}24 19:39:07 UTC"}
\end{Highlighting}
\end{Shaded}

当需要处理unix时间戳时应.POSIXct()函数转化.

\href{https://unixtime.51240.com/}{unix在线转换}

\begin{Shaded}
\begin{Highlighting}[]
\FunctionTok{.POSIXct}\NormalTok{(}\DecValTok{1591709615}\NormalTok{)}
\CommentTok{\#\textgreater{} [1] "2020{-}06{-}09 21:33:35 CST"}
\FunctionTok{ymd\_hms}\NormalTok{(}\FunctionTok{.POSIXct}\NormalTok{(}\DecValTok{1591709615}\NormalTok{))}
\CommentTok{\#\textgreater{} [1] "2020{-}06{-}09 21:33:35 UTC"}
\end{Highlighting}
\end{Shaded}

在使用unix时间戳转换时一定注意R环境和数据系统环境时区是否一直。

\begin{quote}
曾经我在使用阿里云的RDS数据库时没注意时区差异,导致我清洗出来的时间数据错误。
\end{quote}

\begin{Shaded}
\begin{Highlighting}[]
\FunctionTok{ymd\_hms}\NormalTok{(}\FunctionTok{.POSIXct}\NormalTok{(}\DecValTok{1591709615}\NormalTok{),}\AttributeTok{tz =} \StringTok{\textquotesingle{}Asia/Shanghai\textquotesingle{}}\NormalTok{)}
\CommentTok{\#\textgreater{} [1] "2020{-}06{-}09 21:33:35 CST"}
\end{Highlighting}
\end{Shaded}

\hypertarget{extracting-datetime-information}{%
\subsection{提取日期时间成分}\label{extracting-datetime-information}}

\begin{Shaded}
\begin{Highlighting}[]
\CommentTok{\#获取年}
\FunctionTok{year}\NormalTok{(}\FunctionTok{now}\NormalTok{())  }
\CommentTok{\#\textgreater{} [1] 2021}
\CommentTok{\#获取月}
\FunctionTok{month}\NormalTok{(}\FunctionTok{now}\NormalTok{())}
\CommentTok{\#\textgreater{} [1] 5}
\CommentTok{\# 当前时间所在年份天数}
\FunctionTok{yday}\NormalTok{(}\FunctionTok{now}\NormalTok{())}
\CommentTok{\#\textgreater{} [1] 141}
\CommentTok{\# 当前时间所在月天数}
\FunctionTok{mday}\NormalTok{(}\FunctionTok{now}\NormalTok{())}
\CommentTok{\#\textgreater{} [1] 21}
\CommentTok{\# 周几}
\FunctionTok{wday}\NormalTok{(}\FunctionTok{now}\NormalTok{(),}\AttributeTok{label =} \ConstantTok{TRUE}\NormalTok{,}\AttributeTok{week\_start =} \DecValTok{1}\NormalTok{)}
\CommentTok{\#\textgreater{} [1] 周五}
\CommentTok{\#\textgreater{} Levels: 周一 \textless{} 周二 \textless{} 周三 \textless{} 周四 \textless{} 周五 \textless{} 周六 \textless{} 周日}
\CommentTok{\# 所在时刻}
\FunctionTok{hour}\NormalTok{(}\FunctionTok{now}\NormalTok{())}
\CommentTok{\#\textgreater{} [1] 18}
\CommentTok{\# 所在时刻}
\FunctionTok{minute}\NormalTok{(}\FunctionTok{now}\NormalTok{())}
\CommentTok{\#\textgreater{} [1] 37}
\CommentTok{\# 所在时刻}
\FunctionTok{second}\NormalTok{(}\FunctionTok{now}\NormalTok{())}
\CommentTok{\#\textgreater{} [1] 50.6}
\end{Highlighting}
\end{Shaded}

\hypertarget{time-zones}{%
\subsection{处理时区}\label{time-zones}}

数据时区与本地R环境一致时,数据中的时区没必要处理,但是当数据是跨时区的或者不同生产系统的时区不一致,我们需要将数据时区处理一致。

1.时区查看

时区和所用系统设置相关

\begin{Shaded}
\begin{Highlighting}[]
\FunctionTok{Sys.timezone}\NormalTok{()}
\CommentTok{\#\textgreater{} [1] "Asia/Taipei"}
\CommentTok{\# windows 系统默认的时区 中国台北}
\CommentTok{\# linux 上是"Asia/Shanghai"}
\CommentTok{\# mac 上是"Asia/Shanghai"}
\end{Highlighting}
\end{Shaded}

这里还有一个奇怪的点,Windows系统下时区设置为\texttt{(UTC+08:00)北京,重庆,香港特别行政区,乌鲁木齐},但是R返回的时区是\texttt{Asia/Taipei}。

\begin{Shaded}
\begin{Highlighting}[]
\FunctionTok{now}\NormalTok{()}
\CommentTok{\#\textgreater{} [1] "2021{-}05{-}21 18:37:50 CST"}
\end{Highlighting}
\end{Shaded}

\texttt{now()}输出结果中,CST是时区概念。

CST可以同时代表四个时间

\begin{itemize}
\tightlist
\item
  Central Standard Time (USA) UT-6:00
\item
  Central Standard Time (Australia) UT+9:30
\item
  China Standard Time UT+8:00
\item
  Cuba Standard Time UT-4:00
\end{itemize}

2.时区调整

lubridate中用\texttt{with\_tz()},\texttt{force\_tz()}处理时区问题

\begin{Shaded}
\begin{Highlighting}[]
\NormalTok{time }\OtherTok{\textless{}{-}} \FunctionTok{ymd\_hms}\NormalTok{(}\StringTok{"2020{-}12{-}13 15:30:30"}\NormalTok{)}
\NormalTok{time}
\CommentTok{\#\textgreater{} [1] "2020{-}12{-}13 15:30:30 UTC"}

\CommentTok{\# Changes printing}
\FunctionTok{with\_tz}\NormalTok{(time, }\StringTok{"Asia/Shanghai"}\NormalTok{)}
\CommentTok{\#\textgreater{} [1] "2020{-}12{-}13 23:30:30 CST"}
\CommentTok{\# Changes time}
\FunctionTok{force\_tz}\NormalTok{(time, }\StringTok{"Asia/Shanghai"}\NormalTok{)}
\CommentTok{\#\textgreater{} [1] "2020{-}12{-}13 15:30:30 CST"}
\end{Highlighting}
\end{Shaded}

\begin{enumerate}
\def\labelenumi{\arabic{enumi}.}
\setcounter{enumi}{2}
\tightlist
\item
  时区差异
\end{enumerate}

从下面三个时间观察时区,CST时间:中央标准时间;UTC时间:世界协调时间(UTC)是世界上不同国家用来调节时钟和时间的主要时间标准。

如:当UTC时间为0点时,中国CST时间为8点,因为零时区和中国北京时区相差8个时区.

\begin{itemize}
\tightlist
\item
  \url{https://home.kpn.nl/vanadovv/time/TZworld.html\#asi}
\end{itemize}

\begin{Shaded}
\begin{Highlighting}[]
\NormalTok{lubridate}\SpecialCharTok{::}\FunctionTok{now}\NormalTok{() }\CommentTok{\# now函数调用系统默认时区}
\CommentTok{\#\textgreater{} [1] "2021{-}05{-}21 18:37:50 CST"}
\FunctionTok{as\_datetime}\NormalTok{(}\FunctionTok{now}\NormalTok{()) }\CommentTok{\#as\_datetime默认是UTC}
\CommentTok{\#\textgreater{} [1] "2021{-}05{-}21 10:37:50 UTC"}
\FunctionTok{as\_datetime}\NormalTok{(}\FunctionTok{now}\NormalTok{(),}\AttributeTok{tz =} \StringTok{\textquotesingle{}asia/shanghai\textquotesingle{}}\NormalTok{)}
\CommentTok{\#\textgreater{} [1] "2021{-}05{-}21 18:37:50 CST"}
\end{Highlighting}
\end{Shaded}

\hypertarget{interveal}{%
\subsection{时间间隔}\label{interveal}}

\texttt{lubridate}中将时间间隔保存为\texttt{interveal}类对象。

\begin{Shaded}
\begin{Highlighting}[]
\NormalTok{arrive }\OtherTok{\textless{}{-}} \FunctionTok{ymd\_hms}\NormalTok{(}\StringTok{"2020{-}12{-}04 12:00:00"}\NormalTok{, }\AttributeTok{tz =} \StringTok{"asia/shanghai"}\NormalTok{)}
\NormalTok{arrive}
\CommentTok{\#\textgreater{} [1] "2020{-}12{-}04 12:00:00 CST"}

\NormalTok{leave }\OtherTok{\textless{}{-}} \FunctionTok{ymd\_hms}\NormalTok{(}\StringTok{"2020{-}12{-}10 14:00:00"}\NormalTok{, }\AttributeTok{tz =} \StringTok{"asia/shanghai"}\NormalTok{)}
\NormalTok{leave}
\CommentTok{\#\textgreater{} [1] "2020{-}12{-}10 14:00:00 CST"}

\NormalTok{res }\OtherTok{\textless{}{-}} \FunctionTok{interval}\NormalTok{(arrive, leave) }
\CommentTok{\# same above}
\NormalTok{res }\OtherTok{\textless{}{-}}\NormalTok{ arrive }\SpecialCharTok{\%{-}{-}\%}\NormalTok{ leave}
\end{Highlighting}
\end{Shaded}

查看类

\begin{Shaded}
\begin{Highlighting}[]
\FunctionTok{class}\NormalTok{(res)}
\CommentTok{\#\textgreater{} [1] "Interval"}
\CommentTok{\#\textgreater{} attr(,"package")}
\CommentTok{\#\textgreater{} [1] "lubridate"}
\end{Highlighting}
\end{Shaded}

两个时间间隔是否重复

\begin{Shaded}
\begin{Highlighting}[]
\NormalTok{jsm }\OtherTok{\textless{}{-}} \FunctionTok{interval}\NormalTok{(}\FunctionTok{ymd}\NormalTok{(}\DecValTok{20201020}\NormalTok{, }\AttributeTok{tz =} \StringTok{"asia/shanghai"}\NormalTok{), }\FunctionTok{ymd}\NormalTok{(}\DecValTok{20201231}\NormalTok{, }\AttributeTok{tz =} \StringTok{"asia/shanghai"}\NormalTok{))}
\NormalTok{jsm}
\CommentTok{\#\textgreater{} [1] 2020{-}10{-}20 CST{-}{-}2020{-}12{-}31 CST}
\FunctionTok{int\_overlaps}\NormalTok{(jsm, res)}
\CommentTok{\#\textgreater{} [1] TRUE}
\end{Highlighting}
\end{Shaded}

更多详细用法\texttt{?interveal}

\begin{Shaded}
\begin{Highlighting}[]
\FunctionTok{interval}\NormalTok{(}\AttributeTok{start =} \ConstantTok{NULL}\NormalTok{, }\AttributeTok{end =} \ConstantTok{NULL}\NormalTok{, }\AttributeTok{tzone =} \FunctionTok{tz}\NormalTok{(start))}

\NormalTok{start }\SpecialCharTok{\%{-}{-}\%}\NormalTok{ end}

\FunctionTok{is.interval}\NormalTok{(x)}

\FunctionTok{int\_start}\NormalTok{(int)}

\FunctionTok{int\_start}\NormalTok{(int) }\OtherTok{\textless{}{-}}\NormalTok{ value}

\FunctionTok{int\_end}\NormalTok{(int)}

\FunctionTok{int\_end}\NormalTok{(int) }\OtherTok{\textless{}{-}}\NormalTok{ value}

\FunctionTok{int\_length}\NormalTok{(int)}

\FunctionTok{int\_flip}\NormalTok{(int)}

\FunctionTok{int\_shift}\NormalTok{(int, by)}

\FunctionTok{int\_overlaps}\NormalTok{(int1, int2)}

\FunctionTok{int\_standardize}\NormalTok{(int)}

\FunctionTok{int\_aligns}\NormalTok{(int1, int2)}

\FunctionTok{int\_diff}\NormalTok{(times)}
\end{Highlighting}
\end{Shaded}

\hypertarget{calculate-datetime}{%
\subsection{时间日期计算}\label{calculate-datetime}}

时间日期计算以\texttt{number\ line}为依据计算。原文是\texttt{Because\ the\ timeline\ is\ not\ as\ reliable\ as\ the\ number\ line},我没理解这句话。

\begin{Shaded}
\begin{Highlighting}[]
\FunctionTok{minutes}\NormalTok{(}\DecValTok{2}\NormalTok{)}
\CommentTok{\#\textgreater{} [1] "2M 0S"}
\FunctionTok{dminutes}\NormalTok{(}\DecValTok{2}\NormalTok{)}
\CommentTok{\#\textgreater{} [1] "120s (\textasciitilde{}2 minutes)"}
\FunctionTok{dhours}\NormalTok{(}\DecValTok{2}\NormalTok{)}
\CommentTok{\#\textgreater{} [1] "7200s (\textasciitilde{}2 hours)"}
\end{Highlighting}
\end{Shaded}

注意闰年时计算年份的差异

\begin{Shaded}
\begin{Highlighting}[]
\FunctionTok{leap\_year}\NormalTok{(}\DecValTok{2019}\NormalTok{)}
\CommentTok{\#\textgreater{} [1] FALSE}
\FunctionTok{ymd}\NormalTok{(}\DecValTok{20190101}\NormalTok{) }\SpecialCharTok{+} \FunctionTok{dyears}\NormalTok{(}\DecValTok{1}\NormalTok{)}
\CommentTok{\#\textgreater{} [1] "2020{-}01{-}01 06:00:00 UTC"}
\FunctionTok{ymd}\NormalTok{(}\DecValTok{20190101}\NormalTok{) }\SpecialCharTok{+} \FunctionTok{years}\NormalTok{(}\DecValTok{1}\NormalTok{)}
\CommentTok{\#\textgreater{} [1] "2020{-}01{-}01"}

\FunctionTok{leap\_year}\NormalTok{(}\DecValTok{2020}\NormalTok{)}
\CommentTok{\#\textgreater{} [1] TRUE}
\FunctionTok{ymd}\NormalTok{(}\DecValTok{20200101}\NormalTok{) }\SpecialCharTok{+} \FunctionTok{dyears}\NormalTok{(}\DecValTok{1}\NormalTok{)  }\CommentTok{\# 注意查看闰年时的差异}
\CommentTok{\#\textgreater{} [1] "2020{-}12{-}31 06:00:00 UTC"}
\FunctionTok{ymd}\NormalTok{(}\DecValTok{20200101}\NormalTok{) }\SpecialCharTok{+} \FunctionTok{years}\NormalTok{(}\DecValTok{1}\NormalTok{)}
\CommentTok{\#\textgreater{} [1] "2021{-}01{-}01"}
\end{Highlighting}
\end{Shaded}

\texttt{lubridate}中的函数都已向量化

\begin{Shaded}
\begin{Highlighting}[]
\NormalTok{meeting }\OtherTok{\textless{}{-}} \FunctionTok{ymd\_hms}\NormalTok{(}\StringTok{"2020{-}12{-}01 09:00:00"}\NormalTok{, }\AttributeTok{tz =} \StringTok{"asia/shanghai"}\NormalTok{)}
\NormalTok{meeting }\OtherTok{\textless{}{-}}\NormalTok{ meeting }\SpecialCharTok{+} \FunctionTok{weeks}\NormalTok{(}\DecValTok{0}\SpecialCharTok{:}\DecValTok{5}\NormalTok{)}
\NormalTok{meeting }\SpecialCharTok{\%within\%}\NormalTok{ jsm}
\CommentTok{\#\textgreater{} [1]  TRUE  TRUE  TRUE  TRUE  TRUE FALSE}
\end{Highlighting}
\end{Shaded}

除法计算

\begin{Shaded}
\begin{Highlighting}[]
\NormalTok{res }\SpecialCharTok{/} \FunctionTok{ddays}\NormalTok{(}\DecValTok{1}\NormalTok{)}
\CommentTok{\#\textgreater{} [1] 6.08}
\NormalTok{res }\SpecialCharTok{/} \FunctionTok{dminutes}\NormalTok{(}\DecValTok{1}\NormalTok{)}
\CommentTok{\#\textgreater{} [1] 8760}


\NormalTok{res }\SpecialCharTok{\%/\%} \FunctionTok{months}\NormalTok{(}\DecValTok{1}\NormalTok{)}
\CommentTok{\#\textgreater{} [1] 0}
\NormalTok{res }\SpecialCharTok{\%\%} \FunctionTok{months}\NormalTok{(}\DecValTok{1}\NormalTok{)}
\CommentTok{\#\textgreater{} [1] 2020{-}12{-}04 12:00:00 CST{-}{-}2020{-}12{-}10 14:00:00 CST}
\end{Highlighting}
\end{Shaded}

\texttt{as.period}用法

\begin{Shaded}
\begin{Highlighting}[]
\FunctionTok{as.period}\NormalTok{(res }\SpecialCharTok{\%\%} \FunctionTok{months}\NormalTok{(}\DecValTok{1}\NormalTok{))}
\CommentTok{\#\textgreater{} [1] "6d 2H 0M 0S"}
\end{Highlighting}
\end{Shaded}

对于日期而言,因为月天数、年天数不一致,导致不能直接加减天数,如下:

\begin{Shaded}
\begin{Highlighting}[]
\NormalTok{jan31 }\OtherTok{\textless{}{-}} \FunctionTok{ymd}\NormalTok{(}\StringTok{"2020{-}01{-}31"}\NormalTok{)}
\NormalTok{jan31 }\SpecialCharTok{+} \FunctionTok{months}\NormalTok{(}\DecValTok{0}\SpecialCharTok{:}\DecValTok{11}\NormalTok{)}
\CommentTok{\#\textgreater{}  [1] "2020{-}01{-}31" NA           "2020{-}03{-}31" NA           "2020{-}05{-}31"}
\CommentTok{\#\textgreater{}  [6] NA           "2020{-}07{-}31" "2020{-}08{-}31" NA           "2020{-}10{-}31"}
\CommentTok{\#\textgreater{} [11] NA           "2020{-}12{-}31"}
\end{Highlighting}
\end{Shaded}

\texttt{lubridate}中不存在的日期返回\texttt{NA}

解决方案是:\texttt{\%m+\%}或\texttt{\%m-\%}

\begin{Shaded}
\begin{Highlighting}[]
\NormalTok{jan31 }\SpecialCharTok{\%m+\%} \FunctionTok{months}\NormalTok{(}\DecValTok{0}\SpecialCharTok{:}\DecValTok{11}\NormalTok{)}
\CommentTok{\#\textgreater{}  [1] "2020{-}01{-}31" "2020{-}02{-}29" "2020{-}03{-}31" "2020{-}04{-}30" "2020{-}05{-}31"}
\CommentTok{\#\textgreater{}  [6] "2020{-}06{-}30" "2020{-}07{-}31" "2020{-}08{-}31" "2020{-}09{-}30" "2020{-}10{-}31"}
\CommentTok{\#\textgreater{} [11] "2020{-}11{-}30" "2020{-}12{-}31"}
\NormalTok{jan31 }\SpecialCharTok{\%m{-}\%} \FunctionTok{months}\NormalTok{(}\DecValTok{0}\SpecialCharTok{:}\DecValTok{11}\NormalTok{)}
\CommentTok{\#\textgreater{}  [1] "2020{-}01{-}31" "2019{-}12{-}31" "2019{-}11{-}30" "2019{-}10{-}31" "2019{-}09{-}30"}
\CommentTok{\#\textgreater{}  [6] "2019{-}08{-}31" "2019{-}07{-}31" "2019{-}06{-}30" "2019{-}05{-}31" "2019{-}04{-}30"}
\CommentTok{\#\textgreater{} [11] "2019{-}03{-}31" "2019{-}02{-}28"}
\end{Highlighting}
\end{Shaded}

\hypertarget{datetime:application}{%
\section{综合运用}\label{datetime:application}}

\hypertarget{ux65e5ux62a5ux540cux73afux6bd4ux8ba1ux7b97}{%
\subsection{日报同环比计算}\label{ux65e5ux62a5ux540cux73afux6bd4ux8ba1ux7b97}}

零售行业基本都存在日报,作为数据分析师大概率是需要出日报的,但根据所在部门情况会有所不同。很多人都已经在sql或exel中实现了,本案例不完全实现日报,主要是教大家为了实现同环比,怎么利用R做日期范围筛选。

首先我们看看R里面怎么做日期的同环比计算:

\begin{itemize}
\tightlist
\item
  常用函数
\end{itemize}

\texttt{round\_date()}函数根据要求周期回滚日期

\begin{Shaded}
\begin{Highlighting}[]
\FunctionTok{floor\_date}\NormalTok{(}\FunctionTok{today}\NormalTok{(),}\AttributeTok{unit =} \StringTok{\textquotesingle{}year\textquotesingle{}}\NormalTok{)}
\CommentTok{\#\textgreater{} [1] "2021{-}01{-}01"}
\FunctionTok{floor\_date}\NormalTok{(}\FunctionTok{today}\NormalTok{(),}\AttributeTok{unit =} \StringTok{\textquotesingle{}month\textquotesingle{}}\NormalTok{) }
\CommentTok{\#\textgreater{} [1] "2021{-}05{-}01"}
\FunctionTok{floor\_date}\NormalTok{(}\FunctionTok{today}\NormalTok{(),}\AttributeTok{unit =} \StringTok{\textquotesingle{}week\textquotesingle{}}\NormalTok{)}
\CommentTok{\#\textgreater{} [1] "2021{-}05{-}16"}
\end{Highlighting}
\end{Shaded}

以上同系列函数从名字就能大概看出端倪,其中关键参数是unit,可选想如下:
1s,second,minute,5 mins,hour,dat,week,months,bimonth,quarter,season,halfyear,year。

\begin{Shaded}
\begin{Highlighting}[]
\FunctionTok{round\_date}\NormalTok{(}
\NormalTok{  x,}
  \AttributeTok{unit =} \StringTok{"second"}\NormalTok{,}
  \AttributeTok{week\_start =} \FunctionTok{getOption}\NormalTok{(}\StringTok{"lubridate.week.start"}\NormalTok{, }\DecValTok{7}\NormalTok{)}
\NormalTok{)}

\FunctionTok{floor\_date}\NormalTok{(}
\NormalTok{  x,}
  \AttributeTok{unit =} \StringTok{"seconds"}\NormalTok{,}
  \AttributeTok{week\_start =} \FunctionTok{getOption}\NormalTok{(}\StringTok{"lubridate.week.start"}\NormalTok{, }\DecValTok{7}\NormalTok{)}
\NormalTok{)}

\FunctionTok{ceiling\_date}\NormalTok{(}
\NormalTok{  x,}
  \AttributeTok{unit =} \StringTok{"seconds"}\NormalTok{,}
  \AttributeTok{change\_on\_boundary =} \ConstantTok{NULL}\NormalTok{,}
  \AttributeTok{week\_start =} \FunctionTok{getOption}\NormalTok{(}\StringTok{"lubridate.week.start"}\NormalTok{, }\DecValTok{7}\NormalTok{)}
\NormalTok{)}
\end{Highlighting}
\end{Shaded}

change\_on\_boundary参数

\begin{Shaded}
\begin{Highlighting}[]
\FunctionTok{ceiling\_date}\NormalTok{(}\FunctionTok{ymd\_hms}\NormalTok{(}\StringTok{\textquotesingle{}2021{-}01{-}1 00:00:00\textquotesingle{}}\NormalTok{),}\StringTok{\textquotesingle{}month\textquotesingle{}}\NormalTok{)}
\CommentTok{\#\textgreater{} [1] "2021{-}01{-}01 UTC"}
\FunctionTok{ceiling\_date}\NormalTok{(}\FunctionTok{ymd\_hms}\NormalTok{(}\StringTok{\textquotesingle{}2021{-}01{-}1 00:00:00\textquotesingle{}}\NormalTok{),}\StringTok{\textquotesingle{}month\textquotesingle{}}\NormalTok{,}\AttributeTok{change\_on\_boundary =}\NormalTok{ T)}
\CommentTok{\#\textgreater{} [1] "2021{-}02{-}01 UTC"}
\end{Highlighting}
\end{Shaded}

\begin{itemize}
\tightlist
\item
  计算年同比
\end{itemize}

\begin{Shaded}
\begin{Highlighting}[]
\NormalTok{n }\OtherTok{\textless{}{-}} \DecValTok{1} 
\NormalTok{date }\OtherTok{\textless{}{-}} \FunctionTok{today}\NormalTok{()}
\CommentTok{\# current }
\NormalTok{current\_start\_date }\OtherTok{\textless{}{-}}  \FunctionTok{floor\_date}\NormalTok{(date,}\AttributeTok{unit =} \StringTok{\textquotesingle{}year\textquotesingle{}}\NormalTok{)}
\NormalTok{current\_start\_date}
\CommentTok{\#\textgreater{} [1] "2021{-}01{-}01"}
\NormalTok{date }
\CommentTok{\#\textgreater{} [1] "2021{-}05{-}21"}

\CommentTok{\# last year}
\NormalTok{last\_start\_date }\OtherTok{\textless{}{-}} \FunctionTok{floor\_date}\NormalTok{(date,}\AttributeTok{unit =} \StringTok{\textquotesingle{}year\textquotesingle{}}\NormalTok{) }\SpecialCharTok{\%m{-}\%} \FunctionTok{years}\NormalTok{(n)}
\NormalTok{last\_start\_date}
\CommentTok{\#\textgreater{} [1] "2020{-}01{-}01"}
\NormalTok{last\_end\_date }\OtherTok{\textless{}{-}}\NormalTok{ date }\SpecialCharTok{\%m{-}\%} \FunctionTok{years}\NormalTok{(n)}
\NormalTok{last\_end\_date}
\CommentTok{\#\textgreater{} [1] "2020{-}05{-}21"}
\end{Highlighting}
\end{Shaded}

以上,n表示间隔年数,大部分时候都是1。但特殊时候,比如2021年同比2020年2-4月(新冠疫情)基本没有同比意义,所以在此设置为参数。

\begin{itemize}
\tightlist
\item
  计算月同比
\end{itemize}

\texttt{rollback()}函数返回上个月的最后一天或当前月的第一天

\begin{Shaded}
\begin{Highlighting}[]
\FunctionTok{rollback}\NormalTok{(}\FunctionTok{today}\NormalTok{())}
\CommentTok{\#\textgreater{} [1] "2021{-}04{-}30"}
\FunctionTok{rollback}\NormalTok{(}\FunctionTok{today}\NormalTok{(),}\AttributeTok{roll\_to\_first =} \ConstantTok{TRUE}\NormalTok{)}
\CommentTok{\#\textgreater{} [1] "2021{-}05{-}01"}
\end{Highlighting}
\end{Shaded}

\begin{itemize}
\tightlist
\item
  计算月环比
\end{itemize}

计算环比时,\texttt{\%m+\%}或\texttt{\%m-\%}可以很好解决月份天数不一的问题

\begin{Shaded}
\begin{Highlighting}[]
\FunctionTok{as\_date}\NormalTok{(}\StringTok{\textquotesingle{}2020{-}03{-}30\textquotesingle{}}\NormalTok{) }\SpecialCharTok{\%m{-}\%} \FunctionTok{months}\NormalTok{(}\DecValTok{1}\NormalTok{)}
\CommentTok{\#\textgreater{} [1] "2020{-}02{-}29"}

\CommentTok{\# 环比月截止日}
\FunctionTok{today}\NormalTok{()}
\CommentTok{\#\textgreater{} [1] "2021{-}05{-}21"}
\FunctionTok{today}\NormalTok{() }\SpecialCharTok{\%m{-}\%} \FunctionTok{months}\NormalTok{(}\DecValTok{1}\NormalTok{)}
\CommentTok{\#\textgreater{} [1] "2021{-}04{-}21"}
\end{Highlighting}
\end{Shaded}

经过以上计算,得到一对对时间周期,然后在订单或者其它数据中筛选即可获得同环比维度数据。

\begin{itemize}
\tightlist
\item
  模拟计算
\end{itemize}

\begin{Shaded}
\begin{Highlighting}[]
\CommentTok{\# 构造数据}
\NormalTok{bill\_date }\OtherTok{\textless{}{-}} \FunctionTok{as\_date}\NormalTok{((}\FunctionTok{as\_date}\NormalTok{(}\StringTok{\textquotesingle{}2019{-}01{-}01\textquotesingle{}}\NormalTok{)}\SpecialCharTok{:}\FunctionTok{as\_date}\NormalTok{(}\StringTok{\textquotesingle{}2020{-}12{-}01\textquotesingle{}}\NormalTok{)))}
\NormalTok{area }\OtherTok{\textless{}{-}}  \FunctionTok{sample}\NormalTok{(}\FunctionTok{c}\NormalTok{(}\StringTok{\textquotesingle{}华东\textquotesingle{}}\NormalTok{,}\StringTok{\textquotesingle{}华西\textquotesingle{}}\NormalTok{,}\StringTok{\textquotesingle{}华南\textquotesingle{}}\NormalTok{,}\StringTok{\textquotesingle{}华北\textquotesingle{}}\NormalTok{),}\AttributeTok{size =} \FunctionTok{length}\NormalTok{(bill\_date),}\AttributeTok{replace =} \ConstantTok{TRUE}\NormalTok{)}
\NormalTok{category }\OtherTok{\textless{}{-}} \FunctionTok{sample}\NormalTok{(}\FunctionTok{c}\NormalTok{(}\StringTok{\textquotesingle{}品类A\textquotesingle{}}\NormalTok{,}\StringTok{\textquotesingle{}品类B\textquotesingle{}}\NormalTok{,}\StringTok{\textquotesingle{}品类C\textquotesingle{}}\NormalTok{,}\StringTok{\textquotesingle{}品类D\textquotesingle{}}\NormalTok{),}\AttributeTok{size =} \FunctionTok{length}\NormalTok{(bill\_date),}\AttributeTok{replace =} \ConstantTok{TRUE}\NormalTok{)}
\NormalTok{dt }\OtherTok{\textless{}{-}}\NormalTok{ tibble}\SpecialCharTok{::}\FunctionTok{tibble}\NormalTok{(}\AttributeTok{bill\_date =}\NormalTok{ bill\_date ,}\AttributeTok{money =} \FunctionTok{sample}\NormalTok{(}\DecValTok{80}\SpecialCharTok{:}\DecValTok{150}\NormalTok{,}\AttributeTok{size =} \FunctionTok{length}\NormalTok{(bill\_date),}\AttributeTok{replace =} \ConstantTok{TRUE}\NormalTok{),}\AttributeTok{area =}\NormalTok{ area,}\AttributeTok{category =}\NormalTok{ category)}
\FunctionTok{head}\NormalTok{(dt)}
\CommentTok{\#\textgreater{} \# A tibble: 6 x 4}
\CommentTok{\#\textgreater{}   bill\_date  money area  category}
\CommentTok{\#\textgreater{}   \textless{}date\textgreater{}     \textless{}int\textgreater{} \textless{}chr\textgreater{} \textless{}chr\textgreater{}   }
\CommentTok{\#\textgreater{} 1 2019{-}01{-}01   138 华东  品类C   }
\CommentTok{\#\textgreater{} 2 2019{-}01{-}02    80 华南  品类D   }
\CommentTok{\#\textgreater{} 3 2019{-}01{-}03   111 华北  品类B   }
\CommentTok{\#\textgreater{} 4 2019{-}01{-}04    97 华南  品类C   }
\CommentTok{\#\textgreater{} 5 2019{-}01{-}05   126 华东  品类D   }
\CommentTok{\#\textgreater{} 6 2019{-}01{-}06   149 华西  品类A}
\end{Highlighting}
\end{Shaded}

\begin{itemize}
\tightlist
\item
  自定义函数
\end{itemize}

\begin{Shaded}
\begin{Highlighting}[]
\FunctionTok{library}\NormalTok{(dplyr,}\AttributeTok{warn.conflicts =} \ConstantTok{FALSE}\NormalTok{)}
\FunctionTok{library}\NormalTok{(lubridate)}
\NormalTok{y\_to\_y }\OtherTok{\textless{}{-}} \ControlFlowTok{function}\NormalTok{(.dt,date,}\AttributeTok{n =} \DecValTok{1}\NormalTok{,...)\{}
  
\NormalTok{  date }\OtherTok{\textless{}{-}} \FunctionTok{ymd}\NormalTok{(date)}
  
  \ControlFlowTok{if}\NormalTok{(}\FunctionTok{is.na}\NormalTok{(date))\{}
    \FunctionTok{stop}\NormalTok{(}\StringTok{\textquotesingle{}请输入正确日期格式,如20200101\textquotesingle{}}\NormalTok{)}
\NormalTok{  \} }
  
  \CommentTok{\# current }
\NormalTok{ current\_start\_date }\OtherTok{\textless{}{-}}  \FunctionTok{floor\_date}\NormalTok{(date,}\AttributeTok{unit =} \StringTok{\textquotesingle{}year\textquotesingle{}}\NormalTok{)}
 
 \CommentTok{\# last year}
\NormalTok{ last\_start\_date }\OtherTok{\textless{}{-}} \FunctionTok{floor\_date}\NormalTok{(date,}\AttributeTok{unit =} \StringTok{\textquotesingle{}year\textquotesingle{}}\NormalTok{) }\SpecialCharTok{\%m{-}\%} \FunctionTok{years}\NormalTok{(n)}
\NormalTok{ last\_end\_date }\OtherTok{\textless{}{-}}\NormalTok{ date }\SpecialCharTok{\%m{-}\%} \FunctionTok{years}\NormalTok{(n)}
 
\NormalTok{ .dt }\SpecialCharTok{\%\textgreater{}\%} \FunctionTok{mutate}\NormalTok{( 类型 }\OtherTok{=} \FunctionTok{case\_when}\NormalTok{(}\FunctionTok{between}\NormalTok{(bill\_date,current\_start\_date,date) }\SpecialCharTok{\textasciitilde{}} \StringTok{"当前"}\NormalTok{,}
               \FunctionTok{between}\NormalTok{(bill\_date,last\_start\_date,last\_end\_date) }\SpecialCharTok{\textasciitilde{}} \StringTok{"同期"}\NormalTok{,}
               \ConstantTok{TRUE} \SpecialCharTok{\textasciitilde{}} \StringTok{"其他"}\NormalTok{)) }\SpecialCharTok{\%\textgreater{}\%} 
   \FunctionTok{filter}\NormalTok{(类型 }\SpecialCharTok{!=} \StringTok{"其他"}\NormalTok{) }\SpecialCharTok{\%\textgreater{}\%} 
   \FunctionTok{group\_by}\NormalTok{(...) }\SpecialCharTok{\%\textgreater{}\%} 
   \FunctionTok{summarise}\NormalTok{(金额 }\OtherTok{=} \FunctionTok{sum}\NormalTok{(money,}\AttributeTok{na.rm =} \ConstantTok{TRUE}\NormalTok{)) }\SpecialCharTok{\%\textgreater{}\%} 
   \FunctionTok{ungroup}\NormalTok{() }
 
 \CommentTok{\#\%\textgreater{}\% pivot\_wider(names\_from = \textquotesingle{}类型\textquotesingle{},values\_from = \textquotesingle{}金额\textquotesingle{})}
 
\NormalTok{\}}
\end{Highlighting}
\end{Shaded}

\begin{Shaded}
\begin{Highlighting}[]
\FunctionTok{y\_to\_y}\NormalTok{(dt,}\AttributeTok{date =} \StringTok{\textquotesingle{}20201001\textquotesingle{}}\NormalTok{,}\AttributeTok{n =} \DecValTok{1}\NormalTok{,area,类型) }\SpecialCharTok{\%\textgreater{}\%} 
\NormalTok{  tidyr}\SpecialCharTok{::}\FunctionTok{pivot\_wider}\NormalTok{(}\AttributeTok{id\_cols =} \StringTok{\textquotesingle{}area\textquotesingle{}}\NormalTok{,}\AttributeTok{names\_from =} \StringTok{\textquotesingle{}类型\textquotesingle{}}\NormalTok{,}\AttributeTok{values\_from =} \StringTok{\textquotesingle{}金额\textquotesingle{}}\NormalTok{) }\SpecialCharTok{\%\textgreater{}\%} 
  \FunctionTok{mutate}\NormalTok{(增长率 }\OtherTok{=}\NormalTok{ 当前 }\SpecialCharTok{/}\NormalTok{ 同期)}
\CommentTok{\#\textgreater{} \textasciigrave{}summarise()\textasciigrave{} has grouped output by \textquotesingle{}area\textquotesingle{}. You can override using the \textasciigrave{}.groups\textasciigrave{} argument.}
\CommentTok{\#\textgreater{} \# A tibble: 4 x 4}
\CommentTok{\#\textgreater{}   area   当前  同期 增长率}
\CommentTok{\#\textgreater{}   \textless{}chr\textgreater{} \textless{}int\textgreater{} \textless{}int\textgreater{}  \textless{}dbl\textgreater{}}
\CommentTok{\#\textgreater{} 1 华北   7572  7674  0.987}
\CommentTok{\#\textgreater{} 2 华东   8713  8344  1.04 }
\CommentTok{\#\textgreater{} 3 华南   6279  8150  0.770}
\CommentTok{\#\textgreater{} 4 华西   8805  7161  1.23}

\FunctionTok{y\_to\_y}\NormalTok{(dt,}\AttributeTok{date =} \StringTok{\textquotesingle{}20201001\textquotesingle{}}\NormalTok{,}\AttributeTok{n =} \DecValTok{1}\NormalTok{,area,类型,category) }\SpecialCharTok{\%\textgreater{}\%} 
\NormalTok{  tidyr}\SpecialCharTok{::}\FunctionTok{pivot\_wider}\NormalTok{(}\AttributeTok{id\_cols =} \FunctionTok{c}\NormalTok{(}\StringTok{\textquotesingle{}area\textquotesingle{}}\NormalTok{,}\StringTok{\textquotesingle{}category\textquotesingle{}}\NormalTok{),}\AttributeTok{names\_from =} \StringTok{\textquotesingle{}类型\textquotesingle{}}\NormalTok{,}\AttributeTok{values\_from =} \StringTok{\textquotesingle{}金额\textquotesingle{}}\NormalTok{) }\SpecialCharTok{\%\textgreater{}\%} 
  \FunctionTok{mutate}\NormalTok{(增长率 }\OtherTok{=}\NormalTok{ 当前 }\SpecialCharTok{/}\NormalTok{ 同期)}
\CommentTok{\#\textgreater{} \textasciigrave{}summarise()\textasciigrave{} has grouped output by \textquotesingle{}area\textquotesingle{}, \textquotesingle{}类型\textquotesingle{}. You can override using the \textasciigrave{}.groups\textasciigrave{} argument.}
\CommentTok{\#\textgreater{} \# A tibble: 16 x 5}
\CommentTok{\#\textgreater{}   area  category  当前  同期 增长率}
\CommentTok{\#\textgreater{}   \textless{}chr\textgreater{} \textless{}chr\textgreater{}    \textless{}int\textgreater{} \textless{}int\textgreater{}  \textless{}dbl\textgreater{}}
\CommentTok{\#\textgreater{} 1 华北  品类A     1515  1268  1.19 }
\CommentTok{\#\textgreater{} 2 华北  品类B     1807  2600  0.695}
\CommentTok{\#\textgreater{} 3 华北  品类C     1956  1837  1.06 }
\CommentTok{\#\textgreater{} 4 华北  品类D     2294  1969  1.17 }
\CommentTok{\#\textgreater{} 5 华东  品类A     1797  2400  0.749}
\CommentTok{\#\textgreater{} 6 华东  品类B     2507  2064  1.21 }
\CommentTok{\#\textgreater{} \# ... with 10 more rows}
\end{Highlighting}
\end{Shaded}

\hypertarget{ux6e05ux6d17ux4e0dux540cux7c7bux578bux65e5ux671fux683cux5f0f}{%
\subsection{清洗不同类型日期格式}\label{ux6e05ux6d17ux4e0dux540cux7c7bux578bux65e5ux671fux683cux5f0f}}

如将\texttt{c(\textquotesingle{}2001/2/13\ 10:33\textquotesingle{},\textquotesingle{}1/24/13\ 11:16\textquotesingle{})}转换为相同格式的日期格式;

通过一个简单自定义函数解决,本质是区分不同类型日期后采用不同函数去解析日期格式

\begin{Shaded}
\begin{Highlighting}[]

\FunctionTok{library}\NormalTok{(lubridate)}
\FunctionTok{library}\NormalTok{(tidyverse)}

\NormalTok{date1 }\OtherTok{\textless{}{-}} \FunctionTok{c}\NormalTok{(}\StringTok{\textquotesingle{}2001/2/13 10:33\textquotesingle{}}\NormalTok{,}\StringTok{\textquotesingle{}1/24/13 11:16\textquotesingle{}}\NormalTok{)}

\NormalTok{myfun }\OtherTok{\textless{}{-}} \ControlFlowTok{function}\NormalTok{(x)\{}
  
\NormalTok{  n\_length }\OtherTok{\textless{}{-}} \FunctionTok{length}\NormalTok{(x)}
\NormalTok{  res }\OtherTok{\textless{}{-}} \FunctionTok{vector}\NormalTok{(}\AttributeTok{length =}\NormalTok{ n\_length)}
  
  \ControlFlowTok{for}\NormalTok{(i }\ControlFlowTok{in} \DecValTok{1}\SpecialCharTok{:}\NormalTok{n\_length)\{}
\NormalTok{    n }\OtherTok{\textless{}{-}} \FunctionTok{strsplit}\NormalTok{(x[i],}\StringTok{\textquotesingle{}/\textquotesingle{}}\NormalTok{) }\SpecialCharTok{\%\textgreater{}\%} \StringTok{\textasciigrave{}}\AttributeTok{[[}\StringTok{\textasciigrave{}}\NormalTok{(}\DecValTok{1}\NormalTok{) }\SpecialCharTok{\%\textgreater{}\%} \StringTok{\textasciigrave{}}\AttributeTok{[[}\StringTok{\textasciigrave{}}\NormalTok{(}\DecValTok{1}\NormalTok{)}
    \ControlFlowTok{if}\NormalTok{(}\FunctionTok{str\_length}\NormalTok{(n)}\SpecialCharTok{==}\DecValTok{4}\NormalTok{)\{}
\NormalTok{      res[i] }\OtherTok{\textless{}{-}} \FunctionTok{ymd\_hm}\NormalTok{(x[i],}\AttributeTok{tz =} \StringTok{\textquotesingle{}Asia/Shanghai\textquotesingle{}}\NormalTok{)}
\NormalTok{    \} }\ControlFlowTok{else}\NormalTok{ \{}
\NormalTok{      res[i] }\OtherTok{\textless{}{-}} \FunctionTok{mdy\_hm}\NormalTok{(x[i],}\AttributeTok{tz =} \StringTok{\textquotesingle{}Asia/Shanghai\textquotesingle{}}\NormalTok{)}
\NormalTok{    \}}
\NormalTok{  \}}
  \FunctionTok{as\_datetime}\NormalTok{(res,}\AttributeTok{tz =} \StringTok{\textquotesingle{}Asia/Shanghai\textquotesingle{}}\NormalTok{)}
\NormalTok{\}}

\FunctionTok{myfun}\NormalTok{(date1)}
\CommentTok{\#\textgreater{} [1] "2001{-}02{-}13 10:33:00 CST" "2013{-}01{-}24 11:16:00 CST"}
\end{Highlighting}
\end{Shaded}

\hypertarget{ux626bux7801ux540eux4e2dux5956ux65f6ux95f4ux5339ux914d}{%
\subsection{扫码后中奖时间匹配}\label{ux626bux7801ux540eux4e2dux5956ux65f6ux95f4ux5339ux914d}}

假定有两张表,一张是用户扫码表,一张是用户中奖表,如下所示:

\begin{figure}
\centering
\includegraphics{picture/datetime/p1.png}
\caption{数据源视图}
\end{figure}

由于中奖时间和扫码时间不完全一致,导致没办法直接通过\texttt{客户ID}以及\texttt{时间}关联匹配找到客户每次中奖时的积分码,现在要求找到客户每次中奖时对应的积分码?

思路:通过观察数据,发现扫码后如果中奖,一般几秒钟内会有中奖记录,那我们就可以通过``每次中奖时间最近的一次扫码时间的积分码''就是该次中奖对应的积分码解决问题。这样我们通过简单编写自定义函数即可获取答案,即一个时间点从一串时间中找到离自己最近时间点。

\begin{Shaded}
\begin{Highlighting}[]
\NormalTok{testfun }\OtherTok{\textless{}{-}} \ControlFlowTok{function}\NormalTok{(x,y)\{}
\NormalTok{  result }\OtherTok{\textless{}{-}} \FunctionTok{data.frame}\NormalTok{() }\CommentTok{\#应采用列表存储结果向量化}
\NormalTok{  n  }\OtherTok{\textless{}{-}}  \FunctionTok{length}\NormalTok{(x)}
  \ControlFlowTok{for}\NormalTok{( i }\ControlFlowTok{in} \DecValTok{1}\SpecialCharTok{:}\NormalTok{n)\{}
\NormalTok{    res }\OtherTok{\textless{}{-}}\NormalTok{ x[i]}\SpecialCharTok{{-}}\NormalTok{y}
\NormalTok{    res }\OtherTok{\textless{}{-}} \FunctionTok{abs}\NormalTok{(res) }\SpecialCharTok{\%\textgreater{}\%} \FunctionTok{which.min}\NormalTok{() }\CommentTok{\#本处不对,应该判断res大于0的部分中谁最小}
\NormalTok{    kong }\OtherTok{\textless{}{-}} \FunctionTok{data.frame}\NormalTok{(中奖时间 }\OtherTok{=}\NormalTok{ x[i],扫的时间 }\OtherTok{=}\NormalTok{ y[res])}
\NormalTok{    result }\OtherTok{\textless{}{-}} \FunctionTok{rbind}\NormalTok{(kong,result)}
    
\NormalTok{  \}}
  \FunctionTok{return}\NormalTok{(result)}
\NormalTok{\}}
\NormalTok{res }\OtherTok{\textless{}{-}} \FunctionTok{testfun}\NormalTok{(dt}\SpecialCharTok{$}\NormalTok{时间,scan\_dt}\SpecialCharTok{$}\NormalTok{时间)}
\end{Highlighting}
\end{Shaded}

改进代码

\begin{Shaded}
\begin{Highlighting}[]
\NormalTok{testfun }\OtherTok{\textless{}{-}} \ControlFlowTok{function}\NormalTok{(x,y)\{}
\NormalTok{  n  }\OtherTok{\textless{}{-}}  \FunctionTok{length}\NormalTok{(x)}
\NormalTok{  result }\OtherTok{\textless{}{-}} \FunctionTok{list}\NormalTok{()}
  
  \ControlFlowTok{for}\NormalTok{( i }\ControlFlowTok{in} \DecValTok{1}\SpecialCharTok{:}\NormalTok{n)\{}
\NormalTok{    y }\OtherTok{\textless{}{-}}\NormalTok{ y[x}\SpecialCharTok{\textgreater{}}\NormalTok{y]}
\NormalTok{    res }\OtherTok{\textless{}{-}}\NormalTok{ x[i]}\SpecialCharTok{{-}}\NormalTok{y}
\NormalTok{    res }\OtherTok{\textless{}{-}}\NormalTok{ res }\SpecialCharTok{\%\textgreater{}\%} \FunctionTok{which.min}\NormalTok{() }
\NormalTok{    kong }\OtherTok{\textless{}{-}} \FunctionTok{data.frame}\NormalTok{(中奖时间 }\OtherTok{=}\NormalTok{ x[i],扫的时间 }\OtherTok{=}\NormalTok{ y[res])}
\NormalTok{    result[[i]] }\OtherTok{\textless{}{-}}\NormalTok{ kong}
\NormalTok{  \}}
  \FunctionTok{return}\NormalTok{(result)}
\NormalTok{\}}

\NormalTok{res }\OtherTok{\textless{}{-}} \FunctionTok{testfun}\NormalTok{(dt}\SpecialCharTok{$}\NormalTok{时间,scan\_dt}\SpecialCharTok{$}\NormalTok{时间)}
\end{Highlighting}
\end{Shaded}

理论上不同用户可以在同一时间扫码且同时中奖,那上面的代码即不可以获取正确答案。但是我们只要通过按照用户ID切割数据框后稍微改造上面的自定义函数即可。

\begin{Shaded}
\begin{Highlighting}[]
\NormalTok{testfun }\OtherTok{\textless{}{-}} \ControlFlowTok{function}\NormalTok{(dt)\{}
  
\NormalTok{  x }\OtherTok{\textless{}{-}}\NormalTok{ dt}\SpecialCharTok{$}\NormalTok{中奖时间}
\NormalTok{  y }\OtherTok{\textless{}{-}}\NormalTok{ dt}\SpecialCharTok{$}\NormalTok{扫的时间}
\NormalTok{  n  }\OtherTok{\textless{}{-}}  \FunctionTok{length}\NormalTok{(x)}
\NormalTok{  result }\OtherTok{\textless{}{-}} \FunctionTok{list}\NormalTok{()}
  
  \ControlFlowTok{for}\NormalTok{( i }\ControlFlowTok{in} \DecValTok{1}\SpecialCharTok{:}\NormalTok{n)\{}
\NormalTok{    y }\OtherTok{\textless{}{-}}\NormalTok{ y[x}\SpecialCharTok{\textgreater{}}\NormalTok{y]}
\NormalTok{    res }\OtherTok{\textless{}{-}}\NormalTok{ x[i]}\SpecialCharTok{{-}}\NormalTok{y}
\NormalTok{    res }\OtherTok{\textless{}{-}}\NormalTok{ res }\SpecialCharTok{\%\textgreater{}\%} \FunctionTok{which.min}\NormalTok{() }
\NormalTok{    kong }\OtherTok{\textless{}{-}} \FunctionTok{data.frame}\NormalTok{(中奖时间 }\OtherTok{=}\NormalTok{ x[i],扫的时间 }\OtherTok{=}\NormalTok{ y[res])}
\NormalTok{    result[[i]] }\OtherTok{\textless{}{-}}\NormalTok{ kong}
\NormalTok{  \}}
\NormalTok{  result }\OtherTok{\textless{}{-}}\NormalTok{ dplyr}\SpecialCharTok{::}\FunctionTok{bind\_rows}\NormalTok{(result)}
  \FunctionTok{return}\NormalTok{(result)}
\NormalTok{\}}
\NormalTok{dtlist }\OtherTok{\textless{}{-}} \FunctionTok{split}\NormalTok{(alldt,}\StringTok{\textquotesingle{}客户ID\textquotesingle{}}\NormalTok{)}
\NormalTok{purrr}\SpecialCharTok{::}\FunctionTok{map\_dfr}\NormalTok{(dtlist,testfun)}
\end{Highlighting}
\end{Shaded}

虽然可以通过寻找最近一次的扫码记录判断积分码,但是因为网络延迟或中途接电话等各种原因导致扫码时间和中奖时间相差并不是几秒,导致情景复杂,那我们就应该在设计系统时就设计好锁定对应关系,从根本上解决问题。

\hypertarget{datetime:additional-information}{%
\section{补充资料}\label{datetime:additional-information}}

\hypertarget{excel-and-r}{%
\subsection{Excel and R}\label{excel-and-r}}

Excel是我们天天打交道的工具,但是R与Excel都有自己的时间系统,而且还不统一,在计算时会给我们带来误解。

\href{https://support.microsoft.com/zh-cn/office/excel-\%e4\%b8\%ad\%e7\%9a\%84\%e6\%97\%a5\%e6\%9c\%9f\%e7\%b3\%bb\%e7\%bb\%9f-e7fe7167-48a9-4b96-bb53-5612a800b487?ui=zh-CN\&rs=zh-CN\&ad=CN}{Excle日期系统}

下面就Excel和R中的差异做简单阐述。

\hypertarget{ux5deeux5f02}{%
\subsubsection{差异}\label{ux5deeux5f02}}

\texttt{R}中日期起始时间是\texttt{1970-01-01},Excel中起始日期\footnote{Excel中存在两套日期系统}(Windows)是\texttt{1900-01-01},转化成数字两者相差25568。如下所示:

\begin{Shaded}
\begin{Highlighting}[]
\FunctionTok{as.Date}\NormalTok{(}\StringTok{\textquotesingle{}1970{-}01{-}01\textquotesingle{}}\NormalTok{)}
\CommentTok{\#\textgreater{} [1] "1970{-}01{-}01"}
\FunctionTok{as.Date}\NormalTok{(}\DecValTok{25568}\NormalTok{,}\AttributeTok{origin=}\StringTok{\textquotesingle{}1900{-}01{-}01\textquotesingle{}}\NormalTok{) }\CommentTok{\# 1970{-}01{-}02}
\CommentTok{\#\textgreater{} [1] "1970{-}01{-}02"}
\FunctionTok{as.Date}\NormalTok{(}\DecValTok{25568}\NormalTok{,}\AttributeTok{origin=}\StringTok{\textquotesingle{}1899{-}12{-}31\textquotesingle{}}\NormalTok{) }\CommentTok{\# 1970{-}01{-}01}
\CommentTok{\#\textgreater{} [1] "1970{-}01{-}01"}
\end{Highlighting}
\end{Shaded}

Excel中1900-01-01代表数字1,但是R中1970-01-01代表0。这也是比较怪异的点,毕竟R一般都是从1开始。

\begin{Shaded}
\begin{Highlighting}[]
\FunctionTok{as.numeric}\NormalTok{(}\FunctionTok{as.Date}\NormalTok{(}\StringTok{\textquotesingle{}1970{-}01{-}01\textquotesingle{}}\NormalTok{))}
\CommentTok{\#\textgreater{} [1] 0}
\end{Highlighting}
\end{Shaded}

这样导致:
R日期2021-05-21转化成数字是18768,
Excel中日期2021-05-21转化成数字是44337,\textbf{两者相差25569}。

\sout{在用R读取Excel文件时,涉及到数字日期转化时需要注意其中差异。}

\hypertarget{excelux65f6ux95f4ux51fdux6570}{%
\subsubsection{Excel时间函数}\label{excelux65f6ux95f4ux51fdux6570}}

在\texttt{Excel}的\texttt{Power\ Pivot}中有一组DAX智能函数,如:

\begin{itemize}
\tightlist
\item
  基础函数
\end{itemize}

\texttt{date},\texttt{datediff},\texttt{datevalue},\texttt{edate},\texttt{eomonth},\texttt{quarter},\texttt{TIMEVALUE}等等

\begin{itemize}
\tightlist
\item
  智能函数
\end{itemize}

\texttt{dateadd},\texttt{DATESBETWEEN},\texttt{DATESMTD},\texttt{TOTALMTD},\texttt{TOTALQTD},\texttt{TOTALYTD}等等

\texttt{Excel}中因为有了以上时间智能函数,用度量值在透视表中计算同环比变得简单。
假如熟悉DAX时间智能函数,在\texttt{R}中设计相关功能或实现时可以借鉴参考DAX函数的思路。比如在R中写自动化报表时,涉及到同环比计算时就可以按照这个模式设计。

\hypertarget{ux53c2ux8003ux8d44ux6599-1}{%
\subsection{参考资料}\label{ux53c2ux8003ux8d44ux6599-1}}

\begin{enumerate}
\def\labelenumi{\arabic{enumi}.}
\item
  lubridate \url{https://cran.r-project.org/web/packages/lubridate/vignettes/lubridate.html}
\item
  date and time \url{https://www.stat.berkeley.edu/~s133/dates.html}
\item
  dax时间函数 \url{https://docs.microsoft.com/en-us/dax/time-intelligence-functions-dax}
\item
  Excel日期系统 \url{https://support.microsoft.com/zh-cn/office/excel-\%e4\%b8\%ad\%e7\%9a\%84\%e6\%97\%a5\%e6\%9c\%9f\%e7\%b3\%bb\%e7\%bb\%9f-e7fe7167-48a9-4b96-bb53-5612a800b487?ui=zh-CN\&rs=zh-CN\&ad=CN}
\end{enumerate}

\begin{itemize}
\item
  \url{https://www.rdocumentation.org/packages/lubridate/versions/1.7.8}
\item
  pdf 下载 \url{https://rawgit.com/rstudio/cheatsheets/master/lubridate.pdf}
\end{itemize}

\hypertarget{forcats}{%
\chapter{forcats}\label{forcats}}

我在实际工作中因子数据类型使用较少,forcats软件包用来处理因子,该软件包是tidyverse的一部分.

因子是用于对数据进行分类的R的一种数据类型. 它们可以存储字符串和整数.它们在具有有限数量的唯一值的列中很有用. 像``男性'',``女性''和True,False等。它们在统计建模的数据分析中很有用.

因子变量会占用更小空间,R4.0改变了字符默认为因子的方式.想了解更多请参考 \url{https://r4ds.had.co.nz/factors.html}

\begin{Shaded}
\begin{Highlighting}[]
\FunctionTok{object.size}\NormalTok{(}\FunctionTok{rep}\NormalTok{(letters,}\DecValTok{100000}\NormalTok{))}
\CommentTok{\#\textgreater{} 20801504 bytes}
\FunctionTok{object.size}\NormalTok{(}\FunctionTok{rep}\NormalTok{(forcats}\SpecialCharTok{::}\FunctionTok{as\_factor}\NormalTok{(letters),}\DecValTok{100000}\NormalTok{))}
\CommentTok{\#\textgreater{} 10402096 bytes}
\end{Highlighting}
\end{Shaded}

\hypertarget{ux521bux5efaux56e0ux5b50}{%
\section{创建因子}\label{ux521bux5efaux56e0ux5b50}}

实际工作中,可能各个事业部或部门之间没有实际顺序,但是在数据处理过程中需要指定顺序可以用因子.

\begin{Shaded}
\begin{Highlighting}[]
\FunctionTok{library}\NormalTok{(forcats)}
\NormalTok{vec1 }\OtherTok{\textless{}{-}} \FunctionTok{c}\NormalTok{(}\StringTok{\textquotesingle{}部门a\textquotesingle{}}\NormalTok{,}\StringTok{\textquotesingle{}部门b\textquotesingle{}}\NormalTok{,}\StringTok{\textquotesingle{}部门d\textquotesingle{}}\NormalTok{,}\StringTok{\textquotesingle{}部门f\textquotesingle{}}\NormalTok{)}
\FunctionTok{sort}\NormalTok{(vec1)}
\CommentTok{\#\textgreater{} [1] "部门a" "部门b" "部门d" "部门f"}
\NormalTok{vec2 }\OtherTok{\textless{}{-}} \FunctionTok{as\_factor}\NormalTok{(}\FunctionTok{c}\NormalTok{(}\StringTok{\textquotesingle{}部门f\textquotesingle{}}\NormalTok{,}\StringTok{\textquotesingle{}部门d\textquotesingle{}}\NormalTok{,}\StringTok{\textquotesingle{}部门a\textquotesingle{}}\NormalTok{,}\StringTok{\textquotesingle{}部门b\textquotesingle{}}\NormalTok{))}
\FunctionTok{sort}\NormalTok{(vec2)}
\CommentTok{\#\textgreater{} [1] 部门f 部门d 部门a 部门b}
\CommentTok{\#\textgreater{} Levels: 部门f 部门d 部门a 部门b}
\end{Highlighting}
\end{Shaded}

如上所示:实际工作中可以通过指定因子水平从而达到排序效果,在可视化中也可以运用,像指定X轴的顺序.

\hypertarget{data.table}{%
\chapter{data.table}\label{data.table}}

data.table包是我数据处理最常用的R包,是我目前觉得最好用的数据处理包,大部分我需要用到的功能集成在包里,不需要很多的依赖包。我简单接触过python,julia两种语言,并没有深入比较,所以我这个好用的印象仅仅是个人感受。

data.table包是我用了较长一段时间tidyverse系列后发现的``数据处理包''。已经忘记最初是什么吸引了我,我猜测可能是``大数据处理利器''之类的标签吸引了我,因为我喜欢``快''。但是和大部分人可能不同的是,初次接触时,语法的``怪异''并没有给我带来多少麻烦,因为我本来就没有编程基础以及很深的R语言基础。

所以我死记硬背data.table里一些常用用法,尤其喜欢拿Excle的一些用法参照,去实现Excle上面的部分操作,从读取、增、改、删除、筛选、计算列等常规操作入手。慢慢熟悉data.table语法之后,将会享受data.table带来的便利,其简洁的语法以及高效的计算速度(相比tidyverse系列)。

另外,Python中也有该包,目前正在积极开发中,期待ing,毕竟python也是很好用,在不同需求下选择不同的语言实现功能。

官方关于data.table的基础介绍请参阅:

\url{https://cran.r-project.org/web/packages/data.table/vignettes/datatable-intro.html}

data.table 优势:

\begin{itemize}
\tightlist
\item
  速度快
\item
  内存效率高
\item
  API生命周期管理好
\item
  语法简洁
\end{itemize}

\begin{quote}
本文会照搬很多官方关于data.table的demo
\end{quote}

\hypertarget{ux57faux7840ux4ecbux7ecd}{%
\section{基础介绍}\label{ux57faux7840ux4ecbux7ecd}}

本部分从data.table安装,内置的案例查看,到data.table的句式语法,实现基础行列筛选和聚合计算。

1.安装

安装详细信息请参考\href{https://github.com/Rdatatable/data.table/wiki/Installation}{the Installation wiki},有关于不同系统安装首次以及相关说明。

\begin{Shaded}
\begin{Highlighting}[]
\FunctionTok{install.packages}\NormalTok{(}\StringTok{"data.table"}\NormalTok{)}
\CommentTok{\# latest development version:}
\NormalTok{data.table}\SpecialCharTok{::}\FunctionTok{update.dev.pkg}\NormalTok{()}
\end{Highlighting}
\end{Shaded}

2.使用说明

通过以下代码查看内置的使用案例。

\begin{Shaded}
\begin{Highlighting}[]
\FunctionTok{library}\NormalTok{(data.table)}
\FunctionTok{example}\NormalTok{(data.table)}
\end{Highlighting}
\end{Shaded}

\hypertarget{ux8bfbux53d6ux6570ux636e}{%
\subsection{读取数据}\label{ux8bfbux53d6ux6570ux636e}}

在我实际工作中接触的数据大部分以数据库,csv,Excel等形式存在,并且CSV格式数据较少。但是data.table包读取数据的\texttt{fread}函数仅接受CSV格式。如果是Excel格式文件,需要通过如\texttt{readxl},\texttt{openxlsx}等包读入后转换为\texttt{data.table}格式数据。

fread 函数可以直接读取CSV格式文件,无论是本地文件或者在线文件,如下所示:

\begin{quote}
案例中使用的数据集是R包\texttt{nycflights13}带的flights数据集。
\end{quote}

\begin{Shaded}
\begin{Highlighting}[]
\FunctionTok{library}\NormalTok{(data.table)}
\NormalTok{input }\OtherTok{\textless{}{-}} \ControlFlowTok{if}\NormalTok{ (}\FunctionTok{file.exists}\NormalTok{(}\StringTok{"./data/flights.csv"}\NormalTok{)) \{}
   \StringTok{"./data/flights.csv"} \CommentTok{\#本地文件}
\NormalTok{\} }\ControlFlowTok{else}\NormalTok{ \{}
  \StringTok{"https://raw.githubusercontent.com/Rdatatable/data.table/master/vignettes/flights.csv"} \CommentTok{\#在线文件需翻墙}
\NormalTok{\}}
\NormalTok{flights }\OtherTok{\textless{}{-}} \FunctionTok{fread}\NormalTok{(input) }\CommentTok{\#具体参数请参照文档  实际工作中可能会用到的encoding参数,编码 encoding=\textquotesingle{}UTF{-}8\textquotesingle{}}

\FunctionTok{head}\NormalTok{(flights)}
\CommentTok{\#\textgreater{}    year month day dep\_delay arr\_delay carrier origin dest air\_time distance}
\CommentTok{\#\textgreater{} 1: 2014     1   1        14        13      AA    JFK  LAX      359     2475}
\CommentTok{\#\textgreater{} 2: 2014     1   1        {-}3        13      AA    JFK  LAX      363     2475}
\CommentTok{\#\textgreater{} 3: 2014     1   1         2         9      AA    JFK  LAX      351     2475}
\CommentTok{\#\textgreater{} 4: 2014     1   1        {-}8       {-}26      AA    LGA  PBI      157     1035}
\CommentTok{\#\textgreater{} 5: 2014     1   1         2         1      AA    JFK  LAX      350     2475}
\CommentTok{\#\textgreater{} 6: 2014     1   1         4         0      AA    EWR  LAX      339     2454}
\CommentTok{\#\textgreater{}    hour}
\CommentTok{\#\textgreater{} 1:    9}
\CommentTok{\#\textgreater{} 2:   11}
\CommentTok{\#\textgreater{} 3:   19}
\CommentTok{\#\textgreater{} 4:    7}
\CommentTok{\#\textgreater{} 5:   13}
\CommentTok{\#\textgreater{} 6:   18}
\end{Highlighting}
\end{Shaded}

本文读取本地文件,如果该数据集下载失败,可更改地址为(\url{http://www.zhongyufei.com/datatable/data/flights.csv})

\begin{Shaded}
\begin{Highlighting}[]
\NormalTok{flights }\OtherTok{\textless{}{-}} \FunctionTok{fread}\NormalTok{(}\StringTok{"http://www.zhongyufei.com/Rbook/data/flights.csv"}\NormalTok{)}
\end{Highlighting}
\end{Shaded}

数据集记录的是 2014 年,纽约市3大机场(分别为:JFK 肯尼迪国际机场、 LGA 拉瓜迪亚机场,和 EWR 纽瓦克自由国际机场)起飞的航班信息。

具体的记录信息(特征列),包括起飞时间、到达时间、延误时长、航空公司、始发机场、目的机场、飞行时长,和飞行距离等。

\hypertarget{ux57faux672cux683cux5f0f}{%
\subsection{基本格式}\label{ux57faux672cux683cux5f0f}}

\texttt{DT{[}i,\ j,\ by{]}}是data.table的基本样式,在不同位置上实现不同功能。

\begin{figure}
\centering
\includegraphics{https://gitee.com/zhongyufei/photo-bed/raw/pic/img/data.table-i-j-by\%E4\%BB\%8B\%E7\%BB\%8D.png}
\caption{i-j-by}
\end{figure}

\begin{Shaded}
\begin{Highlighting}[]
\NormalTok{DT[i, j, by]}
\DocumentationTok{\#\#   R:                 i                 j        by}
\DocumentationTok{\#\# SQL:  where | order by   select | update  group by}
\end{Highlighting}
\end{Shaded}

data.table个人理解主要有三大类参数,i参数做筛选,j参数做计算,by参数做分组.

拿Excel透视表类别,i位置参数当作『筛选』,by位置用来做汇总字段『行』,j位置当作『值』,如下所示:

\begin{figure}
\centering
\includegraphics{./picture/data-table/01picture.png}
\caption{透视表截图}
\end{figure}

1.代码实例

代码求2014年6月,从各始发机场到各目的机场的飞行距离求和.

\begin{Shaded}
\begin{Highlighting}[]
\FunctionTok{library}\NormalTok{(data.table)}
\NormalTok{flights }\OtherTok{\textless{}{-}} \FunctionTok{fread}\NormalTok{(}\StringTok{"./data/flights.csv"}\NormalTok{)}
\NormalTok{flights[year}\SpecialCharTok{==}\DecValTok{2014} \SpecialCharTok{\&}\NormalTok{ month}\SpecialCharTok{==}\DecValTok{6}\NormalTok{,.(求和项}\AttributeTok{distance=}\FunctionTok{sum}\NormalTok{(distance)),by}\OtherTok{=}\NormalTok{.(origin,dest)]}
\CommentTok{\#\textgreater{}      origin dest 求和项distance}
\CommentTok{\#\textgreater{}   1:    JFK  LAX        2663100}
\CommentTok{\#\textgreater{}   2:    JFK  DFW          82069}
\CommentTok{\#\textgreater{}   3:    JFK  LAS         795792}
\CommentTok{\#\textgreater{}   4:    JFK  SFO        1967946}
\CommentTok{\#\textgreater{}   5:    JFK  SAN         349778}
\CommentTok{\#\textgreater{}  {-}{-}{-}                           }
\CommentTok{\#\textgreater{} 191:    EWR  ANC          13480}
\CommentTok{\#\textgreater{} 192:    EWR  BZN          15056}
\CommentTok{\#\textgreater{} 193:    LGA  TVC           7205}
\CommentTok{\#\textgreater{} 194:    LGA  BZN           3788}
\CommentTok{\#\textgreater{} 195:    JFK  HYA            980}
\end{Highlighting}
\end{Shaded}

2.代码解释

i 的部分:条件year==2014 和 month==6 ;

j 的部分:求和项distance=sum(distance),写在.()中或者list()中;

by 的部分.(origin,dest),重点是写在.()中,和Excel透视表一一对应。

至于为什么要用.()包裹起来,最开始默认为格式强制要求。就这个问题我想说:大部分人可能觉得是比较``怪异''的用法,并且不理解,从而可能留下data.table不好用,很古怪的印象,但是我觉得任何东西存在即合理,你学一个东西总得接受一些你可能不认可的东西,这样可能才是真正的学习,就像拿Python来做数据分析,我刚开始觉得pandas很难用,很反人类,但是后来知道python代码可以直接打包封装成exe后,觉得真香,说这么多主要是想表达我们学会挑选合适的工具用,适应它,用好它就可以了。

\hypertarget{i-j-by-ux4f7fux7528}{%
\subsection{i j by 使用}\label{i-j-by-ux4f7fux7528}}

使用data.table处理数据,接下来我们就用该函数读取数据演示i,j,by的简单使用。

\hypertarget{iux884cux7b5bux9009}{%
\subsubsection{i行筛选}\label{iux884cux7b5bux9009}}

行筛选是一种很常见的数据操作行为,类似我们Excel中的筛选,即按照一定条件筛选符合要求的数据。条件筛选一般分为单条件筛选、多条件筛选;

在筛选时涉及到条件判断,R语言中常用的条件判断分为逻辑运算、关系运算。常用的关系运算符 \textgreater、 \textless、==、!=、\textgreater=、\textless=分别代表大于、小于、等于、不等于、大于等于、小于等于。常用的逻辑运算符 \&、\textbar、!等。

\begin{Shaded}
\begin{Highlighting}[]
\CommentTok{\#单条件筛选}
\NormalTok{filghts[year }\SpecialCharTok{==} \DecValTok{2014}\NormalTok{] }\CommentTok{\#筛选year==2014}
\CommentTok{\#多条件筛选 用 \& 链接}
\NormalTok{flights[ year }\SpecialCharTok{==} \DecValTok{2014} \SpecialCharTok{\&}\NormalTok{ month }\SpecialCharTok{==} \DecValTok{6}\NormalTok{] }
\CommentTok{\# | 相当于中文条件或 }
\NormalTok{flights[ month }\SpecialCharTok{==} \DecValTok{5} \SpecialCharTok{|}\NormalTok{ month }\SpecialCharTok{==} \DecValTok{6}\NormalTok{] }
\CommentTok{\# \%in\% 类似sql中in用法}
\NormalTok{flights[month }\SpecialCharTok{\%in\%} \FunctionTok{c}\NormalTok{(}\DecValTok{1}\NormalTok{,}\DecValTok{3}\NormalTok{,}\DecValTok{5}\NormalTok{,}\DecValTok{7}\NormalTok{,}\DecValTok{9}\NormalTok{)] }
\CommentTok{\# \%between\% 类似sql中between and 用法}
\NormalTok{flights[month }\SpecialCharTok{\%between\%} \FunctionTok{c}\NormalTok{(}\DecValTok{1}\NormalTok{,}\DecValTok{7}\NormalTok{)]}
\end{Highlighting}
\end{Shaded}

\hypertarget{jux5217ux64cdux4f5c}{%
\subsubsection{j列操作}\label{jux5217ux64cdux4f5c}}

数据集较大、字段较多时,由于无效信息较多可以做适当精选,这时需要我们筛选列。与sql中的select用法一致,即保留想要的字段。

.()或list()是data.table中的比较特殊的实现列筛选的用法。常规数字索引,字符向量索引同样有效。

\begin{Shaded}
\begin{Highlighting}[]
\CommentTok{\#注意前面的. .()}
\NormalTok{flights[,.(year,month,day,dep\_delay,carrier,origin)] }
\CommentTok{\#\textgreater{}         year month day dep\_delay carrier origin}
\CommentTok{\#\textgreater{}      1: 2014     1   1        14      AA    JFK}
\CommentTok{\#\textgreater{}      2: 2014     1   1        {-}3      AA    JFK}
\CommentTok{\#\textgreater{}      3: 2014     1   1         2      AA    JFK}
\CommentTok{\#\textgreater{}      4: 2014     1   1        {-}8      AA    LGA}
\CommentTok{\#\textgreater{}      5: 2014     1   1         2      AA    JFK}
\CommentTok{\#\textgreater{}     {-}{-}{-}                                        }
\CommentTok{\#\textgreater{} 253312: 2014    10  31         1      UA    LGA}
\CommentTok{\#\textgreater{} 253313: 2014    10  31        {-}5      UA    EWR}
\CommentTok{\#\textgreater{} 253314: 2014    10  31        {-}8      MQ    LGA}
\CommentTok{\#\textgreater{} 253315: 2014    10  31        {-}4      MQ    LGA}
\CommentTok{\#\textgreater{} 253316: 2014    10  31        {-}5      MQ    LGA}
\CommentTok{\# flights[,list(year,month,day,dep\_delay,carrier,origin)]  same above}

\CommentTok{\# not run}
\CommentTok{\# flights[,1:3]}

\CommentTok{\# not run}
\CommentTok{\# flights[,c(\textquotesingle{}year\textquotesingle{},\textquotesingle{}month\textquotesingle{},\textquotesingle{}day\textquotesingle{})]}
\end{Highlighting}
\end{Shaded}

setcolorder函数可以调整列的顺序,将常用的字段信息排在前面可以用过该函数实现。

\begin{Shaded}
\begin{Highlighting}[]
\CommentTok{\# not run}
\CommentTok{\# setcolorder(x = flights,neworder = c( "month","day","dep\_delay" ,"arr\_delay","carrier" )) }
\CommentTok{\# 按照指定列顺序排序 其余字段保持不变,不是建立副本,是直接修改了flights 数据的列顺序}
\end{Highlighting}
\end{Shaded}

\begin{itemize}
\tightlist
\item
  常规计算
\end{itemize}

根据最开始的Excel透视表截图,我们想要获得如截图一样的结果该怎么实现呢?代码如下:

\begin{Shaded}
\begin{Highlighting}[]
\NormalTok{flights[year}\SpecialCharTok{==}\DecValTok{2014} \SpecialCharTok{\&}\NormalTok{ month}\SpecialCharTok{==}\DecValTok{6}\NormalTok{,.(求和项}\AttributeTok{distance=}\FunctionTok{sum}\NormalTok{(distance),平均距离}\OtherTok{=}\FunctionTok{mean}\NormalTok{(distance)),by}\OtherTok{=}\NormalTok{.(origin,dest)]}
\end{Highlighting}
\end{Shaded}

在i的位置做筛选,j的位置做计算,by指定分组字段。在j的位置可以做各种各样的计算,R中自带的函数,或者是自己定义的函数。

\begin{Shaded}
\begin{Highlighting}[]
\NormalTok{myfun }\OtherTok{\textless{}{-}} \ControlFlowTok{function}\NormalTok{(x)\{}
\NormalTok{    x}\SpecialCharTok{\^{}}\DecValTok{2}\SpecialCharTok{/}\DecValTok{2}
\NormalTok{\}}
\NormalTok{flights[year}\SpecialCharTok{==}\DecValTok{2014} \SpecialCharTok{\&}\NormalTok{ month}\SpecialCharTok{==}\DecValTok{6}\NormalTok{,.(}\FunctionTok{myfun}\NormalTok{(distance)),by}\OtherTok{=}\NormalTok{.(origin,dest)]}
\CommentTok{\#\textgreater{}        origin dest      V1}
\CommentTok{\#\textgreater{}     1:    JFK  LAX 3062813}
\CommentTok{\#\textgreater{}     2:    JFK  LAX 3062813}
\CommentTok{\#\textgreater{}     3:    JFK  LAX 3062813}
\CommentTok{\#\textgreater{}     4:    JFK  LAX 3062813}
\CommentTok{\#\textgreater{}     5:    JFK  LAX 3062813}
\CommentTok{\#\textgreater{}    {-}{-}{-}                    }
\CommentTok{\#\textgreater{} 26484:    JFK  HYA   19208}
\CommentTok{\#\textgreater{} 26485:    JFK  HYA   19208}
\CommentTok{\#\textgreater{} 26486:    JFK  HYA   19208}
\CommentTok{\#\textgreater{} 26487:    JFK  HYA   19208}
\CommentTok{\#\textgreater{} 26488:    JFK  HYA   19208}
\end{Highlighting}
\end{Shaded}

\hypertarget{by-ux5206ux7ec4}{%
\subsubsection{by 分组}\label{by-ux5206ux7ec4}}

分组是按照某种分组实现一定条件下某种聚合方式的计算。分组可以是单字段,多字段以及条件字段等。

1.按月分组

\begin{Shaded}
\begin{Highlighting}[]
\NormalTok{flights[,.(}\FunctionTok{sum}\NormalTok{(distance)),by}\OtherTok{=}\NormalTok{.(month)]}
\CommentTok{\#\textgreater{}     month       V1}
\CommentTok{\#\textgreater{}  1:     1 25112563}
\CommentTok{\#\textgreater{}  2:     2 22840391}
\CommentTok{\#\textgreater{}  3:     3 28716598}
\CommentTok{\#\textgreater{}  4:     4 27816797}
\CommentTok{\#\textgreater{}  5:     5 28030020}
\CommentTok{\#\textgreater{}  6:     6 29093557}
\CommentTok{\#\textgreater{}  7:     7 30059175}
\CommentTok{\#\textgreater{}  8:     8 30322047}
\CommentTok{\#\textgreater{}  9:     9 27615097}
\CommentTok{\#\textgreater{} 10:    10 28900834}
\end{Highlighting}
\end{Shaded}

2.多条件分组

\begin{Shaded}
\begin{Highlighting}[]
\NormalTok{dt }\OtherTok{\textless{}{-}}\NormalTok{ flights[,.(}\FunctionTok{sum}\NormalTok{(distance)),by}\OtherTok{=}\NormalTok{.(carrier,origin)]}
\FunctionTok{head}\NormalTok{(dt)}
\CommentTok{\#\textgreater{}    carrier origin       V1}
\CommentTok{\#\textgreater{} 1:      AA    JFK 20492213}
\CommentTok{\#\textgreater{} 2:      AA    LGA 12365282}
\CommentTok{\#\textgreater{} 3:      AA    EWR  3550217}
\CommentTok{\#\textgreater{} 4:      AS    EWR  1378748}
\CommentTok{\#\textgreater{} 5:      B6    JFK 38117662}
\CommentTok{\#\textgreater{} 6:      B6    EWR  4508574}
\CommentTok{\#可直接重新命名}
\NormalTok{dt }\OtherTok{\textless{}{-}}\NormalTok{ flights[,.(}\FunctionTok{sum}\NormalTok{(distance)),by}\OtherTok{=}\NormalTok{.(}\AttributeTok{newcol1 =}\NormalTok{ carrier,}\AttributeTok{newcol2 =}\NormalTok{ origin)]}
\FunctionTok{head}\NormalTok{(dt)}
\CommentTok{\#\textgreater{}    newcol1 newcol2       V1}
\CommentTok{\#\textgreater{} 1:      AA     JFK 20492213}
\CommentTok{\#\textgreater{} 2:      AA     LGA 12365282}
\CommentTok{\#\textgreater{} 3:      AA     EWR  3550217}
\CommentTok{\#\textgreater{} 4:      AS     EWR  1378748}
\CommentTok{\#\textgreater{} 5:      B6     JFK 38117662}
\CommentTok{\#\textgreater{} 6:      B6     EWR  4508574}
\end{Highlighting}
\end{Shaded}

3.按月份是否大于6分组

即得到是否大于6的两类分组

\begin{Shaded}
\begin{Highlighting}[]
\NormalTok{dt }\OtherTok{\textless{}{-}}\NormalTok{ flights[,.(}\FunctionTok{sum}\NormalTok{(distance)),by}\OtherTok{=}\NormalTok{.(month}\SpecialCharTok{\textgreater{}}\DecValTok{6}\NormalTok{)] }\CommentTok{\#by里面可以做计算}
\FunctionTok{head}\NormalTok{(dt)}
\CommentTok{\#\textgreater{}    month        V1}
\CommentTok{\#\textgreater{} 1: FALSE 161609926}
\CommentTok{\#\textgreater{} 2:  TRUE 116897153}
\end{Highlighting}
\end{Shaded}

\hypertarget{ux884cux5217ux7b5bux9009ux603bux7ed3}{%
\subsection{行列筛选总结}\label{ux884cux5217ux7b5bux9009ux603bux7ed3}}

行筛选在 i 的位置上进行, 列筛选在 j 的位置上进行;data.table中j的位置比较灵活多变,但是i的位置大部分时候都是进行条件筛选。我们通过上述的行列筛选已经大概知道data.table中i,j的用法。也就是我们常规数据清洗过程中的数据筛选过程,筛选符合要求的数据记录。

\begin{Shaded}
\begin{Highlighting}[]

\NormalTok{dt }\OtherTok{\textless{}{-}}\NormalTok{ flights[ year }\SpecialCharTok{==} \DecValTok{2014} \SpecialCharTok{\&}\NormalTok{ month }\SpecialCharTok{==} \DecValTok{6} \SpecialCharTok{\&}\NormalTok{ day }\SpecialCharTok{\textgreater{}=}\DecValTok{15}\NormalTok{,.(year,month,day,dep\_delay,carrier,origin)] }
\FunctionTok{head}\NormalTok{(dt)}
\CommentTok{\#\textgreater{}    year month day dep\_delay carrier origin}
\CommentTok{\#\textgreater{} 1: 2014     6  15        {-}4      AA    JFK}
\CommentTok{\#\textgreater{} 2: 2014     6  15        {-}8      AA    JFK}
\CommentTok{\#\textgreater{} 3: 2014     6  15       {-}12      AA    JFK}
\CommentTok{\#\textgreater{} 4: 2014     6  15        {-}4      AA    LGA}
\CommentTok{\#\textgreater{} 5: 2014     6  15        {-}3      AA    JFK}
\CommentTok{\#\textgreater{} 6: 2014     6  15         5      AA    JFK}
\end{Highlighting}
\end{Shaded}

\hypertarget{ux5e38ux89c4ux64cdux4f5c}{%
\section{常规操作}\label{ux5e38ux89c4ux64cdux4f5c}}

\hypertarget{ux884cux7b5bux9009}{%
\subsection{行筛选}\label{ux884cux7b5bux9009}}

上文已经大致讲过行筛选,但是行筛选使用有一定的技巧,涉及到运算的快慢。主要是逻辑条件的设置,交集并集之间的差异。除了上文中的关系运算筛选,逻辑运算筛选除外,data.table中还有几个常用的筛选函数。

\begin{itemize}
\tightlist
\item
  数字向量筛选
\end{itemize}

\%in\%用法与 sql 中 in 用法类似。

\begin{Shaded}
\begin{Highlighting}[]
\CommentTok{\# 筛选 \%in\% }
\NormalTok{flights[ hour }\SpecialCharTok{\%in\%} \FunctionTok{seq}\NormalTok{(}\DecValTok{1}\NormalTok{,}\DecValTok{24}\NormalTok{,}\DecValTok{2}\NormalTok{) ]}
\end{Highlighting}
\end{Shaded}

\begin{itemize}
\tightlist
\item
  字符向量筛选
\end{itemize}

\%chin\%用法与 \%in\% 类似,但仅仅针对字符。

\begin{Shaded}
\begin{Highlighting}[]
\CommentTok{\# 字符筛选}
\NormalTok{flights[ origin }\SpecialCharTok{\%chin\%} \FunctionTok{c}\NormalTok{(}\StringTok{\textquotesingle{}JFK\textquotesingle{}}\NormalTok{,}\StringTok{\textquotesingle{}LGA\textquotesingle{}}\NormalTok{)]}
\CommentTok{\# not run 同上 \%chin\% 对字符速度筛选速度更快}
\CommentTok{\#flights[ origin \%in\% c(\textquotesingle{}JFK\textquotesingle{},\textquotesingle{}LGA\textquotesingle{})]}
\end{Highlighting}
\end{Shaded}

\begin{itemize}
\tightlist
\item
  between 筛选
\end{itemize}

该函数的新特性矢量化挺实用。

\begin{Shaded}
\begin{Highlighting}[]
\CommentTok{\#between 函数参数}
\CommentTok{\#between(x, lower, upper, incbounds=TRUE, NAbounds=TRUE, check=FALSE)}
\NormalTok{X }\OtherTok{\textless{}{-}}  \FunctionTok{data.table}\NormalTok{(}\AttributeTok{a=}\DecValTok{1}\SpecialCharTok{:}\DecValTok{5}\NormalTok{, }\AttributeTok{b=}\DecValTok{6}\SpecialCharTok{:}\DecValTok{10}\NormalTok{, }\AttributeTok{c=}\FunctionTok{c}\NormalTok{(}\DecValTok{5}\SpecialCharTok{:}\DecValTok{1}\NormalTok{))}
\NormalTok{X[b }\SpecialCharTok{\%between\%} \FunctionTok{c}\NormalTok{(}\DecValTok{7}\NormalTok{,}\DecValTok{9}\NormalTok{)]}
\CommentTok{\#\textgreater{}    a b c}
\CommentTok{\#\textgreater{} 1: 2 7 4}
\CommentTok{\#\textgreater{} 2: 3 8 3}
\CommentTok{\#\textgreater{} 3: 4 9 2}
\NormalTok{X[}\FunctionTok{between}\NormalTok{(b, }\DecValTok{7}\NormalTok{, }\DecValTok{9}\NormalTok{)] }\CommentTok{\#效果同上}
\CommentTok{\#\textgreater{}    a b c}
\CommentTok{\#\textgreater{} 1: 2 7 4}
\CommentTok{\#\textgreater{} 2: 3 8 3}
\CommentTok{\#\textgreater{} 3: 4 9 2}
\NormalTok{X[c }\SpecialCharTok{\%between\%} \FunctionTok{list}\NormalTok{(a,b)] }\CommentTok{\# 矢量化}
\CommentTok{\#\textgreater{}    a b c}
\CommentTok{\#\textgreater{} 1: 1 6 5}
\CommentTok{\#\textgreater{} 2: 2 7 4}
\CommentTok{\#\textgreater{} 3: 3 8 3}
\end{Highlighting}
\end{Shaded}

\begin{itemize}
\tightlist
\item
  like 筛选
\end{itemize}

\%like\% 用法与SQL中 like 类似。

\begin{Shaded}
\begin{Highlighting}[]
\CommentTok{\# \%like\% 用法与SQL中 like 类似}
\NormalTok{DT }\OtherTok{=} \FunctionTok{data.table}\NormalTok{(}\AttributeTok{Name=}\FunctionTok{c}\NormalTok{(}\StringTok{"Mary"}\NormalTok{,}\StringTok{"George"}\NormalTok{,}\StringTok{"Martha"}\NormalTok{), }\AttributeTok{Salary=}\FunctionTok{c}\NormalTok{(}\DecValTok{2}\NormalTok{,}\DecValTok{3}\NormalTok{,}\DecValTok{4}\NormalTok{))}
\NormalTok{DT[Name }\SpecialCharTok{\%like\%} \StringTok{"\^{}Mar"}\NormalTok{]}
\CommentTok{\#\textgreater{}      Name Salary}
\CommentTok{\#\textgreater{} 1:   Mary      2}
\CommentTok{\#\textgreater{} 2: Martha      4}
\end{Highlighting}
\end{Shaded}

\hypertarget{ux65b0ux589eux66f4ux65b0ux5217}{%
\subsection{新增更新列}\label{ux65b0ux589eux66f4ux65b0ux5217}}

新增或删除或更新列是我们数据清洗过程中的常规操作,\texttt{data.table中}实现该类功能是通过\texttt{:=}符号实现。

\begin{itemize}
\tightlist
\item
  选择列
\end{itemize}

\begin{Shaded}
\begin{Highlighting}[]
\NormalTok{dt }\OtherTok{\textless{}{-}} \FunctionTok{data.table}\NormalTok{(}\AttributeTok{col1=}\DecValTok{1}\SpecialCharTok{:}\DecValTok{10}\NormalTok{,}\AttributeTok{col2=}\NormalTok{letters[}\DecValTok{1}\SpecialCharTok{:}\DecValTok{10}\NormalTok{],}\AttributeTok{col3=}\NormalTok{LETTERS[}\DecValTok{1}\SpecialCharTok{:}\DecValTok{10}\NormalTok{],}\AttributeTok{col4=}\DecValTok{1}\SpecialCharTok{:}\DecValTok{10}\NormalTok{)}
\NormalTok{dt[,.(col1,col2)]}
\CommentTok{\#\textgreater{}     col1 col2}
\CommentTok{\#\textgreater{}  1:    1    a}
\CommentTok{\#\textgreater{}  2:    2    b}
\CommentTok{\#\textgreater{}  3:    3    c}
\CommentTok{\#\textgreater{}  4:    4    d}
\CommentTok{\#\textgreater{}  5:    5    e}
\CommentTok{\#\textgreater{}  6:    6    f}
\CommentTok{\#\textgreater{}  7:    7    g}
\CommentTok{\#\textgreater{}  8:    8    h}
\CommentTok{\#\textgreater{}  9:    9    i}
\CommentTok{\#\textgreater{} 10:   10    j}
\CommentTok{\# same above}
\NormalTok{dt[,}\FunctionTok{list}\NormalTok{(col1,col2)]}
\CommentTok{\#\textgreater{}     col1 col2}
\CommentTok{\#\textgreater{}  1:    1    a}
\CommentTok{\#\textgreater{}  2:    2    b}
\CommentTok{\#\textgreater{}  3:    3    c}
\CommentTok{\#\textgreater{}  4:    4    d}
\CommentTok{\#\textgreater{}  5:    5    e}
\CommentTok{\#\textgreater{}  6:    6    f}
\CommentTok{\#\textgreater{}  7:    7    g}
\CommentTok{\#\textgreater{}  8:    8    h}
\CommentTok{\#\textgreater{}  9:    9    i}
\CommentTok{\#\textgreater{} 10:   10    j}
\end{Highlighting}
\end{Shaded}

\begin{itemize}
\tightlist
\item
  新增列
\end{itemize}

如下所示:新增addcol列,最后的{[}{]}是为了显示新增列的数据框,可不增加。

\begin{Shaded}
\begin{Highlighting}[]
\CommentTok{\#data.table()函数创建data.table数据框}
\NormalTok{dt }\OtherTok{\textless{}{-}} \FunctionTok{data.table}\NormalTok{(}\AttributeTok{col1=}\DecValTok{1}\SpecialCharTok{:}\DecValTok{10}\NormalTok{,}\AttributeTok{col2=}\NormalTok{letters[}\DecValTok{1}\SpecialCharTok{:}\DecValTok{10}\NormalTok{],}\AttributeTok{col3=}\NormalTok{LETTERS[}\DecValTok{1}\SpecialCharTok{:}\DecValTok{10}\NormalTok{],}\AttributeTok{col4=}\DecValTok{1}\SpecialCharTok{:}\DecValTok{10}\NormalTok{)}
\CommentTok{\# 新增列 :=}
\NormalTok{dt[,addcol}\SpecialCharTok{:}\ErrorTok{=}\FunctionTok{rep}\NormalTok{(}\StringTok{\textquotesingle{}新列\textquotesingle{}}\NormalTok{,}\DecValTok{10}\NormalTok{)][] }\CommentTok{\#最后的[]是为了显示新增列的数据框,可不增加}
\CommentTok{\#\textgreater{}     col1 col2 col3 col4 addcol}
\CommentTok{\#\textgreater{}  1:    1    a    A    1   新列}
\CommentTok{\#\textgreater{}  2:    2    b    B    2   新列}
\CommentTok{\#\textgreater{}  3:    3    c    C    3   新列}
\CommentTok{\#\textgreater{}  4:    4    d    D    4   新列}
\CommentTok{\#\textgreater{}  5:    5    e    E    5   新列}
\CommentTok{\#\textgreater{}  6:    6    f    F    6   新列}
\CommentTok{\#\textgreater{}  7:    7    g    G    7   新列}
\CommentTok{\#\textgreater{}  8:    8    h    H    8   新列}
\CommentTok{\#\textgreater{}  9:    9    i    I    9   新列}
\CommentTok{\#\textgreater{} 10:   10    j    J   10   新列}
\CommentTok{\#dt[,addcol:=rep(\textquotesingle{}新列\textquotesingle{},10)] 不会显示返回结果,加上[]会显示返回}
\CommentTok{\# 新增多列}
\NormalTok{dt[,}\StringTok{\textasciigrave{}}\AttributeTok{:=}\StringTok{\textasciigrave{}}\NormalTok{(}\AttributeTok{newcol1=}\FunctionTok{rep}\NormalTok{(}\StringTok{\textquotesingle{}newcol1\textquotesingle{}}\NormalTok{,}\DecValTok{10}\NormalTok{),}\AttributeTok{newcol2=}\FunctionTok{rep}\NormalTok{(}\StringTok{\textquotesingle{}newcol2\textquotesingle{}}\NormalTok{,}\DecValTok{10}\NormalTok{))][]}
\CommentTok{\#\textgreater{}     col1 col2 col3 col4 addcol newcol1 newcol2}
\CommentTok{\#\textgreater{}  1:    1    a    A    1   新列 newcol1 newcol2}
\CommentTok{\#\textgreater{}  2:    2    b    B    2   新列 newcol1 newcol2}
\CommentTok{\#\textgreater{}  3:    3    c    C    3   新列 newcol1 newcol2}
\CommentTok{\#\textgreater{}  4:    4    d    D    4   新列 newcol1 newcol2}
\CommentTok{\#\textgreater{}  5:    5    e    E    5   新列 newcol1 newcol2}
\CommentTok{\#\textgreater{}  6:    6    f    F    6   新列 newcol1 newcol2}
\CommentTok{\#\textgreater{}  7:    7    g    G    7   新列 newcol1 newcol2}
\CommentTok{\#\textgreater{}  8:    8    h    H    8   新列 newcol1 newcol2}
\CommentTok{\#\textgreater{}  9:    9    i    I    9   新列 newcol1 newcol2}
\CommentTok{\#\textgreater{} 10:   10    j    J   10   新列 newcol1 newcol2}
\end{Highlighting}
\end{Shaded}

\begin{itemize}
\tightlist
\item
  删除列
\end{itemize}

删除列即将列赋值NULL即可

\begin{Shaded}
\begin{Highlighting}[]
\CommentTok{\# 删除列}
\NormalTok{dt[,col1}\SpecialCharTok{:}\ErrorTok{=}\ConstantTok{NULL}\NormalTok{][]}
\CommentTok{\#\textgreater{}     col2 col3 col4 addcol newcol1 newcol2}
\CommentTok{\#\textgreater{}  1:    a    A    1   新列 newcol1 newcol2}
\CommentTok{\#\textgreater{}  2:    b    B    2   新列 newcol1 newcol2}
\CommentTok{\#\textgreater{}  3:    c    C    3   新列 newcol1 newcol2}
\CommentTok{\#\textgreater{}  4:    d    D    4   新列 newcol1 newcol2}
\CommentTok{\#\textgreater{}  5:    e    E    5   新列 newcol1 newcol2}
\CommentTok{\#\textgreater{}  6:    f    F    6   新列 newcol1 newcol2}
\CommentTok{\#\textgreater{}  7:    g    G    7   新列 newcol1 newcol2}
\CommentTok{\#\textgreater{}  8:    h    H    8   新列 newcol1 newcol2}
\CommentTok{\#\textgreater{}  9:    i    I    9   新列 newcol1 newcol2}
\CommentTok{\#\textgreater{} 10:    j    J   10   新列 newcol1 newcol2}
\CommentTok{\# 删除多列}
\NormalTok{dt[,}\FunctionTok{c}\NormalTok{(}\StringTok{\textquotesingle{}newcol1\textquotesingle{}}\NormalTok{,}\StringTok{\textquotesingle{}newcol2\textquotesingle{}}\NormalTok{)}\SpecialCharTok{:}\ErrorTok{=}\ConstantTok{NULL}\NormalTok{][]}
\CommentTok{\#\textgreater{}     col2 col3 col4 addcol}
\CommentTok{\#\textgreater{}  1:    a    A    1   新列}
\CommentTok{\#\textgreater{}  2:    b    B    2   新列}
\CommentTok{\#\textgreater{}  3:    c    C    3   新列}
\CommentTok{\#\textgreater{}  4:    d    D    4   新列}
\CommentTok{\#\textgreater{}  5:    e    E    5   新列}
\CommentTok{\#\textgreater{}  6:    f    F    6   新列}
\CommentTok{\#\textgreater{}  7:    g    G    7   新列}
\CommentTok{\#\textgreater{}  8:    h    H    8   新列}
\CommentTok{\#\textgreater{}  9:    i    I    9   新列}
\CommentTok{\#\textgreater{} 10:    j    J   10   新列}
\end{Highlighting}
\end{Shaded}

\begin{itemize}
\tightlist
\item
  更新
\end{itemize}

更新即重新赋值,将现有列参与计算等于是重新赋值,可以看成是更新列。

\begin{Shaded}
\begin{Highlighting}[]
\CommentTok{\# 更新列}
\NormalTok{dt[,col1}\SpecialCharTok{:}\ErrorTok{=}\DecValTok{11}\SpecialCharTok{:}\DecValTok{20}\NormalTok{][]}
\CommentTok{\#\textgreater{}     col2 col3 col4 addcol col1}
\CommentTok{\#\textgreater{}  1:    a    A    1   新列   11}
\CommentTok{\#\textgreater{}  2:    b    B    2   新列   12}
\CommentTok{\#\textgreater{}  3:    c    C    3   新列   13}
\CommentTok{\#\textgreater{}  4:    d    D    4   新列   14}
\CommentTok{\#\textgreater{}  5:    e    E    5   新列   15}
\CommentTok{\#\textgreater{}  6:    f    F    6   新列   16}
\CommentTok{\#\textgreater{}  7:    g    G    7   新列   17}
\CommentTok{\#\textgreater{}  8:    h    H    8   新列   18}
\CommentTok{\#\textgreater{}  9:    i    I    9   新列   19}
\CommentTok{\#\textgreater{} 10:    j    J   10   新列   20}
\CommentTok{\# not run }
\CommentTok{\# 两列间计算 也可以理解为更新}
\NormalTok{dt[,newcol}\SpecialCharTok{:}\ErrorTok{=}\NormalTok{col1}\SpecialCharTok{/}\NormalTok{col4]}
\end{Highlighting}
\end{Shaded}

\begin{quote}
Note: DT{[}a \textgreater{} 4, b := c{]} is different from DT{[}a \textgreater{} 4{]}{[}, b := c{]}
\end{quote}

\hypertarget{ux6392ux5e8f}{%
\subsection{排序}\label{ux6392ux5e8f}}

当我们清洗数据时,我们需要将数据框排序,我们可以使用\texttt{setorder}或\texttt{setorderv}函数实现排序。函数是\texttt{data.table}包的函数,比base R 中的\texttt{order}函数要节省内存。
注意:按照函数文档说法:Note that queries like x{[}order(.){]} are optimised internally to use data.table's fast order。即x{[}order(.){]}这样的用法会被优化为data.table的排序方法。

\begin{Shaded}
\begin{Highlighting}[]
\FunctionTok{set.seed}\NormalTok{(45L)}
\NormalTok{DT }\OtherTok{=} \FunctionTok{data.table}\NormalTok{(}\AttributeTok{A=}\FunctionTok{sample}\NormalTok{(}\DecValTok{3}\NormalTok{, }\DecValTok{10}\NormalTok{, }\ConstantTok{TRUE}\NormalTok{),}
         \AttributeTok{B=}\FunctionTok{sample}\NormalTok{(letters[}\DecValTok{1}\SpecialCharTok{:}\DecValTok{3}\NormalTok{], }\DecValTok{10}\NormalTok{, }\ConstantTok{TRUE}\NormalTok{), }\AttributeTok{C=}\FunctionTok{sample}\NormalTok{(}\DecValTok{10}\NormalTok{))}

\FunctionTok{setorder}\NormalTok{(DT, A, }\SpecialCharTok{{-}}\NormalTok{B) }\CommentTok{\#将DT按照A、B排序 A 升序,{-}B降序}

\CommentTok{\# 和上面同样的效果 但是函数变成 setorderv}
\FunctionTok{setorderv}\NormalTok{(DT, }\FunctionTok{c}\NormalTok{(}\StringTok{"A"}\NormalTok{, }\StringTok{"B"}\NormalTok{), }\FunctionTok{c}\NormalTok{(}\DecValTok{1}\NormalTok{, }\SpecialCharTok{{-}}\DecValTok{1}\NormalTok{))}
\end{Highlighting}
\end{Shaded}

\hypertarget{ux5e38ux7528ux51fdux6570-1}{%
\section{常用函数}\label{ux5e38ux7528ux51fdux6570-1}}

常用函数指我们常用功能的函数,如排名、排序、非重复计数、判断、表连接、长宽转换等功能。

\hypertarget{ux7279ux6b8aux7b26ux53f7}{%
\subsection{特殊符号}\label{ux7279ux6b8aux7b26ux53f7}}

.SD,.BY,.N,.I,.NGRP和.GRP,.SDcols等,只能用在 j 的位置,.N 可以用在 i 的位置。

如果想要记住用法需要自己多尝试练习,对于我来说.N使用较多。

\begin{Shaded}
\begin{Highlighting}[]
\NormalTok{DT }\OtherTok{=} \FunctionTok{data.table}\NormalTok{(}\AttributeTok{x=}\FunctionTok{rep}\NormalTok{(}\FunctionTok{c}\NormalTok{(}\StringTok{"b"}\NormalTok{,}\StringTok{"a"}\NormalTok{,}\StringTok{"c"}\NormalTok{),}\AttributeTok{each=}\DecValTok{3}\NormalTok{), }\AttributeTok{v=}\FunctionTok{c}\NormalTok{(}\DecValTok{1}\NormalTok{,}\DecValTok{1}\NormalTok{,}\DecValTok{1}\NormalTok{,}\DecValTok{2}\NormalTok{,}\DecValTok{2}\NormalTok{,}\DecValTok{1}\NormalTok{,}\DecValTok{1}\NormalTok{,}\DecValTok{2}\NormalTok{,}\DecValTok{2}\NormalTok{), }\AttributeTok{y=}\FunctionTok{c}\NormalTok{(}\DecValTok{1}\NormalTok{,}\DecValTok{3}\NormalTok{,}\DecValTok{6}\NormalTok{), }\AttributeTok{a=}\DecValTok{1}\SpecialCharTok{:}\DecValTok{9}\NormalTok{, }\AttributeTok{b=}\DecValTok{9}\SpecialCharTok{:}\DecValTok{1}\NormalTok{)}
\NormalTok{DT}
\CommentTok{\#\textgreater{}    x v y a b}
\CommentTok{\#\textgreater{} 1: b 1 1 1 9}
\CommentTok{\#\textgreater{} 2: b 1 3 2 8}
\CommentTok{\#\textgreater{} 3: b 1 6 3 7}
\CommentTok{\#\textgreater{} 4: a 2 1 4 6}
\CommentTok{\#\textgreater{} 5: a 2 3 5 5}
\CommentTok{\#\textgreater{} 6: a 1 6 6 4}
\CommentTok{\#\textgreater{} 7: c 1 1 7 3}
\CommentTok{\#\textgreater{} 8: c 2 3 8 2}
\CommentTok{\#\textgreater{} 9: c 2 6 9 1}
\NormalTok{X }\OtherTok{=} \FunctionTok{data.table}\NormalTok{(}\AttributeTok{x=}\FunctionTok{c}\NormalTok{(}\StringTok{"c"}\NormalTok{,}\StringTok{"b"}\NormalTok{), }\AttributeTok{v=}\DecValTok{8}\SpecialCharTok{:}\DecValTok{7}\NormalTok{, }\AttributeTok{foo=}\FunctionTok{c}\NormalTok{(}\DecValTok{4}\NormalTok{,}\DecValTok{2}\NormalTok{))}
\NormalTok{X}
\CommentTok{\#\textgreater{}    x v foo}
\CommentTok{\#\textgreater{} 1: c 8   4}
\CommentTok{\#\textgreater{} 2: b 7   2}

\CommentTok{\# 用在i的位置}
\NormalTok{DT[.N] }\CommentTok{\#取DT最后一行,.N 计数函数}
\CommentTok{\#\textgreater{}    x v y a b}
\CommentTok{\#\textgreater{} 1: c 2 6 9 1}
\NormalTok{DT[,.N] }\CommentTok{\#DT 共有多少行记录 返回一个整数}
\CommentTok{\#\textgreater{} [1] 9}
\NormalTok{DT[, .N, by}\OtherTok{=}\NormalTok{x]  }\CommentTok{\#分组计数}
\CommentTok{\#\textgreater{}    x N}
\CommentTok{\#\textgreater{} 1: b 3}
\CommentTok{\#\textgreater{} 2: a 3}
\CommentTok{\#\textgreater{} 3: c 3}
\NormalTok{DT[, .SD, .SDcols}\OtherTok{=}\NormalTok{x}\SpecialCharTok{:}\NormalTok{y]  }\CommentTok{\# 选择x 到y 列}
\CommentTok{\#\textgreater{}    x v y}
\CommentTok{\#\textgreater{} 1: b 1 1}
\CommentTok{\#\textgreater{} 2: b 1 3}
\CommentTok{\#\textgreater{} 3: b 1 6}
\CommentTok{\#\textgreater{} 4: a 2 1}
\CommentTok{\#\textgreater{} 5: a 2 3}
\CommentTok{\#\textgreater{} 6: a 1 6}
\CommentTok{\#\textgreater{} 7: c 1 1}
\CommentTok{\#\textgreater{} 8: c 2 3}
\CommentTok{\#\textgreater{} 9: c 2 6}
\CommentTok{\#DT[, .SD, .SDcols=c("x","y")] 与上面不一样}

\NormalTok{DT[, .SD[}\DecValTok{1}\NormalTok{]] }\CommentTok{\#取第一行}
\CommentTok{\#\textgreater{}    x v y a b}
\CommentTok{\#\textgreater{} 1: b 1 1 1 9}
\NormalTok{DT[, .SD[}\DecValTok{1}\NormalTok{], by}\OtherTok{=}\NormalTok{x] }\CommentTok{\#按x列分组后}
\CommentTok{\#\textgreater{}    x v y a b}
\CommentTok{\#\textgreater{} 1: b 1 1 1 9}
\CommentTok{\#\textgreater{} 2: a 2 1 4 6}
\CommentTok{\#\textgreater{} 3: c 1 1 7 3}
\NormalTok{DT[, }\FunctionTok{c}\NormalTok{(.N, }\FunctionTok{lapply}\NormalTok{(.SD, sum)), by}\OtherTok{=}\NormalTok{x] }\CommentTok{\#按照x分组后 行数计数和每列求和}
\CommentTok{\#\textgreater{}    x N v  y  a  b}
\CommentTok{\#\textgreater{} 1: b 3 3 10  6 24}
\CommentTok{\#\textgreater{} 2: a 3 5 10 15 15}
\CommentTok{\#\textgreater{} 3: c 3 5 10 24  6}
\end{Highlighting}
\end{Shaded}

\hypertarget{ux6392ux5e8fux51fdux6570-1}{%
\subsection{排序函数}\label{ux6392ux5e8fux51fdux6570-1}}

\texttt{frank}和\texttt{frankv}函数参数如下:

\begin{Shaded}
\begin{Highlighting}[]
\FunctionTok{frank}\NormalTok{(x, ..., }\AttributeTok{na.last=}\ConstantTok{TRUE}\NormalTok{, }\AttributeTok{ties.method=}\FunctionTok{c}\NormalTok{(}\StringTok{"average"}\NormalTok{,}
  \StringTok{"first"}\NormalTok{, }\StringTok{"last"}\NormalTok{, }\StringTok{"random"}\NormalTok{, }\StringTok{"max"}\NormalTok{, }\StringTok{"min"}\NormalTok{, }\StringTok{"dense"}\NormalTok{))}

\FunctionTok{frankv}\NormalTok{(x, }\AttributeTok{cols=}\FunctionTok{seq\_along}\NormalTok{(x), }\AttributeTok{order=}\NormalTok{1L, }\AttributeTok{na.last=}\ConstantTok{TRUE}\NormalTok{,}
      \AttributeTok{ties.method=}\FunctionTok{c}\NormalTok{(}\StringTok{"average"}\NormalTok{, }\StringTok{"first"}\NormalTok{, }\StringTok{"random"}\NormalTok{,}
        \StringTok{"max"}\NormalTok{, }\StringTok{"min"}\NormalTok{, }\StringTok{"dense"}\NormalTok{))}
\end{Highlighting}
\end{Shaded}

官方案例,如下所示:

\begin{Shaded}
\begin{Highlighting}[]
\CommentTok{\# on vectors}
\NormalTok{x }\OtherTok{=} \FunctionTok{c}\NormalTok{(}\DecValTok{4}\NormalTok{, }\DecValTok{1}\NormalTok{, }\DecValTok{4}\NormalTok{, }\ConstantTok{NA}\NormalTok{, }\DecValTok{1}\NormalTok{, }\ConstantTok{NA}\NormalTok{, }\DecValTok{4}\NormalTok{)}
\CommentTok{\# NAs are considered identical (unlike base R)}
\CommentTok{\# default is average}
\FunctionTok{frankv}\NormalTok{(x) }\CommentTok{\# na.last=TRUE}
\CommentTok{\#\textgreater{} [1] 4.0 1.5 4.0 6.5 1.5 6.5 4.0}
\FunctionTok{frankv}\NormalTok{(x, }\AttributeTok{na.last=}\ConstantTok{FALSE}\NormalTok{)}
\CommentTok{\#\textgreater{} [1] 6.0 3.5 6.0 1.5 3.5 1.5 6.0}

\CommentTok{\# on data.table}
\NormalTok{DT }\OtherTok{=} \FunctionTok{data.table}\NormalTok{(x, }\AttributeTok{y=}\FunctionTok{c}\NormalTok{(}\DecValTok{1}\NormalTok{, }\DecValTok{1}\NormalTok{, }\DecValTok{1}\NormalTok{, }\DecValTok{0}\NormalTok{, }\ConstantTok{NA}\NormalTok{, }\DecValTok{0}\NormalTok{, }\DecValTok{2}\NormalTok{))}
\FunctionTok{frankv}\NormalTok{(DT, }\AttributeTok{cols=}\StringTok{"x"}\NormalTok{) }\CommentTok{\# same as frankv(x) from before}
\CommentTok{\#\textgreater{} [1] 4.0 1.5 4.0 6.5 1.5 6.5 4.0}
\FunctionTok{frankv}\NormalTok{(DT, }\AttributeTok{cols=}\StringTok{"x"}\NormalTok{, }\AttributeTok{na.last=}\StringTok{"keep"}\NormalTok{)}
\CommentTok{\#\textgreater{} [1] 4.0 1.5 4.0  NA 1.5  NA 4.0}
\FunctionTok{frankv}\NormalTok{(DT, }\AttributeTok{cols=}\StringTok{"x"}\NormalTok{, }\AttributeTok{ties.method=}\StringTok{"dense"}\NormalTok{, }\AttributeTok{na.last=}\ConstantTok{NA}\NormalTok{)}
\CommentTok{\#\textgreater{} [1] 2 1 2 1 2}
\FunctionTok{frank}\NormalTok{(DT, x, }\AttributeTok{ties.method=}\StringTok{"dense"}\NormalTok{, }\AttributeTok{na.last=}\ConstantTok{NA}\NormalTok{) }\CommentTok{\# equivalent of above using frank}
\CommentTok{\#\textgreater{} [1] 2 1 2 1 2}
\end{Highlighting}
\end{Shaded}

\begin{itemize}
\tightlist
\item
  frankv在排序时,NA被认为是一样的,基础base R 中认为不一样.
\end{itemize}

\begin{Shaded}
\begin{Highlighting}[]
\NormalTok{x }\OtherTok{\textless{}{-}}  \FunctionTok{c}\NormalTok{(}\DecValTok{4}\NormalTok{, }\DecValTok{1}\NormalTok{, }\DecValTok{4}\NormalTok{, }\ConstantTok{NA}\NormalTok{, }\DecValTok{1}\NormalTok{, }\ConstantTok{NA}\NormalTok{, }\DecValTok{4}\NormalTok{) }
\FunctionTok{frankv}\NormalTok{(x)}
\CommentTok{\#\textgreater{} [1] 4.0 1.5 4.0 6.5 1.5 6.5 4.0}
\FunctionTok{rank}\NormalTok{(x)}
\CommentTok{\#\textgreater{} [1] 4.0 1.5 4.0 6.0 1.5 7.0 4.0}
\end{Highlighting}
\end{Shaded}

\begin{itemize}
\tightlist
\item
  升序降序选择
\end{itemize}

order参数只能为1或者-1.默认为1代表升序

\begin{Shaded}
\begin{Highlighting}[]
\FunctionTok{frankv}\NormalTok{(x,}\AttributeTok{order =}\NormalTok{ 1L)}
\CommentTok{\#\textgreater{} [1] 4.0 1.5 4.0 6.5 1.5 6.5 4.0}
\FunctionTok{frankv}\NormalTok{(x,}\AttributeTok{order =} \SpecialCharTok{{-}}\NormalTok{1L)}
\CommentTok{\#\textgreater{} [1] 2.0 4.5 2.0 6.5 4.5 6.5 2.0}
\end{Highlighting}
\end{Shaded}

\begin{itemize}
\tightlist
\item
  排序方式选择
\end{itemize}

默认 average,还有dense,random,first,last,max,min等方式。其中dense是紧凑排名,random是随机让相同的随机排列后排名

\begin{Shaded}
\begin{Highlighting}[]
\NormalTok{x }\OtherTok{\textless{}{-}} \FunctionTok{c}\NormalTok{(}\DecValTok{1}\NormalTok{,}\DecValTok{1}\NormalTok{,}\DecValTok{1}\NormalTok{,}\DecValTok{2}\NormalTok{,}\DecValTok{3}\NormalTok{)}
\FunctionTok{frankv}\NormalTok{(x)  }\CommentTok{\#大小相同 排名相同,下一位排名除以2}
\FunctionTok{frankv}\NormalTok{(x,}\AttributeTok{ties.method =} \StringTok{\textquotesingle{}min\textquotesingle{}}\NormalTok{)  }\CommentTok{\#大小相同 排名相同,取最小排名}
\FunctionTok{frankv}\NormalTok{(x,}\AttributeTok{ties.method =} \StringTok{\textquotesingle{}max\textquotesingle{}}\NormalTok{)  }\CommentTok{\#大小相同 排名相同,取最大排名}
\FunctionTok{frankv}\NormalTok{(x,}\AttributeTok{ties.method =} \StringTok{\textquotesingle{}first\textquotesingle{}}\NormalTok{) }\CommentTok{\#相同大小排名以后往后递增 根据实际情况决定}
\FunctionTok{frankv}\NormalTok{(x,}\AttributeTok{ties.method =} \StringTok{\textquotesingle{}dense\textquotesingle{}}\NormalTok{)}
\FunctionTok{frankv}\NormalTok{(x,}\AttributeTok{ties.method =} \StringTok{\textquotesingle{}random\textquotesingle{}}\NormalTok{)}
\end{Highlighting}
\end{Shaded}

\begin{itemize}
\tightlist
\item
  NA处理
\end{itemize}

默认是将NA排在最后,NAs是相同的,与base R 不一样。

na.last参数等于TRUE时,缺失值被排最后;如果等于FALSE,放在前面;如果等于NA,将被移除;如果等于``keep'',将会保留NA.

\begin{Shaded}
\begin{Highlighting}[]
\FunctionTok{frankv}\NormalTok{(}\FunctionTok{c}\NormalTok{(}\ConstantTok{NA}\NormalTok{,}\ConstantTok{NA}\NormalTok{,}\DecValTok{1}\NormalTok{,}\DecValTok{2}\NormalTok{,}\DecValTok{3}\NormalTok{), }\AttributeTok{na.last =} \ConstantTok{TRUE}\NormalTok{,}\AttributeTok{ties.method =} \StringTok{\textquotesingle{}first\textquotesingle{}}\NormalTok{)}
\CommentTok{\#\textgreater{} [1] 4 5 1 2 3}
\FunctionTok{frankv}\NormalTok{(}\FunctionTok{c}\NormalTok{(}\ConstantTok{NA}\NormalTok{,}\ConstantTok{NA}\NormalTok{,}\DecValTok{1}\NormalTok{,}\DecValTok{2}\NormalTok{,}\DecValTok{3}\NormalTok{), }\AttributeTok{na.last =} \ConstantTok{FALSE}\NormalTok{,}\AttributeTok{ties.method =} \StringTok{\textquotesingle{}first\textquotesingle{}}\NormalTok{)}
\CommentTok{\#\textgreater{} [1] 1 2 3 4 5}
\FunctionTok{frankv}\NormalTok{(}\FunctionTok{c}\NormalTok{(}\ConstantTok{NA}\NormalTok{,}\ConstantTok{NA}\NormalTok{,}\DecValTok{1}\NormalTok{,}\DecValTok{2}\NormalTok{,}\DecValTok{3}\NormalTok{), }\AttributeTok{na.last =} \ConstantTok{NA}\NormalTok{,}\AttributeTok{ties.method =} \StringTok{\textquotesingle{}first\textquotesingle{}}\NormalTok{)}
\CommentTok{\#\textgreater{} [1] 1 2 3}
\FunctionTok{frankv}\NormalTok{(}\FunctionTok{c}\NormalTok{(}\ConstantTok{NA}\NormalTok{,}\ConstantTok{NA}\NormalTok{,}\DecValTok{1}\NormalTok{,}\DecValTok{2}\NormalTok{,}\DecValTok{3}\NormalTok{), }\AttributeTok{na.last =} \StringTok{\textquotesingle{}keep\textquotesingle{}}\NormalTok{,}\AttributeTok{ties.method =} \StringTok{\textquotesingle{}first\textquotesingle{}}\NormalTok{)}
\CommentTok{\#\textgreater{} [1] NA NA  1  2  3}
\end{Highlighting}
\end{Shaded}

\hypertarget{ux975eux91cdux590dux8ba1ux6570}{%
\subsection{非重复计数}\label{ux975eux91cdux590dux8ba1ux6570}}

\texttt{uniqueN}相当于\texttt{length(unique(x))},但是计算更快,内存效率更高。

\begin{Shaded}
\begin{Highlighting}[]
\NormalTok{x }\OtherTok{\textless{}{-}}\FunctionTok{sample}\NormalTok{(}\DecValTok{1}\SpecialCharTok{:}\DecValTok{10}\NormalTok{,}\DecValTok{50}\NormalTok{,}\AttributeTok{replace =} \ConstantTok{TRUE}\NormalTok{)}
\FunctionTok{uniqueN}\NormalTok{(x)}
\CommentTok{\#\textgreater{} [1] 10}

\NormalTok{DT }\OtherTok{\textless{}{-}} \FunctionTok{data.table}\NormalTok{(}\AttributeTok{A =} \FunctionTok{rep}\NormalTok{(}\DecValTok{1}\SpecialCharTok{:}\DecValTok{3}\NormalTok{, }\AttributeTok{each=}\DecValTok{4}\NormalTok{), }\AttributeTok{B =} \FunctionTok{rep}\NormalTok{(}\DecValTok{1}\SpecialCharTok{:}\DecValTok{4}\NormalTok{, }\AttributeTok{each=}\DecValTok{3}\NormalTok{),}
                 \AttributeTok{C =} \FunctionTok{rep}\NormalTok{(}\DecValTok{1}\SpecialCharTok{:}\DecValTok{2}\NormalTok{, }\DecValTok{6}\NormalTok{), }\AttributeTok{key =} \StringTok{"A,B"}\NormalTok{)}

\FunctionTok{uniqueN}\NormalTok{(DT, }\AttributeTok{by =} \FunctionTok{key}\NormalTok{(DT))}
\CommentTok{\#\textgreater{} [1] 6}
\FunctionTok{uniqueN}\NormalTok{(DT)}
\CommentTok{\#\textgreater{} [1] 10}
\end{Highlighting}
\end{Shaded}

\hypertarget{ux5224ux65adux51fdux6570}{%
\subsection{判断函数}\label{ux5224ux65adux51fdux6570}}

\begin{itemize}
\tightlist
\item
  fifelse
\end{itemize}

fifelse()类似\texttt{dplyr::if\_else()}函数,相比base::ifelse() 更快。

\begin{Shaded}
\begin{Highlighting}[]
\NormalTok{x }\OtherTok{\textless{}{-}}  \FunctionTok{c}\NormalTok{(}\DecValTok{1}\SpecialCharTok{:}\DecValTok{4}\NormalTok{, }\DecValTok{3}\SpecialCharTok{:}\DecValTok{2}\NormalTok{, }\DecValTok{1}\SpecialCharTok{:}\DecValTok{4}\NormalTok{,}\DecValTok{5}\NormalTok{)}
\FunctionTok{fifelse}\NormalTok{(x }\SpecialCharTok{\textgreater{}}\NormalTok{ 2L, x, x }\SpecialCharTok{{-}}\NormalTok{ 1L)}
\CommentTok{\#\textgreater{}  [1] 0 1 3 4 3 1 0 1 3 4 5}

\FunctionTok{fifelse}\NormalTok{(x }\SpecialCharTok{\textgreater{}}\NormalTok{ 2L,}\FunctionTok{fifelse}\NormalTok{(x }\SpecialCharTok{\textgreater{}=}\NormalTok{ 4L,x }\SpecialCharTok{+}\NormalTok{ 1L,x),x}\SpecialCharTok{{-}}\NormalTok{1L)}
\CommentTok{\#\textgreater{}  [1] 0 1 3 5 3 1 0 1 3 5 6}
\end{Highlighting}
\end{Shaded}

\begin{itemize}
\tightlist
\item
  fcase
\end{itemize}

与sql中的case when,与dplyr中的\texttt{case\_when()}函数用法相似。相比fifelse相比,嵌套更加方便。

\begin{Shaded}
\begin{Highlighting}[]
\NormalTok{x }\OtherTok{=} \DecValTok{1}\SpecialCharTok{:}\DecValTok{10}
\FunctionTok{fcase}\NormalTok{(}
\NormalTok{    x }\SpecialCharTok{\textless{}}\NormalTok{ 5L, 1L,}
\NormalTok{    x }\SpecialCharTok{\textgreater{}}\NormalTok{ 5L, 3L}
\NormalTok{)}
\CommentTok{\#\textgreater{}  [1]  1  1  1  1 NA  3  3  3  3  3}

\CommentTok{\# not run 两种函数实现方式}
\FunctionTok{fifelse}\NormalTok{(x }\SpecialCharTok{\textgreater{}} \DecValTok{5}\NormalTok{,}\FunctionTok{fifelse}\NormalTok{(x }\SpecialCharTok{\textgreater{}}\DecValTok{8}\NormalTok{,}\DecValTok{2}\NormalTok{,}\DecValTok{1}\NormalTok{),}\DecValTok{0}\NormalTok{)}
\CommentTok{\#\textgreater{}  [1] 0 0 0 0 0 1 1 1 2 2}
\FunctionTok{fcase}\NormalTok{(}
\NormalTok{  x }\SpecialCharTok{\textgreater{}} \DecValTok{8}\NormalTok{,}\DecValTok{2}\NormalTok{,}
\NormalTok{  x }\SpecialCharTok{\textgreater{}} \DecValTok{5}\NormalTok{,}\DecValTok{1}\NormalTok{,}
  \AttributeTok{default =} \DecValTok{0}
\NormalTok{)}
\CommentTok{\#\textgreater{}  [1] 0 0 0 0 0 1 1 1 2 2}
\end{Highlighting}
\end{Shaded}

\hypertarget{ux4ea4ux96c6-ux5deeux96c6-ux5408ux5e76}{%
\subsection{交集 差集 合并}\label{ux4ea4ux96c6-ux5deeux96c6-ux5408ux5e76}}

相当于base R 中 union(),intersect(),setdiff() 和setequal() 功能.all参数控制如何处理重复的行,和SQL中不同的是,data.table将保留行顺序.

\begin{Shaded}
\begin{Highlighting}[]

\FunctionTok{fintersect}\NormalTok{(x, y, }\AttributeTok{all =} \ConstantTok{FALSE}\NormalTok{)}
\FunctionTok{fsetdiff}\NormalTok{(x, y, }\AttributeTok{all =} \ConstantTok{FALSE}\NormalTok{)}
\FunctionTok{funion}\NormalTok{(x, y, }\AttributeTok{all =} \ConstantTok{FALSE}\NormalTok{)}
\FunctionTok{fsetequal}\NormalTok{(x, y, }\AttributeTok{all =} \ConstantTok{TRUE}\NormalTok{)}

\NormalTok{x }\OtherTok{\textless{}{-}}  \FunctionTok{data.table}\NormalTok{(}\FunctionTok{c}\NormalTok{(}\DecValTok{1}\NormalTok{,}\DecValTok{2}\NormalTok{,}\DecValTok{2}\NormalTok{,}\DecValTok{2}\NormalTok{,}\DecValTok{3}\NormalTok{,}\DecValTok{4}\NormalTok{,}\DecValTok{4}\NormalTok{))}
\NormalTok{x2 }\OtherTok{\textless{}{-}}  \FunctionTok{data.table}\NormalTok{(}\FunctionTok{c}\NormalTok{(}\DecValTok{1}\NormalTok{,}\DecValTok{2}\NormalTok{,}\DecValTok{3}\NormalTok{,}\DecValTok{4}\NormalTok{)) }\CommentTok{\# same set of rows as x}
\NormalTok{y }\OtherTok{\textless{}{-}}  \FunctionTok{data.table}\NormalTok{(}\FunctionTok{c}\NormalTok{(}\DecValTok{2}\NormalTok{,}\DecValTok{3}\NormalTok{,}\DecValTok{4}\NormalTok{,}\DecValTok{4}\NormalTok{,}\DecValTok{4}\NormalTok{,}\DecValTok{5}\NormalTok{))}

\FunctionTok{fintersect}\NormalTok{(x, y)            }\CommentTok{\# intersect}
\FunctionTok{fintersect}\NormalTok{(x, y, }\AttributeTok{all=}\ConstantTok{TRUE}\NormalTok{)  }\CommentTok{\# intersect all}

\FunctionTok{fsetdiff}\NormalTok{(x, y)              }\CommentTok{\# except}
\FunctionTok{fsetdiff}\NormalTok{(x, y, }\AttributeTok{all=}\ConstantTok{TRUE}\NormalTok{)    }\CommentTok{\# except all}
\FunctionTok{funion}\NormalTok{(x, y)                }\CommentTok{\# union}
\FunctionTok{funion}\NormalTok{(x, y, }\AttributeTok{all=}\ConstantTok{TRUE}\NormalTok{)      }\CommentTok{\# union all}
\FunctionTok{fsetequal}\NormalTok{(x, x2, }\AttributeTok{all=}\ConstantTok{FALSE}\NormalTok{) }\CommentTok{\# setequal}
\FunctionTok{fsetequal}\NormalTok{(x, x2)            }\CommentTok{\# setequal all}
\end{Highlighting}
\end{Shaded}

\hypertarget{ux957fux5bbdux8f6cux6362}{%
\subsection{长宽转换}\label{ux957fux5bbdux8f6cux6362}}

主要是两个函数\texttt{dcast}以及\texttt{melt}实现长宽转换,实现Excel中部分透视表功能。具体的函数参数请自行查阅文档。

\begin{itemize}
\tightlist
\item
  dcast函数能实现长转宽
\end{itemize}

参数如下:fun.aggregate函数指定聚合函数,value.var参数指定参与聚合的字段。formula指定聚合维度,格式用x+y\textasciitilde z,其中x,y在行的位置,z在列的位置。

\begin{Shaded}
\begin{Highlighting}[]
\FunctionTok{dcast}\NormalTok{(data, formula, }\AttributeTok{fun.aggregate =} \ConstantTok{NULL}\NormalTok{, }\AttributeTok{sep =} \StringTok{"\_"}\NormalTok{,}
\NormalTok{    ..., }\AttributeTok{margins =} \ConstantTok{NULL}\NormalTok{, }\AttributeTok{subset =} \ConstantTok{NULL}\NormalTok{, }\AttributeTok{fill =} \ConstantTok{NULL}\NormalTok{,}
    \AttributeTok{drop =} \ConstantTok{TRUE}\NormalTok{, }\AttributeTok{value.var =} \FunctionTok{guess}\NormalTok{(data),}
    \AttributeTok{verbose =} \FunctionTok{getOption}\NormalTok{(}\StringTok{"datatable.verbose"}\NormalTok{))}
\end{Highlighting}
\end{Shaded}

示例如下:

\begin{Shaded}
\begin{Highlighting}[]
\NormalTok{dt }\OtherTok{\textless{}{-}} \FunctionTok{data.table}\NormalTok{(分公司}\OtherTok{=}\FunctionTok{rep}\NormalTok{(}\FunctionTok{c}\NormalTok{(}\StringTok{\textquotesingle{}华东\textquotesingle{}}\NormalTok{,}\StringTok{\textquotesingle{}华南\textquotesingle{}}\NormalTok{,}\StringTok{\textquotesingle{}华西\textquotesingle{}}\NormalTok{,}\StringTok{\textquotesingle{}华北\textquotesingle{}}\NormalTok{),}\DecValTok{1000}\NormalTok{),}
\NormalTok{              季度}\OtherTok{=}\FunctionTok{rep}\NormalTok{(}\FunctionTok{c}\NormalTok{(}\StringTok{\textquotesingle{}一季度\textquotesingle{}}\NormalTok{,}\StringTok{\textquotesingle{}二季度\textquotesingle{}}\NormalTok{,}\StringTok{\textquotesingle{}三季度\textquotesingle{}}\NormalTok{,}\StringTok{\textquotesingle{}四季度\textquotesingle{}}\NormalTok{),}\DecValTok{1000}\NormalTok{),}
\NormalTok{              销售额}\OtherTok{=}\FunctionTok{sample}\NormalTok{(}\DecValTok{100}\SpecialCharTok{:}\DecValTok{200}\NormalTok{,}\DecValTok{4000}\NormalTok{,}\AttributeTok{replace =} \ConstantTok{TRUE}\NormalTok{))}
\FunctionTok{dcast}\NormalTok{(dt,分公司}\SpecialCharTok{\textasciitilde{}}\NormalTok{季度,}\AttributeTok{value.var =} \StringTok{"销售额"}\NormalTok{,}\AttributeTok{fun.aggregate =}\NormalTok{ sum)}
\CommentTok{\#\textgreater{}    分公司 一季度 三季度 二季度 四季度}
\CommentTok{\#\textgreater{} 1:   华东 149470      0      0      0}
\CommentTok{\#\textgreater{} 2:   华北      0      0      0 149343}
\CommentTok{\#\textgreater{} 3:   华南      0      0 150489      0}
\CommentTok{\#\textgreater{} 4:   华西      0 150698      0      0}
\end{Highlighting}
\end{Shaded}

从版本V1.9.6起可以同时对多个值实现不同聚合后的长转宽。

fun参数即 fun.aggregate的简写,可以是自定义的函数。

\begin{Shaded}
\begin{Highlighting}[]
\NormalTok{dt }\OtherTok{\textless{}{-}}  \FunctionTok{data.table}\NormalTok{(}\AttributeTok{x=}\FunctionTok{sample}\NormalTok{(}\DecValTok{5}\NormalTok{,}\DecValTok{20}\NormalTok{,}\ConstantTok{TRUE}\NormalTok{), }\AttributeTok{y=}\FunctionTok{sample}\NormalTok{(}\DecValTok{2}\NormalTok{,}\DecValTok{20}\NormalTok{,}\ConstantTok{TRUE}\NormalTok{),}
                \AttributeTok{z=}\FunctionTok{sample}\NormalTok{(letters[}\DecValTok{1}\SpecialCharTok{:}\DecValTok{2}\NormalTok{], }\DecValTok{20}\NormalTok{,}\ConstantTok{TRUE}\NormalTok{), }\AttributeTok{d1 =} \FunctionTok{runif}\NormalTok{(}\DecValTok{20}\NormalTok{), }\AttributeTok{d2=}\NormalTok{1L)}
\FunctionTok{dcast}\NormalTok{(dt, x }\SpecialCharTok{+}\NormalTok{ y }\SpecialCharTok{\textasciitilde{}}\NormalTok{ z, }\AttributeTok{fun=}\FunctionTok{list}\NormalTok{(sum,mean), }\AttributeTok{value.var=}\FunctionTok{c}\NormalTok{(}\StringTok{"d1"}\NormalTok{,}\StringTok{"d2"}\NormalTok{))}
\CommentTok{\#\textgreater{}     x y d1\_sum\_a d1\_sum\_b d2\_sum\_a d2\_sum\_b d1\_mean\_a d1\_mean\_b d2\_mean\_a}
\CommentTok{\#\textgreater{}  1: 1 1   0.8454   0.3717        1        2    0.8454    0.1858         1}
\CommentTok{\#\textgreater{}  2: 1 2   0.3769   0.0000        1        0    0.3769       NaN         1}
\CommentTok{\#\textgreater{}  3: 2 1   1.3419   0.4810        2        1    0.6709    0.4810         1}
\CommentTok{\#\textgreater{}  4: 2 2   0.0000   0.6666        0        1       NaN    0.6666       NaN}
\CommentTok{\#\textgreater{}  5: 3 1   0.8703   0.1866        1        1    0.8703    0.1866         1}
\CommentTok{\#\textgreater{}  6: 3 2   0.0000   0.9074        0        2       NaN    0.4537       NaN}
\CommentTok{\#\textgreater{}  7: 4 1   0.0957   0.0000        1        0    0.0957       NaN         1}
\CommentTok{\#\textgreater{}  8: 4 2   0.1103   0.1867        1        1    0.1103    0.1867         1}
\CommentTok{\#\textgreater{}  9: 5 1   1.1344   0.0000        2        0    0.5672       NaN         1}
\CommentTok{\#\textgreater{} 10: 5 2   1.0839   0.0707        2        1    0.5419    0.0707         1}
\CommentTok{\#\textgreater{}     d2\_mean\_b}
\CommentTok{\#\textgreater{}  1:         1}
\CommentTok{\#\textgreater{}  2:       NaN}
\CommentTok{\#\textgreater{}  3:         1}
\CommentTok{\#\textgreater{}  4:         1}
\CommentTok{\#\textgreater{}  5:         1}
\CommentTok{\#\textgreater{}  6:         1}
\CommentTok{\#\textgreater{}  7:       NaN}
\CommentTok{\#\textgreater{}  8:         1}
\CommentTok{\#\textgreater{}  9:       NaN}
\CommentTok{\#\textgreater{} 10:         1}
\FunctionTok{dcast}\NormalTok{(dt, x }\SpecialCharTok{+}\NormalTok{ y }\SpecialCharTok{\textasciitilde{}}\NormalTok{ z, }\AttributeTok{fun=}\FunctionTok{list}\NormalTok{(sum,mean), }\AttributeTok{value.var=}\FunctionTok{list}\NormalTok{(}\StringTok{"d1"}\NormalTok{,}\StringTok{"d2"}\NormalTok{)) }\CommentTok{\#注意value.var是向量和列表时的区别}
\CommentTok{\#\textgreater{}     x y d1\_sum\_a d1\_sum\_b d2\_mean\_a d2\_mean\_b}
\CommentTok{\#\textgreater{}  1: 1 1   0.8454   0.3717         1         1}
\CommentTok{\#\textgreater{}  2: 1 2   0.3769   0.0000         1       NaN}
\CommentTok{\#\textgreater{}  3: 2 1   1.3419   0.4810         1         1}
\CommentTok{\#\textgreater{}  4: 2 2   0.0000   0.6666       NaN         1}
\CommentTok{\#\textgreater{}  5: 3 1   0.8703   0.1866         1         1}
\CommentTok{\#\textgreater{}  6: 3 2   0.0000   0.9074       NaN         1}
\CommentTok{\#\textgreater{}  7: 4 1   0.0957   0.0000         1       NaN}
\CommentTok{\#\textgreater{}  8: 4 2   0.1103   0.1867         1         1}
\CommentTok{\#\textgreater{}  9: 5 1   1.1344   0.0000         1       NaN}
\CommentTok{\#\textgreater{} 10: 5 2   1.0839   0.0707         1         1}
\end{Highlighting}
\end{Shaded}

\begin{itemize}
\tightlist
\item
  melt函数实现宽转长
\end{itemize}

\begin{Shaded}
\begin{Highlighting}[]
\FunctionTok{melt}\NormalTok{(data, id.vars, measure.vars,}
    \AttributeTok{variable.name =} \StringTok{"variable"}\NormalTok{, }\AttributeTok{value.name =} \StringTok{"value"}\NormalTok{,}
\NormalTok{    ..., }\AttributeTok{na.rm =} \ConstantTok{FALSE}\NormalTok{, }\AttributeTok{variable.factor =} \ConstantTok{TRUE}\NormalTok{,}
    \AttributeTok{value.factor =} \ConstantTok{FALSE}\NormalTok{,}
    \AttributeTok{verbose =} \FunctionTok{getOption}\NormalTok{(}\StringTok{"datatable.verbose"}\NormalTok{))}
\end{Highlighting}
\end{Shaded}

示例如下:

\begin{Shaded}
\begin{Highlighting}[]
\NormalTok{ChickWeight }\OtherTok{=} \FunctionTok{as.data.table}\NormalTok{(ChickWeight)}
\FunctionTok{setnames}\NormalTok{(ChickWeight, }\FunctionTok{tolower}\NormalTok{(}\FunctionTok{names}\NormalTok{(ChickWeight)))}
\NormalTok{DT }\OtherTok{\textless{}{-}} \FunctionTok{melt}\NormalTok{(}\FunctionTok{as.data.table}\NormalTok{(ChickWeight), }\AttributeTok{id=}\DecValTok{2}\SpecialCharTok{:}\DecValTok{4}\NormalTok{) }\CommentTok{\# calls melt.data.table}
\NormalTok{DT}
\CommentTok{\#\textgreater{}      time chick diet variable value}
\CommentTok{\#\textgreater{}   1:    0     1    1   weight    42}
\CommentTok{\#\textgreater{}   2:    2     1    1   weight    51}
\CommentTok{\#\textgreater{}   3:    4     1    1   weight    59}
\CommentTok{\#\textgreater{}   4:    6     1    1   weight    64}
\CommentTok{\#\textgreater{}   5:    8     1    1   weight    76}
\CommentTok{\#\textgreater{}  {-}{-}{-}                               }
\CommentTok{\#\textgreater{} 574:   14    50    4   weight   175}
\CommentTok{\#\textgreater{} 575:   16    50    4   weight   205}
\CommentTok{\#\textgreater{} 576:   18    50    4   weight   234}
\CommentTok{\#\textgreater{} 577:   20    50    4   weight   264}
\CommentTok{\#\textgreater{} 578:   21    50    4   weight   264}
\end{Highlighting}
\end{Shaded}

\hypertarget{ux8868ux8fdeux63a5}{%
\subsection{表连接}\label{ux8868ux8fdeux63a5}}

两个数据框之间左连,右连等操作,类似数据库中的left\_join right\_join,inner\_join 等函数.

键入?merge()查看函数帮助,data.table 包中和base R 中都有merge 函数,当第一个数据框是data.table格式时启用data.table::merge().

\begin{Shaded}
\begin{Highlighting}[]
\NormalTok{?}\FunctionTok{merge}\NormalTok{()}
\FunctionTok{merge}\NormalTok{(x, y, }\AttributeTok{by =} \ConstantTok{NULL}\NormalTok{, }\AttributeTok{by.x =} \ConstantTok{NULL}\NormalTok{, }\AttributeTok{by.y =} \ConstantTok{NULL}\NormalTok{, }\AttributeTok{all =} \ConstantTok{FALSE}\NormalTok{,}
\AttributeTok{all.x =}\NormalTok{ all, }\AttributeTok{all.y =}\NormalTok{ all, }\AttributeTok{sort =} \ConstantTok{TRUE}\NormalTok{, }\AttributeTok{suffixes =} \FunctionTok{c}\NormalTok{(}\StringTok{".x"}\NormalTok{, }\StringTok{".y"}\NormalTok{), }\AttributeTok{no.dups =} \ConstantTok{TRUE}\NormalTok{,}
\AttributeTok{allow.cartesian=}\FunctionTok{getOption}\NormalTok{(}\StringTok{"datatable.allow.cartesian"}\NormalTok{),  }\CommentTok{\# default FALSE}
\NormalTok{...)}
\end{Highlighting}
\end{Shaded}

x.y为连个数据框,当两个数据框连接字段相同时,用by=c('`,'')连接,不同时采用,by.x=,by.y= ,all,all.x,all.y等参数决定连接方式,sort 默认为排序,当不需要排序时更改参数,allow.cartesian=是否允许笛卡尔,默认不允许,当需要时设置为TURE.

\hypertarget{ux9ad8ux7ea7ux51fdux6570}{%
\section{高级函数}\label{ux9ad8ux7ea7ux51fdux6570}}

高级函数并不是指使用难度,而是使用频率可能不高,但在实现某些功能时特别便利的函数。

如分组聚合的\texttt{groupingsets},前后移动的\texttt{shift}等函数。

\hypertarget{groupingsets}{%
\subsection{groupingsets}\label{groupingsets}}

产生多个层次的合计数据,与\texttt{sql}中的\href{https://www.postgresql.org/docs/9.5/queries-table-expressions.html\#QUERIES-GROUPING-SETS}{grouping set}功能相似。

\textbf{用法}

\begin{Shaded}
\begin{Highlighting}[]
\FunctionTok{rollup}\NormalTok{(x, j, by, .SDcols, }\AttributeTok{id =} \ConstantTok{FALSE}\NormalTok{, ...)}
\FunctionTok{groupingsets}\NormalTok{(x, j, by, sets, .SDcols, }\AttributeTok{id =} \ConstantTok{FALSE}\NormalTok{, jj, ...)}

\CommentTok{\# rollup}
\FunctionTok{rollup}\NormalTok{(DT, }\AttributeTok{j =} \FunctionTok{lapply}\NormalTok{(.SD, sum), }\AttributeTok{by =} \FunctionTok{c}\NormalTok{(}\StringTok{"color"}\NormalTok{,}\StringTok{"year"}\NormalTok{,}\StringTok{"status"}\NormalTok{), }\AttributeTok{id=}\ConstantTok{TRUE}\NormalTok{, }\AttributeTok{.SDcols=}\StringTok{"value"}\NormalTok{)}
\FunctionTok{rollup}\NormalTok{(DT, }\AttributeTok{j =} \FunctionTok{c}\NormalTok{(}\FunctionTok{list}\NormalTok{(}\AttributeTok{count=}\NormalTok{.N), }\FunctionTok{lapply}\NormalTok{(.SD, sum)), }\AttributeTok{by =} \FunctionTok{c}\NormalTok{(}\StringTok{"color"}\NormalTok{,}\StringTok{"year"}\NormalTok{,}\StringTok{"status"}\NormalTok{), }\AttributeTok{id=}\ConstantTok{TRUE}\NormalTok{)}
\end{Highlighting}
\end{Shaded}

如果要达到像Excel中透视表一样的效果,如下所示:

\begin{figure}
\centering
\includegraphics{./picture/data-table/Excel-pivot-groupingsets.png}
\caption{Excel groupingsets透视表}
\end{figure}

\begin{itemize}
\tightlist
\item
  rollup
\end{itemize}

\begin{Shaded}
\begin{Highlighting}[]
\FunctionTok{library}\NormalTok{(magrittr)}
\CommentTok{\#\textgreater{} }
\CommentTok{\#\textgreater{} 载入程辑包:\textquotesingle{}magrittr\textquotesingle{}}
\CommentTok{\#\textgreater{} The following object is masked from \textquotesingle{}package:purrr\textquotesingle{}:}
\CommentTok{\#\textgreater{} }
\CommentTok{\#\textgreater{}     set\_names}
\CommentTok{\#\textgreater{} The following object is masked from \textquotesingle{}package:tidyr\textquotesingle{}:}
\CommentTok{\#\textgreater{} }
\CommentTok{\#\textgreater{}     extract}
\NormalTok{DT }\OtherTok{\textless{}{-}} \FunctionTok{fread}\NormalTok{(}\StringTok{\textquotesingle{}./data/data{-}table{-}groupingsets.csv\textquotesingle{}}\NormalTok{,}\AttributeTok{encoding =} \StringTok{\textquotesingle{}UTF{-}8\textquotesingle{}}\NormalTok{)}
\NormalTok{(}\FunctionTok{rollup}\NormalTok{(DT,}\AttributeTok{j =}\FunctionTok{list}\NormalTok{(以下项目的总和 }\OtherTok{=}\FunctionTok{sum}\NormalTok{(value)),}\AttributeTok{by =} \FunctionTok{c}\NormalTok{(}\StringTok{"area"}\NormalTok{,}\StringTok{"store\_type"}\NormalTok{),}\AttributeTok{id =} \ConstantTok{TRUE}\NormalTok{) }\SpecialCharTok{\%\textgreater{}\%} \FunctionTok{setorderv}\NormalTok{(}\AttributeTok{cols=}\FunctionTok{c}\NormalTok{(}\StringTok{\textquotesingle{}area\textquotesingle{}}\NormalTok{,}\StringTok{\textquotesingle{}grouping\textquotesingle{}}\NormalTok{),}\AttributeTok{na.last =} \ConstantTok{TRUE}\NormalTok{))}
\CommentTok{\#\textgreater{}     grouping area store\_type 以下项目的总和}
\CommentTok{\#\textgreater{}  1:        0 华东     不可比              9}
\CommentTok{\#\textgreater{}  2:        0 华东       可比            309}
\CommentTok{\#\textgreater{}  3:        1 华东       \textless{}NA\textgreater{}            318}
\CommentTok{\#\textgreater{}  4:        0 华北     不可比             72}
\CommentTok{\#\textgreater{}  5:        0 华北       可比            173}
\CommentTok{\#\textgreater{}  6:        1 华北       \textless{}NA\textgreater{}            245}
\CommentTok{\#\textgreater{}  7:        0 华南       可比             86}
\CommentTok{\#\textgreater{}  8:        0 华南     不可比             79}
\CommentTok{\#\textgreater{}  9:        1 华南       \textless{}NA\textgreater{}            165}
\CommentTok{\#\textgreater{} 10:        0 华西       可比              2}
\CommentTok{\#\textgreater{} 11:        0 华西     不可比            198}
\CommentTok{\#\textgreater{} 12:        1 华西       \textless{}NA\textgreater{}            200}
\CommentTok{\#\textgreater{} 13:        3 \textless{}NA\textgreater{}       \textless{}NA\textgreater{}            928}
\end{Highlighting}
\end{Shaded}

通过上述计算,发现计算结果与Excel透视表一样。

\begin{itemize}
\tightlist
\item
  cube
\end{itemize}

观察\texttt{cube()}计算结果与\texttt{rollup()}差异,发现\texttt{cube()}聚合层次更多。

\begin{Shaded}
\begin{Highlighting}[]
\FunctionTok{cube}\NormalTok{(DT,}\AttributeTok{j =} \FunctionTok{sum}\NormalTok{(value),}\AttributeTok{by =} \FunctionTok{c}\NormalTok{(}\StringTok{"area"}\NormalTok{,}\StringTok{"store\_type"}\NormalTok{),}\AttributeTok{id =} \ConstantTok{TRUE}\NormalTok{)}
\CommentTok{\#\textgreater{}     grouping area store\_type  V1}
\CommentTok{\#\textgreater{}  1:        0 华东     不可比   9}
\CommentTok{\#\textgreater{}  2:        0 华东       可比 309}
\CommentTok{\#\textgreater{}  3:        0 华西       可比   2}
\CommentTok{\#\textgreater{}  4:        0 华西     不可比 198}
\CommentTok{\#\textgreater{}  5:        0 华南       可比  86}
\CommentTok{\#\textgreater{}  6:        0 华北     不可比  72}
\CommentTok{\#\textgreater{}  7:        0 华南     不可比  79}
\CommentTok{\#\textgreater{}  8:        0 华北       可比 173}
\CommentTok{\#\textgreater{}  9:        1 华东       \textless{}NA\textgreater{} 318}
\CommentTok{\#\textgreater{} 10:        1 华西       \textless{}NA\textgreater{} 200}
\CommentTok{\#\textgreater{} 11:        1 华南       \textless{}NA\textgreater{} 165}
\CommentTok{\#\textgreater{} 12:        1 华北       \textless{}NA\textgreater{} 245}
\CommentTok{\#\textgreater{} 13:        2 \textless{}NA\textgreater{}     不可比 358}
\CommentTok{\#\textgreater{} 14:        2 \textless{}NA\textgreater{}       可比 570}
\CommentTok{\#\textgreater{} 15:        3 \textless{}NA\textgreater{}       \textless{}NA\textgreater{} 928}
\end{Highlighting}
\end{Shaded}

\begin{itemize}
\tightlist
\item
  groupingsets
\end{itemize}

根据需要指定指定聚合的层次。

\begin{Shaded}
\begin{Highlighting}[]
\CommentTok{\# 与本例中rollup 结果一致}
\FunctionTok{groupingsets}\NormalTok{(DT,}\AttributeTok{j =} \FunctionTok{sum}\NormalTok{(value),}\AttributeTok{by =} \FunctionTok{c}\NormalTok{(}\StringTok{"area"}\NormalTok{,}\StringTok{"store\_type"}\NormalTok{),}\AttributeTok{sets =} \FunctionTok{list}\NormalTok{(}\StringTok{\textquotesingle{}area\textquotesingle{}}\NormalTok{,}\FunctionTok{c}\NormalTok{(}\StringTok{"area"}\NormalTok{,}\StringTok{"store\_type"}\NormalTok{), }\FunctionTok{character}\NormalTok{()),}\AttributeTok{id =} \ConstantTok{TRUE}\NormalTok{)}
\CommentTok{\#\textgreater{}     grouping area store\_type  V1}
\CommentTok{\#\textgreater{}  1:        1 华东       \textless{}NA\textgreater{} 318}
\CommentTok{\#\textgreater{}  2:        1 华西       \textless{}NA\textgreater{} 200}
\CommentTok{\#\textgreater{}  3:        1 华南       \textless{}NA\textgreater{} 165}
\CommentTok{\#\textgreater{}  4:        1 华北       \textless{}NA\textgreater{} 245}
\CommentTok{\#\textgreater{}  5:        0 华东     不可比   9}
\CommentTok{\#\textgreater{}  6:        0 华东       可比 309}
\CommentTok{\#\textgreater{}  7:        0 华西       可比   2}
\CommentTok{\#\textgreater{}  8:        0 华西     不可比 198}
\CommentTok{\#\textgreater{}  9:        0 华南       可比  86}
\CommentTok{\#\textgreater{} 10:        0 华北     不可比  72}
\CommentTok{\#\textgreater{} 11:        0 华南     不可比  79}
\CommentTok{\#\textgreater{} 12:        0 华北       可比 173}
\CommentTok{\#\textgreater{} 13:        3 \textless{}NA\textgreater{}       \textless{}NA\textgreater{} 928}

\CommentTok{\# 与本例中cube 结果一致}
\FunctionTok{groupingsets}\NormalTok{(DT,}\AttributeTok{j =} \FunctionTok{sum}\NormalTok{(value),}\AttributeTok{by =} \FunctionTok{c}\NormalTok{(}\StringTok{"area"}\NormalTok{,}\StringTok{"store\_type"}\NormalTok{),}\AttributeTok{sets =} \FunctionTok{list}\NormalTok{(}\StringTok{\textquotesingle{}area\textquotesingle{}}\NormalTok{,}\FunctionTok{c}\NormalTok{(}\StringTok{"area"}\NormalTok{,}\StringTok{"store\_type"}\NormalTok{),}\StringTok{"store\_type"}\NormalTok{, }\FunctionTok{character}\NormalTok{()),}\AttributeTok{id =} \ConstantTok{TRUE}\NormalTok{)}
\CommentTok{\#\textgreater{}     grouping area store\_type  V1}
\CommentTok{\#\textgreater{}  1:        1 华东       \textless{}NA\textgreater{} 318}
\CommentTok{\#\textgreater{}  2:        1 华西       \textless{}NA\textgreater{} 200}
\CommentTok{\#\textgreater{}  3:        1 华南       \textless{}NA\textgreater{} 165}
\CommentTok{\#\textgreater{}  4:        1 华北       \textless{}NA\textgreater{} 245}
\CommentTok{\#\textgreater{}  5:        0 华东     不可比   9}
\CommentTok{\#\textgreater{}  6:        0 华东       可比 309}
\CommentTok{\#\textgreater{}  7:        0 华西       可比   2}
\CommentTok{\#\textgreater{}  8:        0 华西     不可比 198}
\CommentTok{\#\textgreater{}  9:        0 华南       可比  86}
\CommentTok{\#\textgreater{} 10:        0 华北     不可比  72}
\CommentTok{\#\textgreater{} 11:        0 华南     不可比  79}
\CommentTok{\#\textgreater{} 12:        0 华北       可比 173}
\CommentTok{\#\textgreater{} 13:        2 \textless{}NA\textgreater{}     不可比 358}
\CommentTok{\#\textgreater{} 14:        2 \textless{}NA\textgreater{}       可比 570}
\CommentTok{\#\textgreater{} 15:        3 \textless{}NA\textgreater{}       \textless{}NA\textgreater{} 928}
\end{Highlighting}
\end{Shaded}

\begin{quote}
groupingsets: sets参数,用list()包裹想要聚合的字段组合,最后character(),加上该部分相当于不区分层级全部聚合,用法类似sql中``()''.
\end{quote}

\begin{quote}
SELECT brand, size, sum(sales) FROM items\_sold GROUP BY GROUPING SETS ((brand), (size), ());
\end{quote}

\hypertarget{rleid}{%
\subsection{rleid}\label{rleid}}

该函数根据分组生成长度列。

即将0011001110111101类似这种分组成1 1 2 2 3 3 4 4 4 5 6 6 6 6 7 8。在特定时候是很便捷的一个函数。如在计算股票连续上涨或下跌天数时。

\begin{Shaded}
\begin{Highlighting}[]
\FunctionTok{rleid}\NormalTok{(}\FunctionTok{c}\NormalTok{(}\DecValTok{0}\NormalTok{,}\DecValTok{0}\NormalTok{,}\DecValTok{1}\NormalTok{,}\DecValTok{1}\NormalTok{,}\DecValTok{0}\NormalTok{,}\DecValTok{0}\NormalTok{,}\DecValTok{1}\NormalTok{,}\DecValTok{1}\NormalTok{,}\DecValTok{1}\NormalTok{,}\DecValTok{0}\NormalTok{,}\DecValTok{1}\NormalTok{,}\DecValTok{1}\NormalTok{,}\DecValTok{1}\NormalTok{,}\DecValTok{1}\NormalTok{,}\DecValTok{0}\NormalTok{,}\DecValTok{1}\NormalTok{))}
\CommentTok{\#\textgreater{}  [1] 1 1 2 2 3 3 4 4 4 5 6 6 6 6 7 8}
\end{Highlighting}
\end{Shaded}

用法:

\begin{Shaded}
\begin{Highlighting}[]
\FunctionTok{rleid}\NormalTok{(..., }\AttributeTok{prefix=}\ConstantTok{NULL}\NormalTok{)}
\FunctionTok{rleidv}\NormalTok{(x, }\AttributeTok{cols=}\FunctionTok{seq\_along}\NormalTok{(x), }\AttributeTok{prefix=}\ConstantTok{NULL}\NormalTok{)}
\end{Highlighting}
\end{Shaded}

\begin{Shaded}
\begin{Highlighting}[]
\NormalTok{DT }\OtherTok{=} \FunctionTok{data.table}\NormalTok{(}\AttributeTok{grp=}\FunctionTok{rep}\NormalTok{(}\FunctionTok{c}\NormalTok{(}\StringTok{"A"}\NormalTok{, }\StringTok{"B"}\NormalTok{, }\StringTok{"C"}\NormalTok{, }\StringTok{"A"}\NormalTok{, }\StringTok{"B"}\NormalTok{), }\FunctionTok{c}\NormalTok{(}\DecValTok{2}\NormalTok{,}\DecValTok{2}\NormalTok{,}\DecValTok{3}\NormalTok{,}\DecValTok{1}\NormalTok{,}\DecValTok{2}\NormalTok{)), }\AttributeTok{value=}\DecValTok{1}\SpecialCharTok{:}\DecValTok{10}\NormalTok{)}
\FunctionTok{rleid}\NormalTok{(DT}\SpecialCharTok{$}\NormalTok{grp) }\CommentTok{\# get run{-}length ids}
\CommentTok{\#\textgreater{}  [1] 1 1 2 2 3 3 3 4 5 5}
\FunctionTok{rleidv}\NormalTok{(DT, }\StringTok{"grp"}\NormalTok{) }\CommentTok{\# same as above}
\CommentTok{\#\textgreater{}  [1] 1 1 2 2 3 3 3 4 5 5}
\FunctionTok{rleid}\NormalTok{(DT}\SpecialCharTok{$}\NormalTok{grp, }\AttributeTok{prefix=}\StringTok{"grp"}\NormalTok{) }\CommentTok{\# prefix with \textquotesingle{}grp\textquotesingle{}}
\CommentTok{\#\textgreater{}  [1] "grp1" "grp1" "grp2" "grp2" "grp3" "grp3" "grp3" "grp4" "grp5" "grp5"}
\end{Highlighting}
\end{Shaded}

\hypertarget{shift}{%
\subsection{shift}\label{shift}}

向前或向后功能,通俗来说就是向前或向后移动位置。

示例如下:

\begin{Shaded}
\begin{Highlighting}[]
\NormalTok{x }\OtherTok{=} \DecValTok{1}\SpecialCharTok{:}\DecValTok{5}
\CommentTok{\# lag with n=1 and pad with NA (returns vector)}
\FunctionTok{shift}\NormalTok{(x, }\AttributeTok{n=}\DecValTok{1}\NormalTok{, }\AttributeTok{fill=}\ConstantTok{NA}\NormalTok{, }\AttributeTok{type=}\StringTok{"lag"}\NormalTok{)}
\CommentTok{\#\textgreater{} [1] NA  1  2  3  4}
\end{Highlighting}
\end{Shaded}

其中参数n控制偏移量,n正负数和type的参数相对应。, n=-1 and type=`lead' 与 n=1 and type='lag'效果相同。

在data.table上使用:

\begin{Shaded}
\begin{Highlighting}[]
\NormalTok{DT }\OtherTok{=} \FunctionTok{data.table}\NormalTok{(}\AttributeTok{year=}\DecValTok{2010}\SpecialCharTok{:}\DecValTok{2014}\NormalTok{, }\AttributeTok{v1=}\FunctionTok{runif}\NormalTok{(}\DecValTok{5}\NormalTok{), }\AttributeTok{v2=}\DecValTok{1}\SpecialCharTok{:}\DecValTok{5}\NormalTok{, }\AttributeTok{v3=}\NormalTok{letters[}\DecValTok{1}\SpecialCharTok{:}\DecValTok{5}\NormalTok{])}
\NormalTok{cols }\OtherTok{=} \FunctionTok{c}\NormalTok{(}\StringTok{"v1"}\NormalTok{,}\StringTok{"v2"}\NormalTok{,}\StringTok{"v3"}\NormalTok{)}
\NormalTok{anscols }\OtherTok{=} \FunctionTok{paste}\NormalTok{(}\StringTok{"lead"}\NormalTok{, cols, }\AttributeTok{sep=}\StringTok{"\_"}\NormalTok{)}
\NormalTok{DT[, (anscols) }\SpecialCharTok{:}\ErrorTok{=} \FunctionTok{shift}\NormalTok{(.SD, }\DecValTok{1}\NormalTok{, }\DecValTok{0}\NormalTok{, }\StringTok{"lead"}\NormalTok{), .SDcols}\OtherTok{=}\NormalTok{cols]}
\end{Highlighting}
\end{Shaded}

例如求某人连续消费时间间隔天数时:

\begin{Shaded}
\begin{Highlighting}[]
\NormalTok{DT }\OtherTok{=} \FunctionTok{data.table}\NormalTok{(}\AttributeTok{dates =}\NormalTok{lubridate}\SpecialCharTok{::}\FunctionTok{ymd}\NormalTok{(}\FunctionTok{c}\NormalTok{(}\DecValTok{20210105}\NormalTok{,}\DecValTok{20210115}\NormalTok{,}\DecValTok{20210124}\NormalTok{,}\DecValTok{20210218}\NormalTok{,}\DecValTok{20210424}\NormalTok{)))}
\NormalTok{DT[,newdate}\SpecialCharTok{:}\ErrorTok{=}\FunctionTok{shift}\NormalTok{(dates)]}
\NormalTok{DT}
\CommentTok{\#\textgreater{}         dates    newdate}
\CommentTok{\#\textgreater{} 1: 2021{-}01{-}05       \textless{}NA\textgreater{}}
\CommentTok{\#\textgreater{} 2: 2021{-}01{-}15 2021{-}01{-}05}
\CommentTok{\#\textgreater{} 3: 2021{-}01{-}24 2021{-}01{-}15}
\CommentTok{\#\textgreater{} 4: 2021{-}02{-}18 2021{-}01{-}24}
\CommentTok{\#\textgreater{} 5: 2021{-}04{-}24 2021{-}02{-}18}
\end{Highlighting}
\end{Shaded}

通过构造新列newdate,然后将两列相减\texttt{dates-newdate}即可得到每次购物间隔天数。

\hypertarget{j}{%
\subsection{J}\label{j}}

J 是\texttt{.()},\texttt{list()}等的别名。\texttt{SJ}是排序连接,\texttt{CJ}是交叉连接。

用法:

\begin{Shaded}
\begin{Highlighting}[]
\CommentTok{\# DT[J(...)]                          \# J() only for use inside DT[...]}
\CommentTok{\# DT[.(...)]                          \# .() only for use inside DT[...]}
\CommentTok{\# DT[list(...)]                       \# same; .(), list() and J() are identical}
\FunctionTok{SJ}\NormalTok{(...)                             }\CommentTok{\# DT[SJ(...)]}
\FunctionTok{CJ}\NormalTok{(..., }\AttributeTok{sorted=}\ConstantTok{TRUE}\NormalTok{, }\AttributeTok{unique=}\ConstantTok{FALSE}\NormalTok{)  }\CommentTok{\# DT[CJ(...)]}
\end{Highlighting}
\end{Shaded}

\begin{itemize}
\tightlist
\item
  CJ
\end{itemize}

我喜欢用\texttt{CJ()}函数创建笛卡尔积表。例如在商品运营中,时常需要将门店和商品形成笛卡尔积表,相比起\texttt{dplyr::full\_join()} ,\texttt{data.table::merge.data.table(allow.cartesian\ =\ TRUE\ )},\texttt{CJ}更加方便快捷。

\begin{Shaded}
\begin{Highlighting}[]
\CommentTok{\# CJ usage examples}
\FunctionTok{CJ}\NormalTok{(}\FunctionTok{c}\NormalTok{(}\DecValTok{5}\NormalTok{, }\ConstantTok{NA}\NormalTok{, }\DecValTok{1}\NormalTok{), }\FunctionTok{c}\NormalTok{(}\DecValTok{1}\NormalTok{, }\DecValTok{3}\NormalTok{, }\DecValTok{2}\NormalTok{))                 }\CommentTok{\# sorted and keyed data.table}
\CommentTok{\#\textgreater{}    V1 V2}
\CommentTok{\#\textgreater{} 1: NA  1}
\CommentTok{\#\textgreater{} 2: NA  2}
\CommentTok{\#\textgreater{} 3: NA  3}
\CommentTok{\#\textgreater{} 4:  1  1}
\CommentTok{\#\textgreater{} 5:  1  2}
\CommentTok{\#\textgreater{} 6:  1  3}
\CommentTok{\#\textgreater{} 7:  5  1}
\CommentTok{\#\textgreater{} 8:  5  2}
\CommentTok{\#\textgreater{} 9:  5  3}
\CommentTok{\# do.call(CJ, list(c(5, NA, 1), c(1, 3, 2)))  \# same as above}
\CommentTok{\# CJ(c(5, NA, 1), c(1, 3, 2), sorted=FALSE)   \# same order as input, unkeyed}
\end{Highlighting}
\end{Shaded}

\begin{itemize}
\tightlist
\item
  SJ
\end{itemize}

SJ : Sorted Join. The same value as J() but additionally setkey() is called on all columns in the order they were passed to SJ. For efficiency, to invoke a binary merge rather than a repeated binary full search for each row of i.

\hypertarget{ux5c0fux6280ux5de7}{%
\section{小技巧}\label{ux5c0fux6280ux5de7}}

\hypertarget{ux7528ux6291ux5236ux4e2dux95f4ux8fc7ux7a0bux8f93ux51fa}{%
\subsection{用\{\}抑制中间过程输出}\label{ux7528ux6291ux5236ux4e2dux95f4ux8fc7ux7a0bux8f93ux51fa}}

默认只返回未命名花括号中定义的最后一个对象。

\begin{Shaded}
\begin{Highlighting}[]
\NormalTok{dt }\OtherTok{\textless{}{-}} \FunctionTok{data.table}\NormalTok{(mtcars)}
\NormalTok{dt[,\{tmp1}\OtherTok{=}\FunctionTok{mean}\NormalTok{(mpg); tmp2}\OtherTok{=}\FunctionTok{mean}\NormalTok{(}\FunctionTok{abs}\NormalTok{(mpg}\SpecialCharTok{{-}}\NormalTok{tmp1)); tmp3}\OtherTok{=}\FunctionTok{round}\NormalTok{(tmp2, }\DecValTok{2}\NormalTok{)\}, by}\OtherTok{=}\NormalTok{cyl]}
\CommentTok{\#\textgreater{}    cyl   V1}
\CommentTok{\#\textgreater{} 1:   6 1.19}
\CommentTok{\#\textgreater{} 2:   4 3.83}
\CommentTok{\#\textgreater{} 3:   8 1.79}
\end{Highlighting}
\end{Shaded}

在我不知道上述技巧之前,我可能的操作是

\begin{Shaded}
\begin{Highlighting}[]
\NormalTok{dt }\OtherTok{\textless{}{-}} \FunctionTok{data.table}\NormalTok{(mtcars)}
\NormalTok{res }\OtherTok{\textless{}{-}}\NormalTok{ dt[,tmp1}\SpecialCharTok{:}\ErrorTok{=}\FunctionTok{mean}\NormalTok{(mpg), by}\OtherTok{=}\NormalTok{cyl][,.(}\AttributeTok{tmp2=}\FunctionTok{mean}\NormalTok{(}\FunctionTok{abs}\NormalTok{(mpg}\SpecialCharTok{{-}}\NormalTok{tmp1))), by}\OtherTok{=}\NormalTok{.(cyl)]}
\NormalTok{res[,.(}\FunctionTok{round}\NormalTok{(tmp2,}\DecValTok{2}\NormalTok{)),by}\OtherTok{=}\NormalTok{.(cyl)][]}
\CommentTok{\#\textgreater{}    cyl   V1}
\CommentTok{\#\textgreater{} 1:   6 1.19}
\CommentTok{\#\textgreater{} 2:   4 3.83}
\CommentTok{\#\textgreater{} 3:   8 1.79}
\end{Highlighting}
\end{Shaded}

保留中间变量

\begin{Shaded}
\begin{Highlighting}[]
\NormalTok{dt[,\{tmp1}\OtherTok{=}\FunctionTok{mean}\NormalTok{(mpg); tmp2}\OtherTok{=}\FunctionTok{mean}\NormalTok{(}\FunctionTok{abs}\NormalTok{(mpg}\SpecialCharTok{{-}}\NormalTok{tmp1)); tmp3}\OtherTok{=}\FunctionTok{round}\NormalTok{(tmp2, }\DecValTok{2}\NormalTok{); }\FunctionTok{list}\NormalTok{(}\AttributeTok{tmp2=}\NormalTok{tmp2, }\AttributeTok{tmp3=}\NormalTok{tmp3)\}, by}\OtherTok{=}\NormalTok{cyl][]}
\CommentTok{\#\textgreater{}    cyl tmp2 tmp3}
\CommentTok{\#\textgreater{} 1:   6 1.19 1.19}
\CommentTok{\#\textgreater{} 2:   4 3.83 3.83}
\CommentTok{\#\textgreater{} 3:   8 1.79 1.79}
\end{Highlighting}
\end{Shaded}

不写分号的方式

\begin{Shaded}
\begin{Highlighting}[]
\NormalTok{dt[,\{tmp1}\OtherTok{=}\FunctionTok{mean}\NormalTok{(mpg)}
\NormalTok{     tmp2}\OtherTok{=}\FunctionTok{mean}\NormalTok{(}\FunctionTok{abs}\NormalTok{(mpg}\SpecialCharTok{{-}}\NormalTok{tmp1))}
\NormalTok{     tmp3}\OtherTok{=}\FunctionTok{round}\NormalTok{(tmp2, }\DecValTok{2}\NormalTok{)}
     \FunctionTok{list}\NormalTok{(}\AttributeTok{tmp2=}\NormalTok{tmp2, }\AttributeTok{tmp3=}\NormalTok{tmp3)\},}
\NormalTok{   by}\OtherTok{=}\NormalTok{cyl][]}
\CommentTok{\#\textgreater{}    cyl tmp2 tmp3}
\CommentTok{\#\textgreater{} 1:   6 1.19 1.19}
\CommentTok{\#\textgreater{} 2:   4 3.83 3.83}
\CommentTok{\#\textgreater{} 3:   8 1.79 1.79}
\end{Highlighting}
\end{Shaded}

\hypertarget{ux4f7fux7528ux6253ux5370data.table}{%
\subsection{使用{[}{]}打印data.table}\label{ux4f7fux7528ux6253ux5370data.table}}

在测试代码查看结果时很有用。

\begin{Shaded}
\begin{Highlighting}[]
\NormalTok{df }\OtherTok{\textless{}{-}} \FunctionTok{head}\NormalTok{(mtcars) }\CommentTok{\# doesn\textquotesingle{}t print}
\NormalTok{(df }\OtherTok{\textless{}{-}} \FunctionTok{head}\NormalTok{(mtcars)) }\CommentTok{\# does print}
\CommentTok{\#\textgreater{} \# A tibble: 6 x 11}
\CommentTok{\#\textgreater{}     mpg   cyl  disp    hp  drat    wt  qsec    vs    am  gear  carb}
\CommentTok{\#\textgreater{}   \textless{}dbl\textgreater{} \textless{}int\textgreater{} \textless{}dbl\textgreater{} \textless{}int\textgreater{} \textless{}dbl\textgreater{} \textless{}dbl\textgreater{} \textless{}dbl\textgreater{} \textless{}int\textgreater{} \textless{}int\textgreater{} \textless{}int\textgreater{} \textless{}int\textgreater{}}
\CommentTok{\#\textgreater{} 1  21       6   160   110  3.9   2.62  16.5     0     1     4     4}
\CommentTok{\#\textgreater{} 2  21       6   160   110  3.9   2.88  17.0     0     1     4     4}
\CommentTok{\#\textgreater{} 3  22.8     4   108    93  3.85  2.32  18.6     1     1     4     1}
\CommentTok{\#\textgreater{} 4  21.4     6   258   110  3.08  3.22  19.4     1     0     3     1}
\CommentTok{\#\textgreater{} 5  18.7     8   360   175  3.15  3.44  17.0     0     0     3     2}
\CommentTok{\#\textgreater{} 6  18.1     6   225   105  2.76  3.46  20.2     1     0     3     1}
\end{Highlighting}
\end{Shaded}

\begin{Shaded}
\begin{Highlighting}[]
\CommentTok{\# data.table way of printing after an assignment}
\NormalTok{dt }\OtherTok{\textless{}{-}} \FunctionTok{data.table}\NormalTok{(}\FunctionTok{head}\NormalTok{(mtcars)) }\CommentTok{\# doesn\textquotesingle{}t print}
\NormalTok{dt[,hp2wt}\SpecialCharTok{:}\ErrorTok{=}\NormalTok{hp}\SpecialCharTok{/}\NormalTok{wt][] }\CommentTok{\# does print}
\CommentTok{\#\textgreater{}     mpg cyl disp  hp drat   wt qsec vs am gear carb hp2wt}
\CommentTok{\#\textgreater{} 1: 21.0   6  160 110 3.90 2.62 16.5  0  1    4    4  42.0}
\CommentTok{\#\textgreater{} 2: 21.0   6  160 110 3.90 2.88 17.0  0  1    4    4  38.3}
\CommentTok{\#\textgreater{} 3: 22.8   4  108  93 3.85 2.32 18.6  1  1    4    1  40.1}
\CommentTok{\#\textgreater{} 4: 21.4   6  258 110 3.08 3.21 19.4  1  0    3    1  34.2}
\CommentTok{\#\textgreater{} 5: 18.7   8  360 175 3.15 3.44 17.0  0  0    3    2  50.9}
\CommentTok{\#\textgreater{} 6: 18.1   6  225 105 2.76 3.46 20.2  1  0    3    1  30.3}
\end{Highlighting}
\end{Shaded}

\hypertarget{ux8fd0ux7528}{%
\section{运用}\label{ux8fd0ux7528}}

\hypertarget{ux81eaux5b9aux4e49ux51fdux6570ux8ba1ux7b97}{%
\subsection{自定义函数计算}\label{ux81eaux5b9aux4e49ux51fdux6570ux8ba1ux7b97}}

1.自定义函数处理列

按照自定义函数计算修改单列或多列

\begin{Shaded}
\begin{Highlighting}[]
\CommentTok{\# 测试函数}

\NormalTok{fun }\OtherTok{\textless{}{-}} \ControlFlowTok{function}\NormalTok{(x)\{}
\NormalTok{  x }\OtherTok{\textless{}{-}}\NormalTok{ x}\SpecialCharTok{\^{}}\DecValTok{2}\SpecialCharTok{+}\DecValTok{1}
\NormalTok{\}}

\NormalTok{DT }\OtherTok{\textless{}{-}}  \FunctionTok{data.table}\NormalTok{(}\AttributeTok{x=}\FunctionTok{rep}\NormalTok{(}\FunctionTok{c}\NormalTok{(}\StringTok{"b"}\NormalTok{,}\StringTok{"a"}\NormalTok{,}\StringTok{"c"}\NormalTok{),}\AttributeTok{each=}\DecValTok{3}\NormalTok{), }\AttributeTok{v=}\FunctionTok{c}\NormalTok{(}\DecValTok{1}\NormalTok{,}\DecValTok{1}\NormalTok{,}\DecValTok{1}\NormalTok{,}\DecValTok{2}\NormalTok{,}\DecValTok{2}\NormalTok{,}\DecValTok{1}\NormalTok{,}\DecValTok{1}\NormalTok{,}\DecValTok{2}\NormalTok{,}\DecValTok{2}\NormalTok{), }\AttributeTok{y=}\FunctionTok{c}\NormalTok{(}\DecValTok{1}\NormalTok{,}\DecValTok{3}\NormalTok{,}\DecValTok{6}\NormalTok{), }\AttributeTok{a=}\DecValTok{1}\SpecialCharTok{:}\DecValTok{9}\NormalTok{, }\AttributeTok{b=}\DecValTok{9}\SpecialCharTok{:}\DecValTok{1}\NormalTok{)}

\NormalTok{DT[,.(}\AttributeTok{newcol=}\FunctionTok{fun}\NormalTok{(y)),by}\OtherTok{=}\NormalTok{.(x)]}
\CommentTok{\#\textgreater{}    x newcol}
\CommentTok{\#\textgreater{} 1: b      2}
\CommentTok{\#\textgreater{} 2: b     10}
\CommentTok{\#\textgreater{} 3: b     37}
\CommentTok{\#\textgreater{} 4: a      2}
\CommentTok{\#\textgreater{} 5: a     10}
\CommentTok{\#\textgreater{} 6: a     37}
\CommentTok{\#\textgreater{} 7: c      2}
\CommentTok{\#\textgreater{} 8: c     10}
\CommentTok{\#\textgreater{} 9: c     37}

\CommentTok{\#Not run}
\CommentTok{\#DT[,lapply(.SD,fun),.SDcols=c(\textquotesingle{}y\textquotesingle{},\textquotesingle{}a\textquotesingle{}),by=.(x)] \#多列参与计算}


\CommentTok{\# 批量修改列}
\CommentTok{\#Not run}

\CommentTok{\# myfun \textless{}{-} function(x)\{}
\CommentTok{\#   return(x)}
\CommentTok{\# \}}
\CommentTok{\# }
\CommentTok{\# dt \textless{}{-} dt[,colnames(dt):=lapply(.SD[,1:ncol(dt)],myfun)] \#很重要的用法}
\end{Highlighting}
\end{Shaded}

\hypertarget{ux5e26ux6c47ux603bux7684ux805aux5408ux8fd0ux7b97}{%
\subsection{带汇总的聚合运算}\label{ux5e26ux6c47ux603bux7684ux805aux5408ux8fd0ux7b97}}

按照by的字段级别汇总.

\begin{enumerate}
\def\labelenumi{\arabic{enumi}.}
\tightlist
\item
  rollup
\end{enumerate}

分组聚合后设置id=TRUE将各个级别的汇总显示清晰,当by字段只有一个是和正常聚合计算没有区别.以下是官方案例.

\begin{Shaded}
\begin{Highlighting}[]
\CommentTok{\#Usage}
\CommentTok{\#rollup(x, j, by, .SDcols, id = FALSE, ...)}
\NormalTok{n }\OtherTok{=}\NormalTok{ 24L}
\FunctionTok{set.seed}\NormalTok{(}\DecValTok{25}\NormalTok{)}
\NormalTok{DT }\OtherTok{\textless{}{-}} \FunctionTok{data.table}\NormalTok{(}
    \AttributeTok{color =} \FunctionTok{sample}\NormalTok{(}\FunctionTok{c}\NormalTok{(}\StringTok{"green"}\NormalTok{,}\StringTok{"yellow"}\NormalTok{,}\StringTok{"red"}\NormalTok{), n, }\ConstantTok{TRUE}\NormalTok{),}
    \AttributeTok{year =} \FunctionTok{as.Date}\NormalTok{(}\FunctionTok{sample}\NormalTok{(}\FunctionTok{paste0}\NormalTok{(}\DecValTok{2011}\SpecialCharTok{:}\DecValTok{2015}\NormalTok{,}\StringTok{"{-}01{-}01"}\NormalTok{), n, }\ConstantTok{TRUE}\NormalTok{)),}
    \AttributeTok{status =} \FunctionTok{as.factor}\NormalTok{(}\FunctionTok{sample}\NormalTok{(}\FunctionTok{c}\NormalTok{(}\StringTok{"removed"}\NormalTok{,}\StringTok{"active"}\NormalTok{,}\StringTok{"inactive"}\NormalTok{,}\StringTok{"archived"}\NormalTok{), n, }\ConstantTok{TRUE}\NormalTok{)),}
    \AttributeTok{amount =} \FunctionTok{sample}\NormalTok{(}\DecValTok{1}\SpecialCharTok{:}\DecValTok{5}\NormalTok{, n, }\ConstantTok{TRUE}\NormalTok{),}
    \AttributeTok{value =} \FunctionTok{sample}\NormalTok{(}\FunctionTok{c}\NormalTok{(}\DecValTok{3}\NormalTok{, }\FloatTok{3.5}\NormalTok{, }\FloatTok{2.5}\NormalTok{, }\DecValTok{2}\NormalTok{), n, }\ConstantTok{TRUE}\NormalTok{)}
\NormalTok{)}
\FunctionTok{rollup}\NormalTok{(DT, }\AttributeTok{j =} \FunctionTok{sum}\NormalTok{(value), }\AttributeTok{by =} \FunctionTok{c}\NormalTok{(}\StringTok{"color"}\NormalTok{,}\StringTok{"year"}\NormalTok{,}\StringTok{"status"}\NormalTok{)) }\CommentTok{\# default id=FALSE}
\CommentTok{\#\textgreater{}      color       year   status   V1}
\CommentTok{\#\textgreater{}  1:    red 2015{-}01{-}01   active  3.5}
\CommentTok{\#\textgreater{}  2:  green 2015{-}01{-}01 inactive  5.5}
\CommentTok{\#\textgreater{}  3:  green 2014{-}01{-}01 archived  3.5}
\CommentTok{\#\textgreater{}  4:  green 2015{-}01{-}01 archived  2.0}
\CommentTok{\#\textgreater{}  5: yellow 2014{-}01{-}01   active  4.5}
\CommentTok{\#\textgreater{}  6:    red 2013{-}01{-}01 inactive  2.0}
\CommentTok{\#\textgreater{}  7:  green 2011{-}01{-}01   active  6.0}
\CommentTok{\#\textgreater{}  8:    red 2014{-}01{-}01 inactive  2.5}
\CommentTok{\#\textgreater{}  9:  green 2011{-}01{-}01 archived  2.5}
\CommentTok{\#\textgreater{} 10: yellow 2015{-}01{-}01   active  2.0}
\CommentTok{\#\textgreater{} 11:    red 2012{-}01{-}01 archived  2.0}
\CommentTok{\#\textgreater{} 12:    red 2011{-}01{-}01  removed  3.5}
\CommentTok{\#\textgreater{} 13:  green 2014{-}01{-}01 inactive  8.0}
\CommentTok{\#\textgreater{} 14:  green 2011{-}01{-}01  removed  2.0}
\CommentTok{\#\textgreater{} 15: yellow 2012{-}01{-}01 archived  2.5}
\CommentTok{\#\textgreater{} 16:    red 2013{-}01{-}01  removed  3.5}
\CommentTok{\#\textgreater{} 17:  green 2013{-}01{-}01   active  3.0}
\CommentTok{\#\textgreater{} 18:  green 2014{-}01{-}01  removed  2.5}
\CommentTok{\#\textgreater{} 19:    red 2011{-}01{-}01 archived  3.0}
\CommentTok{\#\textgreater{} 20:    red 2015{-}01{-}01     \textless{}NA\textgreater{}  3.5}
\CommentTok{\#\textgreater{} 21:  green 2015{-}01{-}01     \textless{}NA\textgreater{}  7.5}
\CommentTok{\#\textgreater{} 22:  green 2014{-}01{-}01     \textless{}NA\textgreater{} 14.0}
\CommentTok{\#\textgreater{} 23: yellow 2014{-}01{-}01     \textless{}NA\textgreater{}  4.5}
\CommentTok{\#\textgreater{} 24:    red 2013{-}01{-}01     \textless{}NA\textgreater{}  5.5}
\CommentTok{\#\textgreater{} 25:  green 2011{-}01{-}01     \textless{}NA\textgreater{} 10.5}
\CommentTok{\#\textgreater{} 26:    red 2014{-}01{-}01     \textless{}NA\textgreater{}  2.5}
\CommentTok{\#\textgreater{} 27: yellow 2015{-}01{-}01     \textless{}NA\textgreater{}  2.0}
\CommentTok{\#\textgreater{} 28:    red 2012{-}01{-}01     \textless{}NA\textgreater{}  2.0}
\CommentTok{\#\textgreater{} 29:    red 2011{-}01{-}01     \textless{}NA\textgreater{}  6.5}
\CommentTok{\#\textgreater{} 30: yellow 2012{-}01{-}01     \textless{}NA\textgreater{}  2.5}
\CommentTok{\#\textgreater{} 31:  green 2013{-}01{-}01     \textless{}NA\textgreater{}  3.0}
\CommentTok{\#\textgreater{} 32:    red       \textless{}NA\textgreater{}     \textless{}NA\textgreater{} 20.0}
\CommentTok{\#\textgreater{} 33:  green       \textless{}NA\textgreater{}     \textless{}NA\textgreater{} 35.0}
\CommentTok{\#\textgreater{} 34: yellow       \textless{}NA\textgreater{}     \textless{}NA\textgreater{}  9.0}
\CommentTok{\#\textgreater{} 35:   \textless{}NA\textgreater{}       \textless{}NA\textgreater{}     \textless{}NA\textgreater{} 64.0}
\CommentTok{\#\textgreater{}      color       year   status   V1}
\CommentTok{\#rollup(DT, j = sum(value), by = c("color","year","status"), id=TRUE)}
\end{Highlighting}
\end{Shaded}

个人运用,实际工作中常常需要汇总项,汇总项在Excel透视表中很简单,在R中我之前是构造重复的数据源聚合汇总出现汇总项,极大浪费内存,运算速度减慢.

\begin{itemize}
\tightlist
\item
  新方法 rollup
\end{itemize}

\begin{Shaded}
\begin{Highlighting}[]
\FunctionTok{set.seed}\NormalTok{(}\DecValTok{25}\NormalTok{)}
\NormalTok{N }\OtherTok{\textless{}{-}} \DecValTok{1000}
\NormalTok{dt }\OtherTok{\textless{}{-}} \FunctionTok{data.table}\NormalTok{(}\AttributeTok{col1=}\FunctionTok{sample}\NormalTok{(LETTERS[}\DecValTok{1}\SpecialCharTok{:}\DecValTok{5}\NormalTok{],N,}\AttributeTok{replace =}\NormalTok{ T),}\AttributeTok{col2=}\FunctionTok{sample}\NormalTok{(letters[}\DecValTok{1}\SpecialCharTok{:}\DecValTok{5}\NormalTok{],N,}\AttributeTok{replace =}\NormalTok{ T),}\AttributeTok{num=}\DecValTok{1}\SpecialCharTok{:}\NormalTok{N)}

\FunctionTok{rollup}\NormalTok{(dt,}\AttributeTok{j=}\FunctionTok{c}\NormalTok{(}\FunctionTok{list}\NormalTok{(}\FunctionTok{sum}\NormalTok{(num))),}\AttributeTok{by=}\FunctionTok{c}\NormalTok{(}\StringTok{\textquotesingle{}col1\textquotesingle{}}\NormalTok{,}\StringTok{\textquotesingle{}col2\textquotesingle{}}\NormalTok{))}
\CommentTok{\#\textgreater{}     col1 col2     V1}
\CommentTok{\#\textgreater{}  1:    E    a  19926}
\CommentTok{\#\textgreater{}  2:    D    a  20966}
\CommentTok{\#\textgreater{}  3:    A    d  12927}
\CommentTok{\#\textgreater{}  4:    A    b  20862}
\CommentTok{\#\textgreater{}  5:    A    c  15331}
\CommentTok{\#\textgreater{}  6:    B    d  15414}
\CommentTok{\#\textgreater{}  7:    C    e  20794}
\CommentTok{\#\textgreater{}  8:    D    e  16110}
\CommentTok{\#\textgreater{}  9:    C    d  22152}
\CommentTok{\#\textgreater{} 10:    A    a  18378}
\CommentTok{\#\textgreater{} 11:    C    c  19474}
\CommentTok{\#\textgreater{} 12:    E    d  18831}
\CommentTok{\#\textgreater{} 13:    B    b  19941}
\CommentTok{\#\textgreater{} 14:    C    a  19652}
\CommentTok{\#\textgreater{} 15:    E    c  16734}
\CommentTok{\#\textgreater{} 16:    E    e  24137}
\CommentTok{\#\textgreater{} 17:    E    b  21988}
\CommentTok{\#\textgreater{} 18:    D    b  16607}
\CommentTok{\#\textgreater{} 19:    B    c  25720}
\CommentTok{\#\textgreater{} 20:    B    a  22109}
\CommentTok{\#\textgreater{} 21:    A    e  18724}
\CommentTok{\#\textgreater{} 22:    C    b  24323}
\CommentTok{\#\textgreater{} 23:    D    d  20508}
\CommentTok{\#\textgreater{} 24:    D    c  19668}
\CommentTok{\#\textgreater{} 25:    B    e  29224}
\CommentTok{\#\textgreater{} 26:    E \textless{}NA\textgreater{} 101616}
\CommentTok{\#\textgreater{} 27:    D \textless{}NA\textgreater{}  93859}
\CommentTok{\#\textgreater{} 28:    A \textless{}NA\textgreater{}  86222}
\CommentTok{\#\textgreater{} 29:    B \textless{}NA\textgreater{} 112408}
\CommentTok{\#\textgreater{} 30:    C \textless{}NA\textgreater{} 106395}
\CommentTok{\#\textgreater{} 31: \textless{}NA\textgreater{} \textless{}NA\textgreater{} 500500}
\CommentTok{\#\textgreater{}     col1 col2     V1}
\CommentTok{\#同上 添加汇总项名称 total}
\CommentTok{\#rollup(dt,j=c(list(total=sum(num))),by=c(\textquotesingle{}col1\textquotesingle{},\textquotesingle{}col2\textquotesingle{}))}
\CommentTok{\#添加id=TRUE参数,多出的grouping 列显示聚合级别}
\CommentTok{\#rollup(dt,j=c(list(total=sum(num))),by=c(\textquotesingle{}col1\textquotesingle{},\textquotesingle{}col2\textquotesingle{}),id=TRUE)}
\end{Highlighting}
\end{Shaded}

2.groupingsets

按照指定字段聚合.包作者说相同与SQL中的 GROUPING SETS 操作.详情参照\href{http://www.postgresql.org/docs/9.5/static/queries-table-expressions.html\#QUERIES-GROUPING-SETS}{postgresql}

\begin{Shaded}
\begin{Highlighting}[]
\NormalTok{res }\OtherTok{\textless{}{-}} \FunctionTok{groupingsets}\NormalTok{(DT, }\AttributeTok{j =} \FunctionTok{c}\NormalTok{(}\FunctionTok{list}\NormalTok{(}\AttributeTok{count=}\NormalTok{.N), }\FunctionTok{lapply}\NormalTok{(.SD, sum)), }\AttributeTok{by =} \FunctionTok{c}\NormalTok{(}\StringTok{"color"}\NormalTok{,}\StringTok{"year"}\NormalTok{,}\StringTok{"status"}\NormalTok{),}
             \AttributeTok{sets =} \FunctionTok{list}\NormalTok{(}\StringTok{"color"}\NormalTok{, }\FunctionTok{c}\NormalTok{(}\StringTok{"year"}\NormalTok{,}\StringTok{"status"}\NormalTok{), }\FunctionTok{character}\NormalTok{()), }\AttributeTok{id=}\ConstantTok{TRUE}\NormalTok{)}
\FunctionTok{head}\NormalTok{(res)}
\CommentTok{\#\textgreater{}    grouping  color       year   status count amount value}
\CommentTok{\#\textgreater{} 1:        3    red       \textless{}NA\textgreater{}     \textless{}NA\textgreater{}     7     19  20.0}
\CommentTok{\#\textgreater{} 2:        3  green       \textless{}NA\textgreater{}     \textless{}NA\textgreater{}    13     43  35.0}
\CommentTok{\#\textgreater{} 3:        3 yellow       \textless{}NA\textgreater{}     \textless{}NA\textgreater{}     4     10   9.0}
\CommentTok{\#\textgreater{} 4:        4   \textless{}NA\textgreater{} 2015{-}01{-}01   active     2      8   5.5}
\CommentTok{\#\textgreater{} 5:        4   \textless{}NA\textgreater{} 2015{-}01{-}01 inactive     2      5   5.5}
\CommentTok{\#\textgreater{} 6:        4   \textless{}NA\textgreater{} 2014{-}01{-}01 archived     1      3   3.5}
\end{Highlighting}
\end{Shaded}

注意groupingsets函数中sets参数,用list()包裹想要聚合的字段组合,最后还有一个character(),加上该部分相当于全部聚合.当by只有一个字段时,相当于汇总.用法类似sql中``()''.

上述语句结果等同于下面sql.

\begin{Shaded}
\begin{Highlighting}[]
\KeywordTok{select}\NormalTok{ color ,}\DataTypeTok{year}\NormalTok{, status,}\FunctionTok{count}\NormalTok{(}\OperatorTok{*}\NormalTok{) }\FunctionTok{count}\NormalTok{,}\FunctionTok{sum}\NormalTok{(amount) amount,}\FunctionTok{sum}\NormalTok{(}\FunctionTok{value}\NormalTok{) }\FunctionTok{value} 
\KeywordTok{FROM}\NormalTok{ dbo.DT}
\KeywordTok{GROUP} \KeywordTok{BY}
\FunctionTok{GROUPING}\NormalTok{ SETS(}
\NormalTok{(color),}
\NormalTok{(}\DataTypeTok{year}\NormalTok{,status),}
\NormalTok{() }\CommentTok{{-}{-}{-}{-} 类似 character()}
\NormalTok{)}
\end{Highlighting}
\end{Shaded}

最后还有cube()函数,可?cube查看用法

\hypertarget{ux884cux5217ux8f6cux53d8}{%
\subsection{行列转变}\label{ux884cux5217ux8f6cux53d8}}

\begin{itemize}
\tightlist
\item
  一列变多行
\end{itemize}

用tstrsplit()函数实现

\begin{Shaded}
\begin{Highlighting}[]
\NormalTok{n }\OtherTok{\textless{}{-}} \DecValTok{10}
\NormalTok{dt }\OtherTok{\textless{}{-}} \FunctionTok{data.table}\NormalTok{(}\AttributeTok{name=}\NormalTok{LETTERS[}\DecValTok{1}\SpecialCharTok{:}\NormalTok{n],}\AttributeTok{char=}\FunctionTok{rep}\NormalTok{(}\StringTok{\textquotesingle{}我{-}爱{-}R{-}语{-}言\textquotesingle{}}\NormalTok{),n)}
\NormalTok{res }\OtherTok{\textless{}{-}}\NormalTok{ dt[,.(}\AttributeTok{newcol=}\FunctionTok{tstrsplit}\NormalTok{(char,}\StringTok{\textquotesingle{}{-}\textquotesingle{}}\NormalTok{)),by}\OtherTok{=}\NormalTok{.(name)]}
\FunctionTok{head}\NormalTok{(res)}
\CommentTok{\#\textgreater{}    name newcol}
\CommentTok{\#\textgreater{} 1:    A     我}
\CommentTok{\#\textgreater{} 2:    A     爱}
\CommentTok{\#\textgreater{} 3:    A      R}
\CommentTok{\#\textgreater{} 4:    A     语}
\CommentTok{\#\textgreater{} 5:    A     言}
\CommentTok{\#\textgreater{} 6:    B     我}
\end{Highlighting}
\end{Shaded}

\begin{itemize}
\tightlist
\item
  多行变一列
\end{itemize}

\begin{Shaded}
\begin{Highlighting}[]
\NormalTok{res[,.(}\AttributeTok{char=}\FunctionTok{paste0}\NormalTok{(newcol,}\AttributeTok{collapse =} \StringTok{\textquotesingle{}{-}\textquotesingle{}}\NormalTok{)),by}\OtherTok{=}\NormalTok{.(name)]}
\CommentTok{\#\textgreater{}     name          char}
\CommentTok{\#\textgreater{}  1:    A 我{-}爱{-}R{-}语{-}言}
\CommentTok{\#\textgreater{}  2:    B 我{-}爱{-}R{-}语{-}言}
\CommentTok{\#\textgreater{}  3:    C 我{-}爱{-}R{-}语{-}言}
\CommentTok{\#\textgreater{}  4:    D 我{-}爱{-}R{-}语{-}言}
\CommentTok{\#\textgreater{}  5:    E 我{-}爱{-}R{-}语{-}言}
\CommentTok{\#\textgreater{}  6:    F 我{-}爱{-}R{-}语{-}言}
\CommentTok{\#\textgreater{}  7:    G 我{-}爱{-}R{-}语{-}言}
\CommentTok{\#\textgreater{}  8:    H 我{-}爱{-}R{-}语{-}言}
\CommentTok{\#\textgreater{}  9:    I 我{-}爱{-}R{-}语{-}言}
\CommentTok{\#\textgreater{} 10:    J 我{-}爱{-}R{-}语{-}言}
\CommentTok{\# 同上}
\CommentTok{\# res[,.(char=stringr::str\_c(newcol,collapse = \textquotesingle{}{-}\textquotesingle{})),by=.(name)]}
\end{Highlighting}
\end{Shaded}

\hypertarget{database}{%
\chapter{database}\label{database}}

实际工作中,需要从数据库获取数据并清洗,R与数据库有多种交互方式,目前工作中打交道数据库主要是MSSQL,Oracle,mysql等,本文主要从以上数据库介绍记录``R与数据库的连接''。

R中与数据库交互的包主要有DBI,RODBC,RMySQL,ROracle,odbc等包。DBI库在查询或上传工作中效率比RODBC高,特别数据量较大时,上传效率差异巨大,具体\href{https://github.com/r-dbi/odbc}{差异}请点击查看详情。

即使你暂时没有用数据库,也建议你未来用数据库存储数据,尤其是当有一定数据量时;在我最开始接触数据时,数据一般保存在Excel,那时候数据量大概在50万行左右,当公式较多,尤其时需要大批量vlookup时,Excel表格将会很卡顿。

\hypertarget{ux5b89ux88c5ux6570ux636eux5e93}{%
\section{安装数据库}\label{ux5b89ux88c5ux6570ux636eux5e93}}

如果暂时没有数据库使用经验,如果是使用Windows系统,直接去微软官网下载安装数据库即可。如果决定用R做数据分析相关工作,尤其时商业环境下,使用数据库有较强的必要性。安装数据库后,利用数据库做数据分析的练习测试也是不错的体验。另外也可以积累ETL相关经验。

仅简单介绍 MS SQL Server 安装

\begin{itemize}
\tightlist
\item
  Win环境下安装
\end{itemize}

MS\href{https://www.microsoft.com/zh-cn/sql-server/sql-server-downloads}{下载},选择开发版或精简版(Developer、Express)其中一个版本下载即可。

\begin{figure}
\centering
\includegraphics{./picture/chap2/ms install.png}
\caption{数据库下载}
\end{figure}

成功下载后,按照提示一步步确认即可安装成功。另外使用\texttt{SSMS}工具,微软配套的MS SQL SERVER数据库链接工具连接数据库。至于详细的数据库配置尤其是远程连接、账户等信息请自行查阅相关资料。

\begin{itemize}
\tightlist
\item
  Linux环境下安装
\end{itemize}

\href{https://docs.microsoft.com/zh-cn/sql/linux/sql-server-linux-setup?view=sql-server-ver15}{官网安装指南}

以下用于 SQL Server 2019 的命令指向 Ubuntu 20.04 存储库。 如果使用的是 Ubuntu 18.04 或 16.04,请将以下路径更改为 /ubuntu/18.04/ 或 /ubuntu/16.04/,而不是 /ubuntu/20.04/。

\begin{Shaded}
\begin{Highlighting}[]
\CommentTok{\# 导入公共存储库的密钥}
\FunctionTok{wget} \AttributeTok{{-}qO{-}}\NormalTok{ https://packages.microsoft.com/keys/microsoft.asc }\KeywordTok{|} \FunctionTok{sudo}\NormalTok{ apt{-}key add }\AttributeTok{{-}}

\CommentTok{\# 为 SQL Server 2019 注册 Microsoft SQL Server Ubuntu 存储库}
\FunctionTok{sudo}\NormalTok{ add{-}apt{-}repository }\StringTok{"}\VariableTok{$(}\FunctionTok{wget} \AttributeTok{{-}qO{-}}\NormalTok{ https://packages.microsoft.com/config/ubuntu/20.04/mssql{-}server{-}2019.list}\VariableTok{)}\StringTok{"}

\CommentTok{\# sudo add{-}apt{-}repository "$(wget {-}qO{-} https://packages.microsoft.com/config/ubuntu/18.04/mssql{-}server{-}2019.list)"}

\CommentTok{\# 安装 SQL Server}
\FunctionTok{sudo}\NormalTok{ apt{-}get update}
\FunctionTok{sudo}\NormalTok{ apt{-}get install }\AttributeTok{{-}y}\NormalTok{ mssql{-}server}

\CommentTok{\# 验证服务是否运行}
\ExtensionTok{systemctl}\NormalTok{ status mssql{-}server }\AttributeTok{{-}{-}no{-}pager}
\end{Highlighting}
\end{Shaded}

至于其他如安装sql server 命令行工具请\href{https://docs.microsoft.com/zh-cn/sql/linux/quickstart-install-connect-ubuntu?view=sql-server-linux-ver15\&preserve-view=true}{查阅官网安装}。

接下来我们就R语言与数据库的交互包展开介绍。

\hypertarget{dbi}{%
\section{DBI}\label{dbi}}

\hypertarget{ux5b89ux88c5-2}{%
\subsection{安装}\label{ux5b89ux88c5-2}}

\begin{Shaded}
\begin{Highlighting}[]
\FunctionTok{install.packages}\NormalTok{(}\StringTok{\textquotesingle{}DBI\textquotesingle{}}\NormalTok{)}
\end{Highlighting}
\end{Shaded}

\hypertarget{ux8fdeux63a5ux6570ux636eux5e93}{%
\subsection{连接数据库}\label{ux8fdeux63a5ux6570ux636eux5e93}}

\begin{itemize}
\tightlist
\item
  连接MS SQL SERVER
\end{itemize}

通过以下代码即可连接到服务器172.16.88.2(即IP地址)的数据库,成功连接后即可与数据库交互。

\begin{Shaded}
\begin{Highlighting}[]
\FunctionTok{library}\NormalTok{(DBI)}
\NormalTok{con }\OtherTok{\textless{}{-}} \FunctionTok{dbConnect}\NormalTok{(}
  \AttributeTok{drv =}\NormalTok{ odbc}\SpecialCharTok{::}\FunctionTok{odbc}\NormalTok{(), }\AttributeTok{Driver =} \StringTok{"SQL Server"}\NormalTok{, }\AttributeTok{server =} \StringTok{"172.16.88.2"}\NormalTok{,}\AttributeTok{database =} \StringTok{"spb"}\NormalTok{, }\AttributeTok{uid =} \StringTok{"zhongyf"}\NormalTok{, }\AttributeTok{pwd =} \StringTok{"Zyf123456"}
\NormalTok{)}
\end{Highlighting}
\end{Shaded}

如果你用windows系统,通过DBI包连接数据库发现乱码时,根据数据库编码指定encoding参数即可,常规在win下连接sqlserver设置encoding = ``GBK''。

\begin{Shaded}
\begin{Highlighting}[]
\FunctionTok{library}\NormalTok{(DBI)}
\CommentTok{\#根据数据库编码方式指定encoding}
\NormalTok{con }\OtherTok{\textless{}{-}} \FunctionTok{dbConnect}\NormalTok{(}
  \AttributeTok{drv =}\NormalTok{ odbc}\SpecialCharTok{::}\FunctionTok{odbc}\NormalTok{(), }\AttributeTok{Driver =} \StringTok{"SQL Server"}\NormalTok{, }\AttributeTok{server =} \StringTok{"172.16.88.2"}\NormalTok{,}
  \AttributeTok{database =} \StringTok{"spb"}\NormalTok{, }\AttributeTok{uid =} \StringTok{"zhongyf"}\NormalTok{, }\AttributeTok{pwd =} \StringTok{"Zyf123456"}\NormalTok{, }\AttributeTok{encoding =} \StringTok{"GBK"}
\NormalTok{)}
\CommentTok{\# 查看本机可用驱动 如缺少相应驱动则安装,ODBC Driver 17 for SQL Server 就是个人安装的驱动}

\NormalTok{Drivers\_tbl }\OtherTok{\textless{}{-}}\NormalTok{ odbc}\SpecialCharTok{::}\FunctionTok{odbcListDrivers}\NormalTok{() }
\FunctionTok{head}\NormalTok{(Drivers\_tbl)}
\end{Highlighting}
\end{Shaded}

查询数据库编码方式,从而选择连接数据库时相应的编码方式。

\begin{Shaded}
\begin{Highlighting}[]
\NormalTok{con }\OtherTok{\textless{}{-}} \FunctionTok{dbConnect}\NormalTok{(}
  \AttributeTok{drv =}\NormalTok{ odbc}\SpecialCharTok{::}\FunctionTok{odbc}\NormalTok{(), }\AttributeTok{Driver =} \StringTok{"ODBC Driver 17 for SQL Server"}\NormalTok{,}
  \AttributeTok{server =} \StringTok{"172.16.88.2"}\NormalTok{, }\AttributeTok{database =} \StringTok{"spb"}\NormalTok{, }\AttributeTok{uid =} \StringTok{"zhongyf"}\NormalTok{, }\AttributeTok{pwd =} \StringTok{"Zyf123456"}
\NormalTok{)}

\CommentTok{\#查看编码是否是936 代表中文简体}
\NormalTok{sql }\OtherTok{\textless{}{-}} \StringTok{"SELECT COLLATIONPROPERTY( \textquotesingle{}chinese\_prc\_ci\_as\textquotesingle{}, \textquotesingle{}codepage\textquotesingle{} )"}

\FunctionTok{dbGetQuery}\NormalTok{(con,sql)}

\CommentTok{\# same above}
\CommentTok{\# dbExecute(con,sql)}

\CommentTok{\# 用完后记得关闭数据库连接}
\NormalTok{DBI}\SpecialCharTok{::}\FunctionTok{dbDisconnect}\NormalTok{(con)}
\end{Highlighting}
\end{Shaded}

\begin{itemize}
\tightlist
\item
  连接mysql
\end{itemize}

\texttt{MySQL()}函数来源\texttt{RMySQL}包,用来创建\texttt{\textless{}MySQLDriver\textgreater{}}驱动,以下代码可连接到阿里云的MySQL数据库。

\begin{Shaded}
\begin{Highlighting}[]
\FunctionTok{library}\NormalTok{(RMySQL)}
\NormalTok{con }\OtherTok{\textless{}{-}} \FunctionTok{dbConnect}\NormalTok{(}\FunctionTok{MySQL}\NormalTok{(),}
  \AttributeTok{dbname =} \StringTok{"test"}\NormalTok{, }\AttributeTok{user =} \StringTok{"test\_admin"}\NormalTok{, }\AttributeTok{password =} \StringTok{"30HL1234M7\#¥lD6gxjB"}\NormalTok{,}
  \AttributeTok{host =} \StringTok{"prd{-}public{-}mypersonal.mysql.test.zhangjiabei.rds.aliyuncs.com"}
\NormalTok{)}
\end{Highlighting}
\end{Shaded}

或者通过本地已安装驱动连接数据库

\begin{Shaded}
\begin{Highlighting}[]
\NormalTok{con }\OtherTok{\textless{}{-}}\NormalTok{ DBI}\SpecialCharTok{::}\FunctionTok{dbConnect}\NormalTok{(odbc}\SpecialCharTok{::}\FunctionTok{odbc}\NormalTok{(),}
  \AttributeTok{Driver =} \StringTok{"MySQL ODBC 8.0 Unicode Driver"}\NormalTok{,}
  \AttributeTok{Server =} \StringTok{"localhost"}\NormalTok{, }\AttributeTok{UID =} \StringTok{"root"}\NormalTok{, }\AttributeTok{PWD =} \StringTok{"123456"}\NormalTok{, }\AttributeTok{Database =} \StringTok{"mysql"}\NormalTok{,}
  \AttributeTok{Port =} \DecValTok{3306}
\NormalTok{)}
\end{Highlighting}
\end{Shaded}

mysql数据库默认端口是3306,访问不通时记得检查3306端口是否开放。

\hypertarget{ux6267ux884csqlux4efbux52a1}{%
\subsection{执行sql任务}\label{ux6267ux884csqlux4efbux52a1}}

dbGetQuery()函数处理由DBI包创建的con连接查询任务,dbExecute()执行一些数据库任务

\begin{Shaded}
\begin{Highlighting}[]
\CommentTok{\# dbGetQuery 直接查询}
\NormalTok{res\_table }\OtherTok{\textless{}{-}} \FunctionTok{dbGetQuery}\NormalTok{(con,}\StringTok{\textquotesingle{}select * from table\textquotesingle{}}\NormalTok{) }\CommentTok{\#直接获取sql查询结果}

\CommentTok{\#dbReadTable直接读取}
\FunctionTok{dbReadTable}\NormalTok{(con,}\StringTok{\textquotesingle{}tbl\_name\textquotesingle{}}\NormalTok{) }\CommentTok{\#直接读取数据库中某表}

\CommentTok{\# dbSendQuery 执行一个查询任务 }
\NormalTok{res }\OtherTok{\textless{}{-}} \FunctionTok{dbSendQuery}\NormalTok{(}\AttributeTok{conn =}\NormalTok{ con,}\AttributeTok{statement =} \StringTok{\textquotesingle{}select * FROM tab\textquotesingle{}}\NormalTok{)}
\FunctionTok{dbFetch}\NormalTok{(res)}
\FunctionTok{dbClearResult}\NormalTok{(res)}

\CommentTok{\# dbExecute}
\FunctionTok{dbExecute}\NormalTok{(con,}\StringTok{\textquotesingle{}delete from table where num \textless{}=1000\textquotesingle{}}\NormalTok{) }\CommentTok{\#类似任务}

\CommentTok{\# dbWriteTable()}
\CommentTok{\# 上传数据,指定表名,需上传的数据框df,overwrite是否覆盖,append是否可追加}
\FunctionTok{dbWriteTable}\NormalTok{(}\AttributeTok{conn =}\NormalTok{ con,}\AttributeTok{name =} \StringTok{\textquotesingle{}表名\textquotesingle{}}\NormalTok{,}\AttributeTok{value =}\NormalTok{ df,}\AttributeTok{overwrite=}\NormalTok{TURE,}\AttributeTok{append=}\ConstantTok{FALSE}\NormalTok{)}
\end{Highlighting}
\end{Shaded}

\hypertarget{ux51fdux6570ux4ecbux7ecd}{%
\subsection{函数介绍}\label{ux51fdux6570ux4ecbux7ecd}}

查看数据库信息,查看表名,删除表,关闭连接等常用操作.

\begin{Shaded}
\begin{Highlighting}[]
\NormalTok{con }\OtherTok{\textless{}{-}} \FunctionTok{dbConnect}\NormalTok{(}
  \AttributeTok{drv =}\NormalTok{ odbc}\SpecialCharTok{::}\FunctionTok{odbc}\NormalTok{(),}
  \AttributeTok{Driver =} \StringTok{"ODBC Driver 17 for SQL Server"}\NormalTok{, }\AttributeTok{server =} \StringTok{"172.16.88.2"}\NormalTok{, }
  \AttributeTok{database =} \StringTok{"spb"}\NormalTok{, }\AttributeTok{uid =} \StringTok{"zhongyf"}\NormalTok{, }\AttributeTok{pwd =} \StringTok{"Zyf123456"}\NormalTok{, }\AttributeTok{encoding =} \StringTok{"GBK"}
\NormalTok{)}

\CommentTok{\#查看数据版本连接信息}
\FunctionTok{dbGetInfo}\NormalTok{(con)}

\CommentTok{\# 数据库中的全部表名}
\FunctionTok{dbListTables}\NormalTok{(con) }\CommentTok{\#win下中文表名还是会乱码}

\CommentTok{\# 删除表}
\FunctionTok{dbRemoveTable}\NormalTok{(con,}\StringTok{\textquotesingle{}tbl\_name\textquotesingle{}}\NormalTok{)}

\CommentTok{\# 关闭连接}
\FunctionTok{dbDisconnect}\NormalTok{(con)}
\end{Highlighting}
\end{Shaded}

\hypertarget{odbcux5305}{%
\section{odbc包}\label{odbcux5305}}

官方介绍:Connect to ODBC databases (using the DBI interface)

记录到此时,并不时特别清晰\texttt{odbc}与\texttt{DBI}之间的关系。

odbc可以运用于包括(SQL Server, Oracle, MySQL,PostgreSQL,SQLite)等odbc驱动程序于\texttt{DBI}兼容的接口,相比起来\texttt{DBI}包适用范围更广。

1.安装包

\begin{Shaded}
\begin{Highlighting}[]
\CommentTok{\#安装包}
\FunctionTok{install.packages}\NormalTok{(}\StringTok{\textquotesingle{}odbc\textquotesingle{}}\NormalTok{)}
\end{Highlighting}
\end{Shaded}

2.连接数据库

连接数据库需要注意时区、编码,尤其是涉及到时间时时区如果设置有误,可能导致上传数据错误。

当你在Win系统上连接Sql Server时,如果你使用的数据库是中文环境时,最好设置\texttt{encoding}参数。

如果是linux上通过odbc连接SqlServer,一般情况下可以不用设置编码。如果还是乱码,在连接字符中设置字符编码charset=zh\_CN.GBK,设置为gbk会报错。

\begin{Shaded}
\begin{Highlighting}[]
\FunctionTok{library}\NormalTok{(odbc)}
\NormalTok{con }\OtherTok{\textless{}{-}}\NormalTok{ odbc}\SpecialCharTok{::}\FunctionTok{dbConnect}\NormalTok{(}\FunctionTok{odbc}\NormalTok{(),}
  \AttributeTok{Driver =} \StringTok{"SQL Server"}\NormalTok{, }\AttributeTok{Server =} \StringTok{"Vega"}\NormalTok{, }\AttributeTok{Database =} \StringTok{"ghzy"}\NormalTok{,}
  \AttributeTok{Trusted\_Connection =} \StringTok{"True"}
\NormalTok{) }\CommentTok{\# windows身份认证连接}
\CommentTok{\# con \textless{}{-} dbConnect(odbc::odbc(), .connection\_string = "Driver=\{SQL Server\};}
\CommentTok{\#                                 server=Vega;database=ghzy;uid=zhongyf;pwd=Zyf123456;", timeout = 10)}
\NormalTok{con}
\DocumentationTok{\#\# Not run}
\CommentTok{\# Win}
\NormalTok{con\_spb }\OtherTok{\textless{}{-}} \FunctionTok{dbConnect}\NormalTok{(}\FunctionTok{odbc}\NormalTok{(), }\AttributeTok{.connection\_string =} \StringTok{"driver=\{ODBC Driver 17 for SQL Server\};server=172.16.88.2;database=spb;uid=zhongyf;pwd=Zyf123456"}\NormalTok{, }
                     \AttributeTok{timeout =} \DecValTok{10}\NormalTok{, }\AttributeTok{timezone =} \StringTok{"Asia/Shanghai"}\NormalTok{,}\AttributeTok{encoding =} \StringTok{\textquotesingle{}gbk\textquotesingle{}}\NormalTok{)}
\CommentTok{\#Linux}
\NormalTok{con\_dd }\OtherTok{\textless{}{-}} \FunctionTok{dbConnect}\NormalTok{(odbc}\SpecialCharTok{::}\FunctionTok{odbc}\NormalTok{(), }\AttributeTok{.connection\_string =} \StringTok{"driver=\{ODBC Driver 17 for SQL Server\};server=172.16.88.2;}
\StringTok{                 database=aojo\_dd;uid=wj;pwd=12qw\#$ER;charset=zh\_CN.GBK"}\NormalTok{, }\AttributeTok{timeout =} \DecValTok{10}\NormalTok{)}
\end{Highlighting}
\end{Shaded}

3.查询

\begin{Shaded}
\begin{Highlighting}[]
\NormalTok{dt }\OtherTok{\textless{}{-}}\NormalTok{ odbc}\SpecialCharTok{::}\FunctionTok{dbGetQuery}\NormalTok{(con,}\StringTok{\textquotesingle{}select * from DT\textquotesingle{}}\NormalTok{)}
\FunctionTok{head}\NormalTok{(dt)}
\end{Highlighting}
\end{Shaded}

4.写入数据库

\begin{Shaded}
\begin{Highlighting}[]
\NormalTok{odbc}\SpecialCharTok{::}\FunctionTok{dbWriteTable}\NormalTok{(con,}\AttributeTok{name =} \StringTok{\textquotesingle{}表名\textquotesingle{}}\NormalTok{,}\AttributeTok{value =}\NormalTok{ dt,}\AttributeTok{overwrite =}\NormalTok{ T ) }\CommentTok{\# 是否覆盖}
\NormalTok{odbc}\SpecialCharTok{::}\FunctionTok{dbWriteTable}\NormalTok{(con,}\AttributeTok{name =} \StringTok{\textquotesingle{}表名\textquotesingle{}}\NormalTok{,}\AttributeTok{value =}\NormalTok{ dt,}\AttributeTok{append =}\NormalTok{ T ) }\CommentTok{\# 是否追加}
\end{Highlighting}
\end{Shaded}

\hypertarget{rodbcux5305}{%
\section{RODBC包}\label{rodbcux5305}}

RODBC包是R语言对ODBC数据库接口,可以连接所有的ODBC数据库.

1.安装包

\begin{Shaded}
\begin{Highlighting}[]
\FunctionTok{install.packages}\NormalTok{(}\StringTok{\textquotesingle{}RODBC\textquotesingle{}}\NormalTok{)}
\end{Highlighting}
\end{Shaded}

2.SQL SERVER 数据库举例

\begin{Shaded}
\begin{Highlighting}[]
\FunctionTok{library}\NormalTok{(RODBC)}
\NormalTok{con }\OtherTok{\textless{}{-}} \FunctionTok{odbcDriverConnect}\NormalTok{(}\StringTok{"driver=\{SQL Server\};server=192.168.2.62;database=dbname;uid=zhongyf;pwd=Zyf123456"}\NormalTok{)}
\NormalTok{con}
\NormalTok{RODBC}\SpecialCharTok{::}\FunctionTok{sqlQuery}\NormalTok{(con,}\StringTok{\textquotesingle{}select * from test\textquotesingle{}}\NormalTok{)}
\end{Highlighting}
\end{Shaded}

在WINDOWS机器上,需要知道本机是否有相应数据库的驱动程序.

\begin{itemize}
\tightlist
\item
  查看本机上可用驱动
\end{itemize}

\begin{Shaded}
\begin{Highlighting}[]
\NormalTok{odbc}\SpecialCharTok{::}\FunctionTok{odbcListDrivers}\NormalTok{()}
\end{Highlighting}
\end{Shaded}

\begin{itemize}
\tightlist
\item
  怎样安装驱动
\end{itemize}

请参照\href{https://github.com/r-dbi/odbc\#installation}{驱动安装}

ODBC for sql server driver 下载地址\href{https://docs.microsoft.com/zh-cn/sql/connect/odbc/download-odbc-driver-for-sql-server?view=sql-server-ver15}{地址}

3.数据库字符串

请参照\href{https://www.connectionstrings.com/}{数据库连接字符串}

\begin{Shaded}
\begin{Highlighting}[]
\CommentTok{\#ODBC Driver 17 for SQL Server}
\NormalTok{cn }\OtherTok{\textless{}{-}} \FunctionTok{odbcDriverConnect}\NormalTok{(}\StringTok{"Driver=\{ODBC Driver 17 for SQL Server\};Server=localhost;Database=name;UID=username;PWD=123456;"}\NormalTok{) }\CommentTok{\#server 数据库 UID 数据库账户 PWD 数据库账户密码}
\end{Highlighting}
\end{Shaded}

sql server 请参照\href{https://www.connectionstrings.com/microsoft-odbc-driver-17-for-sql-server/}{sql server连接字符串}

\hypertarget{roracleux5305}{%
\section{ROracle包}\label{roracleux5305}}

在第一次安装这个包时遇到了很多困难,首先需要安装oracle客户端,其次配置好环境变量,最后安装包。R与Oracle的连接需要安装\href{https://www.oracle.com/database/technologies/instant-client.html}{Oracle Instant Client},

\begin{enumerate}
\def\labelenumi{\arabic{enumi}.}
\tightlist
\item
  安装客户端
\end{enumerate}

安装oracle客户端,根据电脑的位数选择相应的32位或64位,根据要连接数据库版本,可以去官网自行下载,本机需要下载的\href{https://www.oracle.com/technetwork/database/enterprise-edition/downloads/112010-win64soft-094461.html}{客户端地址}

\begin{enumerate}
\def\labelenumi{\arabic{enumi}.}
\setcounter{enumi}{1}
\tightlist
\item
  配置环境变量
\end{enumerate}

根据自己所使用的系统,配置环境变量

\begin{verbatim}
OCI_INC='D:\app\zhongyf\product\11.2.0\client_1\oci\include'
OCI_LIB64='D:\app\zhongyf\product\11.2.0\client_1\BIN'
\end{verbatim}

linxu上安装\texttt{Roracle}包,可以参考我的

微信公众号:宇飞的世界

\href{https://mp.weixin.qq.com/s/QLwedZ5mTybqSXdHMTGRIw}{公众号文章连接}

\begin{enumerate}
\def\labelenumi{\arabic{enumi}.}
\setcounter{enumi}{2}
\tightlist
\item
  安装包
\end{enumerate}

安装Roracle包需要配置相应版本的Rtools并添加到环境变量,另外配置两个oracle的环境变量。代码中有注释,按照自己安装版本路径修改。

由于ROracle依赖于Oracle Instant Client,安装之前一定要先安装好客户端。

\begin{Shaded}
\begin{Highlighting}[]
\FunctionTok{install.packages}\NormalTok{(}\StringTok{\textquotesingle{}ROracle\textquotesingle{}}\NormalTok{)}
\end{Highlighting}
\end{Shaded}

\begin{enumerate}
\def\labelenumi{\arabic{enumi}.}
\setcounter{enumi}{3}
\tightlist
\item
  连接数据库
\end{enumerate}

\texttt{Roracle}可以通过\texttt{DBI}包链接,除了驱动和连接字符串有差异,其他部分一样。

\begin{Shaded}
\begin{Highlighting}[]
\FunctionTok{library}\NormalTok{(ROracle)}
\NormalTok{drv }\OtherTok{\textless{}{-}}\FunctionTok{dbDriver}\NormalTok{(}\StringTok{"Oracle"}\NormalTok{)}
\NormalTok{connect.string }\OtherTok{\textless{}{-}} \StringTok{\textquotesingle{}(DESCRIPTION =}
\StringTok{                    (ADDRESS = (PROTOCOL = TCP)(HOST = 192.16.88.129)(PORT = 1521))}
\StringTok{                  (CONNECT\_DATA =}
\StringTok{                      (SERVER = DEDICATED)}
\StringTok{                    (SERVICE\_NAME = bidev)}
\StringTok{                  ))\textquotesingle{}} \CommentTok{\#连接字符串}

\NormalTok{con }\OtherTok{\textless{}{-}} \FunctionTok{dbConnect}\NormalTok{(drv,}\AttributeTok{username =} \StringTok{"query"}\NormalTok{, }\AttributeTok{password =} \StringTok{"query"}\NormalTok{,}\AttributeTok{dbname =}\NormalTok{ connect.string)}
\end{Highlighting}
\end{Shaded}

\begin{enumerate}
\def\labelenumi{\arabic{enumi}.}
\setcounter{enumi}{4}
\tightlist
\item
  乱码问题
\end{enumerate}

如果连接oracle数据库,中文乱码设置以下环境变量即可,或者在启动文件配置该环境变量。

linux下可以在文件Renviron中添加,记得引号,路径为{[}/opt/R/4.0.2/lib/R/etc/Renviron{]}

\begin{Shaded}
\begin{Highlighting}[]
\CommentTok{\# 查询数据库编码}
\NormalTok{select }\FunctionTok{userenv}\NormalTok{(}\StringTok{\textquotesingle{}language\textquotesingle{}}\NormalTok{) from dual}
\FunctionTok{Sys.setenv}\NormalTok{(}\AttributeTok{NLS\_LANG=}\StringTok{"SIMPLIFIED CHINESE\_CHINA.AL32UTF8"}\NormalTok{)}
\end{Highlighting}
\end{Shaded}

\hypertarget{rmysqlux5305}{%
\section{RMySQL包}\label{rmysqlux5305}}

RMySQL包的主要作用可以提供驱动与mysql数据库进行连接,在本机未安装mysql的驱动的情况下.该包正在逐渐被淘汰,可以使用RMariaDB包替换。

\hypertarget{ux5b89ux88c5-3}{%
\subsection{安装}\label{ux5b89ux88c5-3}}

Win系统下直接安装即可,其它平台下需提前安装依赖环境。

\begin{Shaded}
\begin{Highlighting}[]
\CommentTok{\#On recent Debian or Ubuntu install libmariadbclient{-}dev}

\FunctionTok{sudo}\NormalTok{ apt{-}get install }\AttributeTok{{-}y}\NormalTok{ libmariadbclient{-}dev}
\CommentTok{\#On Fedora, CentOS or RHEL we need mariadb{-}devel:}

\FunctionTok{sudo}\NormalTok{ yum install mariadb{-}devel}
\CommentTok{\#On OS{-}X use mariadb{-}connector{-}c from Homebrew:}

\ExtensionTok{brew}\NormalTok{ install mariadb{-}connector{-}c}
\end{Highlighting}
\end{Shaded}

\begin{Shaded}
\begin{Highlighting}[]
\FunctionTok{install.packages}\NormalTok{(}\StringTok{\textquotesingle{}RMySQL\textquotesingle{}}\NormalTok{)}
\end{Highlighting}
\end{Shaded}

\hypertarget{ux8fdeux63a5ux4f7fux7528}{%
\subsection{连接使用}\label{ux8fdeux63a5ux4f7fux7528}}

连接数据库,与上述连接方式基本一致。

\begin{Shaded}
\begin{Highlighting}[]
\FunctionTok{library}\NormalTok{(RMySQL)}
\NormalTok{con }\OtherTok{\textless{}{-}}\NormalTok{ RMySQL}\SpecialCharTok{::}\FunctionTok{dbConnect}\NormalTok{(}\AttributeTok{drv =}\NormalTok{ RMySQL}\SpecialCharTok{::}\FunctionTok{MySQL}\NormalTok{(),}\AttributeTok{host=}\StringTok{\textquotesingle{}localhost\textquotesingle{}}\NormalTok{,}\AttributeTok{dbname=}\StringTok{"mysql"}\NormalTok{,}\AttributeTok{username=}\StringTok{"root"}\NormalTok{,}\AttributeTok{password=}\StringTok{\textquotesingle{}123456\textquotesingle{}}\NormalTok{)}
\end{Highlighting}
\end{Shaded}

\texttt{RMariaDB}包与\texttt{RMySQL}包用法基本一致,在连接时注意驱动的选择即可。

\begin{Shaded}
\begin{Highlighting}[]
\FunctionTok{install.packages}\NormalTok{(}\StringTok{\textquotesingle{}RMariaDB\textquotesingle{}}\NormalTok{)}
\FunctionTok{library}\NormalTok{(RMariaDB)}
\NormalTok{con }\OtherTok{\textless{}{-}}\NormalTok{ RMySQL}\SpecialCharTok{::}\FunctionTok{dbConnect}\NormalTok{(}\AttributeTok{drv =}\NormalTok{ RMariaDB}\SpecialCharTok{::}\FunctionTok{MariaDB}\NormalTok{() ,}\AttributeTok{host=}\StringTok{\textquotesingle{}localhost\textquotesingle{}}\NormalTok{,}\AttributeTok{dbname=}\StringTok{"dbtest"}\NormalTok{,}\AttributeTok{username=}\StringTok{"root"}\NormalTok{,}\AttributeTok{password=}\StringTok{\textquotesingle{}123456\textquotesingle{}}\NormalTok{)}
\end{Highlighting}
\end{Shaded}

\hypertarget{ux5e38ux89c1ux95eeux9898-1}{%
\section{常见问题}\label{ux5e38ux89c1ux95eeux9898-1}}

在使用R包连接数据库时有些常见的问题,整理如下:

\hypertarget{ux4e71ux7801ux95eeux9898}{%
\subsection{乱码问题}\label{ux4e71ux7801ux95eeux9898}}

R中中文乱码问题一直都很麻烦,并且常常遇见,尤其是使用win系统时。

\begin{itemize}
\tightlist
\item
  MS SQL SERVER 乱码
\end{itemize}

修改encoding参数,在win系统下,可以考虑使用RODBC包连接查询数据库,因为该包将自动转换编码,不会存在乱码问题。但是上传效率奇慢,为了减少包依赖保持代码一致性使用odbc连接数据库时遇到乱码,在连接数据库时设定encoding即可。

\begin{Shaded}
\begin{Highlighting}[]
\CommentTok{\# win}
\NormalTok{con\_spb }\OtherTok{\textless{}{-}} \FunctionTok{dbConnect}\NormalTok{(}\FunctionTok{odbc}\NormalTok{(),}
  \AttributeTok{.connection\_string =}
    \StringTok{"driver=\{ SQLServer\};server=172.16.88.2;database=spb;uid=zhongyf;pwd=Zyf123456"}\NormalTok{, }
  \AttributeTok{timeout =} \DecValTok{10}\NormalTok{, }\AttributeTok{timezone =} \StringTok{"Asia/Shanghai"}\NormalTok{, }\AttributeTok{encoding =} \StringTok{"gbk"}
\NormalTok{)}

\CommentTok{\# linux }
\NormalTok{con\_spb }\OtherTok{\textless{}{-}} \FunctionTok{dbConnect}\NormalTok{(}\FunctionTok{odbc}\NormalTok{(),}
                     \AttributeTok{.connection\_string =}
                       \StringTok{"driver=\{ODBC Driver 17 for SQL Server\};server=172.16.88.2;database=spb;uid=zhongyf;pwd=Zyf123456"}\NormalTok{, }
                     \AttributeTok{timeout =} \DecValTok{10}\NormalTok{, }\AttributeTok{timezone =} \StringTok{"Asia/Shanghai"}\NormalTok{, }\AttributeTok{encoding =} \StringTok{"utf8"}
\NormalTok{)}
\end{Highlighting}
\end{Shaded}

\begin{itemize}
\tightlist
\item
  MySQL乱码
\end{itemize}

1.代码修改

\begin{Shaded}
\begin{Highlighting}[]
\CommentTok{\#执行查询语句前执行}
\FunctionTok{dbSendQuery}\NormalTok{(con,}\StringTok{\textquotesingle{}SET NAMES gbk\textquotesingle{}}\NormalTok{)}
\end{Highlighting}
\end{Shaded}

2.ODBC配置

如果是通过ODBC数据源连接,可通过配置需改,如下所示:

\begin{figure}
\centering
\includegraphics{./picture/chap2/pic1.png}
\caption{ODBC配置截图}
\end{figure}

\hypertarget{ux65e0ux6cd5ux8fdeux63a5ux95eeux9898}{%
\subsection{无法连接问题}\label{ux65e0ux6cd5ux8fdeux63a5ux95eeux9898}}

首先需要装mysql的驱动,确保\texttt{RMySQL}成功安装 如果是测试自己安装的mysql,可以先用Navicat连接,如果出现Authentication plugin `caching\_sha2\_password' cannot be loaded的错误。

可能是由于 mysql8 之前的版本中加密规则是mysql\_native\_password,而在mysql8之后,加密规则是caching\_sha2\_password,通过修改加密规则可解决无法连接问题。

\begin{Shaded}
\begin{Highlighting}[]

\CommentTok{{-}{-}cmd 登录本地数据}
\NormalTok{mysql }\OperatorTok{{-}}\NormalTok{u root }\OperatorTok{{-}}\NormalTok{p}
\CommentTok{{-}{-}输入密码}
\KeywordTok{password}\NormalTok{: }

\CommentTok{{-}{-}执行命令}
\KeywordTok{ALTER} \FunctionTok{USER} \StringTok{\textquotesingle{}root\textquotesingle{}}\NormalTok{@}\StringTok{\textquotesingle{}localhost\textquotesingle{}} \KeywordTok{IDENTIFIED} \KeywordTok{BY} \StringTok{\textquotesingle{}password\textquotesingle{}} \KeywordTok{PASSWORD} \KeywordTok{EXPIRE} \KeywordTok{NEVER}\NormalTok{;   \#修改加密规则 }
\CommentTok{{-}{-}{-}ALTER USER \textquotesingle{}root\textquotesingle{}@\textquotesingle{}\%\textquotesingle{} IDENTIFIED BY \textquotesingle{}password\textquotesingle{} PASSWORD EXPIRE NEVER; 看账号权限注意与上面的区别}

\KeywordTok{ALTER} \FunctionTok{USER} \StringTok{\textquotesingle{}root\textquotesingle{}}\NormalTok{@}\StringTok{\textquotesingle{}localhost\textquotesingle{}} \KeywordTok{IDENTIFIED} \KeywordTok{WITH}\NormalTok{ mysql\_native\_password }\KeywordTok{BY} \StringTok{\textquotesingle{}password\textquotesingle{}}\NormalTok{; \#更新一下用户的密码 }
\end{Highlighting}
\end{Shaded}

\hypertarget{ux8fdcux7a0bux8fdeux63a5}{%
\subsection{远程连接}\label{ux8fdcux7a0bux8fdeux63a5}}

当你需要远程连接时,需要确保数据库的远程连接已经开启。在数据库中开启某账户远程连接权限,在公司的话,数据库连接问题咨询公司的IT人员。自己个人电脑上安装的MS SQL SERVER数据库需要自行开启远程连接。

另外如果是云服务器上搭建的数据库,需要开启数据库端口,如Mysql默认端口3306;如果是阿里云的Rds数据库,找DBA管理员要数据库地址以及端口信息。

\hypertarget{dbplyr}{%
\section{dbplyr}\label{dbplyr}}

\texttt{dbplyr}将\texttt{dplyr}包的函数转化为\texttt{SQL}语句去服务器获取数据;在数据量较大、计算较多时,可以将远程连接数据库中的表当作内存中的数据框使用,当本机内存不够大时,这样做的好处不言而喻。

至于为什么使用\texttt{dbplyr}而不是直接编写\texttt{SQL},因为:

\begin{itemize}
\item
  \texttt{dbplyr}写起来简洁高效,基本跟用\texttt{dplyr}没有差别
\item
  能利用数据库所在服务器的算力,配合上并行计算,在处理大量数据时,大大加快速度。
\item
  不同数据库的语法存在差异,当源数据存在不同数据库时,用R的\texttt{dbplyr}包清洗数据时能加快效率
\item
  通过\texttt{dplyr}动词方便实现复杂的逻辑,当过程越多越复杂时\texttt{dbplyr}的优势越明显,不用一层层嵌套语句。
\end{itemize}

\hypertarget{ux57faux7840ux7528ux6cd5}{%
\subsection{基础用法}\label{ux57faux7840ux7528ux6cd5}}

\begin{Shaded}
\begin{Highlighting}[]
\FunctionTok{library}\NormalTok{(dplyr)}
\FunctionTok{library}\NormalTok{(dbplyr)}

\NormalTok{mf }\OtherTok{\textless{}{-}} \FunctionTok{memdb\_frame}\NormalTok{(}\AttributeTok{x =} \DecValTok{1}\NormalTok{, }\AttributeTok{y =} \DecValTok{2}\NormalTok{)}

\NormalTok{mf }\SpecialCharTok{\%\textgreater{}\%} 
  \FunctionTok{mutate}\NormalTok{(}
    \AttributeTok{a =}\NormalTok{ y }\SpecialCharTok{*}\NormalTok{ x, }
    \AttributeTok{b =}\NormalTok{ a }\SpecialCharTok{\^{}} \DecValTok{2}\NormalTok{,}
\NormalTok{  ) }\SpecialCharTok{\%\textgreater{}\%} 
  \FunctionTok{show\_query}\NormalTok{()}
\end{Highlighting}
\end{Shaded}

\begin{Shaded}
\begin{Highlighting}[]
\FunctionTok{library}\NormalTok{(dplyr)}
\CommentTok{\#connect database}
\NormalTok{con }\OtherTok{\textless{}{-}}\NormalTok{ DBI}\SpecialCharTok{::}\FunctionTok{dbConnect}\NormalTok{(RSQLite}\SpecialCharTok{::}\FunctionTok{SQLite}\NormalTok{(), }\AttributeTok{path =} \StringTok{":memory:"}\NormalTok{)}
\CommentTok{\# 上传数据}
\FunctionTok{copy\_to}\NormalTok{(con, nycflights13}\SpecialCharTok{::}\NormalTok{flights, }\StringTok{"flights"}\NormalTok{,}
  \AttributeTok{temporary =} \ConstantTok{FALSE}\NormalTok{, }
  \AttributeTok{indexes =} \FunctionTok{list}\NormalTok{(}
    \FunctionTok{c}\NormalTok{(}\StringTok{"year"}\NormalTok{, }\StringTok{"month"}\NormalTok{, }\StringTok{"day"}\NormalTok{), }
    \StringTok{"carrier"}\NormalTok{, }
    \StringTok{"tailnum"}\NormalTok{,}
    \StringTok{"dest"}
\NormalTok{  )}
\NormalTok{)}

\CommentTok{\# 查看库中全部表名}
\CommentTok{\#dbListTables(con)}

\CommentTok{\#tbl()引用表flights}

\NormalTok{flights\_db }\OtherTok{\textless{}{-}} \FunctionTok{tbl}\NormalTok{(con, }\StringTok{"flights"}\NormalTok{)}
\NormalTok{flights\_db}

\CommentTok{\# 开始查询}
\NormalTok{flights\_db }\SpecialCharTok{\%\textgreater{}\%} \FunctionTok{select}\NormalTok{(year}\SpecialCharTok{:}\NormalTok{day, dep\_delay, arr\_delay)}
\NormalTok{flights\_db }\SpecialCharTok{\%\textgreater{}\%} \FunctionTok{filter}\NormalTok{(dep\_delay }\SpecialCharTok{\textgreater{}} \DecValTok{240}\NormalTok{)}
\NormalTok{flights\_db }\SpecialCharTok{\%\textgreater{}\%} 
  \FunctionTok{group\_by}\NormalTok{(dest) }\SpecialCharTok{\%\textgreater{}\%}
  \FunctionTok{summarise}\NormalTok{(}\AttributeTok{delay =} \FunctionTok{mean}\NormalTok{(dep\_time))}
\end{Highlighting}
\end{Shaded}

部分简单不复杂的sql语句可以用dplyr的语法代替.

\begin{Shaded}
\begin{Highlighting}[]
\NormalTok{tailnum\_delay\_db }\OtherTok{\textless{}{-}}\NormalTok{ flights\_db }\SpecialCharTok{\%\textgreater{}\%} 
  \FunctionTok{group\_by}\NormalTok{(tailnum) }\SpecialCharTok{\%\textgreater{}\%}
  \FunctionTok{summarise}\NormalTok{(}
    \AttributeTok{delay =} \FunctionTok{mean}\NormalTok{(arr\_delay,}\AttributeTok{na.rm =}\NormalTok{ T),}
    \AttributeTok{n =} \FunctionTok{n}\NormalTok{()}
\NormalTok{  ) }\SpecialCharTok{\%\textgreater{}\%} 
  \FunctionTok{arrange}\NormalTok{(}\FunctionTok{desc}\NormalTok{(delay)) }\SpecialCharTok{\%\textgreater{}\%}
  \FunctionTok{filter}\NormalTok{(n }\SpecialCharTok{\textgreater{}} \DecValTok{100}\NormalTok{)}
\NormalTok{tailnum\_delay\_db}
\NormalTok{tailnum\_delay\_db }\SpecialCharTok{\%\textgreater{}\%} \FunctionTok{show\_query}\NormalTok{()}
\NormalTok{tailnum\_delay }\OtherTok{\textless{}{-}}\NormalTok{ tailnum\_delay\_db }\SpecialCharTok{\%\textgreater{}\%} \FunctionTok{collect}\NormalTok{() }\CommentTok{\#把数据从数据库加载到R内存中}
\end{Highlighting}
\end{Shaded}

\hypertarget{ux65e0ux6cd5ux6b63ux786eux8f6cux5316}{%
\subsection{无法正确转化}\label{ux65e0ux6cd5ux6b63ux786eux8f6cux5316}}

在使用过程中发现无法识别\texttt{lubridate}包的函数,但是\texttt{dbplyr}对于不认识的函数都将保留。

利用这个特性,可以使用数据库中原生的相关函数:如下所示,在Oracle中\texttt{to\_date}函数

以下的自定义函数可以实现按照想要\texttt{group\_by}的字段汇总金额、数量、吊牌额、折扣率等,其中关于时间周期的筛选就利用了该特性。

\begin{itemize}
\tightlist
\item
  date
\end{itemize}

\begin{Shaded}
\begin{Highlighting}[]
\CommentTok{\#个人写的争对目前公司数仓写的包中获取销售数据的一段代码}
\NormalTok{get\_sales\_data }\OtherTok{\textless{}{-}} \ControlFlowTok{function}\NormalTok{(con,...,start\_date,end\_date,brand\_name,}\AttributeTok{channel\_type =} \ConstantTok{NULL}\NormalTok{ ,}\AttributeTok{area\_name =} \ConstantTok{NULL}\NormalTok{,}\AttributeTok{boss\_name =} \ConstantTok{NULL}\NormalTok{,}\AttributeTok{category\_name =} \ConstantTok{NULL}\NormalTok{,}\AttributeTok{shop\_no =} \ConstantTok{NULL}\NormalTok{)\{}

\NormalTok{  store\_table }\OtherTok{\textless{}{-}} \FunctionTok{store}\NormalTok{(con,}\AttributeTok{brand\_name =}\NormalTok{ brand\_name,}\AttributeTok{channel\_type =}\NormalTok{ channel\_type ,}\AttributeTok{area\_name =}\NormalTok{ area\_name,}\AttributeTok{boss\_name =}\NormalTok{ boss\_name,}\AttributeTok{shop\_no =}\NormalTok{ shop\_no) }\CommentTok{\#门店信息}
  
\NormalTok{  sku\_table }\OtherTok{\textless{}{-}} \FunctionTok{sku}\NormalTok{(con,}\AttributeTok{category\_name =}\NormalTok{  category\_name ) }\CommentTok{\#商品信息}
  
  \FunctionTok{tbl}\NormalTok{(con, }\FunctionTok{in\_schema}\NormalTok{(}\StringTok{"DW"}\NormalTok{, }\StringTok{"DW\_SALE\_SHOP\_F"}\NormalTok{)) }\SpecialCharTok{\%\textgreater{}\%} \CommentTok{\#DW层}
    \FunctionTok{select}\NormalTok{(BILL\_DATE1, SKU\_NO, SHOP\_NO, BILL\_QTY, BILL\_MONEY2, PRICE) }\SpecialCharTok{\%\textgreater{}\%}
    \FunctionTok{filter}\NormalTok{(}\FunctionTok{between}\NormalTok{(}
\NormalTok{      BILL\_DATE1, }\FunctionTok{to\_date}\NormalTok{(start\_date, }\StringTok{"yyyy{-}mm{-}dd"}\NormalTok{),}
      \FunctionTok{to\_date}\NormalTok{(end\_date, }\StringTok{"yyyy{-}mm{-}dd"}\NormalTok{)}
\NormalTok{    )) }\SpecialCharTok{\%\textgreater{}\%}
    \FunctionTok{mutate}\NormalTok{(年 }\OtherTok{=} \FunctionTok{year}\NormalTok{(BILL\_DATE1), 月 }\OtherTok{=} \FunctionTok{month}\NormalTok{(BILL\_DATE1)) }\SpecialCharTok{\%\textgreater{}\%}
    \FunctionTok{inner\_join}\NormalTok{(store\_table) }\SpecialCharTok{\%\textgreater{}\%}
    \FunctionTok{inner\_join}\NormalTok{(sku\_table) }\SpecialCharTok{\%\textgreater{}\%}
    \FunctionTok{group\_by}\NormalTok{(...) }\SpecialCharTok{\%\textgreater{}\%}
    \FunctionTok{summarise}\NormalTok{(}
\NormalTok{      金额 }\OtherTok{=} \FunctionTok{sum}\NormalTok{(BILL\_MONEY2, }\AttributeTok{na.rm =} \ConstantTok{TRUE}\NormalTok{),}
\NormalTok{      数量 }\OtherTok{=} \FunctionTok{sum}\NormalTok{(BILL\_QTY, }\AttributeTok{na.rm =} \ConstantTok{TRUE}\NormalTok{),}
\NormalTok{      吊牌金额 }\OtherTok{=} \FunctionTok{sum}\NormalTok{(BILL\_QTY }\SpecialCharTok{*}\NormalTok{ PRICE, }\AttributeTok{na.rm =} \ConstantTok{TRUE}\NormalTok{)) }\SpecialCharTok{\%\textgreater{}\%}
    \FunctionTok{collect}\NormalTok{() }\SpecialCharTok{\%\textgreater{}\%}
    \FunctionTok{mutate}\NormalTok{(折扣率}\SpecialCharTok{:}\ErrorTok{=}\NormalTok{ 金额 }\SpecialCharTok{/}\NormalTok{ 吊牌金额) }\SpecialCharTok{\%\textgreater{}\%} 
    \FunctionTok{arrange}\NormalTok{(...)}


  \CommentTok{\# return(res)}
\NormalTok{\}}
\end{Highlighting}
\end{Shaded}

\begin{itemize}
\tightlist
\item
  like
\end{itemize}

\begin{Shaded}
\begin{Highlighting}[]
\NormalTok{mf }\SpecialCharTok{\%\textgreater{}\%} 
  \FunctionTok{filter}\NormalTok{(x }\SpecialCharTok{\%LIKE\%} \StringTok{"\%foo\%"}\NormalTok{) }\SpecialCharTok{\%\textgreater{}\%} 
  \FunctionTok{show\_query}\NormalTok{()}
\end{Highlighting}
\end{Shaded}

\begin{itemize}
\tightlist
\item
  特殊用法
\end{itemize}

特殊情况可以使用\texttt{sql()}函数

\begin{Shaded}
\begin{Highlighting}[]
\NormalTok{mf }\SpecialCharTok{\%\textgreater{}\%} 
  \FunctionTok{transmute}\NormalTok{(}\AttributeTok{factorial =} \FunctionTok{sql}\NormalTok{(}\StringTok{"x!"}\NormalTok{)) }\SpecialCharTok{\%\textgreater{}\%} 
  \FunctionTok{show\_query}\NormalTok{()}
\end{Highlighting}
\end{Shaded}

\hypertarget{ux53c2ux8003ux8d44ux6599-2}{%
\section{参考资料}\label{ux53c2ux8003ux8d44ux6599-2}}

\texttt{DBI}包资料\url{https://dbi.r-dbi.org/reference/}

\texttt{dbplyr}包资料\url{https://dbplyr.tidyverse.org/}

rstudio关于数据库介绍 \url{https://db.rstudio.com/databases}

数据库连接字符串介绍 \url{https://www.connectionstrings.com/}

个人博客关于Roracle的安装介绍 \url{http://www.zhongyufei.com/2020/07/25/roracle-install/}

\url{https://www.r-consortium.org/blog/2017/05/15/improving-dbi-a-retrospect}

\hypertarget{loop-structure}{%
\chapter{Loop structure}\label{loop-structure}}

实际场景中,当需要重复做某动作时,可运用循环结构。

\hypertarget{ux7b80ux5355ux793aux4f8b}{%
\section{简单示例}\label{ux7b80ux5355ux793aux4f8b}}

利用循环实现1到100连续相加求和

\begin{Shaded}
\begin{Highlighting}[]
\NormalTok{total }\OtherTok{\textless{}{-}} \DecValTok{0}
\ControlFlowTok{for}\NormalTok{(i }\ControlFlowTok{in} \DecValTok{1}\SpecialCharTok{:}\DecValTok{100}\NormalTok{)\{}
\NormalTok{  total }\OtherTok{\textless{}{-}}\NormalTok{ total}\SpecialCharTok{+}\NormalTok{i}
\NormalTok{\}}
\FunctionTok{print}\NormalTok{(}\FunctionTok{paste0}\NormalTok{(}\StringTok{\textquotesingle{}1到100连续相加求和等于:\textquotesingle{}}\NormalTok{,total))}

\CommentTok{\# loop structure}
\CommentTok{\# for (var in seq) \{expr\}}
\end{Highlighting}
\end{Shaded}

\hypertarget{ux5faaux73afux57faux7840}{%
\section{循环基础}\label{ux5faaux73afux57faux7840}}

\hypertarget{ux5faaux73afux7ed3ux6784}{%
\subsection{循环结构}\label{ux5faaux73afux7ed3ux6784}}

R中有三种循环结构:

\begin{itemize}
\tightlist
\item
  Repeat
\end{itemize}

\begin{Shaded}
\begin{Highlighting}[]
\NormalTok{i }\OtherTok{\textless{}{-}} \DecValTok{1}
\NormalTok{total }\OtherTok{\textless{}{-}} \DecValTok{0}
\ControlFlowTok{repeat}\NormalTok{\{}
\NormalTok{  total }\OtherTok{\textless{}{-}}\NormalTok{ total}\SpecialCharTok{+}\NormalTok{i}
\NormalTok{  i }\OtherTok{\textless{}{-}}\NormalTok{ i}\SpecialCharTok{+}\DecValTok{1}
  \ControlFlowTok{if}\NormalTok{(i }\SpecialCharTok{\textgreater{}} \DecValTok{100}\NormalTok{)\{}
    \FunctionTok{print}\NormalTok{(}\FunctionTok{paste0}\NormalTok{(}\StringTok{\textquotesingle{}连续相加求和等于:\textquotesingle{}}\NormalTok{,total))}
    \ControlFlowTok{break}
\NormalTok{  \}}
\NormalTok{\}}
\end{Highlighting}
\end{Shaded}

\begin{itemize}
\tightlist
\item
  while
\end{itemize}

\begin{Shaded}
\begin{Highlighting}[]
\NormalTok{i }\OtherTok{\textless{}{-}} \DecValTok{1}
\NormalTok{total }\OtherTok{\textless{}{-}} \DecValTok{0}
\ControlFlowTok{while}\NormalTok{(i }\SpecialCharTok{\textless{}=} \DecValTok{1000}\NormalTok{)\{}
\NormalTok{  total }\OtherTok{\textless{}{-}}\NormalTok{ total}\SpecialCharTok{+}\NormalTok{i}
\NormalTok{  i }\OtherTok{\textless{}{-}}\NormalTok{ i}\SpecialCharTok{+}\DecValTok{1}
\NormalTok{\}}
\FunctionTok{print}\NormalTok{(}\FunctionTok{paste0}\NormalTok{(}\StringTok{\textquotesingle{}1到1000连续相加求和等于:\textquotesingle{}}\NormalTok{,total))}
\CommentTok{\# not run}
\CommentTok{\# sum(1:1000)}
\end{Highlighting}
\end{Shaded}

\begin{itemize}
\tightlist
\item
  for
\end{itemize}

代码如示例所示

\begin{Shaded}
\begin{Highlighting}[]
\FunctionTok{library}\NormalTok{(tidyverse)}
\NormalTok{df }\OtherTok{\textless{}{-}} \FunctionTok{tibble}\NormalTok{(}
  \AttributeTok{a =} \FunctionTok{rnorm}\NormalTok{(}\DecValTok{10}\NormalTok{),}
  \AttributeTok{b =} \FunctionTok{rnorm}\NormalTok{(}\DecValTok{10}\NormalTok{),}
  \AttributeTok{c =} \FunctionTok{rnorm}\NormalTok{(}\DecValTok{10}\NormalTok{),}
  \AttributeTok{d =} \FunctionTok{rnorm}\NormalTok{(}\DecValTok{10}\NormalTok{)}
\NormalTok{)}

\NormalTok{output }\OtherTok{\textless{}{-}} \FunctionTok{vector}\NormalTok{(}\StringTok{"double"}\NormalTok{, }\FunctionTok{ncol}\NormalTok{(df))  }\CommentTok{\# 1. output}
\ControlFlowTok{for}\NormalTok{ (i }\ControlFlowTok{in} \FunctionTok{seq\_along}\NormalTok{(df)) \{            }\CommentTok{\# 2. sequence}
\NormalTok{  output[[i]] }\OtherTok{\textless{}{-}} \FunctionTok{median}\NormalTok{(df[[i]])      }\CommentTok{\# 3. body}
\NormalTok{\}}
\NormalTok{output}
\end{Highlighting}
\end{Shaded}

循环中尽可能利用R中的向量化,比如指定output的长度,当数据量大的时候效率提升将比较明显,养成向量化的意识对提高代码效率有显著效果.

上面代码中 \texttt{vector}函数创建一个空向量带指定长度,有两个参数,第一个时向量类型(`逻辑',`整数',`双精度','字符'等),第二个是向量长度 \texttt{vector(length=5)},类型默认是逻辑型.

\texttt{seq\_along}可以\texttt{?seq}查看用法.

hadely 解释如下:

You might not have seen seq\_along() before. It's a safe version of the familiar 1:length(l), with an important difference: if you have a zero-length vector, seq\_along() does the right thing:

\begin{Shaded}
\begin{Highlighting}[]
\CommentTok{\#wrong}
\FunctionTok{seq\_along}\NormalTok{(}\FunctionTok{c}\NormalTok{())}
\DecValTok{1}\SpecialCharTok{:}\FunctionTok{length}\NormalTok{(}\FunctionTok{c}\NormalTok{())}

\CommentTok{\# generates the integer sequence 1, 2, ..., length(along.with). (along.with is usually abbreviated to along, and  seq\_along is much faster.)}
\end{Highlighting}
\end{Shaded}

\hypertarget{next-break-ux7528ux6cd5}{%
\subsection{next break 用法}\label{next-break-ux7528ux6cd5}}

\begin{itemize}
\tightlist
\item
  next 用法
\end{itemize}

\begin{Shaded}
\begin{Highlighting}[]
\ControlFlowTok{for}\NormalTok{(i }\ControlFlowTok{in}\NormalTok{ letters[}\DecValTok{1}\SpecialCharTok{:}\DecValTok{6}\NormalTok{] )\{}
  \ControlFlowTok{if}\NormalTok{(i }\SpecialCharTok{==} \StringTok{"d"}\NormalTok{)\{}
  \ControlFlowTok{next}
\NormalTok{  \}}
  \FunctionTok{print}\NormalTok{(i)}
\NormalTok{\}}
\end{Highlighting}
\end{Shaded}

\begin{itemize}
\tightlist
\item
  break 用法
\end{itemize}

可以当条件满足时跳出循环常常与repeat循环结构配合使用。

\hypertarget{ux5d4cux5957ux5faaux73af}{%
\subsection{嵌套循环}\label{ux5d4cux5957ux5faaux73af}}

\begin{Shaded}
\begin{Highlighting}[]
\CommentTok{\# not run}
\NormalTok{v }\OtherTok{\textless{}{-}} \FunctionTok{vector}\NormalTok{(}\AttributeTok{length =} \DecValTok{100}\NormalTok{)}
\ControlFlowTok{for}\NormalTok{(i }\ControlFlowTok{in} \DecValTok{1}\SpecialCharTok{:}\DecValTok{10}\NormalTok{)\{}
  \ControlFlowTok{for}\NormalTok{(j }\ControlFlowTok{in} \DecValTok{1}\SpecialCharTok{:}\DecValTok{10}\NormalTok{)\{}
\NormalTok{    v[i}\SpecialCharTok{*}\NormalTok{j] }\OtherTok{=}\NormalTok{ i }\SpecialCharTok{*}\NormalTok{ j }
\NormalTok{  \}}
\NormalTok{\}}
\end{Highlighting}
\end{Shaded}

\hypertarget{ux5faaux73afux53d8ux5316}{%
\section{循环变化}\label{ux5faaux73afux53d8ux5316}}

\hypertarget{ux4feeux6539ux5df2ux6709ux5bf9ux8c61}{%
\subsection{修改已有对象}\label{ux4feeux6539ux5df2ux6709ux5bf9ux8c61}}

\begin{Shaded}
\begin{Highlighting}[]
\NormalTok{res }\OtherTok{\textless{}{-}} \DecValTok{1}\SpecialCharTok{:}\DecValTok{100}
\ControlFlowTok{for}\NormalTok{(i }\ControlFlowTok{in} \FunctionTok{seq\_along}\NormalTok{(res))\{}
\NormalTok{  res[i] }\OtherTok{\textless{}{-}}\NormalTok{ res[i] }\SpecialCharTok{*}\NormalTok{ i}
\NormalTok{\}}
\FunctionTok{str}\NormalTok{(res)}
\end{Highlighting}
\end{Shaded}

\hypertarget{ux5faaux73afux6a21ux5f0f}{%
\subsection{循环模式}\label{ux5faaux73afux6a21ux5f0f}}

共有三种遍历向量的方法,之前展示的都是遍历数字索引\texttt{for\ (i\ in\ seq\_along(xs))},并使用提取值\texttt{x{[}{[}i{]}{]}}.还有两种方式:

\begin{itemize}
\tightlist
\item
  循环遍历元素
\end{itemize}

\texttt{for(i\ in\ xs)},例如我们需要保存文件时,可以利用这种循环模式

\begin{itemize}
\tightlist
\item
  遍历名称
\end{itemize}

\texttt{for\ (nm\ in\ names(xs))},我们可以使用\texttt{x{[}{[}nm{]}{]}} 该名称访问.当我们要在文件名中使用名称时会比较方便.

\begin{Shaded}
\begin{Highlighting}[]
\NormalTok{results }\OtherTok{\textless{}{-}} \FunctionTok{vector}\NormalTok{(}\StringTok{"list"}\NormalTok{, }\FunctionTok{length}\NormalTok{(x))}
\FunctionTok{names}\NormalTok{(results) }\OtherTok{\textless{}{-}} \FunctionTok{names}\NormalTok{(x)}
\end{Highlighting}
\end{Shaded}

数字索引的循环模式最常用,因为可以根据位置提取名称和值.

\begin{Shaded}
\begin{Highlighting}[]
\ControlFlowTok{for}\NormalTok{ (i }\ControlFlowTok{in} \FunctionTok{seq\_along}\NormalTok{(x)) \{}
\NormalTok{  name }\OtherTok{\textless{}{-}} \FunctionTok{names}\NormalTok{(x)[[i]]}
\NormalTok{  value }\OtherTok{\textless{}{-}}\NormalTok{ x[[i]]}
\NormalTok{\}}
\end{Highlighting}
\end{Shaded}

\hypertarget{ux672aux77e5ux957fux5ea6ux8f93ux51fa}{%
\subsection{未知长度输出}\label{ux672aux77e5ux957fux5ea6ux8f93ux51fa}}

有时候我们的循环我们不确定输出的长度是多少.这样会逐步增加向量的长度,如下所示:

\begin{Shaded}
\begin{Highlighting}[]
\NormalTok{means }\OtherTok{\textless{}{-}} \FunctionTok{c}\NormalTok{(}\DecValTok{0}\NormalTok{, }\DecValTok{1}\NormalTok{, }\DecValTok{2}\NormalTok{)}

\NormalTok{output }\OtherTok{\textless{}{-}} \FunctionTok{double}\NormalTok{()}
\ControlFlowTok{for}\NormalTok{ (i }\ControlFlowTok{in} \FunctionTok{seq\_along}\NormalTok{(means)) \{}
\NormalTok{  n }\OtherTok{\textless{}{-}} \FunctionTok{sample}\NormalTok{(}\DecValTok{100}\NormalTok{, }\DecValTok{1}\NormalTok{)}
\NormalTok{  output }\OtherTok{\textless{}{-}} \FunctionTok{c}\NormalTok{(output, }\FunctionTok{rnorm}\NormalTok{(n, means[[i]]))}
\NormalTok{\}}
\FunctionTok{str}\NormalTok{(output)}
\end{Highlighting}
\end{Shaded}

但是这种方式浪费时间,当数据量大时候效率会很低下.因为时间复杂度为(\(O(n^2)\)).解决方案是将结果保存在列表中,然后在完成循环后合并为单个向量:

\begin{Shaded}
\begin{Highlighting}[]
\NormalTok{out }\OtherTok{\textless{}{-}} \FunctionTok{vector}\NormalTok{(}\StringTok{"list"}\NormalTok{, }\FunctionTok{length}\NormalTok{(means))}
\ControlFlowTok{for}\NormalTok{ (i }\ControlFlowTok{in} \FunctionTok{seq\_along}\NormalTok{(means)) \{}
\NormalTok{  n }\OtherTok{\textless{}{-}} \FunctionTok{sample}\NormalTok{(}\DecValTok{100}\NormalTok{, }\DecValTok{1}\NormalTok{)}
\NormalTok{  out[[i]] }\OtherTok{\textless{}{-}} \FunctionTok{rnorm}\NormalTok{(n, means[[i]])}
\NormalTok{\}}
\FunctionTok{str}\NormalTok{(out)}
\FunctionTok{str}\NormalTok{(}\FunctionTok{unlist}\NormalTok{(out)) }\CommentTok{\#unlist将列表向量化}
\end{Highlighting}
\end{Shaded}

\hypertarget{iteration}{%
\chapter{Iteration}\label{iteration}}

常常需要重复操作同样的功能函数,这时可以用迭代来实现。purrr包提供了一套完整的函数来处理循环迭代,可以有效减少重复性工作和代码。

\url{https://purrr.tidyverse.org/}

\hypertarget{ux7b80ux5355ux7528ux6cd5}{%
\section{简单用法}\label{ux7b80ux5355ux7528ux6cd5}}

\begin{itemize}
\tightlist
\item
  map
\end{itemize}

用map循环迭代,map函数始终返回list对象。

\begin{Shaded}
\begin{Highlighting}[]
\FunctionTok{library}\NormalTok{(tidyverse)}

\CommentTok{\# define function}
\NormalTok{addTen }\OtherTok{\textless{}{-}} \ControlFlowTok{function}\NormalTok{(.x) \{}
  \FunctionTok{return}\NormalTok{(.x }\SpecialCharTok{+} \DecValTok{10}\NormalTok{)}
\NormalTok{\}}

\FunctionTok{map}\NormalTok{(}\AttributeTok{.x =} \FunctionTok{c}\NormalTok{(}\DecValTok{1}\NormalTok{, }\DecValTok{4}\NormalTok{, }\DecValTok{7}\NormalTok{), }\AttributeTok{.f =}\NormalTok{ addTen)}
\CommentTok{\# not run}
\CommentTok{\# map(c(1, 4, 7), addTen) \# same above}
\end{Highlighting}
\end{Shaded}

\begin{itemize}
\tightlist
\item
  map\_dbl
\end{itemize}

用map\_dbl循环迭代,map\_dbl函数返回vector。

\begin{Shaded}
\begin{Highlighting}[]
\CommentTok{\#library(purrr)}
\NormalTok{add1 }\OtherTok{\textless{}{-}} \ControlFlowTok{function}\NormalTok{(x) \{}
\NormalTok{  (x}\SpecialCharTok{+}\DecValTok{1}\NormalTok{)}\SpecialCharTok{*}\NormalTok{x}
\NormalTok{\}}
\NormalTok{result1 }\OtherTok{\textless{}{-}} \FunctionTok{map\_dbl}\NormalTok{(}\DecValTok{1}\SpecialCharTok{:}\DecValTok{1000}\NormalTok{,add1) }\CommentTok{\# maP\_dbl 输出结果为向量}

\CommentTok{\#for版本}
\NormalTok{result2 }\OtherTok{\textless{}{-}} \FunctionTok{vector}\NormalTok{(}\AttributeTok{length =} \DecValTok{1000}\NormalTok{)}
\ControlFlowTok{for}\NormalTok{(i }\ControlFlowTok{in} \DecValTok{1}\SpecialCharTok{:}\DecValTok{1000}\NormalTok{)\{}
\NormalTok{  result2[i] }\OtherTok{\textless{}{-}}\NormalTok{ (i}\SpecialCharTok{+}\DecValTok{1}\NormalTok{) }\SpecialCharTok{*}\NormalTok{ i}
\NormalTok{\}}
\CommentTok{\# test }
\CommentTok{\#not run}
\CommentTok{\#table(result1 == result2)}
\CommentTok{\# all equal}
\FunctionTok{identical}\NormalTok{(result1,result2)}
\end{Highlighting}
\end{Shaded}

\hypertarget{mapux7cfbux5217ux5e38ux7528ux51fdux6570}{%
\section{map系列常用函数}\label{mapux7cfbux5217ux5e38ux7528ux51fdux6570}}

\begin{itemize}
\tightlist
\item
  map\_chr
\end{itemize}

\texttt{map\_chr(.x,\ .f)} ,map\_chr 返回对象为字符串

\begin{itemize}
\tightlist
\item
  map\_dbl
\end{itemize}

\texttt{map\_dbl(.x,\ .f)} ,map\_dbl 返回数字向量(双精度)

\begin{itemize}
\tightlist
\item
  map\_df
\end{itemize}

\texttt{map\_df(.x,\ .f)},map\_df 返回对象为数据框,类似函数 \texttt{map\_dfr(.x,.f)},\texttt{map\_dfc(.x,.f)}

\begin{itemize}
\tightlist
\item
  map\_gl
\end{itemize}

\texttt{map\_lgl(.x,\ .f)} 返回逻辑向量

\begin{itemize}
\tightlist
\item
  map\_int
\end{itemize}

\texttt{map\_int(.x,\ .f,\ ...)} 返回整数

map\_df()函数示例

\begin{Shaded}
\begin{Highlighting}[]
\CommentTok{\# 采用匿名函数}
\FunctionTok{map\_df}\NormalTok{(}\FunctionTok{c}\NormalTok{(}\DecValTok{1}\NormalTok{, }\DecValTok{4}\NormalTok{, }\DecValTok{7}\NormalTok{), }\ControlFlowTok{function}\NormalTok{(.x) \{}
  \FunctionTok{return}\NormalTok{(}\FunctionTok{data.frame}\NormalTok{(}\AttributeTok{old\_number =}\NormalTok{ .x, }
                    \AttributeTok{new\_number =} \FunctionTok{addTen}\NormalTok{(.x)))}
\NormalTok{\})}

\CommentTok{\#同上}
\CommentTok{\#step1 定义函数}
\NormalTok{make\_dataframe }\OtherTok{\textless{}{-}} \ControlFlowTok{function}\NormalTok{(x)\{}
  \FunctionTok{data.frame}\NormalTok{(}\AttributeTok{old\_number =}\NormalTok{ x,}\AttributeTok{new\_number =} \FunctionTok{addTen}\NormalTok{(x))}
\NormalTok{\}}
\CommentTok{\#step2 计算}
\FunctionTok{map\_df}\NormalTok{(}\FunctionTok{c}\NormalTok{(}\DecValTok{1}\NormalTok{,}\DecValTok{4}\NormalTok{,}\DecValTok{7}\NormalTok{),make\_dataframe)}
\end{Highlighting}
\end{Shaded}

\hypertarget{ux5f52ux7ea6ux7d2fux8ba1ux51fdux6570}{%
\section{归约累计函数}\label{ux5f52ux7ea6ux7d2fux8ba1ux51fdux6570}}

reduce、accumulate()函数用法介绍.

\begin{itemize}
\tightlist
\item
  reduce
\end{itemize}

在实际工作中,我长用reduce函数实现merge()功能。示例如下:

\begin{Shaded}
\begin{Highlighting}[]
\FunctionTok{reduce}\NormalTok{(}\DecValTok{1}\SpecialCharTok{:}\DecValTok{100}\NormalTok{,}\StringTok{\textasciigrave{}}\AttributeTok{+}\StringTok{\textasciigrave{}}\NormalTok{)}
\FunctionTok{reduce}\NormalTok{(}\DecValTok{100}\SpecialCharTok{:}\DecValTok{1}\NormalTok{,}\StringTok{\textasciigrave{}}\AttributeTok{{-}}\StringTok{\textasciigrave{}}\NormalTok{)}
\end{Highlighting}
\end{Shaded}

将函数功能不断运用到list上得到最后结果。

\begin{Shaded}
\begin{Highlighting}[]
\NormalTok{n }\OtherTok{\textless{}{-}} \DecValTok{10}
\NormalTok{dt1 }\OtherTok{\textless{}{-}} \FunctionTok{data.frame}\NormalTok{(}\AttributeTok{a=}\NormalTok{letters[n],}\AttributeTok{b1=}\FunctionTok{rnorm}\NormalTok{(n))}
\NormalTok{dt2 }\OtherTok{\textless{}{-}} \FunctionTok{data.frame}\NormalTok{(}\AttributeTok{a=}\NormalTok{letters[n],}\AttributeTok{b2=}\FunctionTok{rnorm}\NormalTok{(n))}
\NormalTok{dt3 }\OtherTok{\textless{}{-}} \FunctionTok{data.frame}\NormalTok{(}\AttributeTok{a=}\NormalTok{letters[n],}\AttributeTok{b3=}\FunctionTok{rnorm}\NormalTok{(n))}
\NormalTok{dt4 }\OtherTok{\textless{}{-}} \FunctionTok{data.frame}\NormalTok{(}\AttributeTok{a=}\NormalTok{letters[n],}\AttributeTok{b4=}\FunctionTok{rnorm}\NormalTok{(n))}

\FunctionTok{reduce}\NormalTok{(}\FunctionTok{list}\NormalTok{(dt1,dt2,dt3,dt4),merge)}
\CommentTok{\# not run}
\CommentTok{\# reduce(list(dt1,dt2,dt3,dt4),merge,by=\textquotesingle{}a\textquotesingle{}) same above}
\end{Highlighting}
\end{Shaded}

\begin{itemize}
\tightlist
\item
  accumulate
\end{itemize}

\begin{Shaded}
\begin{Highlighting}[]
\DecValTok{1}\SpecialCharTok{:}\DecValTok{5} \SpecialCharTok{\%\textgreater{}\%} \FunctionTok{accumulate}\NormalTok{(}\StringTok{\textasciigrave{}}\AttributeTok{+}\StringTok{\textasciigrave{}}\NormalTok{)}
\FunctionTok{accumulate}\NormalTok{(letters[}\DecValTok{1}\SpecialCharTok{:}\DecValTok{5}\NormalTok{], paste, }\AttributeTok{sep =} \StringTok{"."}\NormalTok{)}
\end{Highlighting}
\end{Shaded}

\hypertarget{ux5b89ux5168ux51fdux6570}{%
\section{安全函数}\label{ux5b89ux5168ux51fdux6570}}

possibly() 和 safely(),当循环时候遇到错误报错导致整个程序停止,这不是我们想要的。

\begin{Shaded}
\begin{Highlighting}[]
\NormalTok{l }\OtherTok{\textless{}{-}} \FunctionTok{list}\NormalTok{(}\DecValTok{1}\NormalTok{,}\DecValTok{2}\NormalTok{,}\DecValTok{3}\NormalTok{,}\DecValTok{4}\NormalTok{,}\StringTok{\textquotesingle{}5\textquotesingle{}}\NormalTok{)}
\FunctionTok{map}\NormalTok{(l,}\ControlFlowTok{function}\NormalTok{(.x) .x}\SpecialCharTok{+}\DecValTok{1}\NormalTok{)}
\end{Highlighting}
\end{Shaded}

以上程序将会报错,不能正确得到结果。

\begin{Shaded}
\begin{Highlighting}[]
\NormalTok{l }\OtherTok{\textless{}{-}} \FunctionTok{list}\NormalTok{(}\DecValTok{1}\NormalTok{,}\DecValTok{2}\NormalTok{,}\DecValTok{3}\NormalTok{,}\DecValTok{4}\NormalTok{,}\StringTok{\textquotesingle{}5\textquotesingle{}}\NormalTok{)}
\NormalTok{test\_fun }\OtherTok{\textless{}{-}} \FunctionTok{safely}\NormalTok{(}\ControlFlowTok{function}\NormalTok{(.x) .x}\SpecialCharTok{+}\DecValTok{1}\NormalTok{)}
\FunctionTok{map}\NormalTok{(l,test\_fun)}
\end{Highlighting}
\end{Shaded}

用safely()函数将原始function包裹起来,即使执行过程中遇到错误也可以完成整个任务,不会因为中途报错停止,在大型循环过程中,如爬虫过程中比较实用。

\hypertarget{ux6620ux5c04ux591aux4e2aux53c2ux6570}{%
\section{映射多个参数}\label{ux6620ux5c04ux591aux4e2aux53c2ux6570}}

map2 和 pmap 函数可以映射两个及以上参数。

\begin{Shaded}
\begin{Highlighting}[]
\NormalTok{li1 }\OtherTok{\textless{}{-}} \FunctionTok{list}\NormalTok{(}\DecValTok{1}\NormalTok{,}\DecValTok{3}\NormalTok{,}\DecValTok{5}\NormalTok{)}
\NormalTok{li2 }\OtherTok{\textless{}{-}} \FunctionTok{list}\NormalTok{(}\DecValTok{2}\NormalTok{,}\DecValTok{4}\NormalTok{,}\DecValTok{6}\NormalTok{)}
\FunctionTok{map2}\NormalTok{(li1,li2,}\StringTok{\textasciigrave{}}\AttributeTok{+}\StringTok{\textasciigrave{}}\NormalTok{)}
\end{Highlighting}
\end{Shaded}

类似函数 map2\_dbl,map2\_chr,map2\_dfr等等。

\begin{Shaded}
\begin{Highlighting}[]
\NormalTok{li1 }\OtherTok{\textless{}{-}} \FunctionTok{list}\NormalTok{(}\DecValTok{1}\NormalTok{,}\DecValTok{3}\NormalTok{,}\DecValTok{5}\NormalTok{)}
\NormalTok{li2 }\OtherTok{\textless{}{-}} \FunctionTok{list}\NormalTok{(}\DecValTok{2}\NormalTok{,}\DecValTok{4}\NormalTok{,}\DecValTok{6}\NormalTok{)}
\NormalTok{li3 }\OtherTok{\textless{}{-}} \FunctionTok{list}\NormalTok{(}\DecValTok{2}\NormalTok{,}\DecValTok{4}\NormalTok{,}\DecValTok{6}\NormalTok{)}
\NormalTok{li1 }\OtherTok{\textless{}{-}} \FunctionTok{c}\NormalTok{(}\DecValTok{1}\NormalTok{,}\DecValTok{3}\NormalTok{,}\DecValTok{5}\NormalTok{)}
\NormalTok{li2 }\OtherTok{\textless{}{-}} \FunctionTok{c}\NormalTok{(}\DecValTok{2}\NormalTok{,}\DecValTok{4}\NormalTok{,}\DecValTok{6}\NormalTok{)}
\NormalTok{li3 }\OtherTok{\textless{}{-}} \FunctionTok{c}\NormalTok{(}\DecValTok{2}\NormalTok{,}\DecValTok{3}\NormalTok{,}\DecValTok{4}\NormalTok{)}
\NormalTok{li }\OtherTok{\textless{}{-}} \FunctionTok{list}\NormalTok{(li1,li2,li3)}
\FunctionTok{pmap}\NormalTok{(li,sum)}
\end{Highlighting}
\end{Shaded}

同上有pmap\_int,pmap\_dbl,pmap\_dfr等函数。

\hypertarget{ux5176ux4ed6ux51fdux6570ux4ecbux7ecd}{%
\section{其他函数介绍}\label{ux5176ux4ed6ux51fdux6570ux4ecbux7ecd}}

\begin{itemize}
\tightlist
\item
  flatten
\end{itemize}

flatten()系列函数可以将列表输出为稳定类型。purrr package 自带Examples。

\begin{Shaded}
\begin{Highlighting}[]
\NormalTok{x }\OtherTok{\textless{}{-}} \FunctionTok{rerun}\NormalTok{(}\DecValTok{2}\NormalTok{, }\FunctionTok{sample}\NormalTok{(}\DecValTok{4}\NormalTok{))}
\NormalTok{x}
\NormalTok{x }\SpecialCharTok{\%\textgreater{}\%} \FunctionTok{flatten}\NormalTok{()}
\NormalTok{x }\SpecialCharTok{\%\textgreater{}\%} \FunctionTok{flatten\_int}\NormalTok{()}
\CommentTok{\# You can use flatten in conjunction with map}
\NormalTok{x }\SpecialCharTok{\%\textgreater{}\%} \FunctionTok{map}\NormalTok{(1L) }\SpecialCharTok{\%\textgreater{}\%} \FunctionTok{flatten\_int}\NormalTok{()}
\CommentTok{\# But it\textquotesingle{}s more efficient to use the typed map instead.}
\NormalTok{x }\SpecialCharTok{\%\textgreater{}\%} \FunctionTok{map\_int}\NormalTok{(1L)}
\end{Highlighting}
\end{Shaded}

\begin{itemize}
\tightlist
\item
  imap
\end{itemize}

imap()系列函数官方描述:

imap\_xxx(x, \ldots), an indexed map, is short hand for map2(x, names(x), \ldots) if x has names, or map2(x, seq\_along(x), \ldots) if it does not. This is useful if you need to compute on both the value and the position of an element.

imap,当x有names(x)或者seq\_along(x)属性,imap是map2的另一种表达方式。

使用公式快捷方式时,第一个参数是值(.x),第二个参数是位置/名称(.y)。

详情请查看:?imap

示例1:

\begin{Shaded}
\begin{Highlighting}[]
\FunctionTok{imap\_chr}\NormalTok{(}\FunctionTok{sample}\NormalTok{(}\DecValTok{10}\NormalTok{), }\SpecialCharTok{\textasciitilde{}} \FunctionTok{paste0}\NormalTok{(.y, }\StringTok{": "}\NormalTok{, .x))}
\end{Highlighting}
\end{Shaded}

sample(10),没有names(),只有长度信息。转化成map2表达如下:

\begin{Shaded}
\begin{Highlighting}[]
\CommentTok{\#same above}

\FunctionTok{map2\_chr}\NormalTok{(}\FunctionTok{sample}\NormalTok{(}\DecValTok{10}\NormalTok{),}\DecValTok{1}\SpecialCharTok{:}\DecValTok{10}\NormalTok{,}\SpecialCharTok{\textasciitilde{}}\FunctionTok{paste0}\NormalTok{(.y,}\StringTok{": "}\NormalTok{,.x)) }\CommentTok{\# 第二个list 为位置信息.}
\end{Highlighting}
\end{Shaded}

\hypertarget{define-function}{%
\chapter{define function}\label{define-function}}

函数功能使我们尽可能避免复制粘贴代码,而且需要更改的时候不需要大面积修改代码仅需要调整函数参数,使代码整体更加模块化.

假设有工作任务需要给商品SKU排名,在代码中需要重复以下代码5次,当区间需要修改的时候就是灾难.

原始代码示例如下:

\begin{Shaded}
\begin{Highlighting}[]
\FunctionTok{library}\NormalTok{(tidyverse)}
\NormalTok{num }\OtherTok{\textless{}{-}} \FunctionTok{sample}\NormalTok{(}\DecValTok{1}\SpecialCharTok{:}\DecValTok{1000}\NormalTok{,}\DecValTok{1000}\NormalTok{)}
\NormalTok{res1 }\OtherTok{\textless{}{-}} \FunctionTok{if\_else}\NormalTok{(num }\SpecialCharTok{\textless{}=} \DecValTok{50}\NormalTok{,}\StringTok{"1{-}50"}\NormalTok{,}
                \FunctionTok{if\_else}\NormalTok{(num }\SpecialCharTok{\textless{}=} \DecValTok{100}\NormalTok{,}\StringTok{"51{-}100"}\NormalTok{,}
                        \FunctionTok{if\_else}\NormalTok{(num }\SpecialCharTok{\textless{}=} \DecValTok{150}\NormalTok{,}\StringTok{"101{-}150"}\NormalTok{,}
                                \FunctionTok{if\_else}\NormalTok{(num }\SpecialCharTok{\textless{}=} \DecValTok{200}\NormalTok{ ,}\StringTok{"151{-}200"}\NormalTok{,}
                                        \FunctionTok{if\_else}\NormalTok{(num }\SpecialCharTok{\textgreater{}}\DecValTok{200}\NormalTok{,}\StringTok{"200以上"}\NormalTok{,}\StringTok{\textquotesingle{}其他\textquotesingle{}}\NormalTok{)))))}


\CommentTok{\# same above}
\CommentTok{\# case\_when(num \textless{}= 50 \textasciitilde{} \textquotesingle{}1{-}50\textquotesingle{},}
\CommentTok{\#           num \textless{}= 100 \textasciitilde{} \textquotesingle{}51{-}100\textquotesingle{},}
\CommentTok{\#           num \textless{}= 150 \textasciitilde{} \textquotesingle{}101{-}150\textquotesingle{},}
\CommentTok{\#           num \textless{}= 200 \textasciitilde{} \textquotesingle{}151{-}200\textquotesingle{},}
\CommentTok{\#           num \textgreater{} 100 \textasciitilde{} \textquotesingle{}200以上\textquotesingle{}}
\CommentTok{\#           )}

\CommentTok{\# 个人倾向data.table }
\CommentTok{\# data.table::fifelse()}
\CommentTok{\# data.table::fcase() 是sql中case when的实现}
\end{Highlighting}
\end{Shaded}

函数化后代码示例如下:

当需要修改区间时候仅仅只需要调整参数,而不必大量修改代码,当在脚本中需要调用多次时,能简洁代码.

\begin{Shaded}
\begin{Highlighting}[]
\CommentTok{\# 排名区间函数}
\CommentTok{\#library(tidyverse)}
\NormalTok{cut\_function }\OtherTok{\textless{}{-}} \ControlFlowTok{function}\NormalTok{(vecto,x,n)\{}
\NormalTok{  vec }\OtherTok{\textless{}{-}} \FunctionTok{c}\NormalTok{(}\DecValTok{0}\NormalTok{)}
  \ControlFlowTok{for}\NormalTok{(i }\ControlFlowTok{in} \DecValTok{1}\SpecialCharTok{:}\NormalTok{n)\{}
\NormalTok{    kong }\OtherTok{\textless{}{-}}\NormalTok{  i}\SpecialCharTok{*}\NormalTok{x}
\NormalTok{    vec }\OtherTok{\textless{}{-}} \FunctionTok{c}\NormalTok{(vec,kong)}
\NormalTok{  \}}
\NormalTok{  vec }\OtherTok{\textless{}{-}} \FunctionTok{c}\NormalTok{(vec,}\ConstantTok{Inf}\NormalTok{)}
\NormalTok{  labels }\OtherTok{\textless{}{-}} \FunctionTok{c}\NormalTok{()}
\NormalTok{  j }\OtherTok{\textless{}{-}} \DecValTok{1}
  
  \ControlFlowTok{while}\NormalTok{ (j}\SpecialCharTok{\textless{}=}\NormalTok{n) \{}
\NormalTok{    labels[j] }\OtherTok{\textless{}{-}} \FunctionTok{str\_c}\NormalTok{(vec[j]}\SpecialCharTok{+}\DecValTok{1}\NormalTok{,}\StringTok{"{-}"}\NormalTok{,vec[j}\SpecialCharTok{+}\DecValTok{1}\NormalTok{])}
\NormalTok{    j }\OtherTok{\textless{}{-}}\NormalTok{ j}\SpecialCharTok{+}\DecValTok{1}
\NormalTok{  \}}
\NormalTok{  labels }\OtherTok{\textless{}{-}} \FunctionTok{c}\NormalTok{(labels,}\FunctionTok{paste0}\NormalTok{(vec[j],}\StringTok{\textquotesingle{}以上\textquotesingle{}}\NormalTok{))}
\NormalTok{  res }\OtherTok{\textless{}{-}} \FunctionTok{cut}\NormalTok{(}\AttributeTok{x =}\NormalTok{ vecto,}\AttributeTok{breaks =}\NormalTok{ vec,}\AttributeTok{labels =}\NormalTok{ labels) }\SpecialCharTok{\%\textgreater{}\%} \FunctionTok{as.character}\NormalTok{()}
\NormalTok{\}}

\NormalTok{res2 }\OtherTok{\textless{}{-}} \FunctionTok{cut\_function}\NormalTok{(num,}\DecValTok{50}\NormalTok{,}\DecValTok{4}\NormalTok{)}

\CommentTok{\# identical(res1,res2)}
\CommentTok{\# \textgreater{} TRUE}
\end{Highlighting}
\end{Shaded}

\href{https://r4ds.had.co.nz/functions.html}{参考资料}

\hypertarget{ux7b80ux5355ux793aux4f8b-1}{%
\section{简单示例}\label{ux7b80ux5355ux793aux4f8b-1}}

给函数取一个合适名字是很难的事情,徐尽可能从函数名称看出你实现的功能.

\begin{Shaded}
\begin{Highlighting}[]
\NormalTok{add\_ten }\OtherTok{\textless{}{-}} \ControlFlowTok{function}\NormalTok{(x)\{}
\NormalTok{  res }\OtherTok{\textless{}{-}}\NormalTok{ x}\SpecialCharTok{+}\DecValTok{10}
  \FunctionTok{return}\NormalTok{(res) }\CommentTok{\#可以不用显示返回}
\NormalTok{\}}
\FunctionTok{add\_ten}\NormalTok{(}\DecValTok{1}\NormalTok{)}
\end{Highlighting}
\end{Shaded}

写函数时需要考虑函数使用情况,尽可能考虑容错情况,当输入不符合预期时能友好提示错误.

\begin{Shaded}
\begin{Highlighting}[]
\NormalTok{add\_ten }\OtherTok{\textless{}{-}} \ControlFlowTok{function}\NormalTok{(x)\{}
  \ControlFlowTok{if}\NormalTok{(}\FunctionTok{is.numeric}\NormalTok{(x)}\SpecialCharTok{==}\ConstantTok{TRUE}\NormalTok{)\{}
\NormalTok{    x}\SpecialCharTok{+}\DecValTok{10}
\NormalTok{  \} }\ControlFlowTok{else}\NormalTok{ \{}
    \FunctionTok{print}\NormalTok{(}\StringTok{\textquotesingle{}Error,请输入数字\textquotesingle{}}\NormalTok{)}
\NormalTok{  \}}
\NormalTok{\}}
\end{Highlighting}
\end{Shaded}

\hypertarget{ux6761ux4ef6ux6267ux884c}{%
\section{条件执行}\label{ux6761ux4ef6ux6267ux884c}}

\begin{Shaded}
\begin{Highlighting}[]
\NormalTok{has\_name }\OtherTok{\textless{}{-}} \ControlFlowTok{function}\NormalTok{(x) \{}
\NormalTok{  nms }\OtherTok{\textless{}{-}} \FunctionTok{names}\NormalTok{(x)}
  \ControlFlowTok{if}\NormalTok{ (}\FunctionTok{is.null}\NormalTok{(nms)) \{}
    \FunctionTok{rep}\NormalTok{(}\ConstantTok{FALSE}\NormalTok{, }\FunctionTok{length}\NormalTok{(x))}
\NormalTok{  \} }\ControlFlowTok{else}\NormalTok{ \{}
    \SpecialCharTok{!}\FunctionTok{is.na}\NormalTok{(nms) }\SpecialCharTok{\&}\NormalTok{ nms }\SpecialCharTok{!=} \StringTok{""}
\NormalTok{  \}}
\NormalTok{\}}
\end{Highlighting}
\end{Shaded}

\hypertarget{ux591aux6761ux4ef6ux6267ux884c}{%
\subsection{多条件执行}\label{ux591aux6761ux4ef6ux6267ux884c}}

\begin{Shaded}
\begin{Highlighting}[]
\ControlFlowTok{if}\NormalTok{ (this) \{}
  \CommentTok{\# do that}
\NormalTok{\} }\ControlFlowTok{else} \ControlFlowTok{if}\NormalTok{ (that) \{}
  \CommentTok{\# do something else}
\NormalTok{\} }\ControlFlowTok{else}\NormalTok{ \{}
  \CommentTok{\# }
\NormalTok{\}}
\end{Highlighting}
\end{Shaded}

当需要很多if时可考虑用switch()功能

\begin{Shaded}
\begin{Highlighting}[]
\ControlFlowTok{function}\NormalTok{(x, y, op) \{}
   \ControlFlowTok{switch}\NormalTok{(op,}
     \AttributeTok{plus =}\NormalTok{ x }\SpecialCharTok{+}\NormalTok{ y,}
     \AttributeTok{minus =}\NormalTok{ x }\SpecialCharTok{{-}}\NormalTok{ y,}
     \AttributeTok{times =}\NormalTok{ x }\SpecialCharTok{*}\NormalTok{ y,}
     \AttributeTok{divide =}\NormalTok{ x }\SpecialCharTok{/}\NormalTok{ y,}
     \FunctionTok{stop}\NormalTok{(}\StringTok{"Unknown op!"}\NormalTok{)}
\NormalTok{   )}
\NormalTok{ \}}
\end{Highlighting}
\end{Shaded}

\hypertarget{ux51fdux6570ux53c2ux6570-1}{%
\section{函数参数}\label{ux51fdux6570ux53c2ux6570-1}}

函数的参数通常分为两大类,一组是提供要计算的参数,另外一组提供计算时的细节参数.

\begin{Shaded}
\begin{Highlighting}[]
\NormalTok{mean\_ci }\OtherTok{\textless{}{-}} \ControlFlowTok{function}\NormalTok{(x, }\AttributeTok{conf =} \FloatTok{0.95}\NormalTok{) \{}
\NormalTok{  se }\OtherTok{\textless{}{-}} \FunctionTok{sd}\NormalTok{(x) }\SpecialCharTok{/} \FunctionTok{sqrt}\NormalTok{(}\FunctionTok{length}\NormalTok{(x))}
\NormalTok{  alpha }\OtherTok{\textless{}{-}} \DecValTok{1} \SpecialCharTok{{-}}\NormalTok{ conf}
  \FunctionTok{mean}\NormalTok{(x) }\SpecialCharTok{+}\NormalTok{ se }\SpecialCharTok{*} \FunctionTok{qnorm}\NormalTok{(}\FunctionTok{c}\NormalTok{(alpha }\SpecialCharTok{/} \DecValTok{2}\NormalTok{, }\DecValTok{1} \SpecialCharTok{{-}}\NormalTok{ alpha }\SpecialCharTok{/} \DecValTok{2}\NormalTok{))}
\NormalTok{\}}
\NormalTok{x }\OtherTok{\textless{}{-}} \FunctionTok{runif}\NormalTok{(}\DecValTok{100}\NormalTok{)}
\FunctionTok{mean\_ci}\NormalTok{(x)}
\FunctionTok{mean\_ci}\NormalTok{(x, }\AttributeTok{conf =} \FloatTok{0.99}\NormalTok{)}
\end{Highlighting}
\end{Shaded}

\hypertarget{ux53c2ux6570ux540dux79f0}{%
\subsection{参数名称}\label{ux53c2ux6570ux540dux79f0}}

参数的名称很重要,方便我们理解参数含义,调用时不会混乱.以下时几个重要的参数名称

\begin{itemize}
\tightlist
\item
  x, y, z: vectors.
\item
  w: a vector of weights.
\item
  df: a data frame.
\item
  i, j: numeric indices (typically rows and columns).
\item
  n: length, or number of rows.
\item
  p: number of columns.
\end{itemize}

\hypertarget{ux68c0ux67e5ux53c2ux6570ux503c}{%
\subsection{检查参数值}\label{ux68c0ux67e5ux53c2ux6570ux503c}}

在写函数时,并不清楚最终函数的输出,在编写函数时进行约束是有必要的.

\begin{Shaded}
\begin{Highlighting}[]
\NormalTok{wt\_mean }\OtherTok{\textless{}{-}} \ControlFlowTok{function}\NormalTok{(x, w) \{}
  \ControlFlowTok{if}\NormalTok{ (}\FunctionTok{length}\NormalTok{(x) }\SpecialCharTok{!=} \FunctionTok{length}\NormalTok{(w)) \{}
    \FunctionTok{stop}\NormalTok{(}\StringTok{"\textasciigrave{}x\textasciigrave{} and \textasciigrave{}w\textasciigrave{} must be the same length"}\NormalTok{, }\AttributeTok{call. =} \ConstantTok{FALSE}\NormalTok{)}
\NormalTok{  \}}
  \FunctionTok{sum}\NormalTok{(w }\SpecialCharTok{*}\NormalTok{ x) }\SpecialCharTok{/} \FunctionTok{sum}\NormalTok{(w)}
\NormalTok{\}}
\end{Highlighting}
\end{Shaded}

\hypertarget{ux53c2ux6570-1}{%
\subsection{\ldots 参数}\label{ux53c2ux6570-1}}

R中的许多函数都能接受任意数量的输入:

\begin{Shaded}
\begin{Highlighting}[]
\FunctionTok{sum}\NormalTok{(}\DecValTok{1}\NormalTok{,}\DecValTok{2}\NormalTok{,}\DecValTok{3}\NormalTok{,}\DecValTok{4}\NormalTok{,}\DecValTok{5}\NormalTok{,}\DecValTok{6}\NormalTok{,}\DecValTok{7}\NormalTok{,}\DecValTok{8}\NormalTok{,}\DecValTok{9}\NormalTok{,}\DecValTok{10}\NormalTok{)}
\NormalTok{stringr}\SpecialCharTok{::}\FunctionTok{str\_c}\NormalTok{(}\StringTok{\textquotesingle{}a\textquotesingle{}}\NormalTok{,}\StringTok{\textquotesingle{}b\textquotesingle{}}\NormalTok{,}\StringTok{\textquotesingle{}d\textquotesingle{}}\NormalTok{,}\StringTok{\textquotesingle{}e\textquotesingle{}}\NormalTok{,}\StringTok{\textquotesingle{}f\textquotesingle{}}\NormalTok{,}\StringTok{\textquotesingle{}g\textquotesingle{}}\NormalTok{,}\StringTok{\textquotesingle{}h\textquotesingle{}}\NormalTok{)}
\end{Highlighting}
\end{Shaded}

下面的例子中

\begin{Shaded}
\begin{Highlighting}[]
\NormalTok{commas }\OtherTok{\textless{}{-}} \ControlFlowTok{function}\NormalTok{(...) stringr}\SpecialCharTok{::}\FunctionTok{str\_c}\NormalTok{(..., }\AttributeTok{collapse =} \StringTok{", "}\NormalTok{)}
\FunctionTok{commas}\NormalTok{(letters[}\DecValTok{1}\SpecialCharTok{:}\DecValTok{10}\NormalTok{])}
\CommentTok{\#\textgreater{} [1] "a, b, c, d, e, f, g, h, i, j"}

\NormalTok{rule }\OtherTok{\textless{}{-}} \ControlFlowTok{function}\NormalTok{(..., }\AttributeTok{pad =} \StringTok{"{-}"}\NormalTok{) \{}
\NormalTok{  title }\OtherTok{\textless{}{-}} \FunctionTok{paste0}\NormalTok{(...)}
\NormalTok{  width }\OtherTok{\textless{}{-}} \FunctionTok{getOption}\NormalTok{(}\StringTok{"width"}\NormalTok{) }\SpecialCharTok{{-}} \FunctionTok{nchar}\NormalTok{(title) }\SpecialCharTok{{-}} \DecValTok{5}
  \FunctionTok{cat}\NormalTok{(title, }\StringTok{" "}\NormalTok{, stringr}\SpecialCharTok{::}\FunctionTok{str\_dup}\NormalTok{(pad, width), }\StringTok{"}\SpecialCharTok{\textbackslash{}n}\StringTok{"}\NormalTok{, }\AttributeTok{sep =} \StringTok{""}\NormalTok{)}
\NormalTok{\}}
\FunctionTok{rule}\NormalTok{(}\StringTok{"Important output"}\NormalTok{)}
\end{Highlighting}
\end{Shaded}

\hypertarget{ux8fd4ux56deux503c}{%
\section{返回值}\label{ux8fd4ux56deux503c}}

\hypertarget{ux663eux5f0fux8fd4ux56de}{%
\subsection{显式返回}\label{ux663eux5f0fux8fd4ux56de}}

函数返回的通常是最后一句代码的计算结果,可以显式利用return()提前返回。但是R for Data Science 中作者说:
`我认为最好不要使用return()来表示,您可以使用更简单的解决方案尽早返回'

\begin{itemize}
\tightlist
\item
  A common reason to do this is because the inputs are empty:
\end{itemize}

\begin{Shaded}
\begin{Highlighting}[]
\NormalTok{complicated\_function }\OtherTok{\textless{}{-}} \ControlFlowTok{function}\NormalTok{(x, y, z) \{}
  \ControlFlowTok{if}\NormalTok{ (}\FunctionTok{length}\NormalTok{(x) }\SpecialCharTok{==} \DecValTok{0} \SpecialCharTok{||} \FunctionTok{length}\NormalTok{(y) }\SpecialCharTok{==} \DecValTok{0}\NormalTok{) \{}
    \FunctionTok{return}\NormalTok{(}\DecValTok{0}\NormalTok{)}
\NormalTok{  \}}
  \CommentTok{\# Complicated code here}
\NormalTok{\}}
\end{Highlighting}
\end{Shaded}

\begin{itemize}
\tightlist
\item
  Another reason is because you have a if statement with one complex block and one simple block. For example, you might write an if statement like this:
\end{itemize}

\begin{Shaded}
\begin{Highlighting}[]
\NormalTok{f }\OtherTok{\textless{}{-}} \ControlFlowTok{function}\NormalTok{() \{}
  \ControlFlowTok{if}\NormalTok{ (x) \{}
    \CommentTok{\# Do }
    \CommentTok{\# something}
    \CommentTok{\# that}
    \CommentTok{\# takes}
    \CommentTok{\# many}
    \CommentTok{\# lines}
    \CommentTok{\# to}
    \CommentTok{\# express}
\NormalTok{  \} }\ControlFlowTok{else}\NormalTok{ \{}
    \CommentTok{\# return something short}
\NormalTok{  \}}
\NormalTok{\}}
\end{Highlighting}
\end{Shaded}

\hypertarget{ux7f16ux5199ux7ba1ux9053ux51fdux6570}{%
\subsection{编写管道函数}\label{ux7f16ux5199ux7ba1ux9053ux51fdux6570}}

管道函数有两种基本类型: transformations and side-effects。使用transformations时,会将对象传递到函数的第一个参数,然后返回修改后的对象。使用side-effects时,不会对传递的对象进行转换。相反,该函数对对象执行操作,例如绘制图或保存文件。副作用函数应该``无形地''返回第一个参数,以便在不打印它们时仍可以在管道中使用它们。例如,以下简单函数在数据框中打印缺失值的数量:

以上从 R for Data Science 中翻译得来。

\begin{Shaded}
\begin{Highlighting}[]
\NormalTok{show\_missings }\OtherTok{\textless{}{-}} \ControlFlowTok{function}\NormalTok{(df) \{}
\NormalTok{  n }\OtherTok{\textless{}{-}} \FunctionTok{sum}\NormalTok{(}\FunctionTok{is.na}\NormalTok{(df))}
  \FunctionTok{cat}\NormalTok{(}\StringTok{"Missing values: "}\NormalTok{, n, }\StringTok{"}\SpecialCharTok{\textbackslash{}n}\StringTok{"}\NormalTok{, }\AttributeTok{sep =} \StringTok{""}\NormalTok{)}
  
  \FunctionTok{invisible}\NormalTok{(df)}
\NormalTok{\}}
\end{Highlighting}
\end{Shaded}

以交互invisible()方式调用它,则意味着输入df不会被打印出来:

\begin{Shaded}
\begin{Highlighting}[]
\FunctionTok{show\_missings}\NormalTok{(mtcars)}
\end{Highlighting}
\end{Shaded}

但是结果仍存在,默认情况下只是不打印显示出来:

\begin{Shaded}
\begin{Highlighting}[]
\NormalTok{x }\OtherTok{\textless{}{-}} \FunctionTok{show\_missings}\NormalTok{(mtcars) }
\FunctionTok{class}\NormalTok{(x)}
\FunctionTok{dim}\NormalTok{(x)}
\end{Highlighting}
\end{Shaded}

在管道中继续使用

\begin{Shaded}
\begin{Highlighting}[]
\NormalTok{mtcars }\SpecialCharTok{\%\textgreater{}\%} 
  \FunctionTok{show\_missings}\NormalTok{() }\SpecialCharTok{\%\textgreater{}\%} 
  \FunctionTok{mutate}\NormalTok{(}\AttributeTok{mpg =} \FunctionTok{ifelse}\NormalTok{(mpg }\SpecialCharTok{\textless{}} \DecValTok{20}\NormalTok{, }\ConstantTok{NA}\NormalTok{, mpg)) }\SpecialCharTok{\%\textgreater{}\%} 
  \FunctionTok{show\_missings}\NormalTok{() }
\end{Highlighting}
\end{Shaded}

\hypertarget{ux73afux5883}{%
\section{环境}\label{ux73afux5883}}

环境是复杂的,建议阅读原文.

The last component of a function is its environment. This is not something you need to understand deeply when you first start writing functions. However, it's important to know a little bit about environments because they are crucial to how functions work. The environment of a function controls how R finds the value associated with a name. For example, take this function:

\begin{Shaded}
\begin{Highlighting}[]
\NormalTok{f }\OtherTok{\textless{}{-}} \ControlFlowTok{function}\NormalTok{(x) \{}
\NormalTok{  x }\SpecialCharTok{+}\NormalTok{ y}
\NormalTok{\} }
\end{Highlighting}
\end{Shaded}

在很多其他的编程语言中这样定义函数是错误的,因为没有定义\texttt{y}.在R中,这是有效的代码,因为R使用称为\texttt{lexical\ scoping}的方式寻找关联值.在函数内部没有定义\texttt{y},将在上一层环境中查看\texttt{y}:

\begin{Shaded}
\begin{Highlighting}[]
\NormalTok{y }\OtherTok{\textless{}{-}} \DecValTok{100}
\FunctionTok{f}\NormalTok{(}\DecValTok{10}\NormalTok{)}

\NormalTok{y }\OtherTok{\textless{}{-}} \DecValTok{1000}
\FunctionTok{f}\NormalTok{(}\DecValTok{10}\NormalTok{)}
\end{Highlighting}
\end{Shaded}

具体详细的资料请查阅:

\url{https://r4ds.had.co.nz/functions.html\#environment}

\url{http://adv-r.had.co.nz/}

\hypertarget{ux62d3ux5c55ux90e8ux5206}{%
\section{拓展部分}\label{ux62d3ux5c55ux90e8ux5206}}

在我之前工作中遇到需要分组计算时,我想要编写一个函数实现某些功能,但是分组的group\_by()字段不一样时,导致代码没办法复用。

参考资料:\url{https://dplyr.tidyverse.org/articles/programming.html}

\begin{Shaded}
\begin{Highlighting}[]
\CommentTok{\#library(tidyverse)}
\NormalTok{mean\_mpg }\OtherTok{=} \ControlFlowTok{function}\NormalTok{(data, group\_col) \{}
\NormalTok{  data }\SpecialCharTok{\%\textgreater{}\%} 
    \FunctionTok{group\_by}\NormalTok{(group\_col) }\SpecialCharTok{\%\textgreater{}\%}
    \FunctionTok{summarize}\NormalTok{(}\AttributeTok{mean\_mpg =} \FunctionTok{mean}\NormalTok{(mpg))}
\NormalTok{\}}
\NormalTok{mtcars }\SpecialCharTok{\%\textgreater{}\%} \FunctionTok{mean\_mpg}\NormalTok{(cyl)}
\NormalTok{mtcars }\SpecialCharTok{\%\textgreater{}\%} \FunctionTok{mean\_mpg}\NormalTok{(gear)}
\end{Highlighting}
\end{Shaded}

当编写如下函数时,代码将成功运行

\begin{Shaded}
\begin{Highlighting}[]
\CommentTok{\#自定义函数}
\NormalTok{my\_summarise3 }\OtherTok{\textless{}{-}} \ControlFlowTok{function}\NormalTok{(data, group\_var,mean\_var, sd\_var) \{}
\NormalTok{  data }\SpecialCharTok{\%\textgreater{}\%} 
    \FunctionTok{group\_by}\NormalTok{(\{\{ group\_var \}\}) }\SpecialCharTok{\%\textgreater{}\%} 
    \FunctionTok{summarise}\NormalTok{(}\AttributeTok{mean =} \FunctionTok{mean}\NormalTok{(\{\{ mean\_var \}\}), }\AttributeTok{sd =} \FunctionTok{mean}\NormalTok{(\{\{ sd\_var \}\}))}
\NormalTok{\}}

\NormalTok{res1 }\OtherTok{\textless{}{-}} \FunctionTok{my\_summarise3}\NormalTok{(}\AttributeTok{data =}\NormalTok{ mtcars,}\AttributeTok{group\_var =}\NormalTok{ cyl,}\AttributeTok{mean\_var =}\NormalTok{ carb,}\AttributeTok{sd\_var =}\NormalTok{ gear)}
\FunctionTok{my\_summarise3}\NormalTok{(}\AttributeTok{data =}\NormalTok{ mtcars,}\AttributeTok{group\_var =}\NormalTok{ am,}\AttributeTok{mean\_var =}\NormalTok{ carb,}\AttributeTok{sd\_var =}\NormalTok{ gear)}
\CommentTok{\#正常写法}
\NormalTok{res2 }\OtherTok{\textless{}{-}}\NormalTok{ mtcars }\SpecialCharTok{\%\textgreater{}\%} 
  \FunctionTok{group\_by}\NormalTok{(cyl) }\SpecialCharTok{\%\textgreater{}\%} 
  \FunctionTok{summarise}\NormalTok{(}\AttributeTok{mean=}\FunctionTok{mean}\NormalTok{(carb),}\AttributeTok{sd=}\FunctionTok{mean}\NormalTok{(gear))}

\FunctionTok{identical}\NormalTok{(res1,res2)}

\CommentTok{\#res1 和res2 结果完全一致}
\end{Highlighting}
\end{Shaded}

以上my\_summarise3()函数可以按照需求任意指定聚合汇总字段。

  \bibliography{book.bib,packages.bib}

\end{document}
